\documentclass[main.tex]{subfiles}

\begin{document}
\chapter{Formulario}
Qui sono elencate alcune formule che risultano utili in generale nello studio e nella risoluzione della materia.
\\
Sebbene dovrebbero essere cose note, ripresento queste piccole formule perchè sono necessarie per lo svolgimento di esercizi.
\section{Formule per la cinematica}
\begin{figure}[H]
  \begin{subfigure}[b]{.5\textwidth}
  \centering
\[
	\vec{v} = \vec{\omega}\times\vec{r} = \omega r \sin\theta
\]
  \caption{Legame cinematico tra $\vec{v}$ e $\vec{\omega}$}
  \end{subfigure}
  \hfill
  \begin{subfigure}[b]{.5\textwidth}
  \centering
\[
	\vec{a}_t = \vec{\dot{\omega}}\times\vec{r} = \dot{\omega} r \sin\theta
\]
  \caption{Legame cinematico tra $\vec{a}_t$ e $\vec{\dot{\omega}}$}
  \end{subfigure}
  \caption{Legami cinematici}
\end{figure}

\begin{figure}[H]
  \centering
\[
	a_n = \omega^2 r = \omega v = \dfrac{v^2}{r}
\]
  \caption{Accelerazione centripeta}
\end{figure}
\subsection{Brevissimo formulario per le derivate}
\begin{figure}[H]
  \begin{subfigure}[b]{.5\textwidth}
  \centering
\[
	\dfrac{d\sin\alpha}{dt} = \cos\alpha
\]
  \caption{Derivata di seno.}
  \end{subfigure}
  \hfill
  \begin{subfigure}[b]{.5\textwidth}
  \centering
\[
	\dfrac{d\cos\alpha}{dt} = -\sin\alpha
\]
  \caption{Derivata di coseno.}
  \end{subfigure}
\end{figure}

\begin{figure}[H]
  \centering
\[
	\dfrac{d(a(t)b(t)c(t))}{dt} = \dot{a}(t)b(t)c(t) + a(t)\dot{b}(t)c(t) + a(t)b(t)\dot{c}(t)
\]
  \caption{Derivata di prodotto.}
\end{figure}

\begin{figure}[H]
  \centering
\[
	\dfrac{a(b(t))}{dt} = \dot{b}(t)\dot{a}(b(t))
\]
  \caption{Derivata di funzione composta.}
\end{figure}

\subsection{Brevissimo formulario per numeri complessi}
\begin{figure}[H]
  \begin{subfigure}[b]{.5\textwidth}
  \centering
\[
	ie^{i\alpha} = e^{i\left (\dfrac{\pi}{2} + \alpha \right )}
\]
  \caption{Traslazione dell'esponente in exp. complessa.}
  \end{subfigure}
  \hfill
  \begin{subfigure}[b]{.5\textwidth}
  \centering
\[
	\dfrac{d(e^{i\alpha})}{dt} = i\dot{\alpha}e^{i\alpha} = \dot{\alpha}e^{i\left (\dfrac{\pi}{2} + \alpha \right )}
\]
  \caption{Derivare exp. complessa.}
  \end{subfigure}
\end{figure}
\section{Formule per la statica}
\begin{figure}[H]
  \centering
\[
	F_{el} = k(l_o - l)
\]
  \caption{Forza elastica}
\end{figure}

\begin{figure}[H]
  \centering
\[
	gdv_{cerniera} = 2(n_{aste} - 1)
\]
  \caption{Gradi di vincolo di cerniera interna}
\end{figure}


\section{Formule per motori}
\begin{figure}[H]
  \begin{subfigure}[b]{.5\textwidth}
  \centering
\[
	W_m + W_r + W_p = \dfrac{dE_c}{dt}
\]
  \caption{Bilancio di potenze}
  \end{subfigure}
  \hfill
  \begin{subfigure}[b]{.5\textwidth}
  \centering
\begin{align*}
	\dot{\omega}_m = 0 \qquad \dfrac{dE_c}{dt} = 0
\end{align*}
  \caption{Condizione di regime}
  \end{subfigure}
\end{figure}

\begin{figure}[H]
  \begin{subfigure}[b]{.5\textwidth}
  \centering
\[
	W_m = C_m\bullet\omega_m
\]
  \caption{Potenza motrice}
  \end{subfigure}
  \hfill
  \begin{subfigure}[b]{.5\textwidth}
  \centering
\begin{align*}
	W_r = 	\sum &(\text{Forze applicate su corpi in moto})\\&\bullet(\text{velocità baricentriche dei corpi})
\end{align*}
  \caption{Potenza resistente o utilizzatore}
  \end{subfigure}
\end{figure}

\begin{figure}[H]
  \begin{subfigure}[b]{.5\textwidth}
  \centering
\[
	\dfrac{dE_c}{dt} = \dfrac{dE_{c_r}}{dt} + \dfrac{dE_{c_m}}{dt}
\]
  \caption{Variazione di energia cinetica}
  \end{subfigure}
  \hfill
  \begin{subfigure}[b]{.5\textwidth}
  \centering
\[
	\dfrac{dE_{c_m}}{dt} = \omega_m \dot{\omega}_m J_m
\]
  \caption{Variazione di energia cinetica motrice}
  \end{subfigure}
\end{figure}

\begin{figure}[H]
  \begin{subfigure}[b]{.5\textwidth}
  \centering
\[
	W_p = -(1-\mu_D)(W_m - \dfrac{dE_{c_m}}{dt})
\]
  \caption{Potenza perduta per moto diretto}
  \end{subfigure}
  \hfill
  \begin{subfigure}[b]{.5\textwidth}
  \centering
\[
	W_p = -(1-\mu_R)(W_r - \dfrac{dE_{c_r}}{dt})
\]
  \caption{Potenza perduta per moto retrogrado}
  \end{subfigure}
\end{figure}


\end{document}
