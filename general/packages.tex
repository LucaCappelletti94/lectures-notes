\documentclass{report} % We use report to allow for chapers etc...
\usepackage[utf8]{inputenc} % This allows for utf support.
\usepackage[T1]{fontenc}  % Defines true type fonts

\usepackage{iftex} % Used for if
\ifLuaTeX
	\usepackage{polyglossia}
	\setdefaultlanguage{italian}
	\setotherlanguage{english}
	\usepackage{fontspec}
\else
	\usepackage[italian, english]{babel}   % Set up supported languages. Last one is default.
\fi

\usepackage{geometry} % More rich support for page layout.
\geometry{
	a4paper,
	total={170mm,250mm},
	left=20mm,
	top=25mm,
}
\usepackage{emptypage} % When a page is empty, Latex won't generate page number or other page elements.
\usepackage{multicol} % For the possibility of using columns with  \begin{multicols}{n}.
\usepackage[colorlinks=true,urlcolor=blue,pdfpagelabels,hyperindex=false]{hyperref}  % Enable table of contents and links.
\usepackage{microtype} % for automatic micro fitting of characters
\usepackage{amsmath} % for math rendering
\usepackage{float} % to enable floating graphics
\usepackage{graphicx} % for images and generally graphics
\usepackage{caption} % enabling of nice captions
\usepackage{subcaption} % and subcaptions of images
\graphicspath{ {images/} } % We'll put all the images in the folder "images"
\usepackage[style=authoryear,sorting=ynt, backend=bibtex]{biblatex} % Package to handle the bibliography
\nocite{*} % This allows for having entried in the bib file that do not have to be necesseraly used
\usepackage{subfiles} % To use subfiles without cruxifying saints
\usepackage{chngcntr} % Has functions to reset counters

\usepackage{parskip} % To leave spaces in paragraphs
\usepackage{bm} % To have bold vectors
\usepackage{\main/../../packages/soul} % To cancel text with a line using the commant \st, to underline text and highlight.

\usepackage{xcolor} % To highlight text
\usepackage{xfrac} % To allow for sideways fractions
\usepackage[cache=false]{minted} % For highlighting code
\usepackage[algoruled,linesnumbered,titlenumbered,vlined]{\main/../../packages/algorithm2e} % To highlight pseudocode
\definecolor{mintedbackground}{rgb}{0.95,0.95,0.95}
\setminted{
	bgcolor=mintedbackground,
	fontfamily=tt,
	linenos=true,
	numberblanklines=true,
	numbersep=5pt,
	gobble=0,
	frame=leftline,
	framerule=0.4pt,
	framesep=2mm,
	funcnamehighlighting=true,
	tabsize=4,
	obeytabs=false,
	mathescape=false
	samepage=false, %with this setting you can force the list to appear on the same page
	showspaces=false,
	showtabs =false,
	texcl=false,
}

% CDQUOTES HAS TO BE LOADED AFTER THE MINTED
\usepackage{csquotes} % Package required by babel AND polyglossia.

%%%%%%%%%%%%%%%%%%%%%%%%%
% GRAPHICAL EXTRAVAGANZA %
%%%%%%%%%%%%%%%%%%%%%%%%%

\usepackage{pgfplots} % to draw 3d graphs
\pgfplotsset{compat=1.7}
\usepackage{tikz}
\definecolor {processblue}{cmyk}{0.96,0,0,0}
\usetikzlibrary {positioning}
\def\checkmark{\tikz\fill[scale=0.4](0,.35) -- (.25,0) -- (1,.7) -- (.25,.15) -- cycle;}

\pgfmathdeclarefunction{gauss}{3}{%
	\pgfmathparse{1/(#3*sqrt(2*pi))*exp(-((#1-#2)^2)/(2*#3^2))}%
}

\usepackage{paralist} % For compacted enumerations
\usepackage[automake,style=long,nonumberlist,toc,acronym,nomain]{glossaries} % for glossaries and acronyms
\makeglossaries % Enables the package above

\usepackage{imakeidx} % instead of makeidx, so you don't need to run MakeIndex
\makeindex[program=makeindex,columns=2,intoc=true,options={-s ../../general/pyro.ist}] % Enables the package above
\indexsetup{firstpagestyle=empty, othercode=\small} % No page number in the first page of analytical index

\usepackage{bookmark} % Enables bookmarks in generated pdf

\usepackage{fourier} % For icons such as \bomb, \noway, \danger and various others. For more info, go here: http://ctan.mirror.garr.it/mirrors/CTAN/fonts/fourier-GUT/doc/latex/fourier/fourier-orns.pdf

\usepackage{mathtools} % To use custom delimiters that Latex did not implement cos duh
\renewcommand*{\arraystretch}{1.5} % Stretching arrays

\usepackage{fancyhdr} % This allows for the headings in the chapters
\pagestyle{fancy} % This activates it

\DeclarePairedDelimiter\abs{\lvert}{\rvert}%
\DeclarePairedDelimiter\norm{\lVert}{\rVert}%

\usepackage{enumitem,amssymb} % For a todo check list
\newlist{todolist}{itemize}{2} % For a  todo check list
\setlist[todolist]{label=$\square$}  % For a todo check list

\counterwithin*{equation}{section}% Reset equation at \section
\counterwithin*{equation}{subsection}% Reset equation at \subsection
%\counterwithin*{equation}{subsubsection}% Reset equation at \subsubsection // Doesn't work for god knows why.

%%%%%%%%%%%%%%%%%%%%%%%%%
% IMPORTANT NOTE ON BIBLIOGRAPHY USE %
%%%%%%%%%%%%%%%%%%%%%%%%%

% To use the bibliography, run BibTex on this file. If everything is correct, it will terminate with no errors.
% The warning "Using fall-back BibTeX(8) backend: (biblatex) functionality may be reduced/unavailable." is expected.

%%%%%%%%%%%%%%%%%%%%%%%%%
% LUATEX CODE %
%%%%%%%%%%%%%%%%%%%%%%%%%

\ifLuaTeX
	\usepackage{etoolbox} % To enable some parts of luacode
	\usepackage{luacode} % Used to script stuff
	\usepackage{luatex85}
	\providecommand{\samples}{80} % Drawing hd graphs when compiling for the final time (production)
	\begin{luacode}
--[[Require library for Lua library]]
require("lualibs.lua")

function tableMerge(t1, t2)
    for k,v in pairs(t2) do
        if type(v) == "table" then
            if type(t1[k] or false) == "table" then
                tableMerge(t1[k] or {}, t2[k] or {})
            else
                t1[k] = v
            end
        else
            t1[k] = v
        end
    end
    return t1
end

--[[Opens the two metadata file]]
local specificFile = io.open('metadata.json')
local folderFile = io.open('../metadata.json')
local genericFile = io.open('../../metadata.json')

--[[Reads the files]]
local specificJsonString = specificFile:read('*a')
local folderJsonString = folderFile:read('*a')
local generalJsonString = genericFile:read('*a')

--[[Closes the files]]
specificFile.close()
folderFile.close()
genericFile.close()

--[[Convert the Json strings in Lua dictionaries]]
local specificJson =  utilities.json.tolua(specificJsonString)
local folderJson =  utilities.json.tolua(folderJsonString)
local generalJson =  utilities.json.tolua(generalJsonString)

--[[Merge top layer of dictionaries, so that the specific one overrides the generic one.]]

metadata = tableMerge(tableMerge(generalJson, folderJson), specificJson)

\end{luacode}
	\begin{luacode}
if true then
	tex.print("\\renewcommand{\chaptername}{Capitolo}
\renewcommand{\contentsname}{Indice}
\newtheorem{theorem}{Teorema}[section]
\newtheorem{corollary}{Corollario}[theorem]
\newtheorem{lemma}[theorem]{Lemma}
\newtheorem{definition}[theorem]{Definizione}")
else
	tex.print("\\renewcommand{\chaptername}{Chapter}% So that it uses the italian version
\renewcommand{\contentsname}{Index}% So that it uses the italian version
\newtheorem{theorem}{Theorem}[section]
\newtheorem{corollary}{Corollary}[theorem]
\newtheorem{lemma}[theorem]{Lemma}
\newtheorem{definition}[theorem]{Definizione}")
end
\end{luacode}
\else
	% Lua not enabled
	\providecommand{\samples}{50} % Drawing ld graphs when compiling for development
	\renewcommand{\chaptername}{Capitolo}
\renewcommand{\contentsname}{Indice}
\newtheorem{theorem}{Teorema}[section]
\newtheorem{corollary}{Corollario}[theorem]
\newtheorem{lemma}[theorem]{Lemma}
\newtheorem{definition}[theorem]{Definizione}
\fi

%%%%%%%%%%%%%%%%%%%%%%%%%
% EXTERNAL FILES %
%%%%%%%%%%%%%%%%%%%%%%%%%

\newacronym{fol}{FOL}{First Order Logic}
\addbibresource{references.bib}

\usepgfplotslibrary{external}
\tikzexternalize