% !TEX options=--shell-escape
% !TEX options=--enable-write18
% !TEX options=-enable-write18
\documentclass{report} % We use report to allow for chapers etc...
\usepackage{etoolbox}

\usepackage{iftex} % Used for if
\ifLuaTeX
	\usepackage{shellesc}
	\usepackage{polyglossia}
	\setotherlanguage{english}
	\setdefaultlanguage{italian}
	\usepackage{fontspec}
	\usepackage{luacode} % Used to script stuff
	\usepackage{luatex85}
	\usepackage{luaotfload}
	\usepackage{shellesc}
\else
	\usepackage[utf8]{inputenc} % This allows for utf support.
	\usepackage[english, italian]{babel}   % Set up supported languages. Last one is default.
\fi

\usepackage[T1]{fontenc}  % Defines true type fonts

\let\oldtextbf\textbf % This is a backup for options may edit \textbf
\let\oldtextit\textit % This is a backup for options may edit \textit
\let\oldemph\emph % This is a backup for options may edit \emph

\usepackage{silence}
%Disable all warnings issued by latex starting with "You have..."
\WarningFilter{latex}{You have requested package}
\WarningFilter{latex}{There were undefined references}
\WarningFilter{latex}{Size substitutions with differences}
\WarningFilter{latex}{Command}
\WarningFilter{biblatex}{Using fall-back BibTeX(8)}
\WarningFilter{biblatex}{Please (re)run BibTeX on the file(s)}
\WarningFilter{auxhook}{Cannot patch}
\WarningFilter{glossaries}{No \printglossary or \printglossaries found.}

\usepackage{geometry} % More rich support for page layout.
\geometry{
	a4paper,
	total={170mm,250mm},
	left=20mm,
	top=25mm,
}

\usepackage[x11names,table]{xcolor} % To highlight text

\usepackage{amsmath,amsthm,amsfonts,amssymb}
\theoremstyle{definition} % Sets the theorem style without italic

\usepackage{emptypage} % When a page is empty, Latex won't generate page number or other page elements.

\usepackage{multicol} % For the possibility of using columns with  \begin{multicols}{n}.
\usepackage[colorlinks=true,urlcolor=blue,pdfpagelabels,hyperindex=false]{hyperref}  % Enable table of contents and links.
\usepackage{microtype} % for automatic micro fitting of characters
\usepackage{centernot}

\usepackage{framed}
\usepackage{mathtools}
\usepackage{float} % to enable floating graphics
\usepackage[style=authoryear,sorting=ynt, backend=bibtex]{biblatex} % Package to handle the bibliography
\nocite{*} % This allows for having entried in the bib file that do not have to be necesseraly used
\usepackage{subfiles} % To use subfiles without cruxifying saints
\usepackage{chngcntr} % Has functions to reset counters

\usepackage{parskip} % To leave spaces in paragraphs
\usepackage{bm} % To have bold vectors
\usepackage{soul} % To cancel text with a line using the commant \st, to underline text and highlight.

\usepackage{sectsty}

\usepackage{xfrac} % To allow for sideways fractions
\usepackage[cache=true]{minted} % For highlighting code
\usepackage[algoruled,linesnumbered,titlenumbered,vlined]{\main/../../packages/algorithm2e} % To highlight pseudocode
\definecolor{mintedbackground}{rgb}{0.95,0.95,0.95}
\setminted{
	bgcolor=mintedbackground,
	fontfamily=tt,
	linenos=true,
	numberblanklines=true,
	numbersep=5pt,
	gobble=0,
	frame=leftline,
	framerule=0.4pt,
	framesep=2mm,
	funcnamehighlighting=true,
	tabsize=4,
	obeytabs=false,
	mathescape=false
	samepage=false, %with this setting you can force the list to appear on the same page
	showspaces=false,
	showtabs =false,
	texcl=false,
}

% CDQUOTES HAS TO BE LOADED AFTER THE MINTED
\usepackage{csquotes} % Package required by babel AND polyglossia.

%%%%%%%%%%%%%%%%%%%%%%%%%
% GRAPHICAL EXTRAVAGANZA %
%%%%%%%%%%%%%%%%%%%%%%%%%

\usepackage{pgfplots} % to draw 3d graphs
\pgfplotsset{compat=1.14}
\usepackage{tikz}
\usepackage{relsize}
\usepackage{circuitikz}

\ctikzset{tripoles/mos style/arrows}
\ctikzset{tripoles/pmos style/emptycircle}
\ctikzset{tripoles/mos style/arrows}
\usetikzlibrary{shapes,arrows,calc,positioning,automata,matrix}
\usepgfplotslibrary{external}

\definecolor {processblue}{cmyk}{0.96,0,0,0}
\def\checkmark{\tikz\fill[scale=0.4](0,.35) -- (.25,0) -- (1,.7) -- (.25,.15) -- cycle;}

\pgfmathdeclarefunction{gauss}{3}{%
	\pgfmathparse{1/(#3*sqrt(2*pi))*exp(-((#1-#2)^2)/(2*#3^2))}%
}

\usepackage{paralist} % For compacted enumerations
\usepackage[automake,style=long,nonumberlist,toc,acronym,nomain]{glossaries} % for glossaries and acronyms
\makeglossaries % Enables the package above

\usepackage{imakeidx} % instead of makeidx, so you don't need to run MakeIndex
\makeindex[program=makeindex,columns=2,intoc=true,options={-s ../../general/pyro.ist}] % Enables the package above
\indexsetup{firstpagestyle=empty, othercode=\small} % No page number in the first page of analytical index

\usepackage{bookmark} % Enables bookmarks in generated pdf

\usepackage{fourier} % For icons such as \bomb, \noway, \danger and various others. For more info, go here: http://ctan.mirror.garr.it/mirrors/CTAN/fonts/fourier-GUT/doc/latex/fourier/fourier-orns.pdf

\usepackage{mathtools} % To use custom delimiters that Latex did not implement cos duh
\renewcommand*{\arraystretch}{1.5} % Stretching arrays

\usepackage{fancyhdr} % This allows for the headings in the chapters
\pagestyle{fancy} % This activates it

\usepackage[avantgarde]{quotchap} % Custom style for chapters

%%%%%%%%%%%%%%%%%%%%%%%%%%%%%%%%%%%%%%%%%%%%%%%%%%%%%%%%%%
% ALL THE PACKAGES FOR FIGURES   %
%%%%%%%%%%%%%%%%%%%%%%%%%%%%%%%%%%%%%%%%%%%%%%%%%%%%%%%%%%

\usepackage{graphicx} % for images and generally graphics
\usepackage{caption} % enabling of nice captions
\usepackage{subcaption} % and subcaptions of images
\graphicspath{ {\main/images/} } % We'll put all the images in the folder "images"

\setlength{\intextsep}{3pt plus 2pt minus 2pt}

%%%%%%%%%%%%%%%%%%%%%%%%%%%%%%%%%%%%%%%%%%%%%%%%%%%%%%%%%
% THE FOLLOWING CENTERS ALL FLOATING ITEMS BY DEFAULT   %
%%%%%%%%%%%%%%%%%%%%%%%%%%%%%%%%%%%%%%%%%%%%%%%%%%%%%%%%%

\makeatletter
\g@addto@macro\@floatboxreset\centering
\makeatother

\makeatletter
\apptocmd\subcaption@minipage{\centering}{}{}
\makeatother
\makeatletter
\providecommand*\setfloatlocations[2]{\@namedef{fps@#1}{#2}}
\makeatother
\setfloatlocations{figure}{H}
\setfloatlocations{table}{H}
\NewDocumentCommand{\fig}{m g g g g g g O{1} g}{%
  %
  % PARAMETERS MEANING
  %
  % #1 (Mandatory)                     -> First figure
  % #2 (Optional, with curly braces)   -> Second figure
  % #3 (Optional, with curly braces)   -> Third figure
  % #4 (Optional, with curly braces)   -> Forth figure
  % #5 (Optional, with curly braces)   -> Fifth figure
  % #6 (Optional, with curly braces)   -> Sixth figure
  % #7 (Optional, with curly braces)   -> Seventh figure
  % #8 (Optional, with squared braces) -> Maximum width, based on percentage of \textwidth.
  % #9 (Optional, with curly braces) -> Elements to add at the bottom, such as labels of caption for the group
  %
  %--------------------------------------
  %
  % USAGE EXAMPLES
  %
  % \fig{<content of figure 1 here>}{<content of figure 2 here>}{...}[Width I want]{<elements for bottom, such as caption>}
  %
  % IMPORTANT: For having the group caption evaluated correctly, one must insert also the width.
  %
  % Example with 1 figure:
  %
  % I want to show a kebab, with a width of 0.5 of the textwidth and as caption and label ``kebab''.
  %
  % \fig{\includegraphics{kebab.jpg}}[0.5]{
  %   \caption{kebab}
  %   \label{kebab}
  % }
  %
  % Example with 2 figures:
  %
  % I want to show two diagrams, one 3d and the projection, width each one occuping 0.5 of textwidth and with common label and caption ``graphs''.
  %
  % \fig{\includegraphics{graph3d.jpg}}{\includegraphics{graph2d.jpg}}[1]{
  %   \caption{graphs}
  %   \label{graphs}
  % }
  %
  %--------------------------------------
  %
  \FPset{\figuresNumber}{1}
  \IfValueT{#2}{
    \FPadd{\figuresNumber}{\figuresNumber}{1}
    \IfValueT{#3}{
      \FPadd{\figuresNumber}{\figuresNumber}{1}
      \IfValueT{#4}{
        \FPadd{\figuresNumber}{\figuresNumber}{1}
        \IfValueT{#5}{
          \FPadd{\figuresNumber}{\figuresNumber}{1}
          \IfValueT{#6}{
            \FPadd{\figuresNumber}{\figuresNumber}{1}
            \IfValueT{#7}{
              \FPadd{\figuresNumber}{\figuresNumber}{1}
            }
          }
        }
      }
    }
  }
  \FPdiv{\figureWidth}{#8}{\figuresNumber}
  \FPclip{\figureWidth}{\figureWidth}
  \smartif{\figuresNumber}{\one}[
    \begin{figure}
      \scalebox{\figureWidth\textwidth}{#1}
      \IfValueT{#9}{
        #9
      }
    \end{figure}
  ][
    \begin{figure}
      \begin{subfigure}{\figureWidth\textwidth}
      #1
      \end{subfigure}%
      \IfValueT{#2}{\begin{subfigure}{\figureWidth\textwidth}
      #2
      \end{subfigure}}%
      \IfValueT{#3}{\begin{subfigure}{\figureWidth\textwidth}#3\end{subfigure}}%
      \IfValueT{#4}{\begin{subfigure}{\figureWidth\textwidth}#4\end{subfigure}}%
      \IfValueT{#5}{\begin{subfigure}{\figureWidth\textwidth}#5\end{subfigure}}%
      \IfValueT{#6}{\begin{subfigure}{\figureWidth\textwidth}#6\end{subfigure}}%
      \IfValueT{#7}{\begin{subfigure}{\figureWidth\textwidth}#7\end{subfigure}}%
      \IfValueT{#9}{
        #9
      }
    \end{figure}
  ]

}%

%%%%%%%%%%%%%%%%%%%%%%%%%%%%%%%%%%%%%%%%%%%%%%%%%%%%%%%%%

\DeclarePairedDelimiter\abs{\lvert}{\rvert}
\DeclarePairedDelimiter\norm{\lVert}{\rVert}

\usepackage[shortlabels]{enumitem} % For a todo check list
\newlist{todolist}{itemize}{2} % For a  todo check list
% \setlist{nolistsep}
\setlist[todolist]{label=$\square$}  % For a todo check list

\counterwithin*{equation}{section}% Reset equation at \section
\counterwithin*{equation}{subsection}% Reset equation at \subsection
%\counterwithin*{equation}{subsubsection}% Reset equation at \subsubsection // Doesn't work for god knows why.

%%%%%%%%%%%%%%%%%%%%%%%%%
% IMPORTANT NOTE ON BIBLIOGRAPHY USE %
%%%%%%%%%%%%%%%%%%%%%%%%%

% To use the bibliography, run BibTex on this file. If everything is correct, it will terminate with no errors.
% The warning "Using fall-back BibTeX(8) backend: (biblatex) functionality may be reduced/unavailable." is expected.

%%%%%%%%%%%%%%%%%%%%%%%%%
% LUATEX CODE %
%%%%%%%%%%%%%%%%%%%%%%%%%

\ifLuaTeX
	\begin{luacode}
function fileExists(name)
   local f=io.open(name,"r")
   if f~=nil then io.close(f) return true else return false end
end

--[[We define 5 levels of recursive attempts]]
local levels = 10

--[[The expected file name is the following]]
local target = '/general/packages.tex'
deapness = '../'

for i=levels,1,-1 do
    --[[if we have a path that exists we exit]]
    if fileExists(deapness..target) then break end

    --[[We increase the path one level up]]
    deapness = '../'..deapness
end
\end{luacode}
	\providecommand{\samples}{150} % Drawing hd graphs when compiling for the final time (production)
	\directlua{dofile(deapness.."/general/lua/metadataLoader.lua")}
	\directlua{dofile(deapness.."/general/lua/languageSwitch.lua")}
	\directlua{dofile(deapness.."/general/lua/highlight.lua")}

	% THE FOLLOWING IS AN EXAMPLE ON HOW TO DEFINE A LUA COMMAND
	% \def\command{\directlua{dofile(deapness.."/general/lua/fileName.lua")}}
	% \command
\else
	% Lua not enabled
	\providecommand{\samples}{50} % Drawing ld graphs when compiling for development
	\renewcommand{\chaptername}{Capitolo}
\renewcommand{\contentsname}{Indice}
\newtheorem{theorem}{Teorema}[section]
\newtheorem{corollary}{Corollario}[theorem]
\newtheorem{lemma}[theorem]{Lemma}
\newtheorem{definition}[theorem]{Definizione}
\fi

%
% THE FOLLOWING CODE WRAPS THEOREMS IN A GRAY BOX
%
\colorlet{shadecolor}{gray!5}
\let\oldTheorem\theorem
\renewenvironment{theorem}{\begin{shaded}\begin{oldTheorem}}{\end{oldTheorem}\end{shaded}\ignorespacesafterend
}
%%%%%%%%%%%%%%%%%%%%%%%%%
% EXTERNAL FILES %
%%%%%%%%%%%%%%%%%%%%%%%%%

\DeclareMathOperator{\var}{Var}
\DeclareMathOperator*{\argmin}{argmin}
\DeclareMathOperator*{\argmax}{argmax}
%\DeclareMathOperator* this one allows to create environment that behave like sum or min

\newcommand{\crs}{\text{\Cross}}

\newcommand{\bra}[1]{\left\langle#1\right\rvert}
\newcommand{\ket}[1]{\left\lvert#1\right\rangle}
\newcommand{\braket}[2]{\left\langle#1\delimsize\vert#2\right\rangle}

\newcommand{\ceil}[1]{\left\lceil#1\right\rceil}
\newcommand{\floor}[1]{\left\lfloor#1\right\rfloor}

\newcommand{\abs}[1]{\left\lvert#1\right\rvert}
\newcommand{\norm}[1]{\left\lVert#1\right\rVert}
\newcommand{\rnd}[1]{\left(#1\right)}
\newcommand{\sqr}[1]{\left[#1\right]}
\newcommand{\crl}[1]{\left\{#1\right\}}

\let\oldbm\bm
\renewcommand{\bar}{\overline}
\renewcommand{\bm}[1]{\oldbm{\underline{#1}}}
\newcommand{\matr}[1]{\boldsymbol{#1}}
\newcommand{\bbm}[1]{\bar{\bm{#1}}}
\renewcommand{\geq}{\geqslant}
\renewcommand{\leq}{\leqslant}

\def\zero{0}
\def\one{1}
\def\negative{-}
\def\positive{+}
\def\false{\zero}
\def\true{\one}
\newcommand{\R}{\mathbb{R}}
\newcommand{\Z}{\mathbb{Z}}
\newcommand{\N}{\mathbb{N}}
\newcommand{\bs}{\backslash}

\NewDocumentCommand{\gmr}{m}{%
  %
  % #1 -> Value to build gomory cut upon
  %
  \rnd{#1-\floor{#1}}
}


\NewDocumentCommand{\buildVectorAliases}{m O{#1}}{
  %
  % #1 -> Letter to use for aliases
  % #2 -> Symbol to use as output
  %
  \expandafter\newcommand\csname bm#1\endcsname{\bm{#2}}%
  \expandafter\newcommand\csname bbm#1\endcsname{\bbm{#2}}%
  \expandafter\newcommand\csname bm#1t\endcsname{\bm{#2}^T}%
  \expandafter\newcommand\csname bbm#1t\endcsname{\bbm{#2}^T}%
}

\renewcommand{\a}{\alpha}
\newcommand{\w}{\omega}
\newcommand{\e}{\epsilon}

\buildVectorAliases{a}[\a]
\buildVectorAliases{w}[\w]
\buildVectorAliases{c}
\buildVectorAliases{b}
\buildVectorAliases{u}
\buildVectorAliases{v}
\buildVectorAliases{x}
\buildVectorAliases{y}
\buildVectorAliases{z}
\newacronym{fol}{FOL}{First Order Logic}
\addbibresource{references.bib}

\tikzexternalize
