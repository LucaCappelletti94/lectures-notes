
\def\taudef{
	\begin{definition}[Costante di tempo del condensatore]
	La Costante di tempo associata ad un condensatore e' definita come:

	\[\tau = R_{eq} C\]

	Dove $C$ e' la capacita' del condensatore e $R_{eq}$ e' la resistenza equivalente di Thevenim vista ai capi del condensatore.
	Cioe' si ottiene sostituendo al condensatore un circuito aperto e calcolando la resistenza equivalente tra questi due morsetti spegnendo i generatori di tensione non controllati. (Cortocircuitando quelli di Tensione ed sostituendo un aperto a quelli di Corrente).
	\end{definition}

}

\def\carattersistcacondensatore{
	\begin{definition}[Eq Caratteristia del condensatore]
	Sperimentalemnte si e' ricavato che la relazione tensione corrente di un condensatore e':
	\[i_c(t) = C \frac{\partial V_c(t)}{\partial t}\]
	\end{definition}
}
\def\caratteristacondensatorethr{
	\begin{theorem}[Risoluzione della eq caratteristica del condensatore]
	L'eq caratteristia del condnensatore e' un equazione differenziale lineare alle derivate parziali
	\[i_c(t) = C \frac{\partial V_c(t)}{\partial t}\]
	che ha come soluzione:
	\[V_c(t)= V_f - \rnd{V_f - V_i}e^{-\frac{t_p}{\tau}}\]
	\end{theorem}

}
