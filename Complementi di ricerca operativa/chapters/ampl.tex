\providecommand{\main}{..}
\documentclass[\main/main.tex]{subfiles}
\begin{document}

%\subfile{\main/chapters/ampl/}

\section{Introduzione alla programmazione lineare}
Per risolvere un problema utilizzando ampl è necessario utilizzare 3 tipi diversi di file:

\begin{enumerate}
\item Model file (.mod)
\item Data file (.dat)
\item Command file (.run)
\end{enumerate}

Ampl carica questi file e li invia al \textit{solver} (cplex, minos, ...), che quindi legge ed elabora il \textit{Command file}.
\\

\vspace{1em}
\textbf{\textit{Gli esempi che seguono sono tratti dal canale youtube "Yong Wang"}}: \url{https://www.youtube.com/channel/UCXEnJBeaJx3P87A_UfZpd0Q}

\subsection{Primo esempio}

\subsubsection{Esempio di Model file}
\inputminted{ampl}{\main/chapters/ampl/es1/example1.mod}

\subsubsection{Esempio di Command file}
\inputminted{ampl}{\main/chapters/ampl/es1/example1.run}

\subsection{Secondo esempio con separazione dei dati dal model}
\subsubsection{Data file}
\inputminted{ampl}{\main/chapters/ampl/es2/example2.dat}
\subsubsection{Model file}
\inputminted{ampl}{\main/chapters/ampl/es2/example2.mod}
\subsubsection{Command file}
\inputminted{ampl}{\main/chapters/ampl/es2/example2.run}

\end{document}