\providecommand{\main}{..}
\documentclass[\main/main.tex]{subfiles}
\usepackage{circuitikz}
\usepackage{tikz}
\usepackage{relsize}
\usetikzlibrary{arrows,calc}
\ctikzset{tripoles/mos style/arrows}

\begin{document}

\section{Diodi}
Il diodo è un componente elettronico composto da una giunzione PN. 
Il lato P è chiamato "anodo", mentre il lato N è chiamato "catodo". 
La presenza delle zone N e P fa sì che gli elettroni liberi e le lacune si muovano, svuotando così la parte centrale del diodo. 
A seconda del segno tensione applicata ai capi del diodo, cambia l'estensione della zona centrale svuotata: 
\begin{itemize} 
    \item Se la tensione è positiva: il diodo lavora in regime di \textbf{polarizzazione dirett}a e la zona svuotata si restringe.  
	\item Se la tensione è negativa: il diodo lavora in regime di \textbf{polarizzazione inversa} e la zona svuotata si allarga.  
\end{itemize}
In un diodo ideale, in regime di polarizzazione diretta nel diodo scorre corrente, mentre in regime di polarizzazione inversa non scorre corrente: in questo caso il diodo (ideale) si comporta infatti come un circuito aperto. 

Consideriamo invece un diodo reale. 
Le cariche all'interno del diodo si muovono di moto casuale dovuto all'agitazione termica. Questo moto causa una corrente di diffusione non nulla, che può essere calcolata attraverso la \textbf{Legge di Fick}:
\[I_{diff} = A \left[(-q)D_p\frac{\partial p(x)}{\partial x} - (-q)D_n\frac{\partial n(x)}{\partial x} \right]\]
Dove  $D_{p}$ e  $D_{n} $ sono detti "coefficienti di diffusione".

Per calcolare la corrente in regime di polarizzazione \textbf{diretta} si usa la seguente formula:
\[I_D = I_0 \left[e^{\frac{V_D}{V_{Th}}}-1 \right]\]
Dove  $I_0$ è detta "corrente inversa di saturazione" e  $V_{Th} $ è detta "Tensione Termica".
La tensione termica viene calcolata come segue:
\[V_{Th} = \frac{KT}{q}\]
essendo K la costante di Bolzmann. A temperatura ambiente (circa 300 K) essa vale circa 26 mV.

Possiamo descrivere un diodo attraverso tre modelli, con diversi gradi di precisione.

\textbf{Modello 0}:
\begin{itemize} 
  	  \item Inversa: I=0  $\Rightarrow$ Il diodo può essere approssimato a un circuito aperto.
  	  \item Diretta: I$\rightarrow\infty$  $\Rightarrow$ Il diodo può essere approssimato a un circuito chiuso.
\end{itemize}
\textbf{Modello 1}:
\begin{itemize} 
  	  \item Inversa:  I=0  $\Rightarrow$  Il diodo può essere approssimato a un circuito aperto come nel modello precedente.
  	  \item Diretta: Il diodo può essere approssimato a un circuito chiuso a cui sia stato aggiunto un generatore che imponga una tensione di 0.7 V. 
\end{itemize} 
\textbf{Modello 2}:
\begin{itemize} 
  	  \item Inversa: I=0  $\Rightarrow$ Il diodo può essere approssimato a un circuito aperto come nei modelli precedenti.
  	  \item Diretta: Oltre a collegare in serie al circuito chiuso un generatore di tensione da 0.7 V, aggiungiamo anche una resistenza $R_D$ tale che \[R_D = \frac{V_{Th}}{I_D}\]. 
\end{itemize} 

In un diodo reale può verificarsi l'\textbf{Effetto di Run-Out}: questo fenomeno consiste nel fatto che, a mano a mano che il circuito si scalda, il diodo porta una quantità maggiore di corrente per Effetto Joule, scaldandosi così maggiormente e dissipando maggiore potenza. Per evitare questo può essere collegata in serie al diodo una resistenza "limite", che limiti, appunto, la corrente passante per il diodo. 


\textbf{Diodi Zener}
Quando il diodo lavora in regime di polarizzazione inversa (quindi con valori negativi di tensione), esiste una soglia di tensione, detta tensione di Break-Down, oltre la quale nel diodo ricomincia a scorrere corrente. 
Lavorando a tensione di Break-Down ($V_{BD}$) nel diodo passerà corrente a tensione costante. Questo può essere utile in alcune circostanze, e per questo esiste un tipo di diodi, detti diodi \textbf{Zener}, che lavorano sempre a $V_{BD}$. 


\subsection{Metodi di risoluzione per circuiti con diodi}
Per risolvere un circuito contenente un diodo come il seguente, esistono tre differenti metodi.
\begin{enumerate} 
	\item Metodo analitico
	\item Metodo grafico
	\item Approssimazione
\end{enumerate}

Prendiamo in esame il seguente circuito:
\begin{center}
\begin{circuitikz} \draw
(0,0) node[ground] {}
(0,0) to[american voltage source, v^>= $V_{IN}$] (0,4) 
      to[resistor = R, i^>=$I_D$,] (4,4) 
      to[diode , v^>=$V_D$, ] node[ground] {} (4,0)
;\end{circuitikz}
\end{center}

\textbf{1. Metodo analitico:}
Si imposta il sistema:
\[I_D = I_0 \left[e^{\frac{V_D}{V_{Th}}}-1 \right]\]
\[V_D = V_{IN} - RI_D\]
Da esso ricaviamo:
\[V_D = V_{IN} - RI_0[e^{\frac{V_D}{V_{Th}}}-1]\]
Questa equazione può essere risolta iterativamente, "provando" diversi valori fino ad arrivare a una convergenza, oppure con un simulatore (ad esempio PSpice).
Si tratta però di un metodo poco efficiente.


\textbf{2. Metodo grafico:}
Si disegna la curva caratteristica del diodo (figura \ref{grafico_1})


\begin{figure}[H]
\center
\begin{tikzpicture}[
        axis/.style={very thick, ->, >=stealth'} ]
\draw[axis] (0,0)  -- (5,0) node(xline)[right] {$V_D$};
\draw[axis] (0,0) -- (0,5) node(yline)[above] {$I_D$};
\draw[scale=1,domain=0:1.8,smooth,variable=\x,blue] plot ({\x},{e^\x-1});  
\end{tikzpicture}
\caption{}
\label{grafico_1}
\end{figure}

INSERIRE CIRCUITO E GRAFICO

\[V_D = V_{IN} - RI_D\]
\[I_D = \frac{V_{IN}-V_D}{R}\]

\textbf{3. Linearizzazione a tratti}

\end{document}