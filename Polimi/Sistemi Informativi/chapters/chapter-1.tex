\providecommand{\main}{..}
\documentclass[\main/main.tex]{subfiles}


\begin{document}

\section{Introduzione}
\subsection{Dati e informazioni}
La conoscenza può essere strutturata su quattro livelli:

\begin{center}
1. Saggezza

2. Conoscenza 

3. Informazione

4. Dato
\end{center}


\begin{enumerate}
\item \textbf{Dato:} è una misura (ad esempio una temperatura, una lunghezza...). Caratterizzato da un tipo e un'unità di misura.
\item \textbf{Informazione:} è il dato contestualizzato (ad esempio associato al luogo in cui è stata misurata la temperatura). 
\item \textbf{Conoscenza:} Aggiunge all'informazione l'esperienza: consiste nel confrontare le informazioni con altre in proprio possesso, 
\item \textbf{Saggezza:} Consiste nell'utilizzare la conoscenza per prendere decisioni appropriate
\end{enumerate}

\subsection{Organizzazioni}
Le organizzazioni (aziende) operano sulle \textbf{risorse}. Le risorse sono tutto ciò su cui un'organizzazione lavora: possono essere interne (prodotti, persone, infrastrutture, disponibilità finanziaria, norme aziendali, ...) o esterne (clienti, mercato, situazione socio-economica).
Per coordinarsi, le organizzazioni usano il secondo livello della piramide della conoscenza: l'informazione. Talvolta le informazioni sono il prodotto stesso generato dall'azienda, talvolta servono per la comunicazione interna, per tenere traccia dei progressi e della situazione. 

\subsection{Processi}
Il processo è l'insieme di azioni intraprese da un'organizzazione per gestire le risorse. 
Possono essere classificati secondo la Piramide di Antony (figura 1) oppure la Catena del valore di Porter (figura 2):

\begin{figure}
\centering
\includegraphics{porter.png}
\caption{La Catena del Valore di Porter}\label{fig:1}

\centering
\includegraphics{anthony.png}
\caption{La Piramide di Anthony}\label{fig:1}
\end{figure}

\subsection{Sistemi Informativi}
Un sistema informativo è come l’insieme dei mezzi, della conoscenza organizzativa e delle competenze tecniche per gestire l'informazione. 
La progettazione e lo sviluppo di sistemi informativi include  la progettazioni di dati, processi, interazione con l'utente. 
In un'organizzazione non viene mai utilizzato un solo sistema informativo per gestire tutti i processi e le risorse.

\subsection{Classificazione dei Sistemi Informativi}
I sistemi informativi si dividono in \textbf{sistemi operazionali} e \textbf{sistemi decisionali}. 
I sistemi operazionali si occupano di svolgere le funzioni di routine di un'azienda, come le transazioni quotidiane, la contabilità, eccetera.
I sistemi decisionali si occupano della strategia aziendale, quindi utilizzano le informazioni per prendere decisioni nel modo più efficiente. 
Una caratteristica importante che distingue i due tipi di sistemi informativi è il modo in cui sono gestiti i dati: per il sistemi operazionali, i dati sono gestiti da basi di dati, sono ben strutturati e in grandi quantità; per i sistemi operazionali, i dati sono più aggregati, più ridotti e potrebbero perdere la loro struttura. 

\subsection{OLTP e OLAP}
I sistemi informativi agiscono direttamente su basi di dati. A seconda di quali dati trattano e di come agiscono su di essi, i sistemi informativi sono divisi in sistemi OLTP e sistemi OLAP.
\begin{itemize}
\item \textbf{Sistemi OLTP:} i sistemi OLTP (On-Line Transaction Processing) agiscono tramite transazione brevi che avvengono online. Devono garantire la rapidità e numerosità delle query sul database. Operano su database contenenti dati attuali e sempre aggiornati. Si occupano di processi di tipo organizzativo e di controllo. 
\item \textbf{Sistemi OLAP:} i sistemi OLAP (On-Line Analytical Processing) si occupano di dati storici. Fanno poche query complesse. Memorizzano i dati in formato aggregato e accedono a grandi quantità di dati. I dati estratti dai sistemi OLAP servono per il supporto alle decisioni e sono utilizzati per i processi di livello di pianificazione e strategico. 
\end{itemize}


\subsection{Progettare un sistema informativo}
Un sistema informativo ha il compito di mettere in comunicazione i vari settori dell'organizzazione: l'informazione, le persone, le procedure. 
Per progettare un sistema informativo bisogna pianificare diverse cose: la sua architettura, i processi aziendali su cui agirà, l'interfaccia utente, l'interazione con gli strumenti che raccolgono le informazioni,...
Per progettera il sistema informativo in modo efficace, è meglio "dividere" i vari aspetti.


\section{Sistemi Informativi e tecnologie}
\subsection{Glossario: nuove tecnologie}
Le nuove tecnologie più importanti nel settore dell'informazione sono:
\begin{itemize}
\item \textbf{Big Data:} sono i dati in enormi quantità, tanto grandi da richiedere di essere immagazzinati in modo distribuito.
\item \textbf{Internet of Things:} Interconnessione, tramite internet, di dispositivi di uso quotidiano. 
\item \textbf{Cloud Computing:} consiste nell'immagazzinare ed elaborare informazioni su server remoti, in modo che l'accesso ad esse sia possibile da qualunque luogo tramite internet.
\item \textbf{Social media:} tecnologie che creano comunità virtuali, mettendo in comunicazione gli utenti.
\item \textbf{Sistemi mobili:} sono sistemi computazionali portabili, che hanno accesso internet.
\end{itemize}
L'industria odierna, detta Industria 4.0, è fortemente basata su queste nuove tecnologie.

\subsection{SI per la pubblica amministrazione}


\end{document}

