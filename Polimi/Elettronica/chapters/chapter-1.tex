\providecommand{\main}{..}
\documentclass[\main/main.tex]{subfiles}


\begin{document}

\section{Semi Conduttori}
\subsection{Introduzione ai Semiconduttori}
I semi conduttori sono una categoria di materiali che hanno una conduttivita' a meta' tra conduttori ed isolanti.

Ci sono due principali tipi di semiconduttori:
\begin{enumerate}  
	\item Semiconduttori ad elemento singolo: ad esempio quelli al silicio o al germanio.
	\item Semiconduttori composti:ad esempio quelli a lega di gallio-arsenico.
\end{enumerate}
Quindi quelli ad elemento singolo sono tutti elementi con 4 elettroni di valenza nel orbitale piu' esterno, mentre quelli composti sono lege di elementi con valenza 3-5 o 2-6 in modo che in media si comporti come se avesse valenza 4 cosi che si possa formare un reticolo di legami covalenti.

Il principale utilizzo di semiconduttori composti e' per i LED.

Il reticolo di legami covalenti non conduce poiche' non vi e' carica libera pero' col aumentare della temperatura i legami si rompono e generano coppie di elettrone-lacuna ,che in quanto cariche libere rendono il materiale capace di condurre, poi si ricombinano.

Le lacune possono essere modellizzate come particelle di carica opposta al elettrone.

Ovviamente il numero di elettroni liberi e di lacune sono uguali e questo numero per $cm^3$ vale:
\[n_i = BT^{\frac{3}{2}}e^{\frac{-E_g}{2kT}}\]
\begin{tabular}{l}
	$T$ e' la temperatura espressa in gradi Kelvin
	
	$B$ e' un valore che dipende dal materiale e nel silicio vale $7.3 \times 10^{15} cm^{-3} K^{-\frac{3}{2}}$
	
	$E_g$ e' l'energia minima per rompere il legame covalente che nel silico vale $1.12 eV$ (ElettronVolt)
	
	$k$ e' la costante di Boltzmann $8.62 \times 10^{-5} \frac{eV}{K}$
\end{tabular}
a temperatura ambiente \[n_i \sim 1.5 \times 10^{10} \frac{tdc}{cm^3}\]
($tdc$ = Trasportatori di Carica)
\clearpage
\end{document}