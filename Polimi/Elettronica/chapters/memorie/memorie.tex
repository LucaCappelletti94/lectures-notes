\providecommand{\main}{../..}
\documentclass[\main/main.tex]{subfiles}
\usetikzlibrary{arrows,calc}
\ctikzset{tripoles/mos style/arrows}

\begin{document}

\section{Elementi di memoria}
\subsection{Classificazione}
Gli elementi di memoria sono utilizzati per salvare informazioni.
Le memorie possono essere classificate secondi diversi criteri.
\begin{enumerate}
  \item \textbf{Accesso:}

        - \textit{Casuale:} Sono memorie in cui i bit sono immagazzinati in una matrice, e l'accesso ai dati non è sequenziale. Ne sono un esempio le RAM.

        - \textit{Sequenziale:} sono memorie in cui i bit sono immagazzinato in modo sequenziale. Ne sono un esempio le memorie di massa (CD, HD-Disk)
  \item \textbf{Volatilità:}

        - \textit{Memorie volatili:} sono memorie che perdono i dati memorizzati allo spegnimento del dispositivo.

        - \textit{Memorie non volatili:} sono memorie che mantengono i dati memorizzati anche quando il dispositivo viene spento.
\end{enumerate}

A loro volta le memorie non volatili possono essere:
- \textit{One Time Programmable:} sono memorie che vengono scritte una sola volta e da quel momento non possono essere modificate.
- \textit{Riprogrammabili:} sono memorie che possono essere modificate anche dopo la prima scrittura.

\subsection{RAM}
Le memorie RAM possono essere statiche (SRAM) oppure dinamiche (DRAM). Le SRAM si basano sulla retroazione, mentre le DRAM utilizzano una capacità.

\subsubsection{Latch}
Le memorie RAM sono basate su due diversi tipi di elementi di memoria: i \textbf{latch} e i \textbf{flip-flop}.
Sono entrambi elementi di tipo sequenziale, cioè aggiornano la loro uscita solo in determinati intervalli scanditi dal clock. La differenza è nella scansione degli intervalli.
A seguito di un cambiamento nell'ingresso, i latch aggiornano l'uscita se il clock è "alto" (cioè il segnale di clock vale "1").
I flip-flop, invece, aggiornano l'uscita solo durante i fronti di commutazione del clock.
Un flip-flop viene realizzato ponendo in cascata due latch.

Abbiamo detto che le RAM utilizzano flip-flop e latch. Dopo averne visto il funzionamento, possiamo ora affermare che, nel dettaglio, le SRAM sono basate sui flip-flop, mentre le DRAM sfruttano i latch.

\subsubsection{Operazioni di lettura e scrittura}
Abbiamo detto che le SRAM utilizzano la retroazione (in particolare, nella realizzazione fisica, questo si traduce nell'impiego di due inverter); questo logicamente corrisponde all'uso dei flip-flop.
Per questo, i cicli di lettura e scrittura avvengono nel seguente modo: viene letto un dato (dalla memoria oppure in ingresso). Se esso non viene modificato prima che il segnale di clock giunga a un fronte di commutazione (salita o discesa), allora il dato viene ritenuto valido.
Se si tratta di un'operazione di scrittura allora il dato viene scritto in memoria, se si tratta di un'operazione di lettura viene fornito in output.

Le DRAM, invece, utilizzano una condensatore per immagazzinare i risultati; questo logicamente corrisponde all'uso di un latch.
Il ciclo di scrittura avviene nel modo seguente: si seleziona la word-line desiderata e si trasmette il bit da memorizzare nella bit-line che ci interessa. Il bit è immagazzinato nel condensatore.
Questo, però, rende l'informazione immagazzinata "a rischio" di cancellazione: nel circuito infatti nascono correnti di perdita che portano il condensatore a scaricarsi. Per questo motivo, le DRAM hanno bisogno di fare un \textbf{refresh periodico}, cioè di ri-scrivere periodicamente, ricaricando il condensatore, gli "1" memorizzati su di esso, per evitare che vadano persi.
Anche l'operazione di lettura nelle DRAM risulta molto delicata. La bit-line, infatti, si comporta a sua volta come se fosse una capacità: questo compromette la lettura del dato immagazzinato sul condensatore, perché riduce di molto la tensione effettiva ai capi del condensatore. Per questo, la tensione che rappresenta "1" è molto piccola, e potrebbe talvolta essere confusa con una tensione nulla che rappresenta invece lo "0" logico.
Inoltre, la lettura è distruttiva: l'operazione di lettura porta a scaricare parte della carica immagazzinata sul condensatore, rendendo l'informazione memorizzata illeggibile.

\subsubsection{Decoder}
Un elemento importante necessario per costruire una RAM è il \textbf{decoder}.
Il decoder è un dispositivo che funziona da multiplexer: da \textit{n} ingressi è in grado di fornire $2^n$ uscite.
Essendo, come abbiamo accennato, la RAM una memoria a struttura matriciale, saranno necessari due decoder: uno per le colonne e uno per le righe.
Quindi, se abbiamo a disposizione \textit{n} bit per ogni parola di memoria, avremo $2^{2n}$ celle di memoria.



\end{document}