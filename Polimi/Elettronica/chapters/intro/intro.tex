\providecommand{\main}{../..}
\documentclass[\main/main.tex]{subfiles}


\begin{document}

\section{Convenzioni}

Per convenzione la tensione $V_{AB}$ e' misurata come $V_{A} - V{B}$ ed indicata con la punta della freccia in $A$ e la coda in $B$.
\begin{center}
  \begin{circuitikz}
    \draw (4,0) to[open, v_>=$V_{AB}$] (0,3);
    \node[] at (4,0) {$B$};
    \node[] at (0,3) {$A$};
  \end{circuitikz}
\end{center}
\[V_{AB} = V_{A} - V_{B}\]

\section{Caratteristiche dei segnali}

I segnali sono caratteristiche fisiche variabili nel tempo che trasportano informazione, tipicamente tensioni o correnti.

Per Segnali sinuosidali:
\begin{center}
  \begin{tikzpicture}[axis/.style={very thick, ->, >=stealth'} ,scale=1]
    \draw[axis] (-1,0)  -- (7,0) node(xline)[right] {$t$};
    \draw[axis] (0,-3) -- (0,3) node(yline)[above] {};

    \draw plot[scale=1, domain=-1:7,smooth] (\x,{sin(\x r)});

    \draw[red,<->] (0,-2) -- (2*pi,-2);
    \draw[dotted] (2*pi,-2) -- (2*pi,0);
    \node[red] at (pi,-2.3) {$T$};

    \draw[red,<->] (pi/2,0) -- (pi/2,1);
    \node[red] at (1.3,0.5) {$ V_p$};

    \draw[dotted] (pi/2,1) -- (3*pi/2,1);
    \draw[red,<->] (3*pi/2,1) -- (3*pi/2,-1);
    \node[red] at  (4,0.5) {$V_{pp}$};

  \end{tikzpicture}
\end{center}

\begin{align*}
  V_p    & \text{  Tensione di picco}    \\
  V_{pp} & \text{  Tensione Picco Picco} \\
  T      & \text{  Periodo}
\end{align*}

Per segnali ad onda quadra:
\begin{center}
  \begin{tikzpicture}[axis/.style={very thick, ->, >=stealth'} ,scale=1]
    \draw[axis] (-1,0)  -- (7,0) node(xline)[right] {$t$};
    \draw[axis] (0,-2.5) -- (0,3) node(yline)[above] {};
    \draw (0,2) -- (2,2);
    \draw (2,2) -- (2,0);
    \draw (2,0) -- (5,0);
    \draw (5,0) -- (5,2);
    \draw (5,2) -- (7,2);

    \draw[<->,red] (0,-1) -- (2,-1);
    \node[red] at (1,-0.75) {$T_{on}$};
    \draw[<->,red] (2,-1) -- (5,-1);
    \node[red] at (3.5,-0.75) {$T_{off}$};

    \draw[<->,red] (0,-2) -- (5,-2);
    \node[red] at (2.5,-1.75) {$T$};

    \draw[dotted] (2,0) -- (2,-1);
    \draw[dotted] (5,0) -- (5,-2);

  \end{tikzpicture}
\end{center}

\begin{align*}
  T_{on}  & \text{  Periodo On}  \\
  T_{off} & \text{  Periodo Off} \\
  T       & \text{  Periodo}     \\
  D       & \text{  Duty-Cycle}
\end{align*}

\[T = T_{on} + T_{off}\]
\[D = \frac{T_{on}}{T}\]
\end{document}