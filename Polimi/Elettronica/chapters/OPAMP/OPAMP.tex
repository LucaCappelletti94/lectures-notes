\providecommand{\main}{../..}
\documentclass[\main/main.tex]{subfiles}

\begin{document}

\section{Introduzione agli Amplificatori Operazionali}
Un amplificatore operazionale (detto anche OPAMP), è un componente circuitale che consente di realizzare un guadagno in tensione elevato; un amplificatore operazionale, dunque, \textit{amplifica} una tensione in ingresso.
L'amplificatore operazionale è rappresentato con il seguente simbolo circuitale:

\begin{figure}[H]
  \centering
  \begin{circuitikz}
    \draw
    (0, 0) node[op amp] {}
    ;
  \end{circuitikz}
\end{figure}

L'OPAMP presenta due morsetti in ingresso: il morsetto positivo è detto morsetto non invertente, mentre il morsetto negativo è detto morsetto invertente.
Quando si analizzano gli amplificatori operazionali, si possono studiare due modelli: il modello ideale e il modello reale.
Il modello ideale rappresenta un OPAMP non realizzabile nella realtà, che presenta le caratteristiche ideali che un amplificatore operazionale dovrebbe avere per svolgere perfettamente il suo compito, senza errori.
Il modello reale, invece, tiene conto delle limitazioni dovute all'utilizzo di componenti reali, che si discostano da quelli ideali (ad esempio, nella realtà non è possibile realizzare un amplifatore operazionale con guadagno infinito).

\section{L'amplificatore Operazionale Ideale}
Un amplificatore operazionale ideale è caratterizzato dalle seguenti proprierà:
\begin{description}
  \item[Corrente in ingresso nulla:] L'OPAMP ideale non assorbe corrente. Questo significa che la corrente misurata ai morsetti d'ingresso (morsetto positivo e morsetto negativo) di un amplificatore operazionale ideale sarà sempre \textbf{nulla}.
        Un altro modo per esprimere questa caratteristica è dire che l'OPAMP ideale ha impedenza infinita in ingresso.
  \item[Guadagno infinito:] un amplificatore operazionale ideale ha un guadagno molto elevato, che può essere definito ``infinito''.
  \item[Differenza di potenziale nulla ai morsetti:] in un OPAMP ideale, la differenza di potenziale fra il morsetto negativo e quello positivo è sempre nulla. Ciò significa che la tensione ai due morsetti è sempre uguale. Scriveremo dunque: \[V_{+} - V_{-} = 0\].

\end{description}

\end{document}