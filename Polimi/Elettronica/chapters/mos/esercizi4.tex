\providecommand{\main}{../..}
\documentclass[\main/main.tex]{subfiles}

%\graphicspath{ {\main/chapters/circuiti/}}

\begin{document}
\subsection{Esercizio 5.1}
Dato il Circuito sottostante,

Al circuito in figura (a) viene applicato il segnale mostrato nella figura (b).
Si considerino porte logiche ideali alimentate a 0V e 5V, caratterizzate da soglia di commutazione pari a $\frac{V_{DD}}{2}$. Si chiede:

a)Disegnare l'andamento dell'uscita considerando $\Delta$ $T_{LOW,IN}=1s$.

a)Disegnare l'andamento dell'uscita considerando $\Delta$ $T_{LOW,IN}=20\mu s$.

\begin{center}
  \begin{circuitikz}
    \draw (0,0) node[american nand port] (mynand1) {}
    (2,0) node[american not port] (mynot) {}
    (mynand1.out) -| (mynot.in)
    (1,0) to[short, *-] (1,1.26)
    (mynot.out) -| (3,0)
    (3,0) to[resistor=$100k\Omega$] (5,0)
    (5,0) to[short, *-] (5,-0.5)
    (5,-0.5) to[capacitor=$10nF$] (5, -2)
    (5,-2) node[ground] {};
    \draw (5,0) to[short] (5.5,0)
    (5.5,0) to[short] (5.5,0.7)
    (6.9,1) node[american nand port] (mynand2) {}
    (mynand2.in 1) -| (1,1.26)
    (mynand2.in 2) -| (5.5,0.7)
    (mynand2.out) -| (7.6, 2) -| (0,2) -| (mynand1.in 1)
    ;
    \draw (7.6, 1) to[short, *-] (8.2, 1)
    (8.2, 1) node[right] {$V_{OUT}$}
    ;
    \draw (mynand1.in 2) -| (-2,-0.2555)
    (-2,-0.2555) node[above] {IN}
    ;
  \end{circuitikz}
\end{center}

\clearpage
\subsection{Risoluzione Esercizio 5.1}
\end{document}