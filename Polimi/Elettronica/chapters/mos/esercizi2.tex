\providecommand{\main}{../..}
\documentclass[\main/main.tex]{subfiles}

%\graphicspath{ {\main/chapters/circuiti/}}

\begin{document}
\clearpage
\subsection{Esercizio 3.1}
Dato il Circuito sottostante,
\begin{enumerate}
    \item Trovare la tabella di verita' della porta (aka Calcolare $V_{out}$ quando $V_{in} = 0V$ e quando $V_{in} = V_{cc}$)
    \item Calcolare la soglia logica $V_{th}$
    \item Calcolare il tempo di propagazione  $t_p$ sul fronte di salita del ingresso
    \item Calcolare le Potenze statiche $P_{STAT}$ e dinamiche $P_{DIN}$ con un clock TTL\footnote{Tensioni secondo lo standard Transistor Transistor Logic, LOW = 0V , HIGH = 5V} ideal di $T_{CLK} = 0.5\mu s$.
\end{enumerate}

\begin{center}
    \begin{circuitikz}
        \draw(0,4)
        node[pmos] (pmos) {}
        (pmos.gate) node[above left] {G}
        (pmos.drain) node[right] {D}
        (pmos.source) node[right] {S};
        \draw(0,1)
        node[nmos] (nmos) {}
        (nmos.gate) node[above] {G}
        (nmos.drain) node[right] {D}
        (nmos.source) node[right] {S};
        \draw (nmos.gate) -- (-1.5,1);
        \draw (pmos.gate) -- (-1.5,4) -- (-1.5,1);
        \draw (-1.5,2.5) to[short, *-o] (-2,2.5) node[left] {$V_{in}$};
        \draw (0,5.5) node[above] {$V_{dd}$} (0,5.5) node[tground] {} (0,5.5) --(pmos.source);
        \draw (pmos.drain) -- (nmos.drain);
        \draw (nmos.source) -- (0,0) node[ground] {};
        \draw (0,2.5) to[short, *-o] (3,2.5) node[above] {$V_{out}$};
        \draw (2,2.5) to[short, *-] (2,2) to[capacitor = C] (2,0) node[ground] {};
        \draw (nmos.gate) to[open, v_<=$V_{GS}$] (nmos.source);
        \draw (pmos.gate) to[open, v^>=$V_{SG}$] (pmos.source);
    \end{circuitikz}
\end{center}

\[V_{cc}= 5V\]
\[C = 100fF\]
\[K_n = |K_p| = 500 \frac{\mu A}{V^2}\]
\[|V_{t,n}| = |V_{t,p}| = 1V\]
\[T_{CLK} = 0.5\mu s\]

\clearpage
\subsection{Risoluzione Esercizio 3.1}

\textbf{Caso $V_{in} = 0V$}

$V_{GS,n} = 0V < V_{t,n}$ quindi non vi e' canale dal lato del source del NMOS.

$V_{DG,n} = -V_{out} $ e poiche' $V_{out}$ puo' assumere solo valori positivi allora $V_{DG,n} = -V_{out} < V_{t,n}$ quindi non vi e' canale neanche dal lato del drain.

Quindi non essendoci canale da nessuno dei due lati allora il NMOS e' spento e quindi la corrente che circola nei due MOS in serie $I = 0A$.


$V_{GS,p} = V_{dd} > V_{t,p}$ quindi vi e' canale dal lato del source del PMOS.

Quindi il PMOS puo' essere o in zona ohmmica o in saturazione.

Ora se il condesatore e' scarico allora il PMOS e' in saturazione poiche' $V_{GD,p} = V_{out} = 0V < V_{t,p}$ quindi non vi e' canale.

quindi il condensatore si carica con la corrente $I_{SAT,p}$ fino a raggiungere $V_{t,p}$ al quale punto il MOS diventa in zona ohmmica
e continua a caricarsi fino a $V_{dd}$ dove il mos e' acceso in zona ohmmica pero' la sua $V_{SD} = 0V$ quindi ha corrente $I = 0$.

In conclusione dopo i transitori il condensatore , quindi $V_{out}$ si carica a $V_{dd}$

\[V_{in} = 0V \Rightarrow V_{out} = 5V\]



\textbf{Caso $V_{in} = 5V$}

$V_{GS,p} = V_{dd} - V_{in} = 0V < V_{t,p}$ quindi non vi e' canale al source.

$V_{DG,p} = V_{out} - V_{in}$ e poiche' $V_{in} = V_{dd}$ e la $V_{out}$ e' la tensione sul condesnatore ,il quale puo' caricarsi al massimo a $V_{dd}$ allora $V_{out} - V_{in} \le 0 < V_{t,p}$

Quindi non vi puo' essere canale al lato del drain  quindi il PMOS e' sicuraemnte spento il che implica che la corrente che circola nei MOS in serie e' $I = 0$.

$V_{GS,n} = V_{in} > V_{t,n}$ quindi vi e' canale dal lato del source del NMOS.

Quindi il NMOS puo' essere in zona ohmmica o in saturazione.

$V_{DG,n} = V_{in} - V_{out}$ ora supponiamo che il condensatore sia carico a $V_{dd}$ in questo caso le $V_{DG,n} = 0V < V_{t,n}$ quindi non vi e' canale e quindi il NMOS e' in saturazione.

Il consenatore si scarica a massa attraverso l'NMOS finche' non arriva alla tensione $V_{out} = V_{dd} - V_{t,n}$ alla quale l'NMOS passa in zona Ohmmica e il condensatore si scarica piu' lentamente fino ad arrivare $V_{out} = 0V$.

Quindi finiti i transitori $V_{out} = 0V$

\[V_{in} = 5V \Rightarrow V_{out} = 0V\]

\textbf{Riassumendo:}
\begin{center}
    \begin{tabular}{ c | c }
        $V_{in}$ & $V_{out}$ \\
        \hline
        1        & 0         \\
        0        & 1         \\
    \end{tabular}
\end{center}
quindi e' una porta NOT!


\clearpage
\textbf{Calcolo della soglia logica $V_{th}$}

La soglia logica e' la tensione $V_{in}$ per la quale $V_{in} = V_{out}$.

Quindi sicuramente $|V_{DG,n}| = |V_{DG,p}| = |V_{in} - V_{out}| = 0V$ quindi entrambi i MOS non hanno canale dal lato del drain quindi sono o spenti o in saturazione.

Ora procediamo per assurdo.

Supponiamo $V_{in} = 1V$

Allora $V_{GS,p} = V_{dd} - V_{in} = 4V > V_{t,p}$ quindi vi e' canale al source e quindi il PMOS e' in sautrazione.

E $V_{GS,n} = V_{in} = 1V = V_{t,n}$ quindi vi e' canale al source e quindi l'NMOS e' in saturazione.

E poiche' non sembra ci siano contraddizioni prendiamo per vero che entrambi i MOS siano in saturazione.

Ora supponendo che il condensatore sia completamente carico esso non assorbe corrente quindi la corrente dei due MOS e' uguale poiche' in serie.
\[I_{SAT,n} = I_{SAT,p}\]
\[K_n \left(V_{in} - V_{t,n} \right )^2 = K_p \left(V_{dd} - V_{in} - |V_{t,p}| \right )^2\]
e da questo bilancio delle correnti si ricava che la tensione
\[V_{in} = V_{th} = 2.5V\]

\textbf{Calcolo del tempo di propagazione $t_p$}

Scelgo di approssimare con l'approssimazione a corrente costante.

\[I_{SAT,n} = K_n \left( V_{GS,n} - V_{t,n} \right)^2 = 16 K_n = 8mA\]

Calcoliamo il differenziale di carica del condensatore d'uscita.

\[ \triangle Q = C \left( V_f  - V_i\right) = 2.5 C = 2.5 fC\]

ora applichiamo la definizione di corrente

\[ I = \frac{Q}{t} = \frac{\triangle Q}{t_p} \]

\[ t_p = \frac{\triangle Q}{I} = 31.25 ps \]


\textbf{Calcolo della potenza dinamica $P_{DIN}$}

\[ P_{DIN} = V_{dd} I = V_{dd} \frac{\triangle Q}{T_{CLK}} = V_{dd} \frac{V_{dd} C }{T_{CLK}} = C \frac{V_{dd}^2}{T_{CLK}} = 5\mu W\]



\clearpage
\subsection{Esercizio 3.2}
Dato il Circuito sottostante,
\begin{enumerate}
    \item Trovare la tabella di verita' della porta
    \item Calcolare il tempo di propagazione  $t_p$  con $EN = 1$ e A che commuta tra $ 1 \rightarrow 0$
    \item Calcolare la Potenza dinamiche $P_{DIN}$ con un clock di $f_{a} = 400kHz$.
\end{enumerate}


\begin{center}
    \begin{circuitikz}
        \draw(0,8)
        node[pmos] (pmos1) {PMOS1};
        \draw(0,6)
        node[pmos] (pmos2) {PMOS2};
        \draw(0,3)
        node[nmos] (nmos1) {NMOS2};
        \draw(0,1)
        node[nmos] (nmos2) {NMOS1};

        \draw (0,9) node[above] {$V_{dd}$} (0,9) node[tground] {} (0,9) -- (pmos1.source);

        \draw (pmos1.drain)  -- (pmos2.source);
        \draw (pmos2.drain)  -- (nmos1.drain);
        \draw (nmos1.source) -- (nmos2.drain);
        \draw (nmos2.source) -- (0,0) node[ground] {} (0,0);

        \draw (pmos1.gate) -- (-2.5,8) -- (-2.5,1);
        \draw (nmos2.gate) -- (-2.5,1);
        \draw (-2.5,4.5) to[short, *-o] (-3,4.5) node[left] {$A$};

        \draw (0,4.5) to[short, *-o] (3,4.5) node[right] {$B$};
        \draw (2,4.5) to[short, *-] (2,4.5) to[capacitor = C] (2,0) node[ground] {} (0,0);

        \draw (pmos2.gate) node[left] {$\neg EN$};
        \draw (nmos1.gate) node[left] {$EN$};


        \draw (nmos1.gate) to[open, v_<=$V_{GS}$] (nmos1.source);
        \draw (pmos1.gate) to[open, v^>=$V_{SG}$] (pmos1.source);
        \draw (nmos2.gate) to[open, v_<=$V_{GS}$] (nmos2.source);
        \draw (pmos2.gate) to[open, v^>=$V_{SG}$] (pmos2.source);

    \end{circuitikz}
\end{center}

\[K_n = K_p = 380\frac{\mu A}{V^2}\]
\[V_{t,n} = |V_{t,p}| = 1V\]
\[V_{dd} = 5V\]
\[C = 4pF\]

\clearpage
\subsection{Risoluzione Esercizio 3.2}
\textbf{Calcolo della tabella di verita'}

Dobbiamo riempire:
\begin{center}
    \begin{tabular}{ c  c | c}
        $EN$ & $A$ & $B$ \\
        \hline
        0    & 0   & ?   \\
        0    & 1   & ?   \\
        1    & 0   & ?   \\
        1    & 1   & ?   \\
    \end{tabular}
\end{center}

\textbf{Casi con $EN = 1$}

Poiche' $EN = 1$ allora sicuramente NMOS2 e PMOS2 saranno accesi, quindi li approssimo a cortocircuito per semplificare i conti e riconosco che il circuito e' l'inverter di esercizio 3.2.

\textbf{Caso $A = 0$}

$A = 0V$ implica che la $V_{GS,N1} = 0V$ quindi NMOS1 e' spento quindi la $I_{NMOS}  = 0A$

$A = 0V$ implica che la $V_{GS,P1} = V_{dd}$ quindi il PMOS1 e' acceso.
E a prescindere dal comportamento degli altri due MOS , una volta che il condensatore' carico la corrente $I_{PMOS} = 0A$ quindi la $V_{DS,P1} = 0V$
\begin{center}
    \begin{tabular}{ c  c | c}
        $EN$ & $A$ & $B$ \\
        \hline
        0    & 0   & ?   \\
        0    & 1   & ?   \\
        1    & 0   & ?   \\
        1    & 1   & 0   \\
    \end{tabular}
\end{center}

\textbf{Caso  $A = 1$}

Il caso e' duale a quello sopra quindi PMOS1 e' spento, NMOS2 ed NMOS1 sono accesi, quindi $V_{out} = 0V$

\begin{center}
    \begin{tabular}{ c  c | c}
        $EN$ & $A$ & $B$ \\
        \hline
        0    & 0   & ?   \\
        0    & 1   & 0   \\
        1    & 0   & ?   \\
        1    & 1   & 1   \\
    \end{tabular}
\end{center}

\textbf{Altri Casi}

In entrambi i $EN = 0$ quindi PMOS2 e NMOS2 saranno spenti di sicuro.
Quindi siamo in uno stato di alta impedenza nel quale l'uscita rimane quella che era.

\textbf{Risultato}
\begin{center}
    \begin{tabular}{ c  c | c}
        $EN$ & $A$ & $B$ \\
        \hline
        0    & 0   & Hiz \\
        0    & 1   & 0   \\
        1    & 0   & Hiz \\
        1    & 1   & 1   \\
    \end{tabular}
\end{center}

\textbf{Calcolo del tempo di propagazione $t_p$}

Il tempo di propagazione e' quello di carica del condesnatore.

Scelgo di approssimare a corrente costante e poiche' i PMOS hanno $V_{GS,P1} = V_{GS,P2}$ e $K_p$ uguali allora li apporssimo come un unico MOS che ha $K_{p,eq} =  \frac{K_p}{2}$.

\[I_{SAT,p} = \frac{K_p}{2} \left( V_{GS,p} - |V_t| \right)^2 = 3mA\]

\[ \triangle Q = C \frac{V_{dd}}{2} \]
\[ I = \frac{Q}{t} = \frac{\triangle Q}{t_p} = C \frac{V_{dd}}{2 t_p} \]
\[ t_p = C \frac{V_{dd}}{2 I_{SAT,P}} = 3.3ns\]

\textbf{Calcolo della potenza dinamica $P_{DIN}$}

Calcoliamo solo la potenza usata per caricare il condensatore perche' per la scarica lo fa attraverso il mos e quindi l'alimentazione non eroga corrente.

\[P_{DIN} = V_{dd} I = V_{dd} \triangle Q f_{CLK} = C V_{dd}^2 f_{CLK} = 40 \mu W\]

\clearpage
\subsection{Esercizio 3.3}
Dato il Circuito sottostante, A e B sono segnali digitali con livelli $0V$ e $3.3V$

\begin{enumerate}
    \item Determinare la tabella di verita' del circuito specificando il valore di tensione $V_{out}$
    \item Quale funzione logica svolge il circuito?
    \item Calcolare il tempo di propagazione della porta logica quando gli ingressi commutano istantaneamente da  $A,B = 1,1 \longrightarrow A,B = 1,0$
    \item Calcolare la potenza dissipata dal circuito quando $A=1$ e B e' un' onda quadra a frequenza $2MHz$ e $D=30\%$
\end{enumerate}


\begin{center}
    \begin{circuitikz}
        \draw(-2,3) node[left] {$A$} (-2,3) to[short, o-](-1,3);
        \draw(-1,3) to[Tnmos, l=$N_1$](0,3);
        \draw (-0.5,4) node[above] {$B$};
        \draw (0,3) -- (1,3) -- (1,1);

        \draw(-1,1) to[Tnmos, l=$N_2$] (0,1);
        \draw (-0.5,2) node[above] {$\neg B$};
        \draw (0,1) -- (1,1);
        \draw(-1,1) -- (-2,1) -- (-2,0) node[ground] {} (-2,0);

        \draw (1,2) to[short, *-] (2,2) node[above] {$V_{out}$} (2,2) to[capacitor = C](2,0) node[ground]{} (2,0);
    \end{circuitikz}
\end{center}

\[K_n = 200 \frac{\mu A}{V^2}\]
\[V_{t,n} = |V_{t,p}| = 1V\]
\[V_{DD} = 3.3V\]
\[C = 0.2pF\]


\clearpage
\subsection{Risoluzione Esercizio 3.3}

\textbf{Calcolo della tabella di verita'}

\begin{center}
    \begin{tabular}{ c  c | c}
        $A$ & $B$ & $V_{out}$ \\
        \hline
        0   & 0   & ?         \\
        0   & 1   & ?         \\
        1   & 0   & ?         \\
        1   & 1   & ?         \\
    \end{tabular}
\end{center}

Analizziamo i casi con $B=0$:

$N_2$ e' acceso poiche' ha la $V_{GS,1} = 3.3V > V_{t,n}$

e $N_1$ e' spento poiche' ha la $V_{GS,2} = B - A = -A$ che poiche' A puo' assumere solo valori positivi allora sicuramente $V_{GS,2} < V_{t,n}$

Quindi se il condensatore e' carico, esso si scarica attraverso $N_2$ a massa
e una volta scarico $N_2$ e' in zona ohmmica con $I=0 \Rightarrow V=0$
Quindi a prescindere dallo stato del condensatore in entrambi i casi converge a $V_{out} = 0V$

\[B=0 \Rightarrow V_{out} = 0\]

\begin{center}
    \begin{tabular}{ c  c | c}
        $A$ & $B$ & $V_{out}$ \\
        \hline
        0   & 0   & 0V        \\
        0   & 1   & ?         \\
        1   & 0   & 0V        \\
        1   & 1   & ?         \\
    \end{tabular}
\end{center}

Caso $A,B = 0,1$:

In Questo caso $N_2$ e' spento per i discorsi fatti prima.
mentre  $N_1$ ha tensione da un lato $B - V_{out} = 3.3V - V_{out}$ e dal altro $B - A = 3.3V$
quindi $N_1$ e' acceso e a seconda del valore di $V_{out}$ e' in zona ohmmica o in saturazione.

Quindi il condensatore se carico si scarica attraverso $N_1$ a Massa attraverso $A$.

Quindi finiti i transitori $V_{out} = 0V$

\[A,B = 0,1 \Rightarrow V_{out} = 0V\]
\begin{center}
    \begin{tabular}{ c  c | c}
        $A$ & $B$ & $V_{out}$ \\
        \hline
        0   & 0   & 0V        \\
        0   & 1   & 0V        \\
        1   & 0   & 0V        \\
        1   & 1   & ?         \\
    \end{tabular}
\end{center}

Caso $A,B = 1,1$:

$N_2$ spento per quanto detto sopra.

$N_1$ dal lato di $A$ ha tensione rispetto al gate di $0V$ quindi non vi e' canale dal lato di $A$.

Dal altro lato la sua tensione e' $B - V_{out} = 3.3V - V_{out}$ quindi finche' $V_{out} \leq B - V_{t} = 2.3V$ Il mos e' in saturazione nel altro caso e' spento.

Quindi se il condensatore e' scarico esso si carichera' finche' $V_{out} \leq 2.3V$.

\[A,B = 1,1 \Rightarrow V_{out} = 2.3V\]
\begin{center}
    \begin{tabular}{ c  c | c}
        $A$ & $B$ & $V_{out}$ \\
        \hline
        0   & 0   & 0V        \\
        0   & 1   & 0V        \\
        1   & 0   & 0V        \\
        1   & 1   & 2.3V      \\
    \end{tabular}
\end{center}

\textbf{Quindi questa porta e' un AND.}

\textbf{Calcolo del tempo di propagazione $t_p$}

Approssimazione a corrente costante.

\[I_{SAT} = K_n \left(V_{GS} - V_t\right)^2 = 1mA\]
\[\Delta Q = C \left(V_f - V_i \right) = C \left(2.3V - \frac{2.3V}{2} \right) = C \frac{2.3V}{2} = 0.23pC\]
\[I = \frac{\Delta Q}{T}\]
\[t_p = \frac{\Delta Q}{I_{SAT}} = 0.23ns\]

\textbf{Calcolo della potenza dinamica}

Calcolo solo la potenza erogata per caricare il condensatore poiche' la scarica del condensatore e' a massa e quindi l'alimentazione non deve erogare nessuna potenza.

\[P_{DIN} = V_{dd} I = V_{dd} f_{CLK} \Delta Q = V_{dd} f_{CLK} C 2.3V = 3 \mu W\]

\clearpage
\subsection{Esercizio 3.4}
Dato il Circuito sottostante,

\begin{center}
    \begin{circuitikz}
        \draw(-2,3) node[left] {$A$} (-2,3) to[short, o-](-1,3);
        \draw(-1,3) to[Tpmos] (0,3);
        \draw (-0.5,4) node[above] {$\neg B$};
        \draw (0,3) -- (1,3) -- (1,1);

        \draw(-1,1) to[Tpmos] (0,1);
        \draw (-0.5,2) node[above] {$B$};
        \draw (0,1) -- (1,1);
        \draw(-1,1) -- (-2,1) -- (-2,0) node[ground] {} (-2,0);

        \draw (1,2) to[short, *-] (2,2) node[above] {$V_{out}$} (2,2) to[capacitor = C](2,0) node[ground]{} (2,0);

    \end{circuitikz}
\end{center}

\clearpage
\subsection{Risoluzione Esercizio 3.4}
\clearpage
\subsection{Esercizio 4.1}
Dato il Circuito sottostante,

\clearpage
\subsection{Risoluzione Esercizio 4.1}
\end{document}