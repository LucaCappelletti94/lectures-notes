\providecommand{\main}{..}
\documentclass[\main/main.tex]{subfiles}


\begin{document}


\section{Semi Conduttori}
\subsection{Introduzione ai Semiconduttori (Saltabile)}
I semi conduttori sono una categoria di materiali che hanno una conduttivita' a meta' tra conduttori ed isolanti.

Ci sono due principali tipi di semiconduttori:
\begin{enumerate}  
	\item Semiconduttori ad elemento singolo: ad esempio quelli al silicio o al germanio.
	\item Semiconduttori composti:ad esempio quelli a lega di gallio-arsenico.
\end{enumerate}
Quindi quelli ad elemento singolo sono tutti elementi con 4 elettroni di valenza nel orbitale piu' esterno, mentre quelli composti sono lege di elementi con valenza 3-5 o 2-6 in modo che in media si comporti come se avesse valenza 4 cosi che si possa formare un reticolo di legami covalenti.

Il principale utilizzo di semiconduttori composti e' per i LED.

Il reticolo di legami covalenti non conduce poiche' non vi e' carica libera pero' col aumentare della temperatura i legami si rompono e generano coppie di elettrone-lacuna ,che in quanto cariche libere rendono il materiale capace di condurre, poi si ricombinano.

Le lacune possono essere modellizzate come particelle di carica opposta al elettrone.

Ovviamente il numero di elettroni liberi e di lacune sono uguali e questo numero per $cm^3$ vale:
\[n_i = BT^{\frac{3}{2}}e^{\frac{-E_g}{2kT}}\]
\begin{tabular}{l}
	$T$ e' la temperatura espressa in gradi Kelvin
	
	$B$ e' un valore che dipende dal materiale e nel silicio vale $7.3 \times 10^{15} cm^{-3} K^{-\frac{3}{2}}$
	
	$E_g$ e' l'energia minima per rompere il legame covalente che nel silico vale $1.12 eV$ (ElettronVolt)
	
	$k$ e' la costante di Boltzmann $8.62 \times 10^{-5} \frac{eV}{K}$
\end{tabular}
a temperatura ambiente \[n_i \sim 1.5 \times 10^{10} \frac{tdc}{cm^3}\]
($tdc$ = Trasportatori di Carica)
\clearpage
\section{MOSFET}
\subsection{NMOS ed PMOS}
Esistono due tipi duali e complementari di MOSFET: NMOS (Piu' usati e con caratteristiche migliori) e i PMOS.

Definisco due costanti:

\[K_n = \frac{1}{2} \mu_n C_{ox}'\left(\frac{W}{L}\right)\]
\[K_n' = \frac{1}{2} \mu_n C_{ox}'\]
\begin{align*}
\mu_n &\text{ e' la costante di mobilita' degli elettroni}\\
C_{ox}' &\text{ e' la capacita' del condesatore che si forma tra il gate ed il canale}\\
W &\text{ e' la larghezza del canale}\\
L &\text{ e' la lunghezza del canale}
\end{align*}
quindi $K_n = K_n' \left(\frac{W}{L}\right)$

\clearpage
Per l'NMOS:

\begin{center}
\begin{circuitikz} \draw
(0,0) node[nmos] (mos) {}
(mos.gate) node[anchor=east] {G}
(mos.drain) node[anchor=south] {D}
(mos.source) node[anchor=north] {S}
;\end{circuitikz}
\end{center}
Il NMOS e' spento se la $V_{GS} < V_t$

e quindi la corrente
 \[I_{DS} = 0\]


Il NMOS e' in regime ohmico o lineare se $V_{DS} < V_{GS} - V_t$

e quindi la corrente 

\[I_{DS} = K_n \left[ 2 \left(V_{GS} - V_t \right)V_{DS} - V_{DS}^2 \right]\]


Il NMOS e' in zona di saturazione se $V_{DS} > V_{GS} - V_t$

e quindi la corrente 

\[ I_{DS} = K_n \left( V_{GS} - V_t \right)^2\]


\begin{center}
\begin{tikzpicture}
\begin{axis}[
 grid=major,
 samples=\samples,
 xlabel=$V_{GS}$,
    ylabel=$V_{DS}$,
    zlabel=$I_{DS}$
]

\addplot3[surf, unbounded coords=jump]
{x^2 + y^2 > 4 && x < 3/2 ? (x-1)^2 + y^2 : NaN};

\end{axis}
\end{tikzpicture}
\end{center}

INSERIRE QUI GRAFICI I/VDS I/VGS VDS/VGS
\clearpage
E Dualmente per il PMOS:


\begin{center}
\begin{circuitikz} \draw
(0,0) node[pmos] (mos) {}
(mos.gate) node[anchor=east] {G}
(mos.drain) node[anchor=north] {D}
(mos.source) node[anchor=south] {S}
;\end{circuitikz}
\end{center}
ATTENZIONE AI SEGNI

Il PMOS e' spento se la $\left|V_{GS}\right| < \left|V_t\right|$

e quindi la corrente
 \[I_{SD} = 0\]


Il PMOS e' in regime ohmico o lineare se $V_{SD} < V_{SG} - |V_t|$

e quindi la corrente 

\[I_{SD} = K_p \left[ 2 \left(V_{GS} - |V_t| \right)V_{SD} - V_{SD}^2 \right]\]


Il PMOS e' in zona di saturazione se $V_{SD} > V_{GS} - |V_t|$

e quindi la corrente 

\[ I_{SD} = K_p \left( |V_{GS}| - |V_t| \right)^2\]

\clearpage

\subsubsection{Come in che stato di funzionamento e' il MOSFET}
Prendiamo un NMOS per comodita'.
Metodo dei Diodi:

\begin{center}
\begin{circuitikz} \draw
(0,0) node[nmos] (mos) {}
(mos.gate) node[anchor=east] {G}
(mos.drain) node[anchor=south] {D}
(mos.source) node[anchor=north] {S}
;\end{circuitikz}
\end{center}

Il Mosfet essendo in fondo una giunzione NPN e' approssimabile a due diodi in antiserie.
Quindi sostanzialmente ci sono 3 fasi di funzionamento del MOSFET: Off,Ohm,Sat (Spento,Ohmmica,Saturazione).

Off e' quando non vi e' canale da nessuno dei due lati.

Ohm e' quando vi e' canale da entrambi i lati.

Sat quando vi e' canale da solo un lato.

Le regole normali sono:

Se la tensione $V_{GS} < V_t$ allora il MOSFET e' Off

Altrimenti se  $V_{DS} < V_{GS} - V_t$ Il MOSFET e' Ohm

Ed in fine se  $V_{DS} > V_{GS} - V_t$ Il Mosfet e' Sat

Ora osserviamo che
\[V_{DS} < V_{GS} - V_t\]
\[V_{DS} - V_{GS} <  - V_t\]
\[-V_{GS} <  - V_t\]
\[V_{DG} > V_t\]

quindi se $V_{GS} < V_t$ allora vi e' canale dal lato del Source

e se $V_{DG} > V_t$ allora vi e' canale dal lato del Drain

ora se sono entrambe vere si e' in zona Ohmmica, se una sola delle due e' vera si e' in Saturazione, quando sono entrambe false il MOSFET e' spento.


Metodo Grafico :
Basta seguire 4 punti:

$(1)$ Verificare che la $V_{GS} > V_t$

$(2)$ Calcolare la corrente $I_{DS}$ del NMOS quando $V_{DS} = V_{ow} = V_{GS} - V_{t}$

$(3)$ Calcolare le correnti ad un nodo a scelta tra SOURCE e DRAIN imponendo che $V_{DS} = V_{ow}$

$(4)$ Confrontare i due valori.

Se la corrente del NMOS e' maggiore della somma di quelle del nodo allora il NMOS e' in zona ohmmica.

Altrimenti Se la somma delle correnti del nodo e' maggiore di quella del NMOS allora esso e' in saturazione.

Dimosrazione:

QUA METTI GRAFICI BELLI PLZ

\subsubsection{Tips and Tricks}
$(1)$ I MOSFET sono simmetrici e quindi non ha senso parlare di Source e Drain pero' per aiutare convenzione si ha che:

La corrente nei MOSFET scorre sempre in senso concorde alla freccia.

La tensione $V_{GS}$ si misura sempre tra il piedino dove vi e' la freccia e il gate ed ha sempre senso contrario alla freccia.

In pratica queste sono convenzioni per suggerire il funzionamento del MOSFET a chi sta studiando il circuito.

$(2)$ Per Piccole $V_{DS}$ si puo' approssimare:

\[I_n = K_n \left[ 2 \left( V_{GS} -V_t \right)V_{DS} - V_{DS}^2 \right] \sim K_n \left[ 2 \left( V_{GS} -V_t \right)V_{DS} \right]\]
Poiche' se $V_{DS}$ e' piccolo $V_{DS}^2$ e' ancora piu' piccolo e quindi si puo' trascurare senza grossi problemi.

La quale e' una equazione lineare e quindi piu' semplice da risolvere.

Per esempio sul circuito del esercizio 1 con la equazione corretta si ottiene 

$V_r = 0.1416V$

mentre con la seconda equazione si ottiene

$V_r = 0.1435V$


\clearpage
\subsection{Come Risolvere gli esercizi sui MOSFET}
\subsubsection{Esercizio 1}
Dato il Circuito sottostante Calcolare $V_{out}$ nei casi 
$(a)$ $V_{in} = 0V$
$(b)$ $V_{in} = 3.3V$

Poi si calcoli Soglia logica $V_{th}$, Potenza Statica $P_{STAT}$.


\begin{center}
\begin{circuitikz}
\draw (0,0)
node[vcc]{3.3V}
to[R=$10k\Omega$](0,-2) -- (0,-4)
 node[nmos] (mos) {}
(mos.gate) node[anchor=east] {G,$V_{in}$}
(mos.drain) node[anchor=south] {D,$V_{out}$}
(mos.source) node[anchor=north] {S}
node[ground] {GND};
\end{circuitikz}
\end{center}

Dati:

\[V_{cc} = 3.3V\]
\[R = 1k\Omega\]
\[K_n = 5 \frac{mA}{V^2}\]
\[|V_t| = 1V\]
\subsubsection{Risoluzione Esercizio 1}
$(a)$ $V_{in} = 0V$

poiche' sia $V_{in}$ che la tensione al SOURCE allora la tensione $V_{GS} = V_G - V_S = 0$ quindi l'NMOS e' spento quindi $I_n = 0$ e poiche' il NMOS si comporta come circuito aperto anche la corrente della resistenza $I_r = I_n = 0$ e di conseguenza anche la caduta di tensione sulla resistenza e' $0$ poiche' la sua eq caratteristica e' $V = RI$ quindi non essendoci caduta di tensione sulla resistenza $V_{out} = V_{cc} = 3.3V$.

Quindi:
 \[V_{in} = 0V \Rightarrow V_{out} = 3.3V\]
 

$(b)$ $V_{in} = V_{GS} = 3.3V$

quindi $V_{ow} = |V_{GS}| - |V_t| = 2,3V$ quindi $V_{GS} > V_t$ quindi l'NMOS e' Acceso.
Ora bisogna stabilire se si trova in regime ohmmico o di saturazione e procediamo per metodo grafico:

$(1)$ Calcoliamo la corrente $I_{DS}$ quando $V_{DS} = V_{ow}$ e possiamo usare una qualunque tra le due equazioni poiche' in corrispondenza di $V_{ow}$ si raccordano entrambe nello stesso punto, quindi usiamo quella in regime di saturazione poiche' piu' semplice.

\[I_n|_{ow} = K_n \left(V_{ow}\right)^2 = 26mA\]

$(2)$ Calcoliamo la corrente del carico $I_{L} = I_{R}$ che in questo caso coincide con quella della resistenza.

\[I_R|_{ow} = \frac{V_{cc} - V_{ow}}{R} = 1mA\]

$(3)$ Ora si confrontano le due correnti:

Poiche' $I_n|_{ow} = 26mA > I_R|_{ow} = 1mA$ ci si trova in zona Ohmmica, nel caso opposto sarebbe in saturazione.

Quindi ora si calcola $V_{DS}$ Col bilancio delle correnti $I_R = I_n$

\[\frac{V_{cc} - V_{DS}}{R} = K_n \left[ 2 \left(V_{GS} - V_t \right)V_{DS} - V_{DS}^2 \right]\]

Che e' una equazione di secondo grado in $V_{DS}$ 

\[\left(K_n R \right) V_{DS}^2 - \left(2K_nR\left(V_{GS}-V_t\right)+1\right)V_{DS}+V_{cc} = 0\]

La quale parabola ha come radici:

\[V_{DS1} = 4.6V\]
\[V_{DS2} = 0.14V\]

Ovviamente ci puo' essere un solo valore vero, quindi uno e' da scartare.
In questo caso Poiche' $V_{DS1} > V_{cc}$ e $V_{DS1} > V_{ow}$ ci porta a scartare $V_{DS1}$

Quindi $V_{DS} = V_{DS2} = 0.14V$

E poiche' $V_{out} = V_{DS}$ allora $V_{out} = 0.14V$

E quindi in sinossi:

\[V_{in} = 3.3V \Rightarrow V_{out} = 0.14V\]

Calcolo della soglia logica $V_{th}$:

La soglia logica e' la tensione che separa la zona che consideriamo ON da quella che consideriamo OFF.

L'ideale sarebbe $V_{th} = \frac{V_{cc}}{2}$

\[V_{th} = \frac{\bigtriangleup V}{2} = \frac{\left | V_{on} - V_{off} \right |}{2} = \frac{3.3V - 0.14V}{2} = 1.58V\]

Calcolo Delle Potenze Statiche $P_{STAT}$:

In questo circuito abbiamo due potenze statiche, quando la porta e' ON e quando e' OFF.

Caso ON $V_{in} = 0V$:

\[P_{STAT,On} = V_{cc} I_{n} = 0W\]
Poiche' non scorre corrente, il consumo di corrente e' 0 watt. Ottimo.


Caso OFF $V_{in} = 3.3V$:

$P_{STAT,Off} = V_{cc} I_{n} = 3.3V * I_{n}$

coi dati prima calcolati possiamo ricavare $I_{n}$

\[I_{n} = I_{r} = \frac{V_{cc} - V_{DS}}{R} = \frac{3.3V - 0.14V}{1k\Omega} = 3.16mA\]
\[P_{STAT,Off} = V_{cc} I_{n} = 3.3V * 3.16mA = 10,4mW\]
Un consumo veramente grande per una porta cosi piccola. SI puo' fare di meglio.

\clearpage
\subsubsection{Esercizio 2}
Dato il Circuito sottostante Calcolare $V_{out}$ nei casi 
$(a)$ $V_{in} = 0V$
$(b)$ $V_{in} = 3.3V$

Poi si calcoli Soglia logica $V_{th}$, Potenza Statica $P_{STAT}$, Potenza Dinamica $P_{DIN}$, e tempo di propagazione $t_p$.



Dati:

\[V_{cc} = 3.3V\]
\[R = 1k\Omega\]
\[C = 1pF\]
\[|K_p| = 2 \frac{mA}{V^2}\]
\[|V_t| = 1V\]
\subsubsection{Risoluzione Esercizio 2}
$(b)$ $V_{in} = 3.3V$

Poiche' $V_{cc} = V_{in} = 3.3V$ allora $V_{SG} = V_{cc} - V_{in} = 0V$ e 
$V_{SG} < |V_t|$ quindi il PMOS e' spento! Quindi $I_p = I_{SD} = 0A$ ora con una KCL al nodo del DRAIN otteniamo che $I_p = I_r + I_c$ quindi $I_r + I_c = 0$ ora poiche' l'eq caratteristica del condensatore e' $i_c(t) = C \frac{d}{dt}V_c$ e si suppone sempre che i transitori siano finiti allora il condensatore e' scarico $V_c = 0$ e quindi la sua corrente $I_c = 0$, il che implica che $I_r + I_c = I_r = 0$ e quindi la tensione $V_r = R I_r = 0$ e di conseguenza: $V_{out} = V_c = V_r = 0V$.

\[V_{in} = 0V \Rightarrow V_{out} = 0V\]

$(a)$ $V_{in} = 0V$

$V_{SG} = V_{in} - V_{cc} = -3.3V$ e $|V_{SG}| > |V_t|$ e $V_{ow} = |V_{SG}| - |V_t| = -2.3V$ quindi il PMOS e' Acceso.
Ora bisogna stabilire in che zona di lavoro sia, procediamo per metodo grafico.

$(1)$ Calcoliamo la corrente del PMOS alla tensione di overdrive $V_{ow}$:

\[I_p |_{ow} = K_p \left(V_{ow}\right)^2 = 10.58mA\]

$(2)$ Calcoliamo la corrente di carico assumendo che $V_{DS} = V_{ow}$

Poiche' la resistenza ed il condensatore sono in parallelo $V_r = V_c$ e cosi con una KVL si ottiene che $V_r = V_c = V_{cc} - V_{DS} = 1V$ poiche' si calcola in condizioni di regime Il condensatore e' completamente carico a $V_c = 1V$  e quindi come sopra poiche' il consensatore e' carico la sua corrente $I_c = 0$.

Quindi dalla KCL al nodo del DRAIN la corrente \[I_{DS}|_{ow} = I_r + I_c = I_r = \frac{V_r}{R}= \frac{ V_{cc} - V_{ow}}{R} =  \frac{ V_{cc} - V_{cc} + V_t}{R} = \frac{V_t}{R} = 1mA\]

$(3)$ Confrontando le due correnti $I_{DS}|_{ow} = 1mA < I_p |_{ow} = 26mA$ quindi il PMOS si trova in zona Ohmmica.


Stabilito cio' si calcola il punto di lavoro col bilancio delle correnti:
$I_r = I_{DS,Ohm}$

\[\frac{V_{cc} - |V_{SD}|}{R} = K_p \left[ 2 \right(|V_{GS}| - |V_t|)V_{SD} - V_{SD}^2\]
e quindi otteniamo una equazione di secondo grado in $V_{SD}$ che risolvendola ha come soluzioni:

$V_{SD1} = 4.75V$ che scarteremo poiche' $V_{SD1} > V_{cc}$ e $V_{SD1} > |V_{GS}| - |V_t|$ quindi dovrebbe essere in saturazione quando abbiamo gia' dimostrato che e' in zona ohmmica.

e

$V_{SD2} = V_{SD} = 0.347V$ che e' la soluzione corretta.

Ora concludiamo con una KVL dalla quale si ottiene $V_{out} = V_{cc} - V_{SD} = 2.96V$

In Sinossi:
\[V_{in} = 0V \Rightarrow V_{out} = 2.96V\]
.

\end{document}