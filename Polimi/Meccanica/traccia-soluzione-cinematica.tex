\documentclass[main.tex]{subfiles}
 
\begin{document}

\section{Traccia generale di un esercizio di cinematica}

\subsection{Avvertenze}
\textbf{Questo NON è un "algoritmo risolutivo", ma una linea guida su come questo tipo di esercizio va svolto.}
\\
Può essere molto utile per identificare quali argomenti sia necessario conoscere e ripassare prima di poter affrontare un esercizio di questo tipo, alcuni magari ovvi, e per questo viene fornita una checklist da riempire.
\\
Ogni esercizio è unico e questa lista non può e non intende sostituire uno studio adeguato dell'argomento, quindi non intendetela come tale.

\subsection{Argomenti}
\begin{todolist}
\item Sapere derivare equazioni nel campo complesso.
\item Sapere realizzare equazioni di chiusura.
\item Sapere risolvere sistemi lineari.
\item Conoscere il teorema dell'energia cinetica, per masse e momenti d'inerzia.
\item Sapere risolvere prodotti scalari.
\item Conoscere il bilancio delle potenze.
\item Avere almeno una base generale di cinematica.
\end{todolist}

\subsection{Traccia primo punto}
\begin{enumerate}
\item Capire come il sistema va a muoversi.
\item Identificare eventuali vettori (velocità, accelerazioni) che vengono forniti indirizzati in senso opposto al moto del sistema.
\item Identificare vincoli cinematici.
\item Identificare un'equazione di chiusura.
\item Trascrivere l'equazione di chiusura in forma complessa, rappresentante lo spostamento.
\item Separare in componenti cartesiane ed ottenere i termini incogniti.
\item Derivare una volta la forma complessa, ottenendo velocità.
\item Separare la velocità in forma cartesiana ed ottenere i termini incogniti.
\item Derivare una seconda volta la forma complessa, ottenendo accelerazione.
\item Separare l'accelerazione in forma cartesiana ed ottenere i termini incogniti.
\end{enumerate}

\subsection{Traccia secondo punto}
\begin{enumerate}
\item Calcolare l'energia cinetica totale sommando le componenti di tutte le masse in moto di traslazione ed i momenti d'inerzia di corpi in rotazione.
\item Derivare energia cinetica.
\item Calcolare potenza totale, sommando i contributi di tutte le forze e coppie applicate a corpi in moto.
\item Usare il bilancio delle potenze.
\end{enumerate}

\end{document}