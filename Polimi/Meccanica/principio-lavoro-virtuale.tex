\documentclass[main.tex]{subfiles}

\begin{document}

\section{Principio del lavoro virtuale (PLV)}
A differenza dell'approccio che studia l'equilibrio di \textit{forze} e \textit{momenti}, esiste un approccio basato su considerazioni di tipo energetico.

\begin{definition}[Spostamento o rotazione virtuale]
Uno spostamento o rotazione virtuale, è rispettivamente uno spostamento o una rotazione che rispetta le seguenti caratteristiche:
\begin{enumerate}
\item È infinitesimo.
\item È arbitrario.
\item È compatibile con i vincoli (non rompe le condizioni di vincolo).
\item È reversibile, cioè può avvenire in entrambi i sensi.
\item Avviene in tempo congelato.
\end{enumerate}
Vengono indicati con il simbolo: $\delta\vec{S}_i$ e $\delta\vec{\theta}_j$.
\end{definition}

\begin{definition}[Principio del lavoro virtuale]
In un sistema meccanico con vincoli fissi e in assenza di attrito, condizione necessaria e sufficiente per l ’equilibrio è che sia nullo il \textit{lavoro virtuale} compiuto dalle forze e dalle coppie attive per qualsiasi spostamento virtuale del sistema.
\\
Viene calcolato come il prodotto scalare di forze e spostamenti virtuali sommati a quello di coppie e rotazioni virtuali.
\[
	\delta L = \sum_i \vec{F}_i\bullet \delta\vec{S}_i + \sum_j \vec{C}_j\bullet \delta\vec{\theta}_j = 0
\]
\end{definition}

\subsection{Cosa cambia dal metodo delle equazioni cardinali della statica}
Per risolvere un esercizio con il metodo delle \textbf{equazioni cardinali della statica} venivano aggiunte come incognite le reazioni vincolari (le reazioni orizzontali, verticali e momenti), ma in questo metodo esse vengono interamente ignorate poichè non compiono lavoro.
\\
\\
Nel caso, per esempio, di un'asta inclinata di massa non trascurabile su cui viene applicata una forza $F_1$ esterna, nel bilancio del \textbf{PLV} sarebbero considerate unicamente la forza peso $F_g$ agente sul centro di massa dell'asta e la forza $F_1$. Le incognite introdotte saranno gli spostamenti e rotazioni virtuali che le forze producono sul sistema.


\end{document}