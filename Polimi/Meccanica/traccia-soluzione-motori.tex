\documentclass[main.tex]{subfiles}
 
\begin{document}

\section{Traccia generale di un esercizio di motori}

\subsection{Avvertenze}
\textbf{Questo NON è un "algoritmo risolutivo", ma una linea guida su come questo tipo di esercizio va svolto.}
\\
Può essere molto utile per identificare quali argomenti sia necessario conoscere e ripassare prima di poter affrontare un esercizio di questo tipo, alcuni magari ovvi, e per questo viene fornita una checklist da riempire.
\\
Ogni esercizio è unico e questa lista non può e non intende sostituire uno studio adeguato dell'argomento, quindi non intendetela come tale.

\subsection{Argomenti}
\begin{todolist}
\item Conoscere il funzionamento di un motore modellizzato, con potenza motrice, resistente, perduta, trasmissione, etc...
\item Conoscere il bilancio di potenze per motori.
\item Conoscere le regole per moto diretto e retrogrado.
\item Conoscere le regole per regime e transitorio.
\item Conoscere le formule per attrito statico, dinamico e volvente.
\item Sapere risolvere prodotti scalari.
\end{todolist}

\subsection{Traccia generale}
\begin{enumerate}
\item Comprendere come si muove il sistema.
\item Individuare eventuali vettori rappresentati in direzione opposta al moto.
\item Individuare legami cinematici.
\item Calcolare potenza motrice.
\item Calcolare potenza resistente.
\item Calcolare e derivare energia cinetica totale.
\item Individuare tipo di moto, se diretto o retrogrado.
\item Calcolare la potenza perduta.
\item Usare bilancio di potenze.
\end{enumerate}

\end{document}