\documentclass[main.tex]{subfiles}
 
\begin{document}

\section{Traccia generale di un esercizio di statica}

\subsection{Avvertenze}
\textbf{Questo NON è un "algoritmo risolutivo", ma una linea guida su come questo tipo di esercizio va svolto.}
\\
Può essere molto utile per identificare quali argomenti sia necessario conoscere e ripassare prima di poter affrontare un esercizio di questo tipo, alcuni magari ovvi, e per questo viene fornita una checklist da riempire.
\\
Ogni esercizio è unico e questa lista non può e non intende sostituire uno studio adeguato dell'argomento, quindi non intendetela come tale.

\subsection{Argomenti}
\begin{todolist}
\item Sapere risolvere sistemi lineari.
\item Sapere cosa significa che un sistema è isostatico.
\item Conoscere TUTTI i tipi di vincoli e le reazioni vincolari per ogni tipo.
\item Sapere risolvere prodotti vettoriali (per il segno nel calcolo dei momenti).
\item Conoscere le convenzioni per taglio, sforzo e momento flettente.
\end{todolist}

\subsection{Traccia primo punto}
\begin{enumerate}
\item Verificare con l'analisi dei vincoli l'isostaticità del sistema.
\item Considerare inizialmente i vincoli esterni, costruendo un sistema con le reazioni di questi vincoli alle forze applicate.
\item Aggiungere al sistema le reazioni vincolari di uno o più dei corpi rigidi del sistema, sino ad avere un numero di equazioni indipendenti pari al numero di incognite.
\end{enumerate}

\subsection{Traccia secondo punto}
\begin{enumerate}
\item Separare le forze applicate sull'asta richiesta nelle componenti di sforzo e taglio.
\item Calcolare momento flettente massimo, partendo da ENTRAMBI i lati, per avere una verifica.
\item Disegnare i grafici.
\end{enumerate}

\end{document}