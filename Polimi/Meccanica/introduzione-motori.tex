\documentclass[main.tex]{subfiles}
 
\begin{document}

\section{Motori}

\subsection{Come identificare il tipo di moto, se diretto o retrogrado}

Definita $W_m$ (o $W_1$) come la potenza motrice, $W_r$ (o $W_2$) come la potenza resistente, chiamata a volte anche potenza dell'utilizzatore $W_u$ e $W_p$ come potenza perduta, procediamo a identificare il tipo del moto in condizioni di regime e di transitorio.

\begin{figure}[H]
	\[
		\dfrac{dE_{c_m}}{dt} = J_m\omega\dot{\omega}
	\]
	\caption{Spesso la derivata dell'$E_{c_m}$ assume questo valore.}
\end{figure}

\begin{figure}[H]
\[
	\dfrac{dE_c}{dt} = \dfrac{dE_{c_m}}{dt} + \dfrac{dE_{c_r}}{dt}
	\Longrightarrow
	\dfrac{dE_{c_r}}{dt} = \dfrac{dE_c}{dt} - \dfrac{dE_{c_m}}{dt} 
\]
\caption{Come si definisce la derivata di $E_{c_r}$.}
\end{figure}

\begin{figure}[H]
  \begin{subfigure}[b]{.5\textwidth}
  \centering
  \[
	\text{Moto diretto}: \begin{cases}
		W_m - \dfrac{dE_{c_m}}{dt} > 0 \\
		W_u - \dfrac{dE_{c_u}}{dt} < 0
	\end{cases}  
  \]
  \caption{Condizioni di moto diretto.}
  \end{subfigure}
  \hfill
  \begin{subfigure}[b]{.5\textwidth}
  \centering
  \[
	\text{Moto retrogrado}: \begin{cases}
		W_m - \dfrac{dE_{c_m}}{dt} < 0 \\
		W_u - \dfrac{dE_{c_u}}{dt} > 0
	\end{cases}    
  \]
  \caption{Condizioni di moto retrogrado.}
  \end{subfigure}
  \caption{Analisi del tipo di moto.}
\end{figure}

\begin{figure}[H]
  \begin{subfigure}[b]{.5\textwidth}
  \centering
  \[
	\text{Moto diretto}: \begin{cases}
		W_m > 0 \\
		W_u < 0
	\end{cases}  
  \]
  \caption{Condizioni di moto diretto a regime.}
  \end{subfigure}
  \hfill
  \begin{subfigure}[b]{.5\textwidth}
  \centering
  \[
	\text{Moto retrogrado}: \begin{cases}
		W_m < 0 \\
		W_u > 0
	\end{cases}    
  \]
  \caption{Condizioni di moto retrogrado a regime.}
  \end{subfigure}
  \caption{Analisi del tipo di moto a regime.}
\end{figure}
\end{document}