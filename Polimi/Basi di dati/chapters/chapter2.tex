\providecommand{\main}{..}
\documentclass[\main/main.tex]{subfiles}
\begin{document}

\chapter{I linguaggi formali}
\section{Algebra}
L'algebra è un linguaggio formale caratterizzato da cinque operazioni fondamentali e tre operazione derivate.
Le operazioni fondamentali sono:
\begin{enumerate}
  \item Selezione $\sigma$
  \item Proiezione $\pi$
  \item Unione $\cup$
  \item Differenza -
  \item Prodotto cartesiano X
\end{enumerate}
Le operazioni derivate sono:
\begin{enumerate}
  \item Intersezione $\cap$
  \item Semijoin $\ltimes$
  \item join $\bowtie$
\end{enumerate}

Data una tabella di studenti chiamata ``STUDENTE'', contenente matricola, nome, città, definiamo le operazioni tramite i seguenti esempi.


\subsubsection{Selezione}

Esempio: $\sigma_{Nome='Paola'}$ STUDENTE

La selezione è un operatore che produce come risultato una tabella che abbia lo stesso schema (cioè le stesse colonne) di studente e come istanze le sole righe che soddisfano il predicate $Nome=Paola$.

\subsubsection{Proiezione}

Esempio: $\pi_{Nome}$ STUDENTE

La proiezione è un operatore che produce come risultato una tabella che abbia come schema soltanto la colonna \textit{nome} e come istanze tutte le righe della tabella STUDENTE.

Da questo punto in poi per comodità considereremo due generiche tabelle, TABELLA1 e TABELLA2 per gli esempi successivi.

\subsubsection{Unione}

Esempio: TABELLA1 $\cup$ TABELLA2.

L'unione è un operatore che produce come risultato una tabella che abbia come schema le colonne di TABELLA1 e come righe le righe delle due tabelle concatenate.
È possibile realizzare questa operazione \textbf{solo} se le tabelle sono compatibili.

\subsubsection{Differenza}

Esempio: TABELLA1 - TABELLA2.

La differenza è un operatore che produce come risultato una tabella che abbia come schema lo schema di TABELLA1 e come righe la differenza delle tuple delle due tabelle (cioè elimina le righe che compaiono in entrambe le tabelle).
È possibile realizzare questa operazione \textbf{solo} se le tabelle sono compatibili.


\subsubsection{Prodotto cartesiano}

Esempio: TABELLA1 X TABELLA2

Il prodotto cartesiano è un operatore che produce come risultato una tabella che abbia come schema le colonne delle due tabelle e come tutte le possibli coppie degli elementi di TABELLA1 e TABELLA2.

\subsubsection{Unione}

Esempio: TABELLA1 $\cap$ TABELLA2.

L'intersezione è un operatore che produce come risultato una tabella che abbia come schema le colonne di TABELLA1 e come righe le righe che compaiono in entrambe le tabelle.
È un'operazione derivata perché è equivalente a TABELLA1 - (TABELLA1 - TABELLA2).
È possibile realizzare questa operazione \textbf{solo} se le tabelle sono compatibili.


\subsubsection{Join}

Esempio: TABELLA1 $\bowtie_{TABELLA1.attributo1=TABELLA.attributo1}$ TABELLA2.

La join è un operatore che produce come risultato una tabella che abbia come schema la concatenazione degli schemi delle due tabelle e come righe la concatenazione delle righe che soddisfano i predicato della join.
È un'operazione derivata perché è equivalente a $\sigma_{TABELLA1.attributo1=TABELLA2.attributo1}$ TABELLA1XTABELLA2.

La join che utilizza come operatore del predicato l'operatore di uguaglianza (=) è detta \textit{equi-join.}
Un'equi-join che voglia selezionare solo gli attributi omonimi è detta \textit{natural join} e viene espressa omettendo il predicato che confronta gli attributi (quindi TABELLA1 $\bowtie$ TABELLA2).


\subsubsection{Semi-join}

Esempio: TABELLA1 $\ltimes_{TABELLA1.attributo1=TABELLA.attributo1}$ TABELLA2.

La semi-join è un operatore che produce come risultato una tabella che abbia come schema lo schema di TABELLA1 e come istanze le tuple ottenute proiettando su TABELLA1 la join di TABELLA1 e TABELLA2.

Esiste anche la semi-join naturale, che proietta su TABELLA1 la join naturale delle due tabelle.


\section{Calcolo relazionale}
Il calcolo relazionale è un linguaggio dichiarativo: esprime cosa si vuole ottenere ma non la ``strada'' per giungere al risultato.
La forma standard di una qualunque query nel calcolo relazionale è ${ t | p(t) }$, dove t è una variabile e p(t) un predicato in funzione di t, detto \textit{formula}.
Ogni formula è costituita da \textit{atomi}, dove ogni atomo può essere:
\begin{itemize}
  \item t \in R, dove R è un dominio;
  \item espressione - comparatore - espressione, dove un comparatore è un operatore come =, <>, >, >=, ... e un'espressione è un'insieme di costanti e di restrizioni. Una restrizione è espressa come t[A] e seleziona l'attributo A della variabile t.
\end{itemize}
\subsection{Proprietà del calcolo relazionale}
Per il calcolo relazionale valgono le seguenti proprietà:
\begin{description}
  \item[La legge di de Morgan:] $ p1 \wedge  p2 \equiv \neg (\neg p1 \vee \neg p2 ) $
  \item [La Corrispondenze tra quantificatori:]  $\forall t \in R (p(t)) \equiv \neg \exists t \in R (\neg p(t))$
  \item[Definizione di implicazione:] $p1  \Rightarrow  p2 \equiv \neg p1 \vee p2$
\end{description}

\subsection{Formule corrette}
Una formula è considerata corretta \textbf{solo} se non dipende dal dominio. Cioè è indipendente dal dominio degli attributo, ma dipende soltanto dall'istanza del database che stiamo analizzando.

\subsection{Relazione con l'algebra}
È possibile esprimere attraverso il calcolo relazionale tutti gli operatori dell'algebra, e viceversa è possibile esprimere qualunque query del calcolo attraverso l'algebra.

\section{Datalog}
Datalog è un linguaggio formale basato sulle \textit{regole}.
Ogni regola è una ``riga'' datalog, formata da una head e da un body, separati dall'operatore :- e formati da predicati.
Ad esempio, per la regola seguente il predicato P è la head, P1 e P2 sono il body:
P :- P1, P2
Ogni predicato è formato da un nome e uno o più argomenti.
Gli argomenti possono essere variabili, costanti oppure ``don't care''. Il ``don't care'' è rappresentato dal simbolo \_ ed significa che qualunque attributo va bene. Non può essere usato nella head.

L'interpretazione delle regole è che la testa è vera se il corpo è vero.

Le query sono realizzate mettendo ``?'' nella testa:
?-Predicato(``Costante'', Variabile)
Questa query cerca una regola che abbia Predicato come testa e cerca valori della Variabile per cui valga ``Costante''.

Una query senza variabile, come ?-Predicato(``costante'') restituisce true o false.

\subsection{Negazione e query ricorsive}
È possibile negare predicati in datalog premettendo ``not'' al predicato.
Introducendo la negazione, è possibile esprimere in datalog tutte le query dell'algebra relazionale e, in aggiunta, anche le query ricorsive (che nell`algebra non abbiamo).
Una query ricorsiva è una query che presenta nel body il predicato della testa.

Perché la regola sia valida, la negazione deve essere \textit{safe}.
Una negazione è safe se:
\begin{itemize}
  \item tutte le variabili di un letterale negato compaiono in un letterale positivo del body della regola.
  \item Il predicato positivo e il suo negato non dipendono l'uno dall'altro. Quindi P(X) :- R(X), \neg P(X) non è una negazione safe.
\end{itemize}


\end{document}