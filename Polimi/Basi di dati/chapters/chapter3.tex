\providecommand{\main}{..}
\documentclass[\main/main.tex]{subfiles}
\begin{document}

\chapter{Le query SQL}

\section{Sintassi delle query SQL}
Ogni query SQL si compone di tre clausole:
\begin{itemize}
  \item select
  \item from
  \item where
\end{itemize}

Ogni query ha sintassi:

select *attributo*

from *tabella*

  [ where *condizione* ]


Dunque \textit{select} indica l'attributo che ci interessa estrarre, \textit{from} la tabella da cui estrarre l'attributo, \textit{where} la condizione che l'attributo deve rispettare affinché sia rilevante per la nostra query.

La condizione indicata dal \textit{where} può essere espressa tramite un'espressione i cui operatori sono:
\begin{itemize}
  \item \textbf{predicati semplici dell'algebra}: uguaglianza, operatori booleani ...
  \item \textbf{between}: indica gli estremi di un intervallo.
  \item \textbf{distinct}: impedisce che vi siano duplicati fra i risultati della query
\end{itemize}

\textbf{like}: indica parte del nome dell'attributo cercato, ad esempio
\begin{minted}{sql}
nome like 'Mar\%o'
\end{minted}
cerca i nomi che iniziano con ``Mar'' e finiscono con ``o''. Con like si usano i simboli \_ e \%, che indicano rispettivamente un carattere e una sottostringa (di lunghezza arbitraria) di cui non conosciamo (o non ci interessa) il valore; dunque qualunque valore essi abbiano nell'attributo, tale attributo sarà ritenuto valido e restituito dalla query.

\subsection{Attributi con valore null}
Esiste un valore speciale \textbf{null}, che viene utilizzato se non si conosce un valore, se un valore non si può applicare a un determinato attributo, se non si sa se tale valore possa essere applicato a un determinato attributo.
/textit{Null} può essere usato nelle query attraverso gli operatori
\begin{minted}{sql}
attributo is null
\end{minted}
e
\begin{minted}{sql}
attributo is null
\end{minted}
che restituiscono rispettivamente le righe con valore nullo per un determinato attributo e quelle con valore non nullo per un determinato attributo.

\subsection{Tabelle e join}
Le tabelle indicate dopo la clausola \textit{where} indicano il dominio della query.
Possono essere indicate più tabelle, separate da una virgola: la query ne farà automaticamente il prodotto cartesiano e cercherà l'attributo richiesto in tale prodotto cartesiano.
Si può anche effettuare la join in modo esplicito, con l'espressione

\begin{minted}{sql}
select attributo from tabella1 join tabella on condizione_su_cui_viene_effettuata_la_join
\end{minted}
La condizione della join è espressa come in algebra.

La \textit{join} può essere di due tipi differenti:
\begin{itemize}
  \item interna
  \item esterna
\end{itemize}
Per indicare la \textit{join} interna si scrive semplicemente ``join'', per indicare una \textit{join} esterna si scrive ``left/right/full join''.
Una \textit{join} esterna restituisce anche le righe per cui la condizione espressa dall''on'' della join restituisca valori nulli.
Dunque con una \textit{left join} prenderò tutte le righe della tabella dichiarata a sinistra della join, anche se in quella di destra alcune di esse avranno attributi con valori nulli.
La \textit{right join} fa lo stesso ma con le righe della tabella a destra dell'operatore join.
La \textit{full join} prende tutte le righe di entrambe le tabelle.

\subsection{Ridenominazione}
È possibile ``dare un nome'' al risultato della query usando l'operatore ``as'', ad esempio:
\begin{minted}{sql}
select * as informatico from STUDENTE where ...
\end{minted}
\textit{As} si usa anche nell'espressione della \textit{from} nel caso si debbano estrarre più variabili dalla stessa tabella.
Ad esempio
\begin{minted}{sql}
select * from STUDENTE as stud1, STUDENTE as stud2, ...
\end{minted}


\section{Modificare il database in SQL}
È possibile:

- Effettuare inserimenti:
\begin{minted}{sql}
  insert into nome tabella values lista valori
\end{minted}
vengono messi a null o al valore di default.

Se mancano dei valori nella lista
Al posto di usare ``values + lista valori'' si può inserire il risultato di una query, scrivendo la query, come una normale query, alla fine del comando:
\begin{minted}{sql}
  insert into nome tabella query
      \end{minted}


- Effettuare cancellazioni:
\begin{minted}{sql}
   delete from nome tabella where condizione che identifica gli elementi da cancellare
\end{minted}

- Modificare i valori degli attributi:
\begin{minted}{sql}
  update nome_tabella set attributo = valore where condizione
\end{minted}

Cancellare intere tabelle: \textbf{drop table *nome tabella*}

\begin{minted}{sql}
update nome_tabella set attributo = valore where condizione
\end{minted}

\section{Query complesse}
\subsection{Ordinamento}
È possibile riordinare i risultati di una query attraverso il comando \textbf{order by}.
Si usa la sintassi:
\begin{minted}{sql}
order by AttributoOrdinamento [ crescente | descrescente ]
\end{minted}

\subsection{Funzioni aggregate}
Le funzioni aggregate sono funzioni, utilizzate all'interno delle espressioni della clausola \textit{where}, che utilizzano operatori complessi, che operano su più elementi del database.
Esistono cinque operatori SQL per realizzate funzioni aggregate:
- \textbf{count:} restituisce il numero di righe per un certo attributo. Se si aggiunge l'operatore \textit{distinct}, restituisce il numero di righe per l'attributo omettendo le righe duplicate.
\begin{minted}{sql}
count( * | [ distinct | all ] ListaAttributi )
\end{minted}
- \textbf{sum:} restituisce la somma dei valori dell'attributo passato come parametro.
- \textbf{max:} restituisce il massimo fra i valori dell'attributo passato come parametro.
- \textbf{min:} restituisce il minimo fra i valori dell'attributo passato come parametro.
- \textbf{avg:} restituisce la media fra i valori dell'attributo passato come parametro.
\begin{minted}{sql}
sum|max|min|avg ([distinct|all] Attributo )
\end{minted}

\subsection{Raggruppamento}
È possibile che ci sia il bisogno di applicare gli operatori appena visti a un sottoinsieme di righe di una tabella, non all'intera tabella.
Allora si usa l'operatore \textit{group by}, che seleziona le righe che ci interessano.
\begin{minted}{sql}
group by attributo1 having operatore_aggregato(attributo2)
\end{minted}

\subsection{Query binarie}
Sono realizzate concatenando due query attraverso gli operatori insiemistici di unione, intersezione e differenza.
\begin{minted}{sql}
Query1  union | intersect | except  [ all ] Query2
\end{minted}
Se si usa l'operatore ``all'' vengono inclusi anche eventuali duplicati, se si omette essi sono esclusi automaticamente.

\subsection{Query nidificate}
Le query nidificate sono query che contengono ulteriori query al proprio interno.
In particolare, si tratta di query che hanno nella propria clausola \textit{where} un predicato che effettua il confronto con il risultato di un'altra query.
Questi predicati sono formati da due elementi:
\begin{description}
  \item[- L`operatore:] è un operatore logico, quindi =,<>,<,<=,>,...
  \item[- any|all:] indicano la quantità di righe con cui fare il confronto.
\end{description}
\textit{All} indica che il confronto effettuato dall'operatore deve valere per ciascuna riga della query espressa dopo l'\textit{all}, \textit{any} invece indica che il confronto deve valere per almeno una riga della query espressa dopo l'\textit{any}.
Ad esempio:
\begin{minted}{sql}
select Nome
from STUDENTE
where Nome = any
  select Nome
  from STUDENTE
\end{minted}
estrae gli studenti con almeno un omonimo.

Esistono anche altre predicati per effettuare le query nidificate.
Questi predicati sono:
-\textbf{in}: è vero se l'espressione è presente almeno in una riga restituita da Query2:
\begin{minted}{sql}
Espressione in Query2
\end{minted}
Esiste anche il \textit{not in}, che è il negato di in e ha la stessa sintassi.
Il predicato \textit{in} è equivalente a \textit{= any}.
-\textbf{exists}: è vero se Query2 ha almeno una tupla.
\begin{minted}{sql}
exists Query2
\end{minted}
Il suo negato è \textit{not exists}.

\subsubsection{Visibilità delle variabili nelle query nidificate}
Le variabili sono visibili \textbf{solo} nelle query in cui sono dichiarate o nelle query nidificate ad esse.
Dunque due query nidificate alla stessa query principale ma non fra loro \textbf{non} vedono le variabili l'una dell'altra, ma entrambe vedono le variabili della query principale.

\subsubsection{Query nidificate nelle modifiche}
Le query nidificate possono essere utilizzate anche con il comando \textbf{update}.
È sufficiente scrivere:
\begin{minted}{sql}
update nome_riga
    set attributo = Query2
\end{minted}

\section{Permessi d'accesso}
È possibile controllare l'accesso di ciascun utente al database. In particolare si può consentire/limitare le operazioni che ogni utente può effettuare su di esso.
Per concedere un permesso, si utilizza il comando \textbf{grant}. Per revocare un permesso, si utilizza il comando \textbf{revoke}.
La sintassi è la seguente per la grant:
\begin{minted}{sql}
grant < nome_privilegi | all privileges > on nome_risorsa nome_utente [withgrantoption]
  \end{minted}
L'espressione \textit{withgrantoption}, che è opzionale, indica che l'utente a sua volta è autorizzato a fornire il privilegio ad altri utenti.

La revoke invece ha sintassi:
\begin{minted}{sql}
revoke nome_privilegi on nome_risorsa from nome_utente [ restrict | cascade ]
\end{minted}

\section{Transazioni}
Le transazioni sono sequenze di comandi SQL che devono essere racchiuse fra l'espressione \textbf{begin transaction} e \textbf{end transaction}. Attraverso queste transazioni è possibile effettuare delle modifiche al database.
La differenza fra l'utilizzare una transazione e non utilizzarla sta nel fatto che, utilizzando la transazione, è possibile decidere se rendere permanenti le modifiche oppure eliminarle e tornare allo stato iniziale.
Infatti, prima di ogni \textit{end transaction} è necessario decidere mantenere o no le modifiche attraverso, rispettivamente, i comandi \textbf{commit work} e \textbf{rollback work}. Il primo rende definitive le modifiche, il secondo le cancella e mantiene il database allo stato precedente la transazione.


\end{document}