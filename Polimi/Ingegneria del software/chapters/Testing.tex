\providecommand{\main}{..}
\documentclass[\main/main.tex]{subfiles}


\begin{document}
\section{Tipi di Testing}
\begin{itemize}
\item White Box: si generano casi di test basandosi sulla specifica.
\item Black Box: si generano casi di test basandosi sul codice.
\end{itemize}
\section{BlackBox / Funzionale}
\begin{itemize}
\item Genero casi di test per ogni ``categoria'' dei parametri
\item Genero casi per ogni corner case fra ``categorie''
\end{itemize}
\section{Tipid i test Black Box}
Test combinatori:
\begin{itemize}
\item Identifico attributi che possono cambiare durante lesecuzione
\item Si generano combinazioni dei possibili valori degli attributi
\item Le combinazioni devono contemplare:
\begin{itemize}
\item ``valori standard''
\item Corner case (= ad esempio se ho valori in un intervallo prendo gli estremi)
\item valori errati
\end{itemize}
\end{itemize}

\section{WhiteBox / Strutturale}
Genero casi di test per ``percorrere'' tutto il programma
Quindi percorrere tutti i cammini possibli.
I cammini sono le strade che il programma puo' prendere, cioe' tutti \textbf{i modi in cui puo' arrivare dall' inizio alla fine}.
Possono anche vederli come la sequenza di righe di codice che eseguo per arrivare alla fine.

\section{Tipi di test White-Box}

\begin{itemize}
\item \textbf{Copertura dei cammini}: Genero casi che portano ad eseguire tutte le psosibili combinazioni di istruzioni.
Devo contare le iterazioni dei cicli ed i metodi in cui posso uscire da essi per avere l numero totale dei cammini.
\item \textbf{Copertura delle istruzioni}: Genero casi per cui eseguo ogni istruzione \textbf{Almeno} una volta.
\item \textbf{Copertura delle Diramazioni}: Genero casi che mi permettono di coprire tutte le diramazioni, cioe' immaginidnando il programma come diagramma di flusso, devo poter toccare ogni nodo.
\item \textbf{Copertura delle condizioni}: genero casi che coproono tutti i possibili \textbf{Risultati} delle condizioni dei cicli e degli if.
\end{itemize}

\section{Altri tipi di test}

\begin{itemize}
\item \textbf{Test di unita'}: testo ogni modulo separatamente.
\item \textbf{Test di integrazione}: testo sottinsieme a mano a mano piu' grandi di moduli vedendo come ingeragiscono i moduli nei sottinsiemi.
\item \textbf{Test di sistema}: testo il programma completo.
\item \textbf{Test di regressione}: se il programma viene aggiornato, testo l'aggiornamento.
\end{itemize}

\section{ScaffHolding}:
E' composto da:
\begin{itemize}

\item Driver $\longrightarrow$ Componente che simula la parte di programma che invoca il modulo da testare. Prepara l'ambiente, i parametr, fa chiamate accessorie e verifica che vadano a buon fine.
\item Stub  $\longrightarrow$  Componente che simula la parte chaimata dal modulo (= cioe` esegue il modulo da testare, verigica l`ambiente creato dal driver e i parametri) restituisce i risultati richiesti dalle specifiche.
\end{itemize}
Di base lo scaffholding serve per automatizzare i Test.

\end{document}
