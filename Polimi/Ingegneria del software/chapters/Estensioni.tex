\providecommand{\main}{..}
\documentclass[\main/main.tex]{subfiles}


\begin{document}

\section{Estensione}
E' un altro modo per dire ereditarieta'.
Un' esntensione e' \textbf{pura} se non modifica la specifica dei metodi.

\section{Rappresentazione per estensioni}
Di solito la sottoclasse mantiene la AF della superclasse.
Non modifica \textbf{MAi} l'RI della superclassse ed aggiunge il proprio RI.

\section{Verificare validita' delle estensioni}
\begin{itemize}
\item \textbf{Principio di sostituzione di Liskov}: ``ogni modulo che usa un oggetto della superclasse deve poter usare un oggetto della sottoclasse senza accorgersi della differenza''
\item \textbf{Regola della segnatura}: ``La sottoclasse deve avere tutti i metodi della superclasse con la stessa intestazione ma puo' avere \textbf{meno} eccezioni'
\item \textbf{Regola dei metodi}: ``La sottoclasse deve avere la stessa specifica della superclasse.'' pero' posso:
\begin{itemize}
\item \textbf{Indebolire la pre-condizione} = aggiungere altre pre-condizioni in \textbf{OR} a quelle della superclasse.
\item \textbf{Rafforzare la post-condizione} = aggiugnere altre post-condizioni in \textbf{AND} a quelle della superclasse.
  \end{itemize}
  \item \textbf{Regola delle proprieta'} la sottoclasse deve conservare tutti i public invariant dela superclasse, e anche le sue proprieta' evolutive.
\end{itemize}

\section{Sintassi JML}
Introduco nuove condizioni con un
\begin{minted}{java}
// @ also ...
\end{minted}


\end{document}