\providecommand{\main}{..}
\documentclass[\main/main.tex]{subfiles}


\begin{document}

\section{Dominio}
In JML posso usare \textbf{SOLO}:
\begin{itemize}
  \item  \bs result
  \item I parametri formali
  \item Attributi pubblici
  \item Metodi Puri
\end{itemize}






\section{Pre-Condizioni}
Introdotte da ``requires''

Sono condizioni dei parametri sotto le quali vale la specifica.

Se non vengono rispettate il metodo puo' avere comportamento indefinito ( spetta al \textbf{chiamante} fare i controlli, non al metodo!).





\section{Post-Condizioni}
Introdotte da ``ensure''

Effetto garantito \textbf{AL TERMINE} del metodo.
Cioe' devo specificare che vincolirispetta il risultato (= se ho un certo risultato cosa deve risultare vero, cosa deve aver fatto il metodo, ecc...).







\section{Eccezioni}
Introdotte da ``signals(*TipoEccezzione* e''

Dice che cosa deve essere vero \textbf{SE} il metodo ha sollevato l'eccezione (= se ho chiamato eccezzione \textbf{ALLORA} so che deve valere quello che il signals indica).





\section{Sintassi}
Le specifiche JML vanno messe prima della dichiarazione del metodo.

Sono contenute in un commento ed ogni riga inizia con una @.

Esempi:
\begin{minted}{java}
// @ ...
/*
  @ ...
  @ ...
  @ ...
*/
\end{minted}



\section{\bs result}
E' il valore ritornato dal metodo.




\section{Omissione}
Omettere pre o post condizione equivale a scrivere rispettivamente ``require true'' o ``ensure true''


\section{Commenti}
Alcune specifice sono piu' comprensibli in linguaggio naturale.
Vanno espresse come (*...*) con la specifica in linguaggio naturale tra gli asterischi.



\section{Operatori JML}
\begin{itemize}
  \item \bs old(x): restituisce il valore del parametro x \textbf{AL MOMENTO DELLA CHIAMATA}
  \item assigable: indica che un \textbf{PARAMETRO} puo' essere modificato, va messo \textbf{prima} della pre-condizione.
  \item \textbf{Metodi pubblici dele collezioni}: equals,contains,containsAll,get,subList...
  \item Quantificatori:
  \begin{itemize}
    \item \bs forall
    \item \bs exist
    \item \bs sum, \bs product, \bs min, \bs max
    \item \bs num-of: restituisce il nimero di volte per cui un parametro valgono le condizioni.
  \end{itemize}
\end{itemize}


\section{Astrazioni}
La specifica definisce loggetto astratto , ad esempio ``insieme di numeri interi'' l'implementazione definisce l'oggetto concreto, ad esempio ``array di int''.

\section{OAT}
Se ho bisogno di specificare metodi non puri o comunque mi serve conoscere la struttura del implementazione uso l'OAT.

Questo significa che in una riga JML scrivero' *tipo del OAT nel implementazione* oat

Esempio:

\begin{minted}{java}
  // @ spec_public List<Int> oat;
\end{minted}

e poi un commento che ridescrive le caratteristiche.

Cosi posso usare metodi che, se non conoscessi l'implementazione non potrei usare, tipo subList per una lista.

\section{Public Invariant}
Definisco propriet' che devo essere sempre vere per l'oggetto Astratto,
posso usare \textbf{SOLO} metodi e attributi pubblici.

\section{Verificare gli invariant}
Uso il metodo di induzione:
\begin{itemize}
\item Verifico che valga \textbf{al termine} del costruttore.
\item \textbf{Suppongo} che valga \textbf{alla chiamata} di ogni modificatore, devo dimostrare che vale anche al termine
\end{itemize}

\section{Rep}
E' Struttura dati che rappresenta i valori degli oggetti dell' ADT.

E' insieme di variabili \textbf{private} e delle operazioni che l'oggetto svolfe (\textbf{Implementate})

\section{Funzione di Astrazione(AF)}
Specifica il legame fra stato concreto ed astratto.

Si usa un \textbf{Private Invariant} per descriverla. In particolare nell'invariant descrivo la relazione fra parti private ed osservatori.

Deve mettere in evidenza il legame fra rappresentatore ed oggetto, descrive ``l'implementazione'' del Rep.

In sostanza \textbf{Associa ad ogni oggetto concreto il suo oggetto Astratto}.

\section{Rep Invariant(RI)}
Specifica le proprieta' che \textbf{la Rep deve rispettare}.
Come a AF e' un private invariant.

\textbf{NB} il RI puo' diventare falso durante il metodo, basta che sia \textbf{Vero alla chiamata ed al ritorno}.

Si puo' \textbf{Implementare} ad esempio un repOk un metodo che verifica che RI sia rispettato.  Si definisce il metodo repOk con tutti i controlli per l'RI, poi (o si chiama nei mutatori e costrutti tramite le \textbf{Asserzioni}(= asserzioni sono condizioni che se non sono verificate a RUNTIME danno \textbf{AssertionError} e bloccano l`esecuzione.))

\textbf{NB}: Non posso fornire `accesso esterno a parti mutabili del rep, altrimenti \textbf{ESPLODE TUTTO}.
Quindi devo evitare:
\begin{itemize}
\item di restituire come ritorno il riferimento ad una componente mutabile del rep.
\item di mettere nel rep una componente mutabile che ha un riferimento esterno all'oggetto.
\item dichiarare attributi pubblici.
\end{itemize}

\section{Specifiche totali}
le specifiche totali hanno:
\begin{itemize}
\item requires true
\item precondizioni nell' ensure
\item violazione delle pre-condizioni nel signal
\end{itemize}


\end{document}