\providecommand{\main}{..}
\documentclass[\main/main.tex]{subfiles}
\begin{document}

\section{Data Mining}
È definito come un insieme di tecniche per la raccolta di modelli per dati. In \textbf{statistica} è stato utilizzato per realizzare \textit{modelli statistici} cercando pattern tra grandi quantità di dati, e ovviamente è in uso nella ricerca del \textbf{machine learning} dove vengono utilizzati sia per allenare i network sia per ricercare nuovi dati. 
Una delle potenziali applicazioni del data mining è la \textbf{summarization}, cioè quando si estrapolano dati rappresentativi di altri grossi gruppi, come nel caso di \textit{PageRank}.

\subsection{Limiti del data mining: il principio di  Bonferroni}
Il principio di Bonferroni afferma che, una volta calcolato il numero atteso delle occorrenze degli eventi che si stanno cercando, considerando i dati randomici, 
 
\end{document}
