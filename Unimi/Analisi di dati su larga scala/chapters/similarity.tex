\providecommand{\main}{..}
\documentclass[\main/main.tex]{subfiles}

\begin{document}

\begin{definition}[Indice di similarità di insiemi di Jaccard]
  Dati due insiemi $A$ e $B$, la similarità dei due insiemi sarà definita come:
  \[
    \text{SIM}(A, B) = \dfrac{\abs{A \cap B}}{\abs{A \cup B}}
  \]
\end{definition}

Es: due utenti sono definibili simile dall'insieme di oggetti che essi hanno acquistato.

\begin{definition}[K-gramma (Shingling)]
  Stringa di $k$ caratteri che appare consecutivamente in un documento. Se volessimo rappresentare un documento tramite il suo k-gramma, il suo indice di Jaggard del k-gramma misura la similarità testuale dei documenti.
\end{definition}

\begin{definition}[Stop word]
  Parole comuni nel linguaggio naturale ma che non aggiungono particolare valore semantico ad un testo.
\end{definition}

In alcuni contesti vengono tolti gli spazi nei documenti per calcolare il k-gramma di un documento, ma questo in alcuni casi può far perdere informazioni sul documento (Es: "Touch down" nel contesto dell'atterraggio di un aereo o di una partita di rugby).

\begin{definition}[Matrice rappresentativa di un Insieme]
  Le colonne della matrice corrispondono agli insiemi, mentre le righe corrispondono agli elementi del set universale da cui i set sono estratti. Viene posto un 1 nella cella sulla riga $r$ e colonna $c$ se l'elemento $r$ è un membro del set $c$, altrimenti è 0.

  \begin{figure}[H]
    \begin{center}
      \begin{tabular}{c|c|c|c|c}
        Elemento & $S_1$ & $S_2$ & $S_3$ & $S_4$ \\
        \hline
        \hline
        a        & 1     & 0     & 0     & 1     \\
        b        & 0     & 0     & 1     & 0     \\
        c        & 0     & 1     & 0     & 1     \\
        d        & 1     & 0     & 1     & 1     \\
        e        & 0     & 0     & 1     & 0     \\
      \end{tabular}
    \end{center}
    \caption{Nella matrice rappresentativa troviamo $\Delta = \{a,b,c,d,e\}, S_1 = \{a,d\}, S_2 = \{c\}, S_3 = \{b,d,e\}, S_4 = \{a,c,d\}$.}
  \end{figure}
\end{definition}

\section{Minhash}

\begin{definition}[Minhash]
  Il valore di minhash di una qualsiasi colonna è il primo numero nella prima colonna, nella data permutazione (le righe della matrice possono essere permutate) che ha come valore 1.
  Ogni calcolo minhash va a costruire la firma di un set, che è composta da un grande numero di questi calcoli.
\end{definition}

\subsection{Connessione tra minhashing e indice di Jaccard}
La probabilità che una funzione di minhash per una permutazione randomica di righe produca lo stesso valore per due insiemi \textbf{è uguale} alla similarità di Jaccard per questi due insiemi.
\end{document}

\subsection{Firma di minhash}