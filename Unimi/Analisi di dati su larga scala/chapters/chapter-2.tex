\providecommand{\main}{..}
\documentclass[\main/main.tex]{subfiles}
\begin{document}

\section{Analisi di complessità di un job map-reduce}

\subsection{Es: moltiplicazione di matrici}

\[
	A_{m\times n} \times B_{m \times o} \\
	(i,j,a_{ij}) \longmapsto{M_A} ((i,j), (A, k, a_{ik})) \forall j = 1,...,o \\
	(k,j,b_{kj}) \longmapsto{M_B} ((i,j), (B, k, b_{ik})) \forall i = 1,...,m \\
	(i,j)[(A, 1, a_{i1}), ..., (A, m, a_{im}), (B, 1, b_{1j}), ..., (B, m, b_{mj})]
\]


\[
	R(A,B) \Join S(B,C) \Join T(C,D) \\
	(a, b) \longmapsto_{M_R} (b, (R,a))\\
	(b, c) \longmapsto_{M_S} (b, (S,a))\\
\]

Quale è il costo di questo algoritmo.
Indichiamo con $r, s, t$ i rispettivi numeri di tuple delle tabelle R, S, T.
Il primo processo $M_R$ riceve tutte e sole le tuple di R, quindi ha costo r. Il secondo, similmente, ha costo s.

Il risultato del costo di complessità sarà quindi un $O(r+s)$. Ma questo è tra due relazioni. Se volessi farlo da 3 relazioni (\textbf{join in cascata}) cosa andrei ad ottenere?

$(R \Join S) \Join T$

Ottengo il costo $O(r+s+t+rsp)$, con $p$ rappresentante la probabilità che due valori di $R$ e $S$ hanno un attributo uguale.

\subsection{Join multi-way}
Date due funzioni di hash, una $h$ per l'attributo $B$ ed una g per l'attributo $C$, con $b$ \textbf{bucket} e $c$ \textbf{bucket}, avendo che $bc = k$.
\\
Nel caso di una tupla $(u,v) \in R$ viene inviata ad un'unica colonna verticale, riducendo i nodi (\textbf{c reducer}).
\\
Nel caso di una tupla $(w,z) \in T$ viene inviata ad un'unica colonna orizzontale, riducendo i nodi (\textbf{b reducer}).
\\
Nel caso di una tupla $(v,w) \in S$ viene inviata ad un'unica cella, riducendo i nodi ad uno soltanto (\textbf{1 reducer}).
\\
Il costo quindi risulta essere:

\[
	O(r+2s+t+cr+bt)
\]

\subsection{Rilassamento lagrangiano}
N.B. Il parametro lambda non può essere negativo.

\[
	L(b,c) = cr+br - \lambda(bc-k)
\]

\[
	\dfrac{dL(b,c)}{db} = 0
\]

\[
	\dfrac{dL(b,c)}{dc} = 0
\]

Ottengo quindi un sistema:

\[
	\begin{cases}
		t - \lambda c = 0 \\
		r - \lambda b = 0 \\
	\end{cases}
	\Longrightarrow
	\begin{cases}
		t = \lambda c \\
		r = \lambda b \\
	\end{cases}
\]

\[
	\lambda = \sqrt{\dfrac{rt}{k}}
\]

\[
	c = \sqrt{\dfrac{kt}{r}}
\]

\[
	b = \sqrt{\dfrac{kr}{t}}
\]

Il \textbf{costo ottimizzato} della \textbf{join multiway} risulta quindi essere:

\[
	O(r+2s+t+2\sqrt{krt})
\]

\subsection{Es: Join sui nodi di facebook}
Prendiamo ad esempio il grafo dei nodi facebook, dotato di $10^9$ nodi.

$R(U_1, U_2)$, $\abs{R} = r = 3\times10^11$ (dati arbitrari)

\[
	R\Join R \Join R
\]

Approccio Multi-way: $r+2r + r + 2r\sqrt{k} = 4r + 2r\sqrt{k} = 1\times 2\times 10^12 + 6\times 10^11 \sqrt{k}$
\\
Approccio cascata (nell'ipotesi che $abs{R \Join R} = 30r$): $r + r + r + r^2\times p = ... = 2r + 60r = 1 \times 2 \times 10^12 + 1\times 86 \times 10^13$

Ottengo quindi che: $6\times 10^11 \sqrt{k} \leq 1\times 86 \times 10^33 \longrightarrow k \leq 961$, e risulta quindi migliore utilizzare l'approccio multi way quando si hanno meno di 961 nodi da allocare a dei reducer.

\subsection{Es: Google pagerank}
Come funziona \textbf{pagerank}:

\begin{center}
\begin{tabular}{ |c|c|c|c|c| } 
 \hline
  & A & B & C & D \\ 
 \hline
 A & 0 & $\sfrac{1}{2}$ & 1 & 0 \\ 
 \hline
 B & $\sfrac{1}{3}$ & 0 & 0 & $\sfrac{1}{2}$ \\ 
 \hline
 C & $\sfrac{1}{3}$ & 0 & 0 & $\sfrac{1}{2}$ \\ 
 \hline
 D & $\sfrac{1}{3}$ & $\sfrac{1}{2}$ & 0 & 0 \\ 
 \hline
\end{tabular}
\end{center}

\[
	v_j(t+1) = P(\text{Trovarsi in j al tempo t+1}) = \sum_i P (\text{trovarsi in i al tempo t})\bullet P(\text{spostarsi da i a j | trovarsi in i al punto t})
\]

\[
	\sum_i v_i(t) m_{ji} = \sum_i m_{ji}v_i(t) = (Mv(t))_j
\]

con $M_{ij} = P \text{Spostarsi da j a i}$.

\[
	\vec{V}(t+1) = M\vec{v}(t)
\]

\subsubsection{Prova di convergenza}

\begin{definition}[Determinante]
	\[
		\text{det}A = \sum_i a_{ij} c_{ij} = \sum_j a_{ij} c_{ij}	
	\]
\end{definition}

\begin{definition}[Determinante di matrice trasposta]
Una matrice e la sua trasposta possiedono lo stesso determinante.
	\[
		\text{det}A^T = \sum_i a^T_{ij} c^T_{ij} = \sum_i a_{ji} c_{ji}	 =  \sum_j a_{ij} c_{ij}	
	\]
\end{definition}

\begin{definition}[Autovalori]
Si ottengono trovando gli zeri dell'equazione caratteristica.
\[
	\text{det}(A-\lambda I) = 0 \leftrightarrow \text{det}(A-\lambda I)^T = 0 \leftrightarrow  \text(A^T-\lambda I) = 0
\]
\end{definition}

\begin{definition}[Matrice stocastica per righe]
Una qualsiasi matrice stocastica per righe ammette $1$ come autovettore.
\[
	\forall i \sum_j a_{ij} = 1
\]
\end{definition}

\begin{definition}[Matrice stocastica per colonne]
Una qualsiasi matrice stocastica per colonne ammette $1$ come autovettore poiché si tratta della trasposta di una matrice stocastica per righe.
\end{definition}

\begin{theorem}[La potenza di una matrice stocastica è sempre stocastica]
Il risultato dell'elevazione a potenza di una matrice stocastica risulta sempre essere stocastico. Dimostriamo per induzione:
\[
	\text{Base: } k=1 \qquad A^k = A
\] 
\[
	\text{Passo: } A^k \text{ stocastica } \rightarrow A^{k+1} \text{ stocastica }
\]
Dimostriamo che ad un generico passo $k$, otteniamo sempre 1 quando andiamo a sommare i termini.
\[
	a_{ij}^{k+1} = \sum_s a^k_{ij} a_{sj}\\
	\sum_j a^k_{ij} = \sum_j \sum_s a^k_{is}a_{sj}
\] 
Inverto le sommatorie ed ottengo:
\[
	\sum_s a_{is}^k \sum_j a_{sj} = \sum_s a^k_{is} = 1
\]
\end{theorem}

\begin{theorem}[Autovalori di una stocastica]
Se A è stocastica per colonne, il suo autovalore massimo è 1.

\textbf{Procediamo a dimostrare per assurdo.}

Sappiamo che 1 è un autovale di A, poiché A è stocastica. Affermiamo che, per assurdo, esista un $\lambda > 1$ autovalore di A (che è anche l'autovalore di $A^T$) e $v$ un autovettore per $\lambda$.

\[
	A^Tv \Rightarrow \lambda v
\]

\[
	A^{2T} v = A^T \times A^T v = A (\lambda v) = \lambda A^T v = \lambda ^2 v \Rightarrow A^{Tk} v = \lambda ^ k v
\]

Maggioriamo / minoriamo la sommatoria:

\[
	\sum_j a^{Tk}_{ij} v_{max} \geq \sum_j a_{ij}^{Tk} v_j = \lambda ^ k v_i > G
\]

\[
	v_{max} \sum_j a_{ij}^{Tk} > G
\]

Ma siccome la sommatoria è dei termini di una matrice stocastica per righe (essendo la trasposta di A), deve essere pari a 1, per cui:

\[
	v_{max} \sum_j a_{ij}^{Tk} = v_{max} > G
\]

\[
	1 > \dfrac{G}{v_{max}} \Rightarrow \text{assurdo!}
\]

\end{theorem}

\begin{theorem}[Autovalori di potenza di matrice]
\[
	v_0 \rightarrow v_1 = A v_0 \rightarrow v_2 = A v_1 = A^2 v_0 \rightarrow ... \rightarrow v_k = A^k v_0	
\]

\textbf{Dimostrazione}

\begin{align*}
	A v_0 &= A(\alpha_1 x_1 + \alpha_2 x_2 + ... + \alpha_n x_n)
			 &= \alpha_1 A x_1 + \alpha_2 A x_2 + ... + \alpha_n A x_n
			 &= \alpha_1 \lambda_1 x_1 + \alpha_2 \lambda_2 x_2 + ... + \alpha_n \lambda_n x_n
\end{align*}

\[
	v_k = A^k v_0 = \alpha_1 \lambda_1^k x_1 +  \alpha_2 x_2  \dfrac{ \lambda_2^k \lambda_1^k}{ \lambda_1^k}+ ... +  \alpha_k x_k  \dfrac{ \lambda_k^k \lambda_1^k}{ \lambda_1^k}
\]

\[
	v_k = A^k v_0 =  \lambda_1^k (\alpha_1 x_1 +  \alpha_2 x_2  \dfrac{ \lambda_2^k}{ \lambda_1^k}+ ... +  \alpha_k x_k  \dfrac{ \lambda_k^k}{ \lambda_1^k})
\]

per $k$ "grande" $v_k = A^k v_0 \approx \lambda^k_1 \alpha_1 x_1$, nel nostro caso $\lambda_1 = 1$, cioè $\lim_{k \rightarrow \infty} \alpha_1 \lambda_1^k x_1 = 1$.

\end{theorem}

\begin{figure}[H]
\begin{center}
\begin {tikzpicture}[-latex ,auto ,node distance =4 cm and 5cm ,on grid ,
semithick ,
state/.style ={ circle ,top color =white , bottom color = processblue!20 ,
draw,processblue , text=blue , minimum width =1 cm}]
\node[state] (A) [above] {A};
\node[state] (B) [ right =of A] {B};
\node[state] (D)[below right =of A]{D};
\node[state] (C)[below =of A]{C};

\path (A) edge [left =25] node[below =0.15 cm] {} (B);
\path (A) edge [left =25] node[below =0.15 cm] {} (C);
\path (A) edge [left =25] node[below =0.15 cm] {} (D);

\path (B) edge [bend left =25] node[below =0.15 cm] {} (D);
\path (B) edge [bend left =25] node[below =0.15 cm] {} (A);

\path (C) edge [bend left =25] node[below =0.15 cm] {} (A);

\path (D) edge [left =25] node[below =0.15 cm] {} (B);
\path (D) edge [left =25] node[below =0.15 cm] {} (C);
\end{tikzpicture}
\end{center}
\caption{In questo grafo, dopo 10 esecuzioni, pagerank assegna assegna ai nodi A, B, C, D rispettivamente $\sfrac{3}{9}, \sfrac{2}{9}, \sfrac{2}{9} \text{e} \sfrac{2}{9}$}.
\end{figure}

\subsection{Trappole per ragni}
Nodi di grafi (figura \ref{trappola_ragni}) con loop, da cui gli spider non possono uscire.

\begin{figure}[H]
\begin{center}
\begin {tikzpicture}[-latex ,auto ,node distance =4 cm and 5cm ,on grid ,
semithick ,
state/.style ={ circle ,top color =white , bottom color = processblue!20 ,
draw,processblue , text=blue , minimum width =1 cm}]
\node[state] (A) [above] {A};
\node[state] (B) [ right =of A] {B};
\node[state] (D)[below right =of A]{D};
\node[state] (C)[below =of A]{C};
\node[state] (E)[below =of C]{E};

\path (A) edge [left =25] node[below =0.15 cm] {} (B);
\path (A) edge [left =25] node[below =0.15 cm] {} (C);
\path (A) edge [left =25] node[below =0.15 cm] {} (D);

\path (B) edge [bend left =25] node[below =0.15 cm] {} (D);
\path (B) edge [bend left =25] node[below =0.15 cm] {} (A);

\path (C) edge [bend left =25] node[below =0.15 cm] {} (A);
\path (C) edge [left =25] node[below =0.15 cm] {} (E);

\path (D) edge [left =25] node[below =0.15 cm] {} (B);
\path (D) edge [left =25] node[below =0.15 cm] {} (C);

\path (E) edge [loop left =25] node[below =0.15 cm] {} (E);
\end{tikzpicture}
\end{center}
\caption{Dopo un certo numero di iterazioni, tutta la "massa" distribuita inizialmente sul grafo sarà finita su E}.
\label{trappola_ragni}
\end{figure}

\subsection{Risolvere le trappole per ragni col teletrasporto}
 Per evitare le trappole per ragni, andiamo a inserire una certa probabilità $\beta$ di teletrasportarci da un nodo ad un altro al posto di continuare a navigare tramite link.
 
 \[
 	\bm{v}_{t+1} = \beta M \bm{v}_t + (1-\beta) \dfrac{1}{n} \bm{1}
 \] 

P(mi trovo in $i$ al tempo $t+1$) = P(spostarsi tramite link in $i$ | scelgo i link) P(scelgo i link) + P (teletrasportarsi in i | mi teletrasporto) P(scegliere teletrasporto)

\end{document}