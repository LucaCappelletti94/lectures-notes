\providecommand{\main}{..}
\documentclass[\main/main.tex]{subfiles}
\begin{document}

\chapter{Notazioni}
\begin{definition}[O grande]
  Con notazione \textbf{O grande} si intende:
  \[
    g(n) = O(f(n)) \quad \exists c \forall n \geq N \quad g(n) \leq c\cdot f(n)
  \]
\end{definition}

\begin{definition}[Problema polinomiale]
  Un problema \(P(n)\) è detto \textbf{polinomiale} se la sua complessità temporale è \(O(P(n))\).
\end{definition}

\begin{definition}[Non Polinomiale (NP)]
  Questa classe di problemi viene definita tramite un \textit{verificatore} con un'instanza ed un \textit{testimone}.
  \begin{align*}
     & \forall x \in L \quad \exists \omega \in 2^* \text{ t.c. il verificatore risponde in tempo P. Si.}     \\
     & \forall x \not\in L \quad \forall \omega \in 2^* \text{ t.c. il verificatore risponde in tempo P. No.}
  \end{align*}
\end{definition}
\end{document}