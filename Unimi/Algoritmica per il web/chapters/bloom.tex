\providecommand{\main}{..}
\documentclass[\main/main.tex]{subfiles}
\begin{document}
\chapter{Filtri di Bloom}
\begin{multicols}{2}
\begin{definition}[Filtro di Bloom]
    Un filtro di Bloom è una struttura dati probabilistica che rappresenta un insieme: permette di aggiungervi elementi e chiedere se un elemento vi appartenga o no, con il \textbf{rischio di ottenere falsi positivi}.
\end{definition}
\begin{observation}[Cosa compone un filtro di Bloom?]
    Un filtro di Bloom per un insieme \(X\) è rappresentato da un vettore \(\bmb\) di \(m\) bit e da \(d\) funzioni di \textbf{hash} \(f_0, f_1, \ldots, f_{d-1}\) da \(X\) in \(m\).
\end{observation}
\begin{observation}[Come si aggiunge un elemento a un filtro di Bloom?]
    Per aggiungere un elemento di \(x \in X\) a un filtro di Bloom, vengono posti a uno i bit \(b_{f_{k}\rnd{x}}, \rnd{0 \leq k < d}\).
\end{observation}
\begin{observation}[Come si verifica se un elemento appartiene al filtro?]
    Per sapere se un elemento appartiene al filtro si controlla se tutti i bit \(b_{f_{k}\rnd{x}}, \rnd{0 \leq k < d}\) sono a uno, e solo in questo caso si risponde positivamente.
\end{observation}
\begin{observation}[Come mai "filtri" di Bloom?]
    Il nome "filtro" deriva dall'idea che la struttura dovrebbe venire usata per filtrare le richieste a una secondaria soggiacente più lenta. Quando si prevede che la maggior parte delle richieste avrà risposta negativa, un filtro di Bloom può ridurre significativamente gli accessi alla struttura soggiacente.
\end{observation}
\vfill\null
\columnbreak
\begin{analysis}[Probabilità di un falso positivo in un filtro di Bloom dopo \(n\) inserimenti.]
    La probabilità che dopo \(n\) inserimenti uno specifico bit sia zero è:
    \[
        \left(1-\frac{1}{m}\right)^{d n}
    \]
    dato che la probabilità che un singolo bit sia zero (se ne imposto uno a caso) è \(1-\frac{1}{m}\) e assumiamo che le funzioni di hash siano uniformemente distributive e perfettamente aleatorie. Per ottenere un positivo, dobbiamo trovare uno in tutti i punti del vettore che controlliamo e questo avviene con probabilità:
    \[
        \varphi=\left(1-\left(1-\frac{1}{m}\right)^{d n}\right)^{d} \approx\left(1-e^{-\frac{d n}{m}}\right)^{d}
    \]
    dove abbiamo utilizzato il fatto che \((1+\frac{\alpha}{n})^{n} \rightarrow e^{\alpha} \text { per } n \rightarrow \infty\).
    
    Poniamo ora \(p=e^{-\frac{n d}{m}}\), sicché \(d=-(\frac{m}{n}) \ln p\). Il nostro scopo è quindi quello di minimizzare:
    \[
        (1-p)^{-(m / n) \ln p}=e^{-(m / n) \ln p \ln (1-p)}
    \]
    La derivata prima è dunque:
    \[
        -\frac{m}{n} e^{-(m / n) \ln p \ln (1-p)}\left(\frac{\ln (1-p)}{p}-\frac{\ln p}{1-p}\right)
    \]
    e si azzera quando:
    \[
        (1-p) \ln (1-p)=p \ln p
    \]
    Dato che sia a sinistra che a destra abbiamo la stessa funzione \(x \ln x\), una soluzione è certamente data da \(1-p = p\), cioè \(p = \frac{1}{2}\).  

Concludiamo che la probabilità di (falsi) positivi è minimizzata da \(d \equiv \frac{m}{n} \ln 2\) e in tal caso la probabilità di un (falso) positivo è \(2^{-d}\). Vale a dire, possiamo migliorare esponenzialmente la probabilità di errore aumentando linearmente il numero di funzioni di \textbf{hash} e il numero di bit a \(m \equiv \frac{dn}{\ln 2} \approx 1.44 dn\).
\end{analysis}
\end{multicols}
\end{document}