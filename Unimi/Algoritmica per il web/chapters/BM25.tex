\providecommand{\main}{..}
\documentclass[\main/main.tex]{subfiles}
\begin{document}
\section{BM25}
\begin{definition}[Modello a due Poisson]
    Il modello delle due Poisson ipotizza che la distribuzione delle frequenze dei termini in un documento è il risultato di una combinazione di due distribuzioni Poisson, una per i termini frequenti e una per i termini rari.
\end{definition}
\begin{definition}[BM25]
    Con BM25 si intende una funzione \(w_{j}(d, C)\) di pesatura dei termini che utilizza informazioni quali la frequenza dei termini \(df_{j}\), la lunghezza del documento \(dl\), la lunghezza media dei documenti della collezione \(avdl\) e la frequenza del termine considerato nel documento \(d_j\):
    \[
        w_{j}(d, C) =\frac{\left(k_{1}+1\right) d_{j}}{k_{1}\left((1-b)+b \frac{d l}{a v d l}\right)+d_{j}} \underbrace{\log \frac{N-d f_{j}+0.5}{d f_{j}+0.5}}_{IDF}
    \]
    I pesi \(k_1\) e \(b\) sono parametri liberi, tipicamente rispettivamente \(\sqr{1.2, 2.0}\) e \(0.75\).
    
    Il punteggio del documento per una data query si ottiene quindi sommando i termini dei pesi corrispondenti ai termini della query:
    \[
        W(\overline{d}, q, C)=\sum_{j} w_{j}(\overline{d}, C) \cdot q_{j}
    \]
    Più termini risultano comuni con la query più il peso sarà maggiore.
    
    Può essere visto come un'approssimazione del modello a due Poisson.
    
    BM25 tende a fallire quando i documenti sono molto lunghi.
\end{definition}
\begin{analysis}[Che significato hanno le varie parti di BM25?]
    Innanzitutto, il membro di destra funge da \textbf{inverse document frequency (IDF)}, cioè le parole comuni risultano meno importanti. Guardando invece il termine di sinistra, iniziamo dal nominatore: qui troviamo il termine \(d_j\), che nei casi in cui viene sommato coincide con la frequenza del termine della query nel documento. Se esso aumenta di valore, allora lo score aumenta.
    
    Guardando invece il denominatore, questo rende la frequenza del termine meno importante che non diverse parole della query, ma più importante (con la parte frazionaria \(\sfrac{d l}{a v d l}\)) se il documento è particolarmente lungo rispetto alla media.
\end{analysis}
\end{document}