\providecommand{\main}{..}
\documentclass[\main/main.tex]{subfiles}
\begin{document}
\chapter{Boolean Retrieval}
\begin{multicols}{2}
\begin{definition}[Dati non strutturati]
    Con dati non strutturati ci si riferisce a documenti che non possiedono una struttura semanticamente esplicita, semplice da comprendere per un computer.
\end{definition}
\begin{definition}[Information retrieval]
    \textbf{Information retrieval} (IR) consiste nel trovare materiale, tipicamente documenti, di tipo non strutturato, tipicamente testi, che soddisfi una informazione richiesta da una collezione più grande.
\end{definition}
\begin{definition}[Modello di boolean retrieval]
    Il modello di \textbf{boolean retrieval} è un modello di IR cui possiamo porre qualsiasi query sotto forma di espressione booleana dei termini, dove i termini sono combinati tramite le espressioni booleane and, or e not.
    
    Questi modelli vedono ogni documento semplicemente come insiemi di parole.
\end{definition}
\begin{definition}[Precisione]
    Quale frazione dei risultati identificati dal sistema IR sono rilevanti rispetto alle informazioni richieste?
\end{definition}
\begin{definition}[Recall]
    Quale frazione dei documenti rilevanti nella collezione sono stati identificati dal sistema IR?
\end{definition}
\begin{definition}[Efficacia di un sistema IR]
    Per valutare l'efficacia di un sistema IR, cioè la qualità dei suoi risultati di ricerca, si utilizzano due metriche statistiche sui risultati che il sistema ritorna per una data query: la \textbf{precisione} e \textbf{recall}.
\end{definition}
\begin{definition}[Indice inverso]
    Per realizzare un \textbf{indice inverso} andiamo a realizzare un dizionario di termini e ad ogni termine associamo una lista (posting list o lista inversa) dei record (posting) dei documenti dove appare.
    
    Si tratta di una struttura dati per rappresentare la matrice sparsa dei termini e documenti.
\end{definition}
\begin{observation}[Come viene costruito un indice inverso?]
    Ad alto livello, per costruire un indice inverso:
    \begin{enumerate}
        \item Si raccolgono i documenti che devono essere indicizzati.
        \item Si tokenizza il testo.
        \item Si procede ad eseguire una fase di preprocessing linguistico dei token realizzati.
        \item Si indicizzano i documenti in cui appare ogni termine.
    \end{enumerate}
\end{observation}
\begin{definition}[Proximity operator]
    Un \textbf{proximity operator} è un modo per specificare che due termini in una query devono occorrere vicino uno all'altro in un documento.
\end{definition}
\begin{definition}[Stemming]
    Il processo di \textbf{stemming} tipicamente si riferisce a un'euristica piuttosto semplice in cui si va a tagliare la fine delle parole in modo tale da cercare di normalizzarle.
\end{definition}
\begin{definition}[Lemmatization]
    Il processo di \textbf{lemmatization} tipicamente si riferisce ad un processo di stemming più preciso, in cui le parole vengono normalizzate utilizzando un dizionario con una forma base delle parole, chiamata \textbf{lemma}.
\end{definition}
\end{multicols}
\clearpage
\section{Rappresentazione di Elias-Fano}
\begin{multicols}{2}
\begin{definition}[Rappresentazione di Elias-Fano]
    Si tratta di una struttura quasi-succinta per sequenze monotone. Dati due termini \(n\) e \(u\), consideriamo la sequenza \(0 \leq x_0, x_1, x_2, \ldots, x_{n-1} \leq u\). Andiamo a salvare gli \(\ell = \log\rnd{\sfrac{u}{n}}\) bit minori esplicitamente. Salviamo i bit maggiori come una sequenza di intervalli unari codificati (dove \(0^k 1\) rappresenta \(k\)).
    
    Utilizziamo al più \(2+\log\rnd{\frac{u}{n}}\) bit per elemento. Siccome è prossimo al limite di una struttura succinta, per cui la struttura è detta quasi-succinta.
\end{definition}
\begin{observation}[Perché la rappresentazione di Elias-Fano è utile?]
    Lo spazio utilizzato è quasi ottimale e non assume nessuna particolare distribuzione dei dati. Inoltre, per essere letta in modo sequenziale richiede pochi operatori logici e restringe il problema di rank/selection ad un array di circa \(2n\) bits con metà zero e metà uno.
\end{observation}
\begin{definition}[Selezione su una rappresentazione Elias-Fano]
    Supponiamo di voler selezionare il \(k\)-esimo elemento rapidamente. Leggiamo i bit superiori (quelli in unario) uno alla volta, contando gli elementi a uno. Quando si arriva alla parola corretta, ne completiamo la sequenza e prendiamo i bit inferiori.
\end{definition}
\begin{definition}[Ranking su una rappresentazione Elias-Fano]
    Si procede analogamente alla selezione ma si contano gli zero, numero che indica quanto i bit superiori stanno aumentando. Saltiamo i primi \(b >> \ell\) zero superiori e completiamo la sequenza.
\end{definition}
\end{multicols}
\end{document}
