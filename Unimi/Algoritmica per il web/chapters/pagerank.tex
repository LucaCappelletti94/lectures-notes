\providecommand{\main}{..}
\documentclass[\main/main.tex]{subfiles}
\begin{document}
\chapter{Ranking}
\begin{multicols}{2}
\begin{definition}[Sistema di reperimento di informazioni]
    Un sistema di reperimento di informazioni è dato da una collezione documentale \(D\) di dimensione \(N\), da un insieme \(Q\) di interrogazioni, e da una funzione di \textbf{ranking} \(r: Q \times D \rightarrow \R\) che assegna a ogni coppia data da un'interrogazione e un documento un numero reale.
    
    I documenti sono tanto più rilevanti quanto il punteggio è alto.
\end{definition}
\begin{definition}[Ranking endogeno]
    Un tipo di algoritmo di ranking che utilizza il contenuto del documento.
\end{definition}
\begin{definition}[Ranking esogeno]
    Un tipo di algoritmo di ranking che utilizzano la struttura esterna.
\end{definition}
\begin{definition}[Ranking statico]
    Un criterio di ranking statico è indipendente dall'interrogazione. Il punteggio assegnato a ciascun documento è fisso e indipendente da \(q\).
\end{definition}
\begin{definition}[Ranking dinamico]
    Un criterio di ranking dinamico è dipendente dall'interrogazione.
\end{definition}
\end{multicols}
\section{Pagerank}
\begin{multicols}{2}
\begin{definition}[PageRank]
    \textbf{PageRank} è un criterio di ranking statico esogeno basato sulla passeggiata naturale del grafo del web \(G\). Fissato un parametro \(\alpha\) tra \(0\) e \(1\), a ogni passo con probabilità \(\alpha\) si segue un arco uscente e con probabilità \(1-\alpha\) si sceglie un qualunque altro nodo del grafo utilizzando una qualche distribuzione \(\bm{v}\), detta \textbf{vettore di preferenza}. Assumendo che non esistano pozzi, la catena è quindi rappresentata dalla combinazione lineare:
    \[
        \alpha G+(1-\alpha) \mathbf{1} v^{T}
    \]
    dove \(G\) è la matrice della passeggiata naturale su \(G\). Un valore tipico scelto per \(\alpha\) è \(0.85\).
    
    Il grafo associato è unicatenario: tutti i nodi hanno archi entranti da ogni altro nodo e sono pertanto tutti essenziali e siccome hanno anche un loop sono tutti aperiodici. Tutti i nodi essenziali stanno nella stessa unica componente terminale e la matrice associata risulta unicatenaria e aperiodica, e ha quindi sempre un'unica distribuzione limite, detta \textbf{PageRank}.
\end{definition}
\begin{definition}[Fattore di attenuazione (damping factor)]
    In effetti è possibile pensare a PageRank come una funzione di \(\alpha\): per \(\alpha\approx 0\) PageRank è pari al vettore di preferenza o per \(\alpha\approx 1\) risulta concentrato nelle componenti terminali.
    
    Esso è essenziale per rendere la matrice aperiodica e unicatenaria e riduce il numero di cifre significative necessario per il calcolo della distribuzione limite e rende inoltre il calcolo stesso molto più rapido.
\end{definition}
\begin{observation}[Come si risolvono i pozzi in PageRank?]
    Solitamente per risolvere i pozzi si aggiunge un arco verso ogni altro nodo, di fatto redistribuendo il ranking dei pozzi in maniera uniforme: giunti a un pozzo si salta in un altro nodo a caso. Altrimenti si può saltare a un altro nodo usando per esempio il vettore di preferenza.
\end{observation}
\end{multicols}
\end{document}