\providecommand{\main}{..}
\documentclass[\main/main.tex]{subfiles}
\begin{document}

\section{Dispense}
Sono disponibili dispense sul sito del corso: \url{https://homes.di.unimi.it/cordone/courses/2017-mmd/2017-mmd.html}.

\section{Flash card}
Sono disponibili flashcards\footnote{Per maggiori informazioni su come usare delle flashcards guarda questo video: https://www.youtube.com/watch?v=mzCEJVtED0U}, realizzate dall'autore della dispensa, dal sito cram, utilizzabili tramite l'omonima applicazione:

\subsubsection*{Problemi decisionali}
\url{http://www.cram.com/flashcards/metodi-e-modelli-per-le-decisioni-problemi-decisionali-9495432}
\subsubsection*{Soluzioni paretiane}
\url{http://www.cram.com/flashcards/metodi-e-modelli-per-le-decisioni-soluzione-paretiana-9495565}
\subsubsection*{Programmazione in condizioni di incertezza}
\url{http://www.cram.com/flashcards/metodi-e-modelli-per-le-decisioni-programmazione-in-condizioni-di-incertezza-9496060}
\subsubsection*{Programmazione in condizioni di rischio}
\url{http://www.cram.com/flashcards/metodi-e-modelli-per-le-decisioni-programmazione-in-condizioni-di-rischio-9496131}
\subsubsection*{Metodi a razionalità debole}
\url{http://www.cram.com/flashcards/metodi-e-modelli-per-le-decisioni-metodi-a-razionalita-debole-9495965}
\subsubsection*{Giochi simmetrici}
\url{http://www.cram.com/flashcards/metodi-e-modelli-per-le-decisioni-giochi-simmetrici-9496162}
\subsubsection*{Decisioni di gruppo}
\url{http://www.cram.com/flashcards/metodi-e-modelli-per-le-decisioni-decisioni-di-gruppo-9496282}



\end{document}