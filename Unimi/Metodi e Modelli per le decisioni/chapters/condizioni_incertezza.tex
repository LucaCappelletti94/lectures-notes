\providecommand{\main}{..}
\documentclass[\main/main.tex]{subfiles}
\begin{document}

\chapter{Programmazione in condizioni di incertezza}
Viene applicata in problemi in cui uno scenario $\omega$ viene svelato dopo che il decisore ha scelto l'alternativa $x$. Un problema di questo tipo può appartenere a una di due categorie:\textbf{decisione in condizioni di ignoranza} o \textbf{decisioni in condizioni di rischio}.

\section{Dominanza forte}

\begin{definition}[Dominanza forte]
  Si dice che un'alternativa $x$ domina fortemente un'alternativa $x'$ quando il suo impatto è almeno altrettanto buono in tutti gli scenari $\omega \in \Omega$ e migliore in almeno uno:
  \[
    x \preceq x' \Leftrightarrow \begin{cases}
      f(x,\omega) \leq f(x',\omega) \quad \forall \omega \in \Omega \\
      \exists \omega' \in \Omega: f(x,\omega') < f(x',\omega')
    \end{cases}
  \]
\end{definition}

\section{Criteri}
\subsection{Criterio del caso pessimo}
Si assume che qualsiasi scelta si faccia condurrà al proprio caso pessimo, per cui si sceglie l'opzione che ha valore migliore nel proprio caso pessimo.

\subsection{Criterio del caso ottimo}
Si assume che qualsiasi scelta si faccia consurrà al proprio caso migliore, per cui si sceglie l'opzione che ha valore migliore nel proprio caso ottimo.

\subsection{Criterio del caso Hurwicz}
Esegue la combinazione convessa del criterio del caso pessimo ed ottimo usando un coefficiente $\alpha$ detto \textbf{coefficiente di pessimismo}. Quindi prima si assume che qualsiasi scelta condurrà al caso indicato dalla combinazione convessa e poi si sceglie la migliore tra queste.

Il coefficiente $\alpha$ può essere determinato ponendo due scelte certamente indifferenti tra loro come uguali nell'equazione della combinazione convessa, andando anche a costruire questa scelta se necessario.

\subsection{Criterio di equiprobabilità o di Laplace}
Combina gli scenari sommandoli con un dato peso (uguale per tutti), per esempio la media degli scenari (caso di equiprobabilità, da cui il nome). Quindi va a scegliere l'opzione che porta al valore ottimo.

\subsection{Criterio del rammarico}
Questo criterio va a minimizzare il rammarico, cioè la differenza tra l'alternativa ottima di uno scenario e quella che andrebbe ad accadere con una determinata scelta. Per ogni scelta si cerca il rammarico massimo, e quindi si sceglie l'opzione che lo minimizza.

\subsection{Criterio delle eccedenze}
Si tratta del complementare al criterio del rammarico: si va a calcolare la differenza tra l'aternativa pessima di uno scenario e quella che andrebbe ad accadere con una determinata scelta. Per ogni scelta si cerca l'eccedenza minima e quindi si sceglie l'opzione che la massimizza.

\end{document}