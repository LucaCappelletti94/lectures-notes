\providecommand{\main}{..}
\documentclass[\main/main.tex]{subfiles}
\begin{document}

La teoria MAUT (multiple attribute utility theory) ipotizza di trovarsi nella situazione in cui un decisore è in grado di ordinare gli indicatori, creando un set $\Pi$ con un ordine debole. Ipotizza inoltre l'esistenza di una funzione valore $u(f)$ conforme a $\Pi$. Spesso la $u(f)$ non si conosce ed è necessario costruire il passaggio da $\pi$ a $u(f)$

\begin{figure}[H]
  \centering
  \includegraphics[width=0.75\textwidth]{\main/images/MAUT.png}
  \caption{MAUT in progress}
\end{figure}

Campiono quuindi gli impatti e connetto quelli che risultano indifferenti preferenzialmente, cercando di identificare forme analitiche che le descrivano.

\[
  u(b) = f_1f_2 \rightarrow u(b) = f_1^{\alpha_{1}}f_2^{\alpha_{2}}
\]

Definita quindi una formula vado a verificare che valga su quei punti:

\[
  u(f_1) = u(f_2) \Rightarrow  f_1^{\alpha_{1}}f_1^{\alpha_{2}} = f_2^{\alpha_{1}}f_2^{\alpha_{2}}
\]

\begin{figure}[H]
  \[
    u(f) = \prod_{l=1}^p f_l^{\alpha_l}
  \]
  \caption{\textbf{Funzioni di Cobb-Douglas:} descrive un consumatore, viene usata in economia}
\end{figure}

Questo procedimento diventa rapidamente insolubile all'aumentare dei decisori e degli indicatori. Viene quindi \textit{ipotizzato} che la funzione di utilità sia \textbf{additiva}, cioè che valga:

\[
  u(f) = \sum_{l=1}^p u_l(f_l)
\]


\end{document}