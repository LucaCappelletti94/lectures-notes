\providecommand{\main}{../../..}
\documentclass[\main/main.tex]{subfiles}
\begin{document}

\subsection{Esercizio 5}
Dato un problema di decisione con alternative $X = {a_1, \ldots , a_5}$ e scenari $\Omega = {\omega_1, \omega_2}$, e con le utilità $u (f (x, \omega))$ e le probabilità $\pi (\omega)$ riportate nella tabella seguente:


\begin{table}
	\begin{tabular}{|L|L|L|L|L|L||L|}
		\hline
		u (f (x, \omega)) & a_1 & a_2 & a_3 & a_4 & a_5 & \pi (\omega) \\
		\hline
		u_1               & 80  & 100 & 50  & 0   & 90  & 0.5          \\
		\hline
		u_2               & 20  & 10  & 50  & 30  & 30  & 0.4          \\
		\hline
		u_3               & 40  & 10  & 50  & 100 & 20  & 0.1          \\
		\hline
	\end{tabular}
\end{table}

\begin{enumerate}[a)]
	\item Si elenchino le eventuali alternative dominate, specificando da quali altre alternative sono dominate.
	\item Si indichi l’alternativa scelta con il criterio del caso medio, spiegando il procedimento.
	\item Si mostri come cambia l’alternativa scelta al variare della probabilità $\pi (\omega_1)$ del primo scenario, ipotizzando che le altre mantengano le reciproche proporzioni.
\end{enumerate}


\end{document}
