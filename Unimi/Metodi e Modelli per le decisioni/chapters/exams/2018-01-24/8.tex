\providecommand{\main}{../../..}
\documentclass[\main/main.tex]{subfiles}
\begin{document}

\subsection{Exercise 8}
\begin{enumerate}[a)]
	\item Describe the concept of constitution as a method to aggregate preferences.
	\item Describe the plurality system method to aggregate preferences.
	\item Describe the concept of decisive set for a tuple of impacts $(f, f')$.
	\item Describe the concept of dependency from irrelevant alternatives, explaining why one would try to avoid it in group decision.
\end{enumerate}

\subsection{Exercise 8 resolution}
\subsubsection*{Constitution}
A constitution is a function that associates to every set of weak orders relationships on a given set of solutions or alternatives a weak order relationship of the group on the same set of solutions.

\subsubsection*{Plurality system}
The alternative that is more commonly accepted as the optimal by the decision makers. In case of a draw, the solution less commonly accepted is eliminated and the process is repeated.

This method has the defect that it allows solutions strongly desired by a minority and just slightly wanted by a majority to win.

\subsubsection*{Decisive set}
The group of decision makers that makes a given impact preferable to a second one in the group weak order.

\subsubsection*{Dependency from weak alternatives}
The concept of dependency from weak alternatives describe the situation when a value function bonds the value of a better alternative to the existence of a worst one, for example the method of Borda. This can lead to the \textbf{rank reversal}, a phenomenon for which the optimal solution can vary when one deletes the worst one.

It is one of the most common issues in politics, when the order a set of laws is approved or not changes which laws are approved.

\end{document}