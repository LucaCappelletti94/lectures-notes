\providecommand{\main}{../../..}
\documentclass[\main/main.tex]{subfiles}
\begin{document}

\subsection{Exercise 4}
\begin{enumerate}
  \item Describes the limits that the \textit{hierarchical analysis} attributes to the classical multiple attributes utility theory.
  \item Which solutions does it propose to resolve them?
  \item What are the limits of hierarchical analysis?
\end{enumerate}

\subsection{Exercise 4 resolution}
\subsubsection*{Limits of classical multiple attributes utility theory}
\begin{enumerate}
  \item The construction of the utilities functions is subject to large approximation errors.
  \item The weight estimation is subject to large approximation errors when the number of the attributes is high.
  \item The approximation errors sum in cascade.
\end{enumerate}

\subsubsection*{Hierarchical analysis proposal}
\begin{enumerate}
  \item It uses \textbf{pairwise comparisons} to value utilities instead of direct measurement.
  \item It uses \textbf{qualitative scales} instead of quantitative scales.
  \item It uses \textbf{pairwise comparisons} to value the weights of attributes.
  \item Structures attributes in a hierarchy.
  \item Weights are recombined to each layer of the hierarchy.
\end{enumerate}

\subsubsection*{Limits of hierarchical analysis}
\begin{enumerate}
  \item The hierarchical analysis can be applied only to finite problems with a small number of solutions.
  \item It is affected from the \textbf{rank reversal}, even though this can be resolved via \textbf{absolute scales} confronting classes of alternatives instead of the solutions.
\end{enumerate}
\end{document}