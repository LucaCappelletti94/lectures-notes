\providecommand{\main}{../../..}
\documentclass[\main/main.tex]{subfiles}
\begin{document}

\subsection{Esercizio 7}
Nel seguente gioco a due persone, i numeri corrispondono a payoff, cioè benefici:

\begin{table}
  \begin{tabular}{|L|L|L|L|}
    \hline
      & a      & b       & c       \\
    \hline
    a & (7,17) & (21,21) & (14,11) \\
    \hline
    b & (10,5) & (14,4)  & (4,3)   \\
    \hline
    c & (4,4)  & (7,3)   & (10,25) \\
    \hline
  \end{tabular}
\end{table}

\begin{enumerate}
  \item Si determinino le strategie dominate, se ve ne sono.
  \item Si determinino i punti di equilibrio, se ve ne sono.
  \item Si spieghi che cosa si intende con gioco della caccia al cervo.
  \item Si spieghi che cosa si intende con gioco del matrimonio perfetto.
\end{enumerate}

\subsection{Soluzione esercizio 7}

\subsubsection*{Strategie dominate}
Per il giocatore di riga la terza riga è dominata dalla prima (tabella \ref{2017_t1}). Per il giocatore di colonna, la seconda colonna domina sia la prima che la terza (tabella \ref{2017_t2}).
Eliminate quest'ultime per il giocatore di riga la strategia $a$ risulta essere dominante (\ref{2017_t3}).

\begin{figure}
  \begin{subfigure}{0.31\textwidth}
    \begin{table}
      \begin{tabular}{|L|L|L|L|}
        \hline
          & a      & b       & c       \\
        \hline
        a & (7,17) & (21,21) & (14,11) \\
        \hline
        b & (10,5) & (14,4)  & (4,3)   \\
        \hline
      \end{tabular}
    \end{table}
    \caption{Elimino riga $c$}
    \label{2017_t1}
  \end{subfigure}
  \begin{subfigure}{0.31\textwidth}
    \begin{table}
      \begin{tabular}{|L|L|}
        \hline
          & b       \\
        \hline
        a & (21,21) \\
        \hline
        b & (14,4)  \\
        \hline
      \end{tabular}
    \end{table}
    \caption{Elimino colonna $a$ e $b$}
    \label{2017_t2}
  \end{subfigure}
  \begin{subfigure}{0.31\textwidth}
    \begin{table}
      \begin{tabular}{|L|L|}
        \hline
          & b       \\
        \hline
        a & (21,21) \\
        \hline
      \end{tabular}
    \end{table}
    \caption{Elimino riga $b$}
    \label{2017_t3}
  \end{subfigure}
  \caption{La strategia dominante risulta essere $(a,b)$}
\end{figure}

\subsubsection*{Punti di equilibrio}

\begin{table}
  \begin{tabular}{|L|L|L|L|}
    \hline
      & a              & b                       & c               \\
    \hline
    a & (7,17)         & (\tilde{21},\tilde{21}) & (\tilde{14},11) \\
    \hline
    b & (\tilde{10},5) & (\tilde{14},4)          & (4,3)           \\
    \hline
    c & (4,4)          & (7,3)                   & (10,\tilde{25}) \\
    \hline
  \end{tabular}
  \caption{Esiste un equilibrio di Nash in $(a,b)$}
\end{table}

\subsubsection*{Gioco della caccia al cervo}
Nel gioco della caccia al cervo sono considerati due cacciatori che devono decidere se cercare di collaborare ed abbattere una preda grossa oppure cacciare indipendentemente e catturare qualcosa di piccolo.

\begin{table}
  \begin{tabular}{|L|L|L|}
    \hline
       & C                      & NC                    \\
    \hline
    C  & (\tilde{3}, \tilde{3}) & (0, 2)                \\
    \hline
    NC & (2, 0)                 & (\tilde{1},\tilde{1}) \\
    \hline
  \end{tabular}
  \caption{Gioco della caccia al cervo}
\end{table}

Se entrambi i cacciatori collaborano hanno il payoff massimo, abbattendo la preda grande. Quando uno cerca di collaborare e l'altro va indipendentemente solo il secondo ha un piccolo payoff ed il primo niente, mentre quando entrambi vanno indipendenti possono ottenere un piccolo payoff.

\subsubsection*{Gioco del matrimonio perfetto}
Nel gioco del matrimonio perfetto si considerano due coniugi che devono decidere se collaborare o meno per questioni di casa.

\begin{table}
  \begin{tabular}{|L|L|L|}
    \hline
       & C                      & NC             \\
    \hline
    C  & (\tilde{3}, \tilde{3}) & (\tilde{1}, 2) \\
    \hline
    NC & (2, \tilde{1})         & (0,0)          \\
    \hline
  \end{tabular}
  \caption{Gioco del matrimonio perfetto}
\end{table}

La soluzione migliore si ha quando entrambi collaborano, ma nel caso in cui uno non collabori e l'altro voglia collaborare, per chi collabora è comunque meglio continuare a collaborare rispetto a smettere di collaborare anche lui.

\end{document}
