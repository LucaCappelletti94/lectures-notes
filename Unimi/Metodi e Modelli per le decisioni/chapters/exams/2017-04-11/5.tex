\providecommand{\main}{../../..}
\documentclass[\main/main.tex]{subfiles}
\begin{document}

\subsection{Esercizio 5}
Viene dato il seguente problema di programmazione in condizioni di incertezza, i cui valori indicano utilità:

\begin{table}
  \begin{tabular}{|L|L|L|L|L|}
    \hline
    u_{\omega a} & a_1 & a_2 & a_3 & a_4 \\
    \hline
    \omega_1     & 50  & 40  & 60  & 90  \\
    \hline
    \omega_1     & 10  & 30  & 20  & 0   \\
    \hline
  \end{tabular}
\end{table}

\begin{enumerate}[a)]
  \item Si indichi se vi sono alternative dominate e quali alternative le dominano.
  \item Si indichi l’alternativa scelta con il criterio del caso pessimo.
  \item Si mostri come cambia l’alternativa scelta con il criterio del caso medio al variare della probabilità $\pi(\omega_1)$ del primo scenario.
\end{enumerate}

\subsection{Soluzione esercizio 5}


\end{document}
