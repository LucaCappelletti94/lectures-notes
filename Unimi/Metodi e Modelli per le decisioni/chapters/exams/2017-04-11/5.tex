\providecommand{\main}{../../..}
\documentclass[\main/main.tex]{subfiles}
\begin{document}

\subsection{Esercizio 5}
Viene dato il seguente problema di programmazione in condizioni di incertezza, i cui valori indicano utilità:

\begin{table}
  \begin{tabular}{|L|L|L|L|L|}
    \hline
    u_{\omega a} & a_1 & a_2 & a_3 & a_4 \\
    \hline
    \omega_1     & 50  & 40  & 60  & 90  \\
    \hline
    \omega_2     & 10  & 30  & 20  & 0   \\
    \hline
  \end{tabular}
\end{table}

\begin{enumerate}[a)]
  \item Si indichi se vi sono alternative dominate e quali alternative le dominano.
  \item Si indichi l’alternativa scelta con il criterio del caso pessimo.
  \item Si mostri come cambia l’alternativa scelta con il criterio del caso medio al variare della probabilità $\pi(\omega_1)$ del primo scenario.
\end{enumerate}

\subsection{Soluzione esercizio 5}
\subsubsection*{Alternative dominate}
Un'alternativa domina un'altra quando, senza applicare particolari criteri, questa è migliore per almeno uno degli scenari ed uguale in tutti gli altri. Cerchiamo la soluzione che dia scenari a valore massimo poichè i valori rappresentano utilità.

In questo caso, l'alternativa solamente $a_3 \prec a_1$, mentre non è possibile affermare nulla delle altre che sono preferibili l'una alle altre per almeno uno scenario.

\subsubsection*{Criterio del caso pessimo}
Nel criterio del caso pessimo, si assume che qualsiasi scelta venga effettuata accadrà il suo corrispondente caso pessimo. La scelta con scenario pessimo migliore è $a_2$.

\begin{table}
  \begin{tabular}{|L|L|L|L|L|}
    \hline
    u_{\omega a} & a_1                   & a_2                   & a_3                   & a_4                 \\
    \hline
    \omega_1     & 50                    & 40                    & 60                    & 90                  \\
    \hline
    \omega_2     & \cellcolor{red!50} 10 & \cellcolor{red!50} 30 & \cellcolor{red!50} 20 & \cellcolor{red!50}0 \\
    \hline
  \end{tabular}
  \caption{Casi pessimi in rosso}
\end{table}

\subsubsection*{Caso medio}
Il caso medio (suppongo sia, dato che non appare nelle dispense) il caso di Hurwicz con coefficiente della combinazione convessa pari $\pi(\omega_1)$ per il primo scenario (che coincide sempre con il caso ottimo) e $1-\pi(\omega_1)$ per il secondo (che coincide sempre con il caso pessimo).

\begin{table}
  \begin{tabular}{|L|L|L|L|L|}
    \hline
        & a_1                                   & a_2                                   & a_3                                   & a_4             \\
    \hline
    u_H & 50\pi(\omega_1) + 10(1-\pi(\omega_1)) & 40\pi(\omega_1) + 30(1-\pi(\omega_1)) & 60\pi(\omega_1) + 20(1-\pi(\omega_1)) & 90\pi(\omega_1) \\
    \hline
  \end{tabular}
\end{table}

\begin{figure}
  \begin{subfigure}{0.31\textwidth}
    \begin{table}[H]
      \begin{tabular}{|L|L|L|L|L|}
        \hline
            & a_1 & a_2 & a_3 & a_4 \\
        \hline
        u_H & 50  & 40  & 60  & 90  \\
        \hline
      \end{tabular}
      \caption{Caso $\pi(\omega_1)=1$: viene scelta $a_4$}
    \end{table}
  \end{subfigure}
  \begin{subfigure}{0.31\textwidth}
    \begin{table}[H]
      \begin{tabular}{|L|L|L|L|L|}
        \hline
            & a_1 & a_2 & a_3 & a_4 \\
        \hline
        u_H & 10  & 30  & 20  & 0   \\
        \hline
      \end{tabular}
      \caption{Caso $\pi(\omega_1)=0$: scelta $a_2$}
    \end{table}
  \end{subfigure}
  \begin{subfigure}{0.31\textwidth}
    \begin{tikzpicture}
      \begin{axis}[
          width= \textwidth,
          xlabel=$f$,
          ylabel=$u_H$,
          domain=0:1,
          ytick = {0,10,...,90},
          legend style={at={(0,1)},anchor=north west}
        ]
        \addplot[mark=none,color=red]{50*x + 10*(1-x)};
        \addplot[mark=none,color=blue]{40*x + 30*(1-x)};
        \addplot[mark=none,color=green]{60*x + 20*(1-x)};
        \addplot[mark=none]{90*x};
        \legend{$a_1$,$a_2$,$a_3$,$a_4$}
      \end{axis}
    \end{tikzpicture}
    \caption{L'utilità $u_H$ al variare della probabilità}
  \end{subfigure}
\end{figure}



\end{document}
