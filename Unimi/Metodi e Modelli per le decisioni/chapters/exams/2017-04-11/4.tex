\providecommand{\main}{../../..}
\documentclass[\main/main.tex]{subfiles}
\begin{document}

\subsection{Esercizio 4}
Si spieghi sotto quali condizioni è possibile ricavare da una matrice di confronti a coppie un vettore di pesi.

Se possibile, lo si faccia sulla seguente matrice.

\[
  M = \begin{bmatrix}
    1            & 3            & \sfrac{1}{2}  & 6  \\
    \sfrac{1}{3} & 1            & \sfrac{1}{6}  & 2  \\
    2            & 6            & 1             & 12 \\
    \sfrac{1}{6} & \sfrac{1}{2} & \sfrac{1}{12} & 1
  \end{bmatrix}
\]

Si spieghi un metodo qualsiasi per ricavare un vettore di pesi da una matrice che non lo consente direttamente.

\subsection{Soluzione esercizio 4}
\subsubsection*{Condizioni per ottenere un vettore di pesi}
Una matrice di confronti a coppie per consentire di ottenere un vettore di pesi deve essere \textbf{coerente}, cioè possedere le seguenti proprietà:

\begin{description}
  \item[Positività] Tutti i pesi della matrice devono essere positivi:
        \[
          \lambda_{lm} > 0
        \]
  \item[Reciprocità] Tutti i pesi devono essere l'inverso del proprio simmetrico:
        \[
          \lambda_{lm} = \frac{1}{\lambda_{ml}}
        \]
  \item[Coerenza] Dati due pesi con indici successivi, da essi si deve poter determinare il terzo:
        \[
          \lambda_{ln} = \lambda_{lm}\lambda_{mn}
        \]
\end{description}

\subsubsection*{Calcolo del vettore di pesi}
La matrice $M$ rispetta le proprietà di positività, di reciprocità e di coerenza, per cui si può procedere a calcolare il vettore dei pesi:

\[
  \omega_i = \frac{\bm{M}_i}{\sum_{j=1}^n \alpha_{ij}} = \begin{bmatrix}
    \sfrac{2}{7}  \\
    \sfrac{2}{21} \\
    \sfrac{4}{7}  \\
    \sfrac{1}{21} \\
  \end{bmatrix}
\]


\subsubsection*{Ricostruzione di matrici coerenti}
Si procede risolvendo il problema di minimizzazione seguente, dove $\Lambda$ è la matrice costituita dai rapporti stimati dal decisore, $W$ la matrice costituita dai rapporti $\frac{\omega_l}{\omega_m}$.

\begin{align*}
  \min_\omega \norm{W - \Lambda}                 \\
  \sum_{l \in P} \omega_l & = 1                  \\
  \omega_l                & \geq 0 \quad l \in P
\end{align*}

Il valore ottimo della funzione obbiettivo si può assumere come misura dell'incoerenza iniziale.

Per tenere conto dell'incoerenza del decisore è stato anche proposto di sostituire i valori stimati $\lambda_{lm}$ con intervalli e di definire la norma componendo le distanze fra ciascun rapporto $\frac{\omega_l}{\omega_m}$ e l'intervallo corrispondente, anziché il valore stimato.


\end{document}
