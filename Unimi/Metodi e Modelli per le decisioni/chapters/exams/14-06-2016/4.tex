\providecommand{\main}{../../..}
\documentclass[\main/main.tex]{subfiles}
\begin{document}

\subsection{Esercizio 4}
Si elenchino le condizioni che rendono coerente una matrice di confronti a coppie.

Si descriva il procedimento per determinare un vettore di pesi corretto a partire da una matrice di confronti a coppie coerente.

Si descriva brevemente un modo per derivare una matrice coerente a partire da una non coerente e un criterio per valutare la qualità di tale derivazione.

\subsection{Soluzione esercizio 4}
\subsubsection*{Condizioni di coerenza}
Una matrice di confronti a coppie è coerente quando ha le seguenti proprietà:
\begin{description}
  \item[Positività] I tassi di sostituzione fra utilità normalizzati sono sempre positivi.
        \[
          \lambda_{lm} > 0
        \]
  \item[Reciprocità] Il tasso di sostituzione di una utilità normalizzata rispetto ad un'altra è il reciproco del tasso della seconda rispetto alla prima.
        \[
          \lambda_{ln} =  \frac{1}{\lambda_{ml}}
        \]
  \item[Coerenza] Due tassi di sostituzione con indici ``in sequenza'' determinano il terzo:
        \[
          \lambda_{ln} = \lambda_{lm}\lambda_{mn}
        \]
\end{description}

\subsubsection*{Calcolo del vettore di pesi}
Il vettore dei pesi di una matrice coerente risulta pari ad una colonna qualsiasi della matrice normalizzata per la somma di sé stessa:

\[
  \omega_i = \frac{\bm{M}_i}{\sum_{j=1}^n \alpha_{ij}}
\]

\subsubsection*{Ricostruzione di matrici coerenti}
Si procede risolvendo il problema di minimizzazione seguente, dove $\Lambda$ è la matrice costituita dai rapporti stimati dal decisore, $W$ la matrice costituita dai rapporti $\frac{\omega_l}{\omega_m}$.

\begin{align*}
  \min_\omega \norm{W - \Lambda}                 \\
  \sum_{l \in P} \omega_l & = 1                  \\
  \omega_l                & \geq 0 \quad l \in P
\end{align*}

Il valore ottimo della funzione obbiettivo si può assumere come misura dell'incoerenza iniziale.

Per tenere conto dell'incoerenza del decisore è stato anche proposto di sostituire i valori stimati $\lambda_{lm}$ con intervalli e di definire la norma componendo le distanze fra ciascun rapporto $\frac{\omega_l}{\omega_m}$ e l'intervallo corrispondente, anziché il valore stimato.

\end{document}
