\providecommand{\main}{../../..}
\documentclass[\main/main.tex]{subfiles}
\begin{document}

\section{Exercise 4}

\fig{
  \[
    \Lambda = \begin{bmatrix}
      1  & \sfrac{1}{12} & \sfrac{1}{4} & \sfrac{1}{20} \\
      12 & 1             & 3            & \sfrac{3}{5}  \\
      4  & \sfrac{1}{3}  & 1            & \sfrac{1}{5}  \\
      20 & \sfrac{5}{3}  & 5            & 1
    \end{bmatrix}
  \]
}{
  \begin{table}
    \begin{tabular}{L|LLLL}
      u_{fa} & a_1  & a_2  & a_3  & a_4  \\
      \hline
      f_1    & 0.82 & 0.88 & 0.75 & 0.50 \\
      f_2    & 0.15 & 0.19 & 0.20 & 0.40
    \end{tabular}
  \end{table}
}

\begin{enumerate}
  \item Explain the necessary and sufficient conditions in order that a pairwise matrix may define a weights vector.
  \item Determine a weight vector from the given matrix or explain why it isn't possible.
  \item Determine the surclassment relationship based on the threshold $\epsilon_1=\epsilon_2=0.1$ from the given valuations table.
  \item Determine the nucleus of the relationship.
\end{enumerate}

\end{document}