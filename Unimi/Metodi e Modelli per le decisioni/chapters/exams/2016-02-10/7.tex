\providecommand{\main}{../../..}
\documentclass[\main/main.tex]{subfiles}
\begin{document}

\subsection{Esercizio 7}
Dato il seguente gioco a due persone:

\begin{table}
  \begin{tabular}{|L|L|L|}
    \hline
      & A     & B      \\
    \hline
    A & (0,5) & (-1,3) \\
    \hline
    B & (0,0) & (-1,3) \\
    \hline
  \end{tabular}
\end{table}

\begin{enumerate}
  \item Determinare se vi siano strategie dominate.
  \item Determinare gli eventuali punti di equilibrio (o dimostrare che non ve ne sono).
  \item Applicare il criterio del caso pessimo al gioco in forma estesa, assumendo che i due giocatori non muovano simultaneamente, ma muova prima $G_r$ e poi $G_c$.
  \item Applicare il criterio del caso pessimo al gioco in forma estesa, assumendo che i due giocatori non muovano simultaneamente, ma muova prima $G_c$ e poi $G_r$.
\end{enumerate}

\subsection{Soluzione esercizio 7}

\subsubsection*{Strategie dominate}
Non esistono strategie dominate.

\subsubsection*{Punti di equilibrio}
L'unico punto di equilibrio è per la strategia $(A,A)$.
\begin{table}
  \begin{tabular}{|L|L|L|}
    \hline
      & A                     & B              \\
    \hline
    A & (\tilde{0},\tilde{5}) & (-1,\tilde{3}) \\
    \hline
    B & (\tilde{0},0)         & (-1,\tilde{3}) \\
    \hline
  \end{tabular}
\end{table}

\subsubsection*{Criterio del caso pessimo}
Nel criterio del caso pessimo un giocatore assume che l'altro cercherà di danneggiarlo il più possibile.
\paragraph*{Prima $G_r$, poi $G_c$}
Per $G_r$ scegliere $A$ o $B$ non porta ad payoff differenti: il controllo sull'esito infatti lo ha $G_c$.

\begin{figure}
  \begin{subfigure}{0.24\textwidth}
    \begin{table}
      \begin{tabular}{|L|L|L|}
        \hline
          & A                         & B                          \\
        \hline
        A & \cellcolor{blue!50} (0,5) & \cellcolor{blue!50} (-1,3) \\
        \hline
        B & (0,0)                     & (-1,3)                     \\
        \hline
      \end{tabular}
    \end{table}
    \caption{Turno 1: $G_r$ sceglie A}
  \end{subfigure}
  \begin{subfigure}{0.24\textwidth}
    \begin{table}
      \begin{tabular}{|L|L|L|}
        \hline
          & A                         & B                            \\
        \hline
        A & \cellcolor{blue!50} (0,5) & \cellcolor{purple!70} (-1,3) \\
        \hline
        B & (0,0)                     & \cellcolor{red!50} (-1,3)    \\
        \hline
      \end{tabular}
    \end{table}
    \caption{Turno 2: $G_c$ sceglie B per garantire massima perdita}
  \end{subfigure}
  \begin{subfigure}{0.24\textwidth}
    \begin{table}
      \begin{tabular}{|L|L|L|}
        \hline
          & A                         & B                         \\
        \hline
        A & (0,5)                     & (-1,3)                    \\
        \hline
        B & \cellcolor{blue!50} (0,0) & \cellcolor{blue!50}(-1,3) \\
        \hline
      \end{tabular}
    \end{table}
    \caption{Turno 1: $G_r$ sceglie B}
  \end{subfigure}
  \begin{subfigure}{0.24\textwidth}
    \begin{table}
      \begin{tabular}{|L|L|L|}
        \hline
          & A                         & B                            \\
        \hline
        A & (0,5)                     & \cellcolor{red!50} (-1,3)    \\
        \hline
        B & \cellcolor{blue!50} (0,0) & \cellcolor{purple!70} (-1,3) \\
        \hline
      \end{tabular}
    \end{table}
    \caption{Turno 2: $G_c$ sceglie B per garantire massima perdita}
  \end{subfigure}
\end{figure}


\paragraph*{Prima $G_c$, poi $G_r$}
$G_c$ sceglierà $B$, poichè, nel caso pessimo, $G_r$ può scegliere $B$ e imporgli un payoff pari a $0$, mentre scegliendo $B$ il payoff sarà sempre $3$.
\begin{figure}
  \begin{subfigure}{0.24\textwidth}
    \begin{table}
      \begin{tabular}{|L|L|L|}
        \hline
          & A     & B                         \\
        \hline
        A & (0,5) & \cellcolor{red!50} (-1,3) \\
        \hline
        B & (0,0) & \cellcolor{red!50}(-1,3)  \\
        \hline
      \end{tabular}
    \end{table}
    \caption{Turno 1: $G_c$ sceglie B}
  \end{subfigure}
  \begin{subfigure}{0.24\textwidth}
    \begin{table}
      \begin{tabular}{|L|L|L|}
        \hline
          & A                         & B                            \\
        \hline
        A & (0,5)                     & \cellcolor{red!50} (-1,3)    \\
        \hline
        B & \cellcolor{blue!50} (0,0) & \cellcolor{purple!70} (-1,3) \\
        \hline
      \end{tabular}
    \end{table}
    \caption{Turno 2: $G_r$ può scegliere sia $A$ che $B$ (nella figura sceglie $B$), senza alterare la massima perdita che può imporre a $G_c$}
  \end{subfigure}
\end{figure}
\end{document}
