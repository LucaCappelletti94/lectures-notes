\providecommand{\main}{../../..}
\documentclass[\main/main.tex]{subfiles}
\begin{document}

\subsection{Esercizio 8}
Considerando il metodo di Condorcet per le decisioni di gruppo:
\begin{enumerate}[a)]
  \item Si descriva il metodo stesso.
  \item Se ne indichi il difetto principale, fornendo un esempio.
  \item Si fornisca un esempio (non banale) nel quale il metodo produce una preferenza di gruppo che sia un ordine debole.
\end{enumerate}

\subsection{Soluzione esercizio 8}

\subsubsection*{Il metodo di Condorcet}
Nel metodo di Condorcet, una scelta $x_1$ domina una seconda $x_2$ quando il numero di decisori per cui $x_1\preceq x_2$ è maggiore del numero di decisori per cui $x_2\preceq x_1$. Questo, generalmente, consente di creare un ordine debole dal quale trarre la decisione di gruppo.

\subsubsection*{Il difetto del metodo di Condorcet}
Nel caso di preferenze circolari il metodo di Condorcet non è in grado di realizzare un ordine debole.

\begin{table}
  \begin{tabular}{|L|L|L|L|L|}
    \hline
    Ordine & d_1 & d_2 & d_3 & d_4 \\
    \hline
    1      & a   & b   & c   & d   \\
    \hline
    2      & b   & c   & d   & c   \\
    \hline
    3      & c   & d   & a   & a   \\
    \hline
    4      & d   & a   & b   & b   \\
    \hline
  \end{tabular}
  \caption{Preferenze circolari: caso in cui Condorcet fallisce}
\end{table}

\subsubsection*{Esempio del metodo di Condorcet}

\begin{figure}
  \begin{subfigure}{0.31\textwidth}
    \begin{table}
      \begin{tabular}{|L|L|L|L|}
        \hline
        Ordine & d_1 & d_2 & d_3 \\
        \hline
        1      & a   & b   & c   \\
        \hline
        2      & b   & c   & b   \\
        \hline
        3      & c   & d   & a   \\
        \hline
        4      & d   & a   & d   \\
        \hline
      \end{tabular}
    \end{table}
    \caption{Tabella delle preferenze per decisore}
  \end{subfigure}
  \begin{subfigure}{0.31\textwidth}
    \begin{table}
      \begin{tabular}{|L|L|L|L|L|}
        \hline
          & a & b & c & d \\
        \hline
        a & 1 & 0 & 0 & 1 \\
        \hline
        b & 1 & 1 & 1 & 1 \\
        \hline
        c & 1 & 0 & 1 & 1 \\
        \hline
        d & 0 & 0 & 0 & 1 \\
        \hline
      \end{tabular}
    \end{table}
    \caption{Tabella delle preferenze di gruppo}
  \end{subfigure}
  \begin{subfigure}{0.31\textwidth}
    \begin{table}
      \begin{tabular}{|L|L|L|L|L|}
        \hline
        Ordine & Scelta \\
        \hline
        1      & b      \\
        \hline
        2      & c      \\
        \hline
        3      & a      \\
        \hline
        4      & d      \\
        \hline
      \end{tabular}
    \end{table}
    \caption{Ordine debole secondo Condorcet}
  \end{subfigure}
  \caption{Esempio del metodo di Condorcet}
\end{figure}






\end{document}
