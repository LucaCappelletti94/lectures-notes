\providecommand{\main}{../../..}
\documentclass[\main/main.tex]{subfiles}
\begin{document}

\subsection{Esercizio 1}
Dato un problema decisionale con insieme di impatti $F = \{a, b, c, d, e, f \}$ e relazione di preferenza $\Pi$:

\[
  \Pi = \{(a, a), (a, b), (a, c), (a, d), (b, b), (b, c), (b, d), (c, b), (c, c), (c,d),(d,d),(e,b),(e,c),(e,d),(e,e),(f,b),(f,c),(f,d),(f,f)\}
\]

\begin{enumerate}
  \item Si spieghi il significato della relazione per il problema decisionale.
  \item Si elenchino le proprietà principali di cui $\Pi$ gode.
  \item Si derivi la relazione di indifferenza associata $\text{Ind}_\Pi$.
  \item Si dica se la relazione è un ordine di qualche genere e quali conseguenze questo ha sul problema decisionale.
\end{enumerate}

\subsection{Soluzione esercizio 1}

Rappresentando la relazione come un grafo si ottiene:

\NewAdigraph{PreferenceRelationship}{%
  a:2;
  b:2;
  c:2;
  d:2;
  e:2;
  f:2;
}{%
  a,a;
  a,b;
  a,c;
  b,b;
  b,c;
  b,d;
  c,b;
  c,c;
  c,d;
  d,d;
  e,b;
  e,c;
  e,d;
  e,e;
  f,b;
  f,c;
  f,d;
  f,f;
}

\begin{figure}
  \PreferenceRelationship{}
\end{figure}

\NewAdigraph{ReflexivePreferenceRelationship}{%
  a:2;
  b:2;
  c:2;
  d:2;
  e:2;
  f:2;
}{%
  a,a,green;
  a,b;
  a,c;
  b,b,green;
  b,c;
  b,d;
  c,b;
  c,c,green;
  c,d;
  d,d,green;
  e,b;
  e,c;
  e,d;
  e,e,green;
  f,b;
  f,c;
  f,d;
  f,f,green;
}

\NewAdigraph{IndifferencePreferenceRelationship}{%
  a:2;
  b:2;
  c:2;
  d:2;
  e:2;
  f:2;
}{%
  a,a,red;
  a,b;
  a,c;
  b,b,red;
  b,c,red;
  b,d;
  c,b,red;
  c,c,red;
  c,d;
  d,d,red;
  e,b;
  e,c;
  e,d;
  e,e,red;
  f,b;
  f,c;
  f,d;
  f,f,red;
}
1. La relazione $\Pi$ è binaria, cioè specifica coppie di impatti $(f_1, f_2): f_1 \preceq f_2$.

\begin{figure}
  \begin{subfigure}{0.49\textwidth}
    \begin{table}
      \begin{tabular}{|c|c|c|c|c|c|c|}
        \hline
          & a                      & b                      & c                      & d                      & e                      & f                      \\
        \hline
        a & \cellcolor{green!25} 1 & 1                      & 1                      & 1                      & 0                      & 0                      \\
        \hline
        b & 0                      & \cellcolor{green!25} 1 & 1                      & 1                      & 0                      & 0                      \\
        \hline
        c & 0                      & 1                      & \cellcolor{green!25} 1 & 1                      & 0                      & 0                      \\
        \hline
        d & 0                      & 0                      & 0                      & \cellcolor{green!25} 1 & 0                      & 0                      \\
        \hline
        e & 0                      & 1                      & 1                      & 1                      & \cellcolor{green!25} 1 & 0                      \\
        \hline
        f & 0                      & 0                      & 1                      & 1                      & 0                      & \cellcolor{green!25} 1 \\
        \hline
      \end{tabular}
      \caption{Rappresentazione a tabella di $\Pi$}
    \end{table}
  \end{subfigure}
  \begin{subfigure}{0.49\textwidth}
    \ReflexivePreferenceRelationship{}
  \end{subfigure}
\end{figure}

2. La relazione $\Pi$ gode della proprietà \textbf{riflessiva} (la diagonale principale della matrice è composta da soli $1$, evidenziati in verde).

\begin{figure}
  \begin{subfigure}{0.49\textwidth}
    \begin{table}
      \begin{tabular}{|c|c|c|c|c|c|c|}
        \hline
          & a                    & b                    & c                    & d                    & e                    & f                    \\
        \hline
        a & \cellcolor{red!25} 1 & 1                    & 1                    & 1                    & 0                    & 0                    \\
        \hline
        b & 0                    & \cellcolor{red!25} 1 & \cellcolor{red!25} 1 & 1                    & 0                    & 0                    \\
        \hline
        c & 0                    & \cellcolor{red!25} 1 & \cellcolor{red!25} 1 & 1                    & 0                    & 0                    \\
        \hline
        d & 0                    & 0                    & 0                    & \cellcolor{red!25} 1 & 0                    & 0                    \\
        \hline
        e & 0                    & 1                    & 1                    & 1                    & \cellcolor{red!25} 1 & 0                    \\
        \hline
        f & 0                    & 0                    & 1                    & 1                    & 0                    & \cellcolor{red!25} 1 \\
        \hline
      \end{tabular}
      \caption{Rappresentazione a tabella di $\Pi$, con $\text{Ind}_\Pi$ evidenziata.}
    \end{table}
  \end{subfigure}
  \begin{subfigure}{0.49\textwidth}
    \IndifferencePreferenceRelationship{}
  \end{subfigure}
\end{figure}

3. La \textbf{relazione di indifferenza} $\text{Ind}_\Pi$ è composta dalle coppie indifferenti, cioè che il decisore accetta di scambiare in entrambe le direzioni (nella tabella evidenziate in rosso).

\[
  \Pi = \{(a, a), (b, b), (b,c), (c, c), (c, b), (d,d),(e,e),(f,f)\}
\]

4. La proprietà transitiva è rispettata.

Si tratta di una relazione di \textbf{pre-ordine}.
\end{document}