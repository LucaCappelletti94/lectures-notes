\providecommand{\main}{../../..}
\documentclass[\main/main.tex]{subfiles}
\begin{document}

\subsection{Esercizio 7}
Si distinguano i concetti di strategia pura e strategia mista nel contesto della teoria dei giochi.

Si definisca il concetto di strategia dominata nel caso dei giochi a due persone.

Si indichi se il gioco rappresentato dalla seguente matrice di payoff ammette strategie dominate oppure no, motivando la risposta.

\begin{table}
  \begin{tabular}{|L|L|L|L|}
    \hline
      & 1     & 2     & 3     \\
    \hline
    1 & (1,4) & (2,3) & (5,0) \\
    \hline
    2 & (0,0) & (1,1) & (3,3) \\
    \hline
    3 & (3,1) & (6,2) & (2,7) \\
    \hline
  \end{tabular}
\end{table}

Nel caso dei giochi a due persone a somma zero, è possibile che vi siano strategie dominate? In caso positivo, si fornisca un esempio; in caso negativo, si spieghi perché no.

\subsection{Soluzione esercizio 7}

\subsubsection*{Strategia pura e mista}
Una strategia pura è composta da un solo impatto certo cioè con probabilità unitaria. Una strategia mista è composta da più strategie pure, ognuna con una determinata probabilità di essere scelta.

\subsubsection*{Strategia dominata}
Una strategia $s_1$ è dominata da una strategia $s_2$ quando ogni possibile payoff di $s_1$ è peggiore o uguale dei rispettivi payoff di $s_2$.

\subsubsection*{Strategie dominate nel gioco}
Per il giocatore di riga, la seconda riga è dominata sia dalla prima che dalla terza, pertanto viene eliminata (tabella \ref{step_1}). Per il giocatore di colonna nessuna strategia risulta dominata.

\begin{table}
  \begin{tabular}{|L|L|L|L|}
    \hline
      & 1     & 2     & 3     \\
    \hline
    1 & (1,4) & (2,3) & (5,0) \\
    \hline
    3 & (3,1) & (6,2) & (2,7) \\
    \hline
  \end{tabular}
  \caption{Elimino seconda riga}
  \label{step_1}
\end{table}

\subsubsection*{Strategie dominate nei giochi a somma zero}
Si, nei giochi a somma zero esistono strategie dominate. Per esempio, nel gioco a somma zero in tabella \ref{t1} il giocatore di riga sceglierà sempre la riga $B$ (tabella \ref{t2}) ed il giocatore di colonna sceglierà sempre la colonna $B$ (tabella \ref{t3}).

\begin{figure}
  \begin{subfigure}{0.31\textwidth}
    \begin{table}
      \begin{tabular}{|L|L|L|}
        \hline
          & A & B  \\
        \hline
        A & 0 & -1 \\
        \hline
        B & 3 & 2  \\
        \hline
      \end{tabular}
    \end{table}
    \caption{Gioco a somma zero con strategie dominate}
    \label{t1}
  \end{subfigure}
  \begin{subfigure}{0.31\textwidth}
    \begin{table}
      \begin{tabular}{|L|L|L|}
        \hline
          & A & B \\
        \hline
        B & 3 & 2 \\
        \hline
      \end{tabular}
    \end{table}
    \caption{Elimino la riga $A$}
    \label{t2}
  \end{subfigure}
  \begin{subfigure}{0.31\textwidth}
    \begin{table}
      \begin{tabular}{|L|L|}
        \hline
          & B \\
        \hline
        B & 2 \\
        \hline
      \end{tabular}
    \end{table}
    \caption{Elimino la colonna $A$}
    \label{t3}
  \end{subfigure}
\end{figure}

\end{document}
