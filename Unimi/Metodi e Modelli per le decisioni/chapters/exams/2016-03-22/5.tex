\providecommand{\main}{../../..}
\documentclass[\main/main.tex]{subfiles}
\begin{document}

\subsection{Esercizio 5}
Si descriva brevemente il criterio del rammarico per la risoluzione di problemi di decisione in condizioni di ignoranza, spiegandone la logica intrinseca.

Si indichi se tale criterio fornisce un ordinamento debole delle alternative del problema, spiegando che cosa si intende con tale proprietà e motivando la risposta.

Si dica se tale criterio garantisce l'indipendenza dalle alternative irrilevanti, spiegando che cosa si intende con tale proprietà e motivando la risposta.

\subsection{Soluzione esercizio 5}
\subsubsection*{Criterio del rammarico}
Il criterio del rammarico identifica il caso ottimo per ogni scenario, quindi calcola per ogni combinazione di scelta e scenario la distanza dal valore ottimo dello scenario, detta rammarico. Si va a determinare quindi, per ogni scelta, il rammarico massimo e si sceglie la scelta con rammarico minimo.

\subsubsection*{Ordinamento}
Fornisce sempre un ordine debole poiché rende lineare il confronto tra i vari scenari. Il rammarico al più risulterà uguale tra più scelte, ma sono certamente ordinate debolmente.

\subsubsection*{Indipendenza tra alternative irrilevanti}
Nessuno dei criteri di scelta rispetta la proprietà di indipendenza, che garantirebbe l'assenza del \textbf{rank reversal}.
\end{document}
