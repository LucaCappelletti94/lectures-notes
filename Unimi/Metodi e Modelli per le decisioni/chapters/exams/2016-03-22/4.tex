\providecommand{\main}{../../..}
\documentclass[\main/main.tex]{subfiles}
\begin{document}

\subsection{Esercizio 4}
Si descrivano brevemente i seguenti aspetti dell'Analisi Gerarchica (AHP), spiegandone le motivazioni:
\begin{enumerate}[a)]
  \item L'uso di scale qualitative.
  \item L'uso di confronti a coppie.
  \item L'uso di una gerarchia di attributi.
\end{enumerate}
Se è possibile farlo con semplici calcoli, si ricavi dalla seguente matrice di confronti a coppie un vettore di pesi; si spieghi perché possibile farlo o perché non è possibile.

\[
  \bm{M} = \begin{bmatrix}
    1  & \sfrac{1}{12} & \sfrac{1}{4} & \sfrac{1}{20} \\
    12 & 1             & 3            & \sfrac{3}{5}  \\
    4  & \sfrac{1}{3}  & 1            & \sfrac{1}{5}  \\
    20 & \sfrac{5}{3}  & 5            & 1
  \end{bmatrix}
\]

\subsection{Risoluzione esercizio 4}
\subsubsection*{Scale qualitative}
Si usano scale qualitative per semplificare la valutazione della preferenza tra due impatti e si assegna un valore numerico arbitrariamente grande in base alla preferenza, per esempio $1$ per l'indifferenza o $7$ se un impatto fossero molto preferibile ad un'altro.
L'idea è di tradurre un giudizio qualitativo in un valore numero. Questo, intrensicamente, impedisce a chi fa uso del modello di affidarsene completamente come modello perfetto della realtà.

\subsubsection*{Confronti a coppie}
I pesi degli attributi vengono costruiti tramite l'elaborazione dei giudizi sul peso relativo degli indicatori forniti dal decisore, partendo da matrici di confronti a coppie costruite con i valori qualitative tratti dalla scala di Saaty.

\subsubsection*{Gerarchia di attributi}
Gli attributi vengono divisi in categorie di ordine e omogeneità simile per facilitare il confronto, considerando che nel caso di molti attributi procedere unicamente per confronti a coppie sarebbe impraticabile, mentre così il numero di confronti totale si riduce sensibilmente.

\subsubsection*{Calcolo dei pesi}
\[
  \bm{M} = \begin{bmatrix}
    1  & \sfrac{1}{12} & \sfrac{1}{4} & \sfrac{1}{20} \\
    12 & 1             & 3            & \sfrac{3}{5}  \\
    4  & \sfrac{1}{3}  & 1            & \sfrac{1}{5}  \\
    20 & \sfrac{5}{3}  & 5            & 1
  \end{bmatrix}
  \Rightarrow
  u_M = \begin{bmatrix}
    1  \\
    12 \\
    4  \\
    20 \\
  \end{bmatrix}/(1+12+4+20)
  = \begin{bmatrix}
    \sfrac{1 }{37} \\
    \sfrac{12}{37} \\
    \sfrac{4 }{37} \\
    \sfrac{20}{37} \\
  \end{bmatrix}
\]

\end{document}
