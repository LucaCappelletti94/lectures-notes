\providecommand{\main}{..}
\documentclass[\main/main.tex]{subfiles}
\begin{document}

Minimizzo $f(x)$, con la condizione di $g_j(x) \leq 0 \forall j = 1...n $.

Supponiamo di voler identificare la posizione migliore di una discarica, e che il punto in cui i rifiuti vengono prodotti sia $R = (1,0)$, che in punto $C = (0,0)$ vi sia in una città e che si debba avere una distanza di almeno $2$ dalla città. Inoltre, la nostra discarica deve trovarsi a sinistra di $\dfrac{3}{2}$, cioè $x_0 < \dfrac{3}{2}$, perché li vi è un confine.

\begin{center}
\begin{tikzpicture}
% axis
\draw[->] (-1,0) -- (3,0) node[right] {$x_1$};
\draw[->] (0,-1) -- (0,3) node[above] {$x_2$};

% points
\foreach \Point/\PointLabel in {(1,0)/R, (0,0)/C}
\draw[fill=black] \Point circle (0.05) node[above right] {$\PointLabel$};

\draw[domain=-1:3,smooth,variable=\y,red]  plot ({3/2},{\y});
\end{tikzpicture}
\end{center}

La funzione di minimo che vado a definire risulta:

\[
	\text{min} f(x) = \begin{cases}
		(x_1-1)^2 + x_2^2\\
		x_1^2 + x_2^2 = 4\\
		x_1  < \dfrac{3}{2}	
	\end{cases}
\]

\begin{center}
\begin{tikzpicture}
\begin{axis}[
	grid=major,
	samples=\samples,
	xlabel=$x_1$,
    ylabel=$x_2$,
    zlabel=$fitness$
]

\addplot3[surf, unbounded coords=jump]
{x^2 + y^2 > 4 && x < 3/2 ? (x-1)^2 + y^2 : NaN};

\end{axis}
\end{tikzpicture}
\end{center}


\section{Programmazione matematica}

\begin{definition}{Ottimo locale}
$\widetilde{x}$ ottimo locale $\Leftrightarrow$ $f(x) \geq f(\widetilde(x)) \forall x \in \mathbb{U}_{\widetilde{x}, \epsilon}$
\end{definition}

Dato $\widetilde{x}$ come un \textbf{ottimo locale}, e $\xi(\alpha)$ un \textbf{arco ammissibile} con la caratteristica di:

\[
	\xi(0) = \widetilde{x} \qquad
	\xi(alpha) \in X \forall \alpha \in [0, \widehat{\alpha})
\]

Allora vale che $\xi$ risulta \textbf{non migliorante}:

\[
	f(\xi(\alpha)) \geq f(\widetilde{x}) = f(\xi(0)) \forall \alpha \in [0, \widehat{\alpha})
\]

La formula sovra riportata può essere espressa più semplicemente tramite:

\[
	[\nabla f(\widetilde{x})]^T P_{\xi} \geq \emptyset
\]

\begin{definition}[Punti non regolari]
\[
	\widetilde{x} \text{ regolare } \Leftrightarrow \nabla g_j (\widetilde{x}) \text{per $ g_j$ attivo, con le varie funzioni $g_j$ linearmente indipendenti}
\]
\end{definition}

\begin{definition}[Punti non regolari]
Sono dei punti per cui non vale 
	\[
		[\nabla g_j (\widetilde)]^T P_\xi (\widetilde{x}) \geq 0 \text{ per $g_j$ attivo } \Leftarrow 	\begin{cases}
				\xi \text{ arco ammissibile}\\
				\widetilde{x} \text{ ottimo locale}
		\end{cases}
		\Rightarrow
		[\nabla f(\widetilde{x})]^T P_{\xi} \geq \emptyset
	\]
\end{definition}

\begin{center}
\begin{tikzpicture}
\begin{axis}[
	grid=major,
	samples=\samples,
	xlabel=$x_1$,
    ylabel=$x_2$,
    zlabel=$fitness$
]

\addplot3[surf, unbounded coords=jump]
{(x-1)^3+y<0 && ((x-1)^3-y<0)? 1 : 0};

\end{axis}
\end{tikzpicture}
\end{center}

\section{Lemma di Farkas}
Non ho capito a che serve

\begin{figure}[H]
\[
	C_j = \{ p \in \mathbb{R}^2: g_j^T p \leq 0 \forall j \} 
\]
\caption{Cono direzioni "opposte" ai vettori $g_j$}
\end{figure}

\begin{figure}[H]
\[
	C_f = \{ p \in \mathbb{R}^2: f^T p \leq 0 \forall j \} 
\]
\caption{Cono direzioni "opposte" a $f$}
\end{figure}

Se $\exists \mu_j \geq 0 : f = \sum_j \mu_j g_j \Leftrightarrow (C_g \subseteq C_f) -f^Tp \leq 0 \forall p: g_j^Tp \leq 0 \forall j$

Posso riscrivere questa formula usando i gradienti:

Se $\exists \mu_j \geq 0 : \nabla f = \sum_j \mu_j \nabla g_j \Leftrightarrow (C_g \subseteq C_f) -\nabla f^Tp \leq 0 \forall p: \nabla g_j^Tp \leq 0 \forall j$

che cosa è la combinazione lineare? e convessa? e conica?

\section{Altra roba che non capisco}

Se $\widetilde(x)$ è un \textbf{ottimo locale} e \textbf{regolare}, allora $\exists \mu_j \geq 0: \nabla f(\widetilde{x}) + \sum_{j:g_j \text{ attivo}} \mu_j \nabla g_j = \emptyset$

Questo viene posto a sistema con $\mu_j g_j (\widetilde{x}) = \emptyset \forall j = 1...n$:

\[
	\begin{cases}
		\exists \mu_j \geq 0: \nabla f(\widetilde{x}) + \sum_{j:g_j \text{ attivo}} \mu_j \nabla g_j = \emptyset\\
		\mu_j g_j (\widetilde{x}) = \emptyset \forall j = 1...n\\
		g_j(\widetilde{x}) <0 \Rightarrow \mu_j =0 \\
		g_j(x) \leq 0 \forall j = 1...m
	\end{cases}
\]

\end{document}