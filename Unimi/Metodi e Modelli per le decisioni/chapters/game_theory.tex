\providecommand{\main}{..}
\documentclass[\main/main.tex]{subfiles}
\begin{document}

\chapter{Teoria dei Giochi}
Studia le situazioni in cui ogni decisore (qui chiamato giocatore) $d \in D$ ha proprie variabili $x \in X$ di decisione che fissa in modo indipendente dagli altri decisori, cercando di massimizzare le funzioni impatto $f^{(d)}$ (qui chiamati payoff).

\begin{figure}
	\begin{align*}
		\max f^{(d)} & = f^{(d)}(x^{(1)}, \ldots, x^{(\abs{D})}) \\
		x^{(d)}      & \in X^{(d)}
	\end{align*}
	\caption{Problema di teoria dei giochi}
\end{figure}

I giochi possono essere rappresentati a \textbf{forma estesa} nella quale il gioco è rappresentato ad albero delle possibili scelte sino al termine del gioco o in \textbf{forma strategica} nella quale il gioco è rappresentato con una tabella dei payoff per ogni combinazione di scelte del giocatore e degli avversari.

\begin{definition}[Strategia]
	Un strategia per il giocare $d \in D$ corrisponde a un sottoinsieme di archi dell'albero di gioco che sia:
	\begin{enumerate}
		\item Costituito da archi associati al giocatore $d$.
		\item Coerente, cioè che non contenga più archi uscenti dallo stesso nodo.
		\item Massimale, cioè che contenga un arco per ogni nodo dotato di archi uscenti.
	\end{enumerate}
\end{definition}

\begin{definition}[Dominanza fra strategie]
	Date due strategie $x^{(d)}$ e $x'^{(d)}$ per il giocatore $d$, si dice che $x^{(d)}$ domina $x'^{(d)}$ quando, per qualsiasi comportamento di un altro giocatore, la strategia $x^{(d)}$ da un payoff $f^{(d)}$ maggiore.
\end{definition}

\begin{definition}[Equilibrio di Nash]
	Un vettore di strategie è detto \textbf{punto di equilibrio} o \textbf{equilibrio di Nash} quando a nessuno dei giocatori conviene cambiare la propria strategia (allontanarsi da quella nel vettore di strategie dell'equilibrio) perché avrebbe un payoff peggiore se nessuno degli altri giocatori cambia la propria strategia.
\end{definition}

\clearpage

\section{Calcolare gli equilibri di Nash}
Nei giochi finiti in forma strategica l'equilibrio di Nash è una (o più) delle caselle della tabella dei payoff. Per individuarlo, se esiste, si procede nel modo seguente:

\begin{enumerate}[a)]
	\item Si marca ogni colonna con utilità massima per il giocatore di riga.
	\item Si marca ogni riga con utilità massima per il giocatore di colonna.
	\item Le celle in cui tutte le utilità sono marcate sono equilibri di Nash.
\end{enumerate}

\section{Criteri per la scelta di strategie}
\subsection{Strategia del caso pessimo}
Il giocatore $d$ considera che l'avversario farà sempre la mossa volta a danneggiarlo maggiormente, e segue la strategia che quindi, nel caso pessimo, gli garantisce il massimo guadagno.

\begin{definition}[Valore di gioco]
	Si definisce \textbf{valore di gioco} (formula \ref{valore_gioco}) per il giocatore $d$ l'utilità massima ottenibile dal giocatore $d$ nel caso pessimo, cioè la miglior garanzia possinile sulla sua prestazione:
	\begin{figure}
		\begin{subfigure}{0.49\textwidth}
			\[
				u^{(d)} = \max_{x^{(d)} \in X^{(d)}} \min_{x^{(j)} \in X^{(j)}, j\neq d} f^{(j)} (x^{(1)}, \ldots, x^{(\abs{D})})
			\]
			\caption{Valore di gioco a $n$ giocatori}
		\end{subfigure}
		\begin{subfigure}{0.49\textwidth}
			\[
				u^{(r)} = \max_{x^{(r)} \in X^{(r)}} \min_{x^{(c)} \in X^{(c)}} f^{(c)} (x^{(r)}, x^{(c)})
			\]
			\caption{Valore di gioco a 2 giocatori ($r$ riga, $c$ colonna)}
		\end{subfigure}
		\caption{Valore di gioco}
		\label{valore_gioco}
	\end{figure}

	Nei \textbf{giochi a somma zero a 2 giocatori} si determina il valore di gioco del giocatore di riga $u^{(r)}$ e si indica quello del giocatore di colonna con $-u^{(r)}$, talvolta lasciato implicito nella notazione.
\end{definition}

\end{document}