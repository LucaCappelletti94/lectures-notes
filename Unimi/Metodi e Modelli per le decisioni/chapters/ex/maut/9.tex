\providecommand{\main}{../../..}
\documentclass[\main/main.tex]{subfiles}
\begin{document}

\subsection{Esercizio 9}
Dato il problema:

\begin{figure}
  \begin{align*}
    \min f_1(x) = x_1^2 + x_2^2 \\
    \max f_2(x) = x_2           \\
    x_2 & \leq 10
  \end{align*}
  \caption{Esercizio 9}
\end{figure}

Si esprima analiticamente l'utilità associata a $f_1(x)$ come:

\[
  u_1(f_1)\begin{cases}
    10 - \frac{f_1}{20} \qquad 0 \leq f_1 \leq 200 \\
    0 \qquad f_1 \geq 200
  \end{cases}
\]
cioè tale utilità decresce linearmente fino ad annullarsi per $f_1 = 200$, mentre l'utilità
associata a $f_2$ coincide con il valore stesso di $f_2$.

Si indichino le coordinate del “punto utopia” nello spazio delle utilità e nello spazio degli indicatori.

\subsection{Soluzione esercizio 9}
Il punto utopia rappresenta il punto in cui le funzioni acquisiscono il valore massimo, ignorando i vincoli di variabili tra le varie funzioni.

Il valore massimo di $u_1$ è $10$, per $f_1=0$.

Il valore massimo di $u_2$ è $10$, per $f_2=10$.

Da cui i due punti utopia risultano $U=(10,10)$ e $F=(0,10)$.



\end{document}