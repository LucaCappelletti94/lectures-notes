\providecommand{\main}{../../..}
\documentclass[\main/main.tex]{subfiles}

\newcommand{\customplottredici}[9]{
  \addplot3[mark=none,color=#9]{
    #1 + x + y <= 1 &&
    #5*(#2)  +#6*(#1)+ #7*(#3) +   #8*(#4) >= max(max(
    0*(#2)  +100*(#1)+ 70*(#3) +   40*(#4),
    100*(#2)+ 83*(#1)+ 20*(#3) +  100*(#4)),
    80*(#2) +  0*(#1)+ 100*(#3) +  20*(#4)
    )?#1:NaN};
}
\newcommand{\partialprojectiontredici}[7]{
  %
  % #1 -> Value of epsilon 2
  % #2 -> Number of specific w as 1 - #1 - x - y
  % #3 -> Number of specific w as x
  % #4 -> Number of specific w as y
  % #5 -> formula for w 1
  % #6 -> formula for w 3
  % #7 -> formula for w 4
  %
  \begin{subfigure}{0.325\textwidth}
    \begin{tikzpicture}
      \begin{axis}[
          width=\textwidth,
          xlabel=$\w_#2$,
          ylabel=$\w_#3$,
          zlabel=$\w_2$,
          domain=0:1-#1,
          y domain=0:1-#1
        ]

        \customplottredici{#1}{#5}{#6}{#7}{0}{100}{70}{40}{red}
        \customplottredici{#1}{#5}{#6}{#7}{100}{83}{20}{100}{green}
        \customplottredici{#1}{#5}{#6}{#7}{80}{0}{100}{20}{blue}

        \legend{$A^*$, $B^*$, $C^*$}
      \end{axis}
    \end{tikzpicture}
    \caption{With $\w_2=#1$ fixed, $\w_#2 = 1 - #1 - \w_#3 - \w_#4$}
  \end{subfigure}
}

\newcommand{\projectiontredici}[1]{
  \partialprojectiontredici{#1}{1}{3}{4}{1-#1-x-y}{x}{y}
  \partialprojectiontredici{#1}{3}{1}{4}{x}{1-#1-x-y}{y}
  \partialprojectiontredici{#1}{4}{1}{3}{x}{y}{1-#1-x-y}
}

\begin{document}

\subsection{Exercise 13}
The following decision problem is given, with 3 alternatives and 4 attributes (utilities), having weights $\w_i$:

\begin{table}
  \begin{tabular}{L|LLL}
    \text{Attributes} & A   & B   & C   \\
    \hline
    u_1               & 0   & 100 & 80  \\
    u_2               & 100 & 83  & 0   \\
    u_3               & 70  & 20  & 100 \\
    u_4               & 40  & 100 & 20
  \end{tabular}
  \begin{tabular}{L|L}
         & \text{Weights} \\
    \hline
    \w_1 & 0.25           \\
    \w_2 & 0.30           \\
    \w_3 & 0.40           \\
    \w_4 & 0.05
  \end{tabular}
\end{table}

Determine the best alternative with the method of the weighted sum.

Carry out a sensitivity analysis on the weight $\w_2 = 0.30$.

\subsection{Exercise 13 resolution}
\subsubsection*{Weighted sum}
\begin{table}
  \begin{tabular}{L|LLL}
    \text{Attributes} & A  & B    & C  \\
    \hline
    u^*               & 60 & 62.9 & 61 \\
  \end{tabular}
\end{table}
The method of the weighted sum suggests the alternative $B$.

\subsubsection*{Sensitivity analysis}
The exercise require us to determine for witch interval $\w_2 \in [\w_2 - \epsilon', \w_2 + \epsilon'']$ the suggested alternative $B$ does not change as optimal solution.

The weights are normalized, meaning that their sum is unitary:

\[
  \sum \w_i = 1 = \w_1+ \w_2 + \w_3 + \w_4
\]
The variation of a weight means the variation of at least another one.

The value of $\w_2$ is our knob to analyze the variation of the other weighs.

The inequality to solve is the following, considering the weights known.

\[
  100\w_1 + 83\w_2 + 20\w_3 + 100\w_4 \geq \max\crl{
    100\w_2 + 70\w_3 + 40\w_4,
    80\w_1 + 100\w_3 + 20\w_4
  }
\]

\begin{figure}
  \projectiontredici{0}
  \projectiontredici{0.05}
  \projectiontredici{0.1}
  \projectiontredici{0.15}
  \projectiontredici{0.2}
  \caption{The supports of solutions with $\w_2 \in \sqr{0,0.2}$}
\end{figure}
\begin{figure}
  \projectiontredici{0.25}
  \projectiontredici{0.3}
  \projectiontredici{0.35}
  \projectiontredici{0.4}
  \projectiontredici{0.45}
  \caption{The supports of solutions with $\w_2 \in \sqr{0.25,0.45}$}
\end{figure}
\begin{figure}
  \projectiontredici{0.5}
  \projectiontredici{0.55}
  \projectiontredici{0.6}
  \projectiontredici{0.65}
  \projectiontredici{0.7}
  \caption{The supports of solutions with $\w_2 \in \sqr{0.5, 0.7}$}
\end{figure}
\begin{figure}
  \projectiontredici{0.75}
  \projectiontredici{0.8}
  \projectiontredici{0.85}
  \projectiontredici{0.9}
  \projectiontredici{0.95}
  \caption{The supports of solutions with $\w_2 \in \sqr{0.75, 0.95}$}
\end{figure}

The range for witch the alternative $B$ is optimal $\w_2$ is obtainable using the intersection of the regions. When a green area exists in the intersection, $B$ is still optimal for some values of the weights.

\end{document}