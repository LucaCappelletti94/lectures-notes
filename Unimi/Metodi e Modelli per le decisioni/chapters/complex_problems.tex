\providecommand{\main}{..}
\documentclass[\main/main.tex]{subfiles}
\begin{document}

\section{Problemi complessi}

\begin{figure}[H]
	\[
		P = (X, \Omega, F, f, D, \Pi)
	\]
	\caption{Definizione formale di problema di decisione.}
\end{figure}

Queste variabili rappresentano:

\begin{enumerate}
	\item $X$ rappresenta l'insieme delle \textbf{alternative}, o delle \textbf{soluzioni} o anche delle \textbf{soluzioni ammissibili}.
	\item $\Omega$ rappresenta insieme degli \textbf{scenari} o \textbf{esiti}.
	\item $F$ rappresenta l'insieme degli \textbf{impatti}.
	\item $f$ rappresenta la \textbf{funzione dell'impatto}.
	\item $D$ rappresenta l'insieme dei \textbf{decisori}, tipicamente un insieme finito e di dimensione bassa. Un decisore è un'entità umana, modellata quanto possibile matematicamente.
	\item $\Pi$ insieme delle \textbf{preferenze}.
\end{enumerate}

$X$ viene definito come: \[X \subseteq \mathbb{R}^n \text{se } \bm{x} \in X \Rightarrow \bm{x} = \begin{bmatrix}x_1\\x_2\\...\\x_n \end{bmatrix}\]
con ogni termine $x_i$ viene chiamato o \textbf{elemento di alternativa} o \textbf{variabile di decisione}.

$\Omega$ viene definito come: \[\Omega \subseteq \mathbb{R}^r \text{se } \bm{\omega} \in \Omega \Rightarrow \bm{\omega} = \begin{bmatrix}\omega_1\\\omega_2\\...\\\omega_r \end{bmatrix}\]
con ogni termine $\omega_i$ viene chiamato o \textbf{elemento di scenario} o \textbf{variabile esogene}, cioè variabili che influiscono sulla configurazione del nostro sistema, non decise arbitrariamente ma provenienti dall'esterno.

$F$ viene definito come: \[F \subseteq \mathbb{R}^p \text{se } \bm{f} \in F \Rightarrow \bm{f} = \begin{bmatrix}f_1\\ f_2\\...\\ f_p \end{bmatrix}\]
Le $f_l \in \mathbb{R}$ vengono ipotizzate ad essere intere e vengono chiamate \textbf{indicatore}, \textbf{attributo}, \textbf{criterio} o \textbf{obbiettivo}. Un \textbf{indicatore} per esempio potrebbe essere un \textit{valore ottimo}.

La $f$ viene definita come: \[ f(\bm{x},\bm{\omega}): X\times\Omega \rightarrow F \]
La matrice di tutte le combinazioni viene chiamata \textbf{matrice delle valutazioni}.

La $\Pi$ viene definita come \[\Pi: D \rightarrow 2^{F\times F}\], dove $\pi_d \subseteq F\times F$. $F\times F$ rappresenta l'insieme delle \textbf{coppie ordinate di impatti}, mentre $2^{F \times F}$ rappresenta l'insieme delle \textbf{relazioni binarie}.

Per esempio, ponendo $F = \{f, f', f''\}$, otteniamo un prodotto cartesiano: \[F \times F = \{ (f,f'), (f,f''), (f', f), (f, f''), (f'', f), (f'', f'), (f, f), (f', f'), (f'', f'') \}\]

La \textbf{preferenza} è la volontà per cui il decisore risulta disponibile a fare uno scambio.

Un esempio di preferenza è: \[ f' \preccurlyeq_d f' \Leftrightarrow (f', f'')\in \Pi_d \]. In un ambiente ingegneristico si usa il $\preccurlyeq_d$, minimizzando i costi, mentre in un ambiente economico si cerca di massimizzare i costi $\succcurlyeq_d$.

\begin{definition}[indifferenza]
	Due preferenze $f'$ e $f''$ sono dette \textbf{indifferenti} quando:

	\[
		f' ~ f'' \Leftrightarrow \begin{cases} f' \preccurlyeq_d f'' \\ f' \succcurlyeq_d f'' \end{cases}
	\]
\end{definition}

\begin{definition}[Preferenza Stretta]
	Una preferenza $f'$ è detta \textbf{preferenza stretta} quando:

	\[
		f' <_d f'' \Leftrightarrow
		\begin{cases}
			f' \preccurlyeq_d f'' \\
			f' \nsucc_d f''
		\end{cases}
	\]
\end{definition}

\begin{definition}[Incomparabilità]
	Due preferenze $f'$ e $f''$ sono dette \textbf{incomparabili} quando:

	\[
		f' \Join_d f'' \Leftrightarrow
		\begin{cases}
			f' \nprec_d f'' \\
			f' \nsucc_d f''
		\end{cases}
	\]
\end{definition}


\section{Proprietà delle preferenze}
\subsubsection{Proprietà riflessiva}
\[
	f \preccurlyeq f \qquad \forall f \in F
\]

\subsubsection{Proprietà di completezza}
Un decisore può sempre concludere una decisione (ipotesi molto forte che talvolta porta a risultati impossibili):
\[
	f \nprec f' \Rightarrow f' \preccurlyeq f \qquad \forall f, f' \in F
\]

\subsubsection{Proprietà di anti-simmetria}
\[
	f \preccurlyeq f' \wedge f' \preccurlyeq f \Rightarrow f' = f \qquad \forall f, f' \in F
\]

\subsubsection{Proprietà Transitiva}
Solitamente i decisori non possiedono questa proprietà, anche perché è necessario modellare lo scorrere del tempo, per cui le proprietà valgono potenzialmente solo in un determinato periodo temporale. Viene generalmente considerata verificata.
\[
	f \preccurlyeq f' \wedge f' \preccurlyeq f'' \Rightarrow f \preccurlyeq f'' \qquad \forall f, f', f'' \in F
\]

\section{Ipotesi funzione del valore}
Un decisore che ha in mente una funzione valore $v$, ha in mente una relazione di preferenza $\Pi$ \textbf{riflessiva}, \textbf{completa}, \textbf{non necessariamente anti simmetrica} e \textbf{transitiva}. Quando una relazione possiede queste proprietà viene chiamata \textbf{ordine debole}, debole perché possono esiste dei \textit{pari merito}. Un campo di applicazione sono i campionati sportivi.
\[
	\exists v: F\rightarrow \mathbb{R}: f \preccurlyeq f' \Leftrightarrow v_{(f)} \succcurlyeq v_{(f')}
\]

\subsubsection{Condizioni di preordine}
Avendo le condizioni di \textbf{riflessività}, \textbf{transitività} si ottiene la condizione di \textbf{preordine}.

\subsubsection{Ordini deboli}
Avendo le condizioni di \textbf{riflessività}, \textbf{transitività} e \textbf{completezza} si ha la condizione di ordine debole, che è molto utilizzata.

\subsubsection{Ordine parziale}
Avendo le condizioni di \textbf{riflessività}, \textbf{transitività} e \textbf{antisimmetria} si ottiene la condizione di \textbf{ordine parziale}.

\subsubsection{Ordine totale}
Avendo le condizioni di \textbf{riflessività}, \textbf{transitività}, \textbf{completezza} e \textbf{antisimmetria} si ottiene la condizione di \textbf{ordine totale}.

\section{Tabella riassuntiva}
\begin{center}
	\begin{tabular}{|c|c|c|c|c|}
		\hline
		\textit{Proprietà}     & \textbf{Preordine} & \textbf{Ordine debole} & \textbf{Ordine parziale} & \textbf{Ordine totale} \\
		\hline
		\textbf{Riflessività}  & \checkmark         & \checkmark             & \checkmark               & \checkmark             \\
		\hline
		\textbf{Transitività}  & \checkmark         & \checkmark             & \checkmark               & \checkmark             \\
		\hline
		\textbf{Completezza}   &                    & \checkmark             &                          & \checkmark             \\
		\hline
		\textbf{Antisimmetria} &                    &                        & \checkmark               & \checkmark             \\
		\hline
	\end{tabular}
\end{center}

\section{Conto di Borda}
La formula in figura \ref{borda} utilizzato per costruire una \textbf{funzione valore}:

\begin{figure}[H]
	\[
		v(f) = \abs{\{f' \in F: f \preccurlyeq f' \}}
	\]
	\caption{Conto di Borda}
	\label{borda}
\end{figure}

Il valore di un impatto è pari al numero di impatti cui esso è preferibile, compreso l'impatto stesso.

Quando la cardinalità dell'insieme è $\mathbb{N}\times\mathbb{R}$ è possibile ottenere una \textbf{funzione valore}, ma quando ci si trova in condizioni come $\mathbb{R}\times\mathbb{R}$ che non risultano più mappabili sull'insieme $\mathbb{R}$ non risulta più possibile realizzare una \textbf{funzione valore}.

\section{Problemi semplici}
Un problema viene detto \textit{semplice} quando essi possiedono queste caratteristiche:

\begin{enumerate}
	\item $\exists v(f) $ conforme
	\item $\abs{\Omega} = 1 \Rightarrow f: X \rightarrow \mathbb{R}$, cioè esiste un $f(x)$
	\item $\abs{D} = 1$
	\item $X = \{x \in \mathbb{R}^n: g_j (x) \leq 0 \forall j = 1,..., n \} \text{ con } g_j \in C^1(\mathbb{R}^n)$
\end{enumerate}
\end{document}