\providecommand{\main}{..}
\documentclass[\main/main.tex]{subfiles}
\begin{document}

\chapter{Decisioni di gruppo}
Nelle decisioni di gruppo il numero di decisori è maggiore di uno. Pertanto, oltre ai meccanismi di selezione delle decisioni visti per i casi a singolo decisore, vanno considerati dei metodi per unire nel modo migliore le scelte dei vari decisori.

\begin{definition}[Costituzione]
  Si dice \textbf{costituzione} o \textbf{funzione di benessere collettivo} una funzione che associa ad ogni $\abs{\mathcal{D}}-upla$ di ordini deboli su un insieme di soluzioni dato un ordine debole (di gruppo) sullo stesso insieme:
  \[
    g: \mathcal{D}(X)^{\abs{\mathcal{D}}} \rightarrow \mathcal{D}(X)
  \]
\end{definition}

\section{Metodo di Condorcet o delle maggioranze}
Una soluzione $x_1$ è preferibile ad una seconda $x_2$ se il gruppo dei decisori che preferisce $x_1$ a $x_2$ è maggiore del gruppo che preferisce $x_2$ a $x_1$.

Questo metodo è soggetto ad un caso limite dove non può costruire un ordine debole, cioè quando le soluzioni sono espresse in modo tale per cui i gruppi creano un ciclo di preferenze.

\begin{table}
  \begin{tabular}{|L|L|L|L|L|}
    \hline
    Ordine & d_1 & d_2 & d_3 & d_4 \\
    \hline
    1      & a   & b   & c   & d   \\
    \hline
    2      & b   & c   & d   & c   \\
    \hline
    3      & c   & d   & a   & a   \\
    \hline
    4      & d   & a   & b   & b   \\
    \hline
  \end{tabular}
  \caption{Preferenze circolari: caso in cui Condorcet fallisce}
\end{table}

\section{Metodo di Borda}
Assegna un punteggio ad una soluzione in base alla posizione che essa assume sulla lista di preferenze di ogni decisore, somma questi punteggi e va a scegliere quella soluzione che ha il punteggio più alto.

Talvolta è soggetto al \textbf{rank reversal}, cioè eliminando una soluzione (anche mediocre) soluzioni precedentemente non ottime lo diventano.

\section{Metodo lessicografico}
Il primo decisore (trattato come monarca o capo) sceglie per tutti la preferenza tra due soluzioni, e solo se è per lui è indifferente si passa al secondo decisore etc..

\section{Sistema di Pluralità}
Si sceglie la soluzione più condivisa come ottima tra i decisori. Qualora vi fosse un pareggio, si elimina la soluzione meno condivisa e si ripete il conteggio, sino ad arrivare ad una maggioranza.

Questo metodo permette però a soluzioni fortemente gradite da una minoranza di prevalere qualora piacesse sufficientemente a tutti gli altri.

\section{Approccio assiomatico}
Arrow, definendo tramite assiomi delle proprietà che sono desiderabili in un sistema di voto democratico dimostra che non è possibile realizzare un metodo di voto che goda di tutti gli assiomi contemporaneamente.

\end{document}