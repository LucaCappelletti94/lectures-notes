\providecommand{\main}{..}
\documentclass[\main/main.tex]{subfiles}
\begin{document}

\chapter{Influencer}

\begin{figure}
	\includegraphics[width=0.6\textwidth]{influencer.jpg}
	\caption{Influencer}
\end{figure}

\section{La necessità di modellare gli influencer}
Nel mondo iper-connesso odierno, non è più possibile pensare ad una persona come un decisore indipendente da altri, ma è necessario considerare come la decisione di altri vada a modificare quella di una determinata persona.

\section{Definizione di influencer}
Un \textbf{influencer} è un decisore $d_i \in D$ in grado di modificare la decisione di almeno un altro decisore $d_j \in D$ in merito ad almeno un impatto $f$.

\section{Funzione di influenza}
La funzione di influenza $\Psi$ (formula \ref{influence_function}) è definita come l'influenza che il decisore $i-esimo$ ha sul decisore $j-esimo$ in merito all'impatto $k-esimo$.

\begin{figure}
	\[
		\Psi(d_i, d_j, f_k): D \times D \times F \rightarrow [0,1]
	\]
	\caption{La funzione di influenza $\Psi$}
	\label{influence_function}
\end{figure}

La somma di tutti le influenze che un decisore $d_j$ può ricevere relativamente ad un impatto $f_k$, cioè la somma del suo \textbf{vettore di influenze} (figura \ref{influence_vector}), è esattamente $1$ (formula \ref{influence_function_column_limit}). Se non vi sono influencer attivi su $d_j$, il decisore $d_i$ che influenza $d_j$ corrisponde con $d_j$.

\begin{figure}
	\begin{subfigure}{0.45\textwidth}
		\[
			\bm{\Psi}_{jk} = \begin{bmatrix}
				\Psi(d_1,d_j,f_k)         \\
				\vdots                    \\
				\Psi(d_i,d_j,f_k)         \\
				\vdots                    \\
				\Psi(d_{\abs{D}},d_j,f_k) \\
			\end{bmatrix}
		\]
		\caption{Vettore di influenze $\bm{\Psi}_{jk}$}
		\label{influence_vector}
	\end{subfigure}
	\begin{subfigure}{0.45\textwidth}
		\[
			\sum_{j=1}^{\abs{D}} \Psi(d_i, d_j, f_k)= 1 \quad \forall d_i \in D, \forall f_k \in F
		\]
		\caption{Vincolo del vettore di influenze $\bm{\Psi}_{jk}$}
		\label{influence_function_column_limit}
	\end{subfigure}
	\caption{Vettore di influenze e vincolo del vettore di influenze}
\end{figure}

La somma di tutte le influenze in una matrice di influenza su impatto $k-esimo$ (figura \ref{matrice_influenza_k}) deve essere pari al numero di decisori $\abs{D}$ (formula \ref{influence_function_matrix_k_limit}).

\begin{figure}
	\[
		\bm{\Psi}_{k} = \begin{bmatrix}
			\Psi(d_1,d_1,f_k)          & \Psi(d_1,d_2,f_k)          & \ldots & \Psi(d_1,d_{\abs{D}},f_k))          \\
			\Psi(d_2,d_1,f_k)          & \Psi(d_2,d_2,f_k)          & \ldots & \Psi(d_2,d_{\abs{D}},f_k))          \\
			\vdots                     & \vdots                     & \ddots & \vdots                              \\
			\Psi(d_{\abs{D}}, d_1,f_k) & \Psi(d_{\abs{D}}, d_2,f_k) & \ldots & \Psi(d_{\abs{D}}, d_{\abs{D}}, f_k)
		\end{bmatrix}
	\]
	\caption{Matrice di influenza su impatto $k-esimo$ $\bm{\Psi}_k$}
	\label{matrice_influenza_k}
\end{figure}

\begin{figure}
	\[
		\sum_{j=1}^{\abs{D}}\sum_{i=1}^{\abs{D}} \Psi(d_i, d_j, f_k) = \abs{D} \quad \forall f_k \in F
	\]
	\caption{Vincolo per matrice di influenza su impatto k-esimo $\bm{\Psi}_k$}
	\label{influence_function_matrix_k_limit}
\end{figure}

\section{Matrice di influenza}
Con matrice di influenza $\bm{\Psi}$ (figura \ref{influence_matrix}) si intende una matrice 3-dimensionale $D \times D \times F$ in cui il valore $\psi_{ijk}$ è definito come il valore della funzione di influenza $\Psi$ per quei decisori e impatto.

\begin{figure}
	\begin{tikzpicture}[every node/.style={anchor=north east,fill=white,minimum width=1.4cm,minimum height=7mm}]
		\matrix (mA) [draw,matrix of math nodes]
		{
			\Psi(d_1,d_1,f_{\abs{F}}) & \Psi(d_1,d_2,f_{\abs{F}}) & \ldots & \Psi(d_1,d_{\abs{D}},f_{\abs{F}})) \\
			\Psi(d_2,d_1,f_{\abs{F}}) & \Psi(d_2,d_2,f_{\abs{F}}) & \ldots & \Psi(d_2,d_{\abs{D}},f_{\abs{F}})) \\
			\vdots                 & \vdots                 & \ddots & \vdots                  \\
			\Psi(d_{\abs{D}}, d_1,f_{\abs{F}}) & \Psi(d_{\abs{D}}, d_2,f_{\abs{F}}) & \ldots & \Psi(d_{\abs{D}}, d_{\abs{D}},f_{\abs{F}})  \\
		};

		\matrix (mB) [draw,matrix of math nodes] at ($(mA.south west)+(5,1.5)$)
		{
			\Psi(d_1,d_1,f_k) & \Psi(d_1,d_2,f_k) & \ldots & \Psi(d_1,d_{\abs{D}},f_k)) \\
			\Psi(d_2,d_1,f_k) & \Psi(d_2,d_2,f_k) & \ldots & \Psi(d_2,d_{\abs{D}},f_k)) \\
			\vdots                 & \vdots                 & \ddots & \vdots                  \\
			\Psi(d_{\abs{D}}, d_1,f_k) & \Psi(d_{\abs{D}}, d_2,f_k) & \ldots & \Psi(d_{\abs{D}}, d_{\abs{D}},f_k)  \\
		};

		\matrix (mC) [draw,matrix of math nodes] at ($(mB.south west)+(5,1.5)$)
		{
			\Psi(d_1,d_1,f_1) & \Psi(d_1,d_2,f_1) & \ldots & \Psi(d_1,d_{\abs{D}},f_1)) \\
			\Psi(d_2,d_1,f_1) & \Psi(d_2,d_2,f_1) & \ldots & \Psi(d_2,d_{\abs{D}},f_1)) \\
			\vdots                 & \vdots                 & \ddots & \vdots                  \\
			\Psi(d_{\abs{D}}, d_1,f_1) & \Psi(d_{\abs{D}}, d_2,f_1) & \ldots & \Psi(d_{\abs{D}}, d_{\abs{D}},f_1)  \\
		};

		\draw[dashed](mA.north east)--(mC.north east);
		\draw[dashed](mA.north west)--(mC.north west);
		\draw[dashed](mA.south east)--(mC.south east);
	\end{tikzpicture}
	\caption{Matrice di Influenza $\bm{\Psi}$}
	\label{influence_matrix}
\end{figure}

\subsection{Casi banali}

\subsubsection{Matrice di influenza in assenza di influencer}
La matrice di influenza in una condizione di assenza di influencers (figura \ref{no_influencer_influence_matrix}) è composta da $\abs{F}$ matrici \textbf{identità}, in cui ogni decisore è convinto solamente della propria decisione.

\begin{figure}
	\begin{tikzpicture}[every node/.style={anchor=north east,fill=white,minimum width=1.4cm,minimum height=7mm}]
		\matrix (mA) [draw,matrix of math nodes]
		{
			1      & 0      & \ldots & 0      \\
			0      & 1      & \ldots & 0      \\
			\vdots & \vdots & \ddots & \vdots \\
			0      & 0      & \ldots & 1      \\
		};

		\matrix (mB) [draw,matrix of math nodes] at ($(mA.south west)+(2,1.5)$)
		{
			1      & 0      & \ldots & 0      \\
			0      & 1      & \ldots & 0      \\
			\vdots & \vdots & \ddots & \vdots \\
			0      & 0      & \ldots & 1      \\
		};

		\matrix (mC) [draw,matrix of math nodes] at ($(mB.south west)+(2,1.5)$)
		{
			1      & 0      & \ldots & 0      \\
			0      & 1      & \ldots & 0      \\
			\vdots & \vdots & \ddots & \vdots \\
			0      & 0      & \ldots & 1      \\
		};

		\draw[dashed](mA.north east)--(mC.north east);
		\draw[dashed](mA.north west)--(mC.north west);
		\draw[dashed](mA.south east)--(mC.south east);
	\end{tikzpicture}
	\caption{Matrice di influenza in assenza di influencer}
	\label{no_influencer_influence_matrix}
\end{figure}

\subsubsection{Matrice di influenza in condizioni di dittatura}
In condizioni di dittatura la matrice di influenza (figura \ref{dictator_influence_matrix}) sarà composto di soli $1$ quando $d_i$ è il dittatore, di soli $0$ altrimenti.

\begin{figure}
	\begin{tikzpicture}[every node/.style={anchor=north east,fill=white,minimum width=1.4cm,minimum height=7mm}]
		\matrix (mA) [draw,matrix of math nodes]
		{
			0      & 0      & \ldots & 0      \\
			0      & 0      & \ldots & 0      \\
			\vdots & \vdots & \vdots & \vdots \\
			1      & 1      & \ldots & 1      \\
			\vdots & \vdots & \ddots & \vdots \\
			0      & 0      & \ldots & 0      \\
		};

		\matrix (mB) [draw,matrix of math nodes] at ($(mA.south west)+(2,2.5)$)
		{
			0      & 0      & \ldots & 0      \\
			0      & 0      & \ldots & 0      \\
			\vdots & \vdots & \vdots & \vdots \\
			1      & 1      & \ldots & 1      \\
			\vdots & \vdots & \ddots & \vdots \\
			0      & 0      & \ldots & 0      \\
		};

		\matrix (mC) [draw,matrix of math nodes] at ($(mB.south west)+(2,2.5)$)
		{
			0      & 0      & \ldots & 0      \\
			0      & 0      & \ldots & 0      \\
			\vdots & \vdots & \vdots & \vdots \\
			1      & 1      & \ldots & 1      \\
			\vdots & \vdots & \ddots & \vdots \\
			0      & 0      & \ldots & 0      \\
		};

		\draw[dashed](mA.north east)--(mC.north east);
		\draw[dashed](mA.north west)--(mC.north west);
		\draw[dashed](mA.south east)--(mC.south east);
	\end{tikzpicture}
	\caption{Matrice di Influenza in condizioni di dittatura}
	\label{dictator_influence_matrix}
\end{figure}

\subsubsection{Matrice di influenza in condizioni di equa oligarchia}
In condizioni di oligarchia la matrice di influenza (figura \ref{oligarchy_influence_matrix}) avrà il gruppo di decisori (nel caso rappresentato 4) a capo dell'oligarchia che si suddividono equamente l'influenza nei confronti degli altri e non si influenzano a vicenda.

\begin{figure}
	\begin{tikzpicture}[every node/.style={anchor=north east,fill=white,minimum width=1.4cm,minimum height=7mm}]
		\matrix (mA) [draw,matrix of math nodes]
		{
			0      & \ldots & 0      & 0      & 0      & 0      & \ldots & 0      \\
			\vdots & \vdots & \vdots & \vdots & \vdots & \vdots & \vdots & \vdots \\
			0.25   & \ldots & 0      & 0      & 0      & 1      & \ldots & 0.25   \\
			0.25   & \ldots & 0      & 0      & 1      & 0      & \ldots & 0.25   \\
			0.25   & \ldots & 0      & 1      & 0      & 0      & \ldots & 0.25   \\
			0.25   & \ldots & 1      & 0      & 0      & 0      & \ldots & 0.25   \\
			\vdots & \vdots & \vdots & \vdots & \vdots & \vdots & \ddots & \vdots \\
			0      & \ldots & 0      & 0      & 0      & 0      & \ldots & 0      \\
		};

		\matrix (mB) [draw,matrix of math nodes] at ($(mA.south west)+(9,5.5)$)
		{
			0      & \ldots & 0      & 0      & 0      & 0      & \ldots & 0      \\
			\vdots & \vdots & \vdots & \vdots & \vdots & \vdots & \vdots & \vdots \\
			0.25   & \ldots & 0      & 0      & 0      & 1      & \ldots & 0.25   \\
			0.25   & \ldots & 0      & 0      & 1      & 0      & \ldots & 0.25   \\
			0.25   & \ldots & 0      & 1      & 0      & 0      & \ldots & 0.25   \\
			0.25   & \ldots & 1      & 0      & 0      & 0      & \ldots & 0.25   \\
			\vdots & \vdots & \vdots & \vdots & \vdots & \vdots & \ddots & \vdots \\
			0      & \ldots & 0      & 0      & 0      & 0      & \ldots & 0      \\
		};

		\matrix (mC) [draw,matrix of math nodes] at ($(mB.south west)+(9,5.5)$)
		{
			0      & \ldots & 0      & 0      & 0      & 0      & \ldots & 0      \\
			\vdots & \vdots & \vdots & \vdots & \vdots & \vdots & \vdots & \vdots \\
			0.25   & \ldots & 0      & 0      & 0      & 1      & \ldots & 0.25   \\
			0.25   & \ldots & 0      & 0      & 1      & 0      & \ldots & 0.25   \\
			0.25   & \ldots & 0      & 1      & 0      & 0      & \ldots & 0.25   \\
			0.25   & \ldots & 1      & 0      & 0      & 0      & \ldots & 0.25   \\
			\vdots & \vdots & \vdots & \vdots & \vdots & \vdots & \ddots & \vdots \\
			0      & \ldots & 0      & 0      & 0      & 0      & \ldots & 0      \\
		};

		\draw[dashed](mA.north east)--(mC.north east);
		\draw[dashed](mA.north west)--(mC.north west);
		\draw[dashed](mA.south east)--(mC.south east);
	\end{tikzpicture}
	\caption{Matrice di Influenza in condizioni di equa oligarchia}
	\label{oligarchy_influence_matrix}
\end{figure}

\clearpage

\subsection{Matrice di influenza del decisore}
La \textbf{matrice di influenza del decisore} è il taglio della matrice di influenze fissato un decisore $d_j$.

\begin{table}
	\begin{tabular}{|c|c|c|c|c|c|}
		\hline
		              & $d_1$                         & $\ldots$ & $d_i$                         & $\ldots$ & $d_{\abs{D}}$                         \\
		\hline
		$f_1$         & $\Psi(d_1, d_j, f_1)$         & $\ldots$ & $\Psi(d_i, d_j, f_1)$         & $\ldots$ & $\Psi(d_{\abs{D}}, d_j, f_1)$         \\
		\hline
		$\vdots$      & $\vdots$                      & $\ddots$ & $\vdots$                      & $\vdots$ & $\vdots$                              \\
		\hline
		$f_k$         & $\Psi(d_1, d_j, f_k)$         & \ldots   & $\Psi(d_i, d_j, f_k)$         & $\ldots$ & $\Psi(d_{\abs{D}}, d_j, f_k)$         \\
		\hline
		$\vdots$      & $\vdots$                      & $\vdots$ & $\vdots$                      & $\ddots$ & $\vdots$                              \\
		\hline
		$f_{\abs{F}}$ & $\Psi(d_1, d_j, f_{\abs{F}})$ & \ldots   & $\Psi(d_i, d_j, f_{\abs{F}})$ & $\ldots$ & $\Psi(d_{\abs{D}}, d_j, f_{\abs{F}})$ \\
		\hline
	\end{tabular}
	\caption{Matrice di influenza del decisore (sulle righe gli impatti, sulle colonne gli influencer)}
\end{table}

\subsubsection{Caso di assenza di influencer}
Nel caso di assenza di influencer, i decisori influenzano unicamente se stessi:

\[
	\begin{cases}
		\Psi(d_j, d_i, f_k) = 1                \\
		\Psi(d_i, d_j, f_k) = 0 \quad i \neq j \\
	\end{cases}
	\forall f_k \in F
\]

\begin{table}
	\begin{tabular}{|c|c|c|c|c|c|}
		\hline
		              & $d_1$    & $\ldots$ & $d_{i=j}$ & $\ldots$ & $d_{\abs{D}}$ \\
		\hline
		$f_1$         & 0        & $\ldots$ & 1         & $\ldots$ & 0             \\
		\hline
		$\vdots$      & $\vdots$ & $\ddots$ & $\vdots$  & $\vdots$ & $\vdots$      \\
		\hline
		$f_k$         & 0        & \ldots   & 1         & $\ldots$ & 0             \\
		\hline
		$\vdots$      & $\vdots$ & $\vdots$ & $\vdots$  & $\ddots$ & $\vdots$      \\
		\hline
		$f_{\abs{F}}$ & 0        & \ldots   & 1         & $\ldots$ & 0             \\
		\hline
	\end{tabular}
	\caption{Caso di assenza di infuencer}
\end{table}

\subsubsection{Caso di dittatura}
Nel caso di dittatura, i decisori sono completamente influenzati dal dittatore $d_d$:

\[
	\begin{cases}
		\Psi(d_d, d_i, f_k) = 1                \\
		\Psi(d_i, d_j, f_k) = 0 \quad i \neq d \\
	\end{cases}
	\forall f_k \in F
\]

\begin{table}
	\begin{tabular}{|c|c|c|c|c|c|}
		\hline
		              & $d_1$    & $\ldots$ & $d_{i=d}$ & $\ldots$ & $d_{\abs{D}}$ \\
		\hline
		$f_1$         & 0        & $\ldots$ & 1         & $\ldots$ & 0             \\
		\hline
		$\vdots$      & $\vdots$ & $\ddots$ & $\vdots$  & $\vdots$ & $\vdots$      \\
		\hline
		$f_k$         & 0        & \ldots   & 1         & $\ldots$ & 0             \\
		\hline
		$\vdots$      & $\vdots$ & $\vdots$ & $\vdots$  & $\ddots$ & $\vdots$      \\
		\hline
		$f_{\abs{F}}$ & 0        & \ldots   & 1         & $\ldots$ & 0             \\
		\hline
	\end{tabular}
	\caption{Caso di dittatura}
\end{table}


\subsection{Matrice di influenza dell'influencer}
La \textbf{matrice di influenza dell'influncer} è il taglio della matrice di influenze fissato un decisore influencer $d_i$.

\begin{table}
	\begin{tabular}{|c|c|c|c|c|c|}
		\hline
		              & $d_1$                         & $\ldots$ & $d_j$                         & $\ldots$ & $d_{\abs{D}}$                         \\
		\hline
		$f_1$         & $\Psi(d_i, d_1, f_1)$         & $\ldots$ & $\Psi(d_i, d_j, f_1)$         & $\ldots$ & $\Psi(d_i, d_{\abs{D}}, f_1)$         \\
		\hline
		$\vdots$      & $\vdots$                      & $\ddots$ & $\vdots$                      & $\vdots$ & $\vdots$                              \\
		\hline
		$f_k$         & $\Psi(d_i, d_1, f_k)$         & \ldots   & $\Psi(d_i, d_j, f_k)$         & $\ldots$ & $\Psi(d_i, d_{\abs{D}}, f_k)$         \\
		\hline
		$\vdots$      & $\vdots$                      & $\vdots$ & $\vdots$                      & $\ddots$ & $\vdots$                              \\
		\hline
		$f_{\abs{F}}$ & $\Psi(d_i, d_1, f_{\abs{F}})$ & \ldots   & $\Psi(d_i, d_j, f_{\abs{F}})$ & $\ldots$ & $\Psi(d_i, d_{\abs{D}}, f_{\abs{F}})$ \\
		\hline
	\end{tabular}
	\caption{Matrice di influenza dell'influencer (sulle righe gli impatti, sulle colonne i decisori)}
\end{table}

\subsubsection{Caso di assenza di influencer}
Nel caso di assenza di influencer, gli ``influencer'' influenzano unicamente se stessi:

\[
	\begin{cases}
		\Psi(d_i, d_i, f_k) = 1                \\
		\Psi(d_i, d_j, f_k) = 0 \quad i \neq j \\
	\end{cases}
	\forall f_k \in F
\]

\begin{table}
	\begin{tabular}{|c|c|c|c|c|c|}
		\hline
		              & $d_1$    & $\ldots$ & $d_{i=j}$ & $\ldots$ & $d_{\abs{D}}$ \\
		\hline
		$f_1$         & 0        & $\ldots$ & 1         & $\ldots$ & 0             \\
		\hline
		$\vdots$      & $\vdots$ & $\ddots$ & $\vdots$  & $\vdots$ & $\vdots$      \\
		\hline
		$f_k$         & 0        & \ldots   & 1         & $\ldots$ & 0             \\
		\hline
		$\vdots$      & $\vdots$ & $\vdots$ & $\vdots$  & $\ddots$ & $\vdots$      \\
		\hline
		$f_{\abs{F}}$ & 0        & \ldots   & 1         & $\ldots$ & 0             \\
		\hline
	\end{tabular}
	\caption{Caso di assenza di infuencer}
\end{table}

\subsubsection{Caso di dittatura}
Nel caso di dittatura, l'unico influencer che può influenzare è $d_d$:

\[
	\begin{cases}
		\Psi(d_d, d_i, f_k) = 1                \\
		\Psi(d_i, d_j, f_k) = 0 \quad i \neq d \\
	\end{cases}
	\forall f_k \in F
\]

Scegliendo come influencer $d_d$ la matrice è formata da soli $1$, mentre scegliendo come influencer $d_{i \neq d}$ la matrice è formata da soli $0$.

\begin{figure}
	\begin{subfigure}{0.48\textwidth}
		\begin{table}
			\begin{tabular}{|c|c|c|c|c|c|}
				\hline
				              & $d_1$    & $\ldots$ & $d_j$    & $\ldots$ & $d_{\abs{D}}$ \\
				\hline
				$f_1$         & 1        & $\ldots$ & 1        & $\ldots$ & 1             \\
				\hline
				$\vdots$      & $\vdots$ & $\ddots$ & $\vdots$ & $\vdots$ & $\vdots$      \\
				\hline
				$f_k$         & 1        & \ldots   & 1        & $\ldots$ & 1             \\
				\hline
				$\vdots$      & $\vdots$ & $\vdots$ & $\vdots$ & $\ddots$ & $\vdots$      \\
				\hline
				$f_{\abs{F}}$ & 1        & \ldots   & 1        & $\ldots$ & 1             \\
				\hline
			\end{tabular}
			\caption{Scegliendo come influencer $d_d$}
		\end{table}
	\end{subfigure}
	\begin{subfigure}{0.48\textwidth}
		\begin{table}
			\begin{tabular}{|c|c|c|c|c|c|}
				\hline
				              & $d_1$    & $\ldots$ & $d_j$    & $\ldots$ & $d_{\abs{D}}$ \\
				\hline
				$f_1$         & 0        & $\ldots$ & 0        & $\ldots$ & 0             \\
				\hline
				$\vdots$      & $\vdots$ & $\ddots$ & $\vdots$ & $\vdots$ & $\vdots$      \\
				\hline
				$f_k$         & 0        & \ldots   & 0        & $\ldots$ & 0             \\
				\hline
				$\vdots$      & $\vdots$ & $\vdots$ & $\vdots$ & $\ddots$ & $\vdots$      \\
				\hline
				$f_{\abs{F}}$ & 0        & \ldots   & 0        & $\ldots$ & 0             \\
				\hline
			\end{tabular}
			\caption{Scegliendo come influencer $d_{i \neq d}$}
		\end{table}
	\end{subfigure}
	\caption{Caso di dittatura}
\end{figure}


\subsection{Matrice di influenza dell'impatto}
La \textbf{matrice di influenza dell'impatto} è il taglio della matrice di influenze fissato un impatto $f_k$.

\begin{table}
	\begin{tabular}{|c|c|c|c|c|c|}
		\hline
		              & $d_1$                         & $\ldots$ & $d_i$                         & $\ldots$ & $d_{\abs{D}}$                         \\
		\hline
		$d_1$         & $\Psi(d_1, d_1, f_k)$         & $\ldots$ & $\Psi(d_i, d_1, f_k)$         & $\ldots$ & $\Psi(d_{\abs{D}}, d_1, f_k)$         \\
		\hline
		$\vdots$      & $\vdots$                      & $\ddots$ & $\vdots$                      & $\vdots$ & $\vdots$                              \\
		\hline
		$d_j$         & $\Psi(d_1, d_j, f_k)$         & \ldots   & $\Psi(d_i, d_j, f_k)$         & $\ldots$ & $\Psi(d_{\abs{D}}, d_j, f_k)$         \\
		\hline
		$\vdots$      & $\vdots$                      & $\vdots$ & $\vdots$                      & $\ddots$ & $\vdots$                              \\
		\hline
		$d_{\abs{D}}$ & $\Psi(d_1, d_{\abs{D}}, f_k)$ & \ldots   & $\Psi(d_i, d_{\abs{D}}, f_k)$ & $\ldots$ & $\Psi(d_{\abs{D}}, d_{\abs{D}}, f_k)$ \\
		\hline
	\end{tabular}
	\caption{Matrice di influenza dell'impatto (sulle righe i decisori, sulle colonne gli influencer)}
\end{table}

\clearpage

\section{Preferenza influenzata}
Ora che abbiamo definito come andiamo a modellare l'influenza, procediamo a definire la \textbf{preferenza influenzata} e quali strumenti essa possa andare a fornire.

Data una preferenza $\Pi$ che forma un \textbf{ordine debole} ed una funzione valore $v$ che assegna un valore ad ogni \textbf{impatto} o \textbf{soluzione}, andiamo a formare una matrice dei valori $V$ che rappresenta l'immagine della funzione $v(\Pi)$:

\begin{table}
	\begin{tabular}{|c|c|c|c|c|}
		\hline
		$d_1$    & $\ldots$ & $d_j$    & $\ldots$ & $d_{\abs{D}}$ \\
		\hline
		$f_1$    & $\ldots$ & $f_2$    & $\ldots$ & $f_5$         \\
		\hline
		$\vdots$ & $\ddots$ & $\vdots$ & $\vdots$ & $\vdots$      \\
		\hline
		$f_3$    & \ldots   & $f_1$    & $\ldots$ & $f_7$         \\
		\hline
		$\vdots$ & $\vdots$ & $\vdots$ & $\ddots$ & $\vdots$      \\
		\hline
		$f_7$    & \ldots   & $f_4$    & $\ldots$ & $f_3$         \\
		\hline
	\end{tabular}
	\caption{Preferenza $\Pi$, ogni decisore esprime il proprio ordinamento di $f_k \in F$}
\end{table}

\begin{table}
	\begin{tabular}{|c|c|c|c|c|c|}
		\hline
		              & $d_1$                      & $\ldots$ & $d_j$                      & $\ldots$ & $d_{\abs{D}}$                      \\
		\hline
		$f_1$         & $v(\Pi_{f_1 d_1})$         & $\ldots$ & $v(\Pi_{f_1 d_j})$         & $\ldots$ & $v(\Pi_{f_1 d_{\abs{D}}})$         \\
		\hline
		$\vdots$      & $\vdots$                   & $\ddots$ & $\vdots$                   & $\vdots$ & $\vdots$                           \\
		\hline
		$f_k$         & $v(\Pi_{f_k d_1})$         & \ldots   & $v(\Pi_{f_k d_j})$         & $\ldots$ & $v(\Pi_{f_k d_{\abs{D}}})$         \\
		\hline
		$\vdots$      & $\vdots$                   & $\vdots$ & $\vdots$                   & $\ddots$ & $\vdots$                           \\
		\hline
		$f_{\abs{F}}$ & $v(\Pi_{f_{\abs{F}} d_1})$ & \ldots   & $v(\Pi_{f_{\abs{F}} d_j})$ & $\ldots$ & $v(\Pi_{f_{\abs{F}} d_{\abs{D}}})$ \\
		\hline
	\end{tabular}
	\caption{Matrice dei valori $V$}
\end{table}

\begin{figure}
	\begin{subfigure}{0.45\textwidth}
		\[
			V_{d_{j}f_{k}} = v(\Pi_{d_{j}f_{k}})
		\]
		\caption{Matrice di valori}
	\end{subfigure}
	\begin{subfigure}{0.45\textwidth}
		\[
			v(\Pi_{d_{j}f_{k}}) = \text{posizione di }f_{k}\text{ nel vettore }\Pi_{d_{j}}
		\]
		\caption{Esempio di funzione valore}
	\end{subfigure}
\end{figure}

Ora procediamo a calcolare una matrice dei valori normalizzata per ogni decisore.

TODO HERE

\subsection{Esempio}
Abbiamo 4 decisori $D = \{d_1, d_2, d_3, d_4 \}$ che devono stabilire una soluzione tra 4 che andremo a rappresentare, per massima comprensibilità, con dei colori $X = \{ \text{red}, \text{green}, \text{blue}, \text{yellow}\}$.

Ogni decisore ha già stabilito l'ordine delle soluzioni.

\begin{table}
	\begin{tabular}{|c|c|c|c|}
		\hline
		\rowcolor{gray!10} $d_1$ & $d_2$                   & $d_3$                   & $d_4$                   \\
		\hline
		\cellcolor{red!50} r     & \cellcolor{green!50} g  & \cellcolor{red!50} r    & \cellcolor{yellow!50} y \\
		\hline
		\cellcolor{green!50} g   & \cellcolor{red!50} r    & \cellcolor{green!50} g  & \cellcolor{green!50} g  \\
		\hline
		\cellcolor{blue!50} b    & \cellcolor{blue!50} b   & \cellcolor{yellow!50} y & \cellcolor{red!50} r    \\
		\hline
		\cellcolor{yellow!50} y  & \cellcolor{yellow!50} y & \cellcolor{blue!50} b   & \cellcolor{blue!50} b   \\
		\hline
	\end{tabular}
	\caption{Ordinamento delle soluzioni}
	\label{not_influenced_example}
\end{table}

Procedendo tramite il \textbf{sistema di pluralità}, che conta i decisori che vedono una decisione buona, si ottiene:

\[
	V(x) = \abs{\{d \in D: x \preceq x' \forall x' \in X \}}
\]

\[
	\Pi_D(x): x \preceq_D x' \Leftrightarrow V_D(x) \geq V_D(x')
\]

\begin{table}
	\begin{tabular}{|c|c|}
		\hline
		$x$ & V(x) \\
		\hline
		r   & 2    \\
		\hline
		g   & 1    \\
		\hline
		b   & 0    \\
		\hline
		y   & 1    \\
		\hline
	\end{tabular}
	\caption{Ordinamento delle soluzioni}
\end{table}

\subsubsection{Identificazione di influenza}
Un possibile caso di utilizzo della preferenza influenzata è \textbf{l'identificazione dell'influenza}: supponiamo di conoscere la posizione ufficiale dei decisori \textit{prima} che questi la ufficializzino, e immaginiamo che essa coincida con quanto visto nella figura \ref{not_influenced_example}. Ora, supponiamo che la posizione ufficiale divenga:

\begin{table}
	\begin{tabular}{|c|c|c|c|}
		\hline
		\rowcolor{gray!10} $d_1$ & $d_2$                   & $d_3$                   & $d_4$                   \\
		\hline
		\cellcolor{green!50} g   & \cellcolor{green!50} g  & \cellcolor{green!50} g  & \cellcolor{yellow!50} y \\
		\hline
		\cellcolor{red!50} r     & \cellcolor{red!50} r    & \cellcolor{red!50} r    & \cellcolor{green!50} g  \\
		\hline
		\cellcolor{blue!50} b    & \cellcolor{blue!50} b   & \cellcolor{yellow!50} y & \cellcolor{red!50} r    \\
		\hline
		\cellcolor{yellow!50} y  & \cellcolor{yellow!50} y & \cellcolor{blue!50} b   & \cellcolor{blue!50} b   \\
		\hline
	\end{tabular}
	\label{influenced_example}
\end{table}

È possibile a ritroso capire come i decisori sono stati influenzati in questa decisione.

Andiamo a costruire per ogni decisore la \textbf{matrice delle preferenze} prima ($\Pi$) e dopo ($\Pi'$) l'ufficializzazione, definendo come nostra funzione valore la posizione della soluzione nella scala (0 il minimo e 3 il massimo, avendo 4 soluzioni) e come funzione di preferenza tra due soluzioni la differenza tra le due funzioni valore.

\clearpage

\section{Proprietà di influenza}
\subsection{Decisore totalmente influenzato}
Quando un decisore $d_j$ ha un valore di influenza nei confronti di sè stesso pari a 0 (Formula \ref{decisore_influenzato}).

\begin{figure}
	\[
		\Psi(d_j, d_j, f_k) = 0
	\]
	\caption{Decisore completamente influenzato}
	\label{decisore_influenzato}
\end{figure}

\subsection{Decisore non influenzato}
Quando è noto che un decisore $d_j$ non è influenzato vale che (Formula \ref{decisore_non_influenzato}):

\begin{figure}
	\[
		\begin{cases}
			\Psi(d_{ni}, d_{ni}, f_k) = 1             \\
			\Psi(d_i, d_{ni}, f_k) = 0 \quad i\neq ni \\
			\Psi(d_{ni}, d_j, f_k) \geq 0
		\end{cases}
		\qquad \forall f_k \in F
	\]
	\caption{Decisore non influenzato}
	\label{decisore_non_influenzato}
\end{figure}

\subsection{Determinabilità dell'influenza}
La matrice di influenza $\bm{\Psi}$ è calcolabile nel caso in cui valga la disequazione \ref{influenza_determinabile}:


\begin{figure}
	\[
		\abs{F} \geq \abs{D}
	\]
	\caption{Vincolo di determinabilità}
	\label{influenza_determinabile}
\end{figure}



\subsubsection{Approssimazione dell'influenza in casi d'indeterminabilità}

\end{document}