\providecommand{\main}{../../}
\documentclass[\main/main.tex]{subfiles}
\begin{document}

\section{Domande su Clustering}

\subsection{Cosa si intende per clustering? In quali famiglie vengono divisi? Esplicitare la differenza tra soft e hard clustering}
Il \textit{clustering} è una tecnica di apprendimento non supervisionato che consiste nel separare un insieme di dati non etichettati in insiemi, i \textit{cluster}, internamente omogenei.

\subsubsection*{In quali famiglie sono divisi?}
Gli algoritmi di clustering si dividono principalmente in due categorie: algoritmi gerarchici e algoritmi partizionali.

Gli algoritmi gerarchici organizzano il dataset in una struttura ad albero dividendo cluster troppo disomogenei (\textbf{algoritmi divisivi}) o unendo cluster simili tra loro (\textbf{algoritmi agglomerativi}).

Gli algoritmi partizionali impongono una suddivisione dello spazio delle \textit{feature} in più sottoinsiemi, che sono i cluster e si dividono tra $hard$ e $soft$.

Infine alcuni algoritmi possono essere basati su teorie probabilistiche: si parla di \textbf{algoritmi statistici}.

\subsubsection{Differenza tra soft e hard clustering}
Negli algoritmi partizionali, se ogni \textit{pattern} può appartenere ad un solo cluster si parla di \textit{hard clustering}, altrimenti, se ogni pattern può appartenere a più cluster con un grado di \textit{membership} si parla di \textit{soft clustering} o \textit{fuzzy clustering}.

\subsection{Che relazione c'è tra clustering e classificazione e quali sono le criticità?}
Clustering e classificazione si pongono di risolvere lo stesso problema, il raggruppamento dei dati in classi, ma agiscono in modo profondamente diverso:

Il clustering procede in modo non supervisionato, agendo unicamente sui dati, mentre la classificazione procede in modo supervisionato, usando informazioni estratte dal training set.

\subsubsection*{Quali sono le criticità?}
Il clustering \textbf{non} è un problema ben posto: i risultati vanno validati ed interpretati. Esistono vari gradi di libertà nella modellazione di un algoritmo di clustering, rappresentazione dei pattern, calcolo delle feature, numero dei cluster attesi, scelta del tipo di distanza da usare tra cluster etc...

\subsubsection*{È possibile usare una mappa di Kohonen per rappresentare una sfera? Perché?}
Si, poiché le SOM distribuiscono una griglia su una superficie man mano che i dati vengono analizzati. Il dato sposta i punti della griglia con i loro vicini in modo da ottenere la migliore distribuzione possibile, nel caso di una sfera in cui i dati rappresentano la superficie la mappa, assunto che vi siano abbastanza dati, ne assumerà la forma.

\end{document}
