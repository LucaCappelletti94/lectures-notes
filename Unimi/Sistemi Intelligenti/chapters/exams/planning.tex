\providecommand{\main}{../../}
\documentclass[\main/main.tex]{subfiles}
\begin{document}
\section{Domande sul Planning}

\subsection{Come è definito il problema dell'esplorazione autonoma di ambienti con robot mobili?}
Il problema dell'esplorazione autonoma di ambienti con robot mobili consiste nel mettere un robot dotato di capacità di movimento in un ambiente inizialmente ignoto.

Il robot, muovendosi all'interno dell'ambiente deve costruire una rappresentazione interna della mappa dell'ambiente in modo da facilitare la navigazione per dei task futuri.

Un robot che ha esplorato efficacemente un ambiente, ad esempio, sarà in grado di muoversi da un punto A a un punto B dell'ambiente nel più breve tempo possibile.

Esistono 3 problemi tipici nell'esplorazione:
\begin{description}
  \item[Problema della ricarica] È importante anche che il robot si muova ottimizzando gli spostamenti, in quanto è dotato di un tempo di carica limitato. Il robot deve essere in grado di pianificare delle soste regolari presso le stazioni di ricarica.
  \item[Problema della posizione] Dato che un robot non può mai conoscere la sua posizione assoluta nell'ambiente, deve essere in grado di ricavare la sua posizione relativa, operazione molto complessa e facilmente soggetta a errori di stima.
  \item[Problema dell'incertezza dell'esito delle azioni] I robot si muovono in un ambiente totalmente stocastico: l'esito di ogni azione non è certo, sia per la presenza di ostacoli non previsti nell'ambiente, sia perché i robot sono molto fragili e tendono a rompersi/cadere/bloccarsi. Un robot mobile deve essere munito di una policy efficace che lo renda in grado di reagire agli ostacoli imprevisti.
\end{description}

\subsection{Cosa è un sistema multi-agente? Se ne dia un esempio in applicazioni robotiche}
Un sistema multi-agente (MAS) è definito come una collezione di agenti che interagiscono tra loro e con l'ambiente per raggiungere un obiettivo comune.

Un esempio di applicazione è la \textbf{sorveglianza multi-agente} che coinvolge un gruppo di agenti incaricato di effettuare un'operazione di "clear" di un dato ambiente, vale a dire una visita "strategica" dell'ambiente che assicura l'identificazione e la cattura delle entità ostili che lo contaminano.

I robot devono affrontare il problema della pianificazione di una strategia da seguire e il problema della comunicazione tra di essi, che spesso prendono vie molto diverse e si "perdono di vista".
\end{document}