\providecommand{\main}{../../../}
\documentclass[\main/main.tex]{subfiles}
\begin{document}
\subsection{Esercizio delle macchine}
Tre macchine, $A$, $B$ e $C$, producono rispettivamente il 50\%, il 40\%, e il 10\% del numero totale dei pezzi prodotti da una fabbrica.

Le percentuali di produzione difettosa di queste macchine sono rispettivamente del 2\%, 1\% e 4\%.

\begin{enumerate}
  \item Determinare la probabilità di estrarre un pezzo difettoso.
  \item Viene estratto a caso un pezzo che risulta difettoso. Determinare la probabilità che quel pezzo sia stato prodotto da $C$.
\end{enumerate}

\subsection{Soluzione esercizio delle macchine}
I valori sono modellati come segue:
\begin{align*}
  \tprob{A} = 0.5             \\
  \tprob{B} = 0.4             \\
  \tprob{C} = 0.1             \\
  \tprob{Difettoso}{A} = 0.02 \\
  \tprob{Difettoso}{B} = 0.01 \\
  \tprob{Difettoso}{C} = 0.04 \\
\end{align*}
Viene chiesto di ottenere la probabilità $\tprob{Difettoso}$. Procediamo con la \textbf{formula delle probabilità totali}:

\formulaProbTot

\begin{align*}
  \ttotProb{Difettoso}{A}{B}{C}                \\
   & = 0.2\cdot0.5 + 0.1\cdot0.4 + 0.4\cdot0.1 \\
   & =0.018
\end{align*}
Viene chiesto di ottenere la probabilità $\tprob{C}{Difettoso}$, che otteniamo tramite il \textbf{teorema di Bayes}:

\bayesTh

\begin{align*}
  \tbayes{C}{Difettoso}            \\
   & = \frac{0.04\cdot 0.1}{0.018}
  = 0.22\bar{2}
\end{align*}

\end{document}