\providecommand{\main}{../../../}
\documentclass[\main/main.tex]{subfiles}
\begin{document}
\subsection{Esercizio sul tumore al seno}
Lo strumento principe per lo screaning per il tumore al seno è la radiografia (mammografia).

Sappiamo che la \textbf{sensitività} della mammografia è intorno al 90\% e che la \textbf{specificità} sia anch'essa del 90\%.

\begin{enumerate}
  \item Qual è la probabilità che l'esame dia risultato positivo, sapendo che le donne malate sono lo 0,01\%?
  \item Qual è la percentuale di donne che hanno uno screening positivo, di essere effettivamente malate?
\end{enumerate}

\subsection{Soluzione esercizio sul tumore al seno}
Sensitività e specificità sono associate rispettivamente ad errore del primo e secondo tipo:

\begin{description}
  \item[Errore di primo tipo] È quando si rifiuta come falsa l'ipotesi vera.
  \item[Errore di secondo tipo] È quando si accetta come vera l'ipotesi falsa.
\end{description}

\begin{description}
  \item[Sensitività] Probabilità che lo strumento non compia un errore di primo tipo.
  \item[Specificità] Probabilità che lo strumento non compia un errore di secondo tipo.
\end{description}


La \textbf{sensitività} di uno strumento è la probabilità che esso dia esito positivo in un caso positivo:

\[
  \prob{\text{Esito positivo}}{\text{Donna malata}} = 0.9
\]
La \textbf{specificità} di uno strumento è la probabilità che esso dia esito negativo in un caso negativo:

\[
  \prob{\text{Esito negativo}}{\text{Donna sana}} = 0.9
\]
Viene chiesta la probabilità $\prob{\text{Esito positivo}}$ sapendo che $\prob{\text{Donna malata}}=0.01$, che possiamo ottenere tramite la formula delle probabilità totali:

\formulaProbTot

\begin{align*}
  \totProb{\text{Esito positivo}}{\text{Donna malata}}{\text{Donna sana}} \\
   & = 0.9\cdot0.01 + (1-0.9)\cdot(1-0.01)                                \\
   & = 0.9\cdot0.01 + 0.1\cdot0.99                                        \\
   & = 0.108
\end{align*}
Viene chiesta la probabilità $\prob{\text{Donna malata}}{\text{Esito positivo}}$, ottenibile tramite il teorema di Bayes:

\bayesTh

\begin{align*}
  \bayes{\text{Donna malata}}{\text{Esito positivo}} \\
   & = \frac{0.9\cdot0.01}{0.108}                    \\
   & = 0.083
\end{align*}
Una donna è malata quando l'esito è positivo nell'8.3\% dei casi. Da questi valori di evince che una sensitivtà del 90\% risulta essere un valore troppo basso quando la probabilità dell'evento positivo (essere malata) è basso.
\end{document}