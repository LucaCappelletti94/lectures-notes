\providecommand{\main}{..}
\documentclass[\main/main.tex]{subfiles}
\begin{document}

Il numero a fianco della domanda rappresenta il numero di volte che è stata posta. Quando esso non è presente, significa che le domande sono state poste solo una volta.

\section{Reinforcemente Learning}

\subsection{Cosa si intende per Apprendimento con Rinforzo?}
\subsection{Quali sono gli attori?}
\subsection{Cosa rappresenta la critica?}
\subsection{Che tipo di architettura si può ipotizzare nell'apprendimento con rinforzo?}
\subsection{Condizionamento classico e condizionamento operante}
\subsection{Quale relazione c'è con l'intelligenza?}
\subsection{Come potreste illustrare: Exploration vs Exploitation?}
\subsection{Cos'è il credit assignement?}
\subsection{Cosa si intende per traccia e quale è il suo ruolo?}
\subsection{Scrivere le equazioni dell'algoritmo Q-learning in cui si consideri anche la traccia.}
\subsection{Dato un problema a piacere si descriva uno degli algoritmi e mostrare due passaggi di addestramento}
\subsection{Quale criterio si sceglie per definire i Reward?}
\subsection{A quali elementi sono associati i reward? Allo stato? All'azione? Allo stato prossimo? Perchè?}

\section{Fuzzy System}

\subsection{Definire i passi per costruire un sistema fuzzy.}
\subsection{Cosa si intende per FAM?}
\subsection{Una FAM memorizza numeri o preposizione logiche? Come?}
\subsection{Definire un problema per FAM a piacere che involva almeno due variabili in ingresso e due in uscita.}
\subsection{Definire tutti i componenti e calcolare l'uscita passo a passo per un valore di input a piacere}

\section{Macchine e intelligenza}

\subsection{Descrivere il test di Turing}
\subsection{Descrivere l'esperimento della stanza cinese}
\subsection{Come mai è stato proposto il test di Turing?}
\subsection{Come mai è stato proposto l'esperimento della scatola cinese?}
\subsection{Cosa voleva dimostrare il test di Turing?}
\subsection{Cosa voleva dimostrare l'esperimento della scatola cinese?}
\subsection{Cosa si intende per ipotesi forte ed ipotesi debole dell'AI?}
\subsection{Riportare almeno due elementi del contraddittorio sulle ipotesi su cui è basata l'ipotesi debole sull'AI}
\subsection{Descrivere il “Brain prosthesis thought experiment” di Moravec e commentarlo.}

\section{Statistica}

\subsection{Eserizio 1}
In una città lavorano due compagnie di taxi: blue e verde, la maggior parte dei taxisti lavorano per la compagnia verde per cui si ha la seguente distribuzione di taxi in città: 85\% di taxi verdi e 15\% di taxi blu. Succede un incidente in cui è coinvolto un taxi. Un testimone dichiara che il taxi era blu. Era sera e buio, c'era anche un po' di nebbia ma il testimone ha una vista acuta, la sua affidabilità è stata valutata del 70\%. Qual è la probabilità che il taxi fosse effettivamente blu? Quale deve essere l'affidabilità del testimone perché la probabilità che il taxi fosse effettivamente blu sia del 99\%? Enunciare il teorema di Bayes. Discutere l'analisi di varianza per un sistema lineare. Dimostrare che la stima ai minimi quadrati è equivalente alla stima a massima verosimiglianza nel caso di errore Gaussiano sui dati. Cosa fornisce? Come?

\section{Apprendimento supervisionato}

\subsection{Definire l'algoritmo di apprendimento di una rete neurale con unità arbitrarie.}
\subsection{Definire la funzione obbiettivo utilizzata}
\subsection{Come si utilizza la funzione obbiettivo nell'algoritmo di apprendimento}
\subsection{Cosa si intende per apprendimento per epoche e per trial?}
\subsection{Qual è il vantaggio di ciascuna delle modalità di apprendimento?}
\subsection{Cosa si intende per training e test set? Perché mai vengono utilizzati? Quali problemi si vogliono evitare?}
\subsection{Una rete neurale con unità sigmoidali e un modello parametrico? È lineare? Perché?}
\subsection{Se i dati sono acquisiti senza errori, è una buona scelta aumentare di molto i parametri del modello in modo da garantirsi che l'errore sul training set vada a zero? Perché?}
\subsection{Cosa si intende per un problema di regressione ed illustrare una possibile soluzione.}
\subsection{Come funziona l'approssimazione incrementale multi-scala, cosa garantisce e quali vantaggi può avere?}

\section{AI}

\subsection{Si descriva il funzionamento della Forward Search.  Perché è considerato un template e non un algoritmo?}
\subsection{Si elenchino due possibili implementazioni di Forward Search elencandone proprietà, vantaggi e svantaggi.}

\section{Clustering}

\subsection{Cosa si intende per clustering? In quali famiglie vengono divisi?}
\subsection{Che relazione c'è tra clustering e classificazione e quali sono le criticità?}

\section{Biologia}

\subsection{Definire il neurone biologico evidenziandone le parti più significative per la trasmissione dell'informazione ed il loro comportamento.}
\subsection{Descrivere il funzionamento complessivo del neurone biologico.}
\subsection{Che differenza c'è tra neuroni motori, neuroni sensoriali ed inter-neuroni?}
\subsection{Come viene trasmessa ed elaborata l'informazione da un neurone?}
\subsection{ Cos'è uno spike?}
\subsection{ Quali sono le aree corticali principali?}
\subsection{Cos'è il codice di popolazione?}
\subsection{Data un'area cerebrale è univoca la funzione implementata in quell'area?}
\subsection{Cosa sono i mirror neurons? Quali implicazioni hanno per i sistemi intelligenti e l’apprendimento?}
\end{document}