\providecommand{\main}{..}
\documentclass[\main/main.tex]{subfiles}
\begin{document}

\section{Introduzione}
Il materiale di questa sezione del corso, erogata dal prof. Basilico, è disponibile sul \href{http://teaching.basilico.di.unimi.it/doku.php/pub/sistemi-intelligenti-2017-2018}{sito personale del docente}. Le slide son generalmente sufficienti, i libri presenti in elenco sono per approfondimenti.

\section{``Flavours'' nella piantificazione discreta}
Si può pianificare per \textbf{fattibilità} o per \textbf{ottimalità}, e rispettivamente vengono chiamati \textit{problem solving} e \textit{ricerca della soluzione ottima}.

\begin{definition}[Soluzione ottima]
  La soluzione tale per cui non è peggiore di nessun'altra.
\end{definition}

\section{Formulazione del problema}
In un problema abbiamo:

\begin{enumerate}
  \item Uno o più agenti
  \item $\bm{X}$ è un set di stati, $x$ lo stato generico. Questa è un'astrazione.
  \item $\bm{U}(x)$ è il set di azioni che possono essere intraprese in un determinato stato $x$. L'azione generica è indicata con $u$.
  \item Se un'azione $u$ è intrapresa in uno stato $x$, allora viene raggiunto lo stato $x' = f(x,u)$ e $f$ viene chiamata \textbf{funzione di transizione}, ed è deterministica. (Questo implica che stiamo modellando un mondo deterministico).
  \item Ci deve essere uno \textbf{stato di partenza} $x_i$ ed un set di \textbf{stati di goal} $\bm{X}_G$.
\end{enumerate}

Un problema viene rappresentato con un grafo chiamato \textbf{grafo di transizioni}.

\section{Ricerca sistematica}
In una ricerca valgono le seguenti proprietà: in un grafo \textbf{finito} un grafo eventualmente visiterà tutti gli stati raggiungibili (evita l`esplorazione ridondante.).

Se il grafo è \textbf{infinito}, se la risposta è \textit{si} l'algoritmo deve terminare, mentre se la risposta è \textit{no} va bene che esso vada avanti all'infinito, però nel limite asintotico tutti gli stati devono essere visitati.

\subsection{Forward Search}

PRENDI ALGORITMO FORWARD SEARCH DA SLIDE

\end{document}