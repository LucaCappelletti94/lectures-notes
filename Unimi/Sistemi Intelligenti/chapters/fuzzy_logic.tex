\providecommand{\main}{..}
\documentclass[\main/main.tex]{subfiles}
\begin{document}

\section{Logica fuzzy vs classica}
\subsection{Le funzioni di appartenenza}
In logica classica la funzione che descrive la verità di un'affermazione è rappresentabile come una funzione impulsiva, per esempio:

\[
	\begin{cases}
		1 \quad x > 0    \\
		0 \quad x \leq 0
	\end{cases}
\]

Mentre la funzione di appartenenza nella logica fuzzy sono più adeguate funzioni come:

\begin{enumerate}
	\item Una lineare che aumenta progressivamente da 0 a 1 in un certo $\Delta x$ determinato.
	\item Un sigmoide.
	\item Funzioni probabilistiche, come una normale.
\end{enumerate}

\subsection{Classi di appartenenza}
In logica classica le classi sono nette, come nel caso della funzione istintiva si ha una condizione del tipo:

\[
	\begin{cases}
		A \quad x \geq 0 \wedge x<1 \\
		B \quad x \geq 1 \wedge x<2 \\
		C \quad x \geq 2 \wedge x<3 \\
		D \quad x \geq 3 \wedge x<4 \\
	\end{cases}
\]

Nella logica fuzzy, vengono descritte per ogni gruppo funzioni che assumono valori anche negli insiemi in cui nella logica classica esse non sono definite. Linearmente esse raggiungono lo 0 mano a mano che esse si sovrappongono con le altre funzioni. In un qualsiasi punto di ascissa, vale la formula:

\[
	\sum_{i=0}^{n} m_i = 1
\]

\section{Logica fuzzy e probabilità}
Descrivono cose diverse: prendendo per esempio le previsioni meteo, la \textbf{probabilità} si occupa di prevedere i mm di pioggia che potrebbero andare a cadere, mentre la \textbf{logica fuzzy} si occuperebbe di descrivere il grado di \textbf{fuzzyness} tramite il quale andiamo a descrivere quanto è "pioggia", con una funzione che in base a quante gocce di pioggia sono cadute si descrive la \textit{funzione di appartenenza fuzzy} tra le classi "piove" e "non piove".

Ulteriormente, una volta che un evento è avvenuto la sua \textbf{probabilità} scompare, nel senso che ora è un dato noto, mentre il valore di \textbf{fuzzyness} mantiene il suo valore descrittivo per l'evento.

\section{Gli operatori logici nella logica fuzzy}

\begin{center}
	\begin{tabular}{ |c|c|c| }
		\hline
		Operatore & Logica Classica & Logica Fuzzy      \\
		\hline
		$\wedge$  & $A \wedge B$    & $min(T(A), T(B))$ \\
		\hline
		$\vee$    & $A \vee B$      & $max(T(A), T(B))$ \\
		\hline
		$ \neg$   & $\neg A$        & $1 - T(A)$        \\
		\hline
	\end{tabular}
\end{center}

\section{Misure in un insieme fuzzy}

\subsection{Norma di un vettore}

\begin{figure}[H]
	\[
		M(A) = \sqrt[p]{\sum_{i=1}^{n} \abs{m_A(x_i)}^p}
	\]
	\caption{Norma di un vettore}
\end{figure}

\subsection{Entropia}
Dato un certo punto $A$, definisco due vettori $\vec{a}$ e $\vec{b}$ che descrivono la posizione del punto $A$ a partire dagli estremi opposti del quadrato.

\textbf{L'entropia minima} risulta pari a 0.

\textbf{L'entropia massima} risulta pari a 1 e si trova nel punto di mezzo (Es. quando una macchina parcheggia tra un posto e l'altro e non è chiaro in quale posto andrebbe vista come pargheggiata). Questa coincide con la \textbf{massima fuzzyness} e in questo punto vale che $A \cup A_c = A \cap A_c$.

\begin{figure}[H]
	\[
		E(A) = \dfrac{a}{b} = \dfrac{l^1(A, A_{vicino})}{l^1(A, A_{lontano})}
	\]
	\caption{Entropia}
\end{figure}

\section{Fuzzy Associative Memory FAM}
Una FAM trasforma uno spazio di input in uno spazio di output. Esse implementano una serie di regole su delle variabile logiche fuzzy in ingresso.

Le regole sono regole della logica classica, mentre le variabili sono fuzzy.

Una FAM va a descrivere un insieme di classi ed assegna un valore di una funzione di appartenenza ad ogni variabile su ogni classe, poi su queste classi vengono eseguite operazioni di logica classica.

\subsection{Come opera il sistema}

\begin{enumerate}
	\item Riceve le classi attivate in input
	\item Riceve il grado di fit per ogni classe
	\item Identifica le regole attivate
	\item Determino le classi in uscita attivate
	\item Determino il grado di fitness per ogni classe in uscita (regola)
	\item Defuzzyficazione
\end{enumerate}

\end{document}