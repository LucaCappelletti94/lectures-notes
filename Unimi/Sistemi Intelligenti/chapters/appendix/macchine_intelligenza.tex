\providecommand{\main}{../../}
\documentclass[\main/main.tex]{subfiles}
\begin{document}

\subsection{Descrivere il test di Turing, l'esperimento della stanza cinese e l'esperimento della stanza di Maxwell [4]}
\subsubsection{Test di Turing}
Il test di Turing, proposto da Alan Turing nel 1950, è un metodo per valutare l'intelligenza di un'intelligenza artificiale. Turing, preso spunto dal gioco dell'imitazione in cui una persona doveva comprendere se un interlocutore nascosto fosse uomo o donna in base ai messaggi che questo inviava, propone un gioco analogo in cui la conversazione avviene con una macchina od una persona, ed il nuovo obbiettivo è determinare se si sta conversando con una macchina o una persona.

Il test è considerato passato quando la macchina è riconosciuta come umana.

\subsubsection{Stanza Cinese}
L'esperimento della stanza cinese è stato proposto da Jhon Searle nel 1980 in contrapposizione all'ipotesi che una macchina possa essere davvero intelligente. Secondo Searle una macchina non può essere intelligente in nessun modo in quanto manca di quella che possiamo definire "coscienza". L'esperimento consiste nel mettere una persona in una stanza con un traduttore di simboli cinesi in un alfabeto conosciuto alla persona e un foglio con delle domande scritte in cinese. L'uomo riuscirà a rispondere alle domande pur non avendo coscienza di quel che sta facendo in quanto sta semplicemente traducendo i simboli.

\subsubsection*{Stanza di Maxwell}
Esperimento mentale da Paul e Patricia Churchland, è una critica alla stanza cinese. Propone assiomi plausibili (elettricità e magnetismo sono forze e le forze non hanno a che fare con la luminosità), ma errati, sulla natura di elettricità, magnetismo e luce ed immagina Maxwell, intento a realizzare luce usando elettricità e magnetismo, costretto a spiegare che questi assiomi non sono validi e che non hanno nessuna giustificazione sulla natura della luce.

Sebbene la stanza cinese di Searle possa apparire "semanticamente buia", non vi è nessunissima giustificazione alla sua pretesa, fondata su quest'apparenza, che la manipolazione di simboli secondo certe regole non potrà mai dar luogo a fenomeni semantici, specie se i lettori hanno soltanto una concezione vaga e basata sul buon senso dei fenomeni semantici e cognitivi di cui si cerca una spiegazione. Invece di sfruttare la comprensione che i lettori hanno di queste cose, l'argomento di Searle sfrutta senza troppi scrupoli la loro ignoranza in proposito.

\subsubsection{Come mai son stati proposti? Cosa volevano dimostrare?}
I 2 esempi sono stati proposti perché Turing sosteneva che una macchina possa essere definita intelligente nel momento una macchina riesce a far credere ad un osservatore di essere una persona, mentre Searle sostiene che una macchina non potrà mai essere definita intelligente in quanto assente di "coscienza".

\subsection{Discutere la relazione tra algoritmo, macchina di Turing ed intelligenza.}
Un algoritmo è una sequenza di passi elementari computabili, usati per risolvere un problema.

La macchina di Turing, proposta nel 1936 dall'omonimo matematico, è un formalismo in grado di eseguire algoritmi computabili ed arrivare in un tempo finito alla soluzione basandosi su regole definite su un alfabeto di simboli.

Una macchina di Turing è intelligente secondo l'ipotesi debole (cioè appare intelligente) ma non secondo l'ipotesi forte.

\subsection{Cosa si intende per ipotesi forte ed ipotesi debole dell'AI? [4]}
Sono due linee di pensiero nella filosofia dell'intelligenza artificiale:
\begin{description}
  \item[Ipotesi forte] È possibile realizzare un'intelligenza artificiale cosciente, senza mostrare necessariamente processi di pensiero umani. Questa ipotesi è spesso accompagnata da proposte di imitazione della struttura fisica del cervello (neuroni, sinapsi, etc..).
  \item[Ipotesi debole] È possibile realizzare un'intelligenza artificiale che appare intelligente (una macchina di Turing che risolve un algoritmo molto complesso) ma non è effettivamente cosciente.
\end{description}

\subsection{Riportare il contraddittorio sulle ipotesi su cui è basata l'ipotesi debole sull'AI [3]}
\begin{enumerate}
  \item Una macchina non può originare nulla di nuovo, esegue dei programmi.
  \item Il comportamento intelligente non può essere completamente replicato, per esempio l'aspetto emotivo.
  \item Il comportamento intelligente non può essere completamente catturato da regole formali (argument for informality), per esempio il subconscio.
  \item Anche se una macchina di Turing riuscisse a superare test di Turing, mancherebbe comunque di una coscienza.
\end{enumerate}

\subsection{Descrivere il "Brain prosthesis thought experiment" di Moravec e commentarlo. [3]}
Proposto da Hans Moravec nel 1988, chiede cosa succederebbe se sostituissimo uno a uno tutti i neuroni con un dispositivo elettronico equivalente. Esistono due risposte:
\begin{description}
  \item[Risposta funzionalistica] La mente è de-facto una scatola nera e modificare i costituenti fisici non comporta modifiche.
  \item[Risposta strutturalista] Ad un certo punto la coscienza svanisce.
\end{description}

\end{document}