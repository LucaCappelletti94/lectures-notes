\providecommand{\main}{../../}
\documentclass[\main/main.tex]{subfiles}
\begin{document}

\section{Descrivere il test di Turing e l'esperimento della stanza cinese. [4]}
Il test di Turing è un criterio per determinare se una macchina sia in grado di pensare. Turing prende spunto da un gioco, chiamato "gioco dell'imitazione" (the imitation game), a tre partecipanti: un uomo A, una donna B, e una terza persona C. Quest'ultimo è tenuto separato dagli altri due e tramite una serie di domande deve stabilire qual è l'uomo e quale la donna. Dal canto loro anche A deve ingannare C e portarlo a fare un'identificazione errata, mentre B deve aiutarlo.
Il test di Turing si basa sul presupposto che una macchina si sostituisca ad A. Se la percentuale di volte in cui C indovina chi sia l'uomo e chi la donna è simile prima e dopo la sostituzione di A con la macchina, allora la macchina stessa dovrebbe essere considerata intelligente, dal momento che sarebbe indistinguibile da un essere umano.
Per macchina intelligente Turing ne intende una in grado di pensare, ossia capace di concatenare idee e di esprimerle. Per Turing, quindi, tutto si limita alla produzione di espressioni non prive di significato.
La Stanza cinese è un esperimento mentale ideato da John Searle dove: si supponga che si possa costruire un computer che si comporti come se capisse il cinese. In altre parole, il computer prenderebbe dei simboli cinesi in ingresso, eseguirebbe un programma e produrrebbe altri simboli cinesi in uscita. Si supponga che il comportamento di questo computer sia così convincente da poter facilmente superare il test di Turing. A tutte le domande dell'umano il computer risponderebbe appropriatamente, in modo che l'umano si convinca di parlare con un altro umano che parla correttamente cinese.
Ora, Searle chiede di supporre che un uomo si sieda all'interno del calcolatore. In altre parole, egli si immagina in una stanza (la stanza cinese) con un libro contenente la versione in inglese del programma utilizzato dal computer e carta e penna in abbondanza. L'uomo potrebbe ricevere scritte in cinese attraverso una finestra di ingresso, elaborarle seguendo le istruzioni del programma, e produrre altri simboli cinesi in uscita, in modo identico a quanto faceva il calcolatore.
Il calcolatore potrebbe dimostrare di essere intelligente al test di Turing senza, comprendere nulla. Non conoscendo il cinese non si può generare la semantica dalla sintassi.

\subsection{Come mai son stati proposti? Cosa volevano dimostrare?}
La macchina di Turing è stata proposta per supportare la teoria seconda la quale una macchina attraverso corrette e complesse funzioni possa governare la struttura delle risposte umane. Il cervello e la macchina di Turing sono molto diversi e infatti si parla di creare funzioni e programmi basati sull’hardware della macchina e che raggiungano un funzionamento equivalente a quello del cervello umano ma non uguale.
La stanza cinese è invece stata proposta per dimostrare che l’ipotesi di Turing non rappresenta una vera intelligenza. Con la stanza cinese infatti si dimostra che una macchina o anche una persona potrebbero riuscire a rispondere a delle domande in modo convincente ma senza possedere un’intelligenza, il solo seguire delle regole predefinite infatti non implica il possedere un’intelligenza.
\section{Discutere la relazione tra algoritmo, macchina di Turing ed intelligenza.}
\section{Cosa si intende per ipotesi forte ed ipotesi debole dell'AI? [4]}
Nella filosofia dell'intelligenza artificiale, l'intelligenza artificiale forte è l'idea che opportune forme di intelligenza artificiale possano veramente ragionare e risolvere problemi; l'intelligenza artificiale forte sostiene che è possibile per le macchine diventare sapienti o coscienti di sé, senza necessariamente mostrare processi di pensiero simili a quelli umani.
In contrasto con l'intelligenza artificiale forte, l'intelligenza artificiale debole si riferisce all'uso di programmi per studiare o risolvere specifici problemi o ragionamenti che non possono essere compresi pienamente nei limiti delle capacità cognitive umane. Diversamente dall'intelligenza artificiale forte, quella debole non realizza un'auto-consapevolezza e non dimostra il largo intervallo di livelli di abilità cognitive proprio dell'uomo, ma è esclusivamente un problem-solver specifico e, parzialmente, intelligente.

\section{Riportare almeno due elementi del contraddittorio sulle ipotesi su cui è basata l'ipotesi debole sull'AI [3]}
\begin{enumerate}
\item Una macchina non può originare nulla di nuovo, esegue dei programmi. Una macchina però può imparare dall’esperienza e quindi costruirsi una cultura in grado di migliorare. 
\item Il comportamento intelligente non può essere completamente replicato. 
\item Il comportamento intelligente non può essere completamente catturato da regole formali (argument for informality). 
\item Anche se un computer si comportasse in modo da superare il test di Turing, non sarebbe comunque classificato come intelligente.
\end{enumerate}

\section{Descrivere il "Brain prosthesis thought experiment" di Moravec e commentarlo. [3]}
Ipotizzando di disporre di neuroni artificiali equivalenti a quelli biologici, l’esperimento consisterebbe nel sostituire uno ad uno ogni neurone, fino ad ottenere un cervello completamente artificiale. Secondo Moravec, questo cervello conterrebbe la stessa mente del cervello di partenza, risposta funzionalista. Secondo Searle, la coscienza sparirebbe ma il comportamento visibile all’esterno sarebbe indistinguibile dall’originale, risposta strutturalista.
Vanno quindi prese in esame le varie possibili conclusioni: 1. i meccanismi causali della coscienza stanno ancora operando nel cervello elettronico, 2. gli eventi mentali non sono presenti nel cervello elettronico, quindi non `e conscio, 3. l’esperimento `e inattuabile, pertanto proporre ipotesi al riguardo non ha alcun senso. Escludiamo la terza possibilità perchè siamo interessati alla questione filosofica e non alla realizzabilità dell’esperimento. La seconda opzione proposta ci porta a ritenere la coscienza come un qualcosa che è ininfluente ai fini della determinazione dell’output del soggetto. Avendo, però, riprodotto il funzionamento di un cervello reale, dovremmo anche asserire che gli eventi mentali consci nel cervello umano non hanno un collegamento casuale con il comportamento. Se quindi accettiamo il fatto che l’esperimento della sostituzione del cervello dimostra che quello elettronico è cosciente, dobbiamo concordare che la coscienza è conservata anche quando l’intero cervello è sostituito da un insieme di chip elettronici; ovvero la prima opzione proposta.

\end{document}