\providecommand{\main}{../../}
\documentclass[\main/main.tex]{subfiles}
\begin{document}

\subsection{Descrivere il test di Turing, l'esperimento della stanza cinese e l'esperimento della stanza di Maxwell [4]}
\subsubsection{Test di Turing}
Il test di Turing si svolge con 3 agenti. A e B sono rispettivamente una macchina ed un uomo mentre C è un esaminatore. A e B sono in una stanza separata rispetto a C. C pone delle domande sia ad A che a B e questi risponderanno. C dovrà capire quale delle 2 risposte proviene dall'uomo. Se C non riesce a riconoscere chi sia l'uomo rispetto alla macchina, allora la macchina si può definire intelligente.
Se si volesse fare un paragone tra questo esperimento ed un altro, si potrebbe introdurre al posto di una macchina una donna. Il funzionamento del "gioco" sarebbe il medesimo e se l'esaminatore riesce a riconoscere chi è l'uomo contro la donna con la stessa percentuale di probabilità dell'uomo contro la macchina, allora la macchina si può definire intelligente.

\subsubsection{Stanza Cinese}
L'esperimento della stanza cinese è invece stato proposto da Jhon Searle in contrapposizione all'ipotesi che una macchina possa essere intelligente. Secondo Searle una macchina non può essere intelligente in nessun modo in quanto manca di quella che possiamo definire "coscienza". L'esperimento consiste nel mettere una persona in una stanza con un traduttore di simboli cinesi in un alfabeto conosciuto alla persona e un foglio con delle domande scritte in cinese. L'uomo riuscirà a rispondere alle domande pur non avendo coscienza di quel che sta facendo in quanto sta semplicemente traducendo i simboli.

\subsubsection*{Stanza di Maxwell}
Prende il nome dall'esperimento di Maxwell:

L'esperimento di Maxwell consiste nel creare luce utilizzando le onde elettromagnetiche. Questo non è possibile in quanto le sole forze non sono in grado di creare luce.

Sebbene la stanza cinese di Searle possa apparire "semanticamente buia", non vi è nessunissima giustificazione alla sua pretesa, fondata su quest'apparenza, che la manipolazione di simboli secondo certe regole non potrà mai dar luogo a fenomeni semantici, specie se i lettori hanno soltanto una concezione vaga e basata sul buon senso dei fenomeni semantici e cognitivi di cui si cerca una spiegazione. Invece di sfruttare la comprensione che i lettori hanno di queste cose, l'argomento di Searle sfrutta senza troppi scrupoli la loro ignoranza in proposito.

\subsubsection{Come mai son stati proposti? Cosa volevano dimostrare?}
I 2 esempi sono stati proposti perché Turing sosteneva che una macchina possa essere definita intelligente nel momento una macchina riesce a far credere ad un osservatore di essere una persona, mentre Searle sostiene che una macchina non potrà mai essere definita intelligente in quanto assente di "coscienza".

\subsection{Discutere la relazione tra algoritmo, macchina di Turing ed intelligenza.}
Un algoritmo è una serie di operazioni semplici che eseguite in sequenza porta a un risultato, la macchina di Turing esegue algoritmi sulla base di regole che vengono definite da una funzione chiamata funzione di transizione e sull’inizializzazione della macchina stessa, la macchina termina quando questa arriva in uno stato terminale. Una macchina come la macchina di Turing non è intelligente perché non conosce cosa sta eseguendo ma si limita a compiere le istruzioni che le sono state dettate, al più la si può istruire in modo che sia capace di adattarsi ed eseguire le operazioni che le sono state dettate in base alle situazioni, senza però formulare niente di nuovo.

\subsection{Cosa si intende per ipotesi forte ed ipotesi debole dell'AI? [4]}
Nella filosofia dell'intelligenza artificiale, l'intelligenza artificiale forte è l'idea che opportune forme di intelligenza artificiale possano veramente ragionare e risolvere problemi; l'intelligenza artificiale forte sostiene che è possibile per le macchine diventare sapienti o coscienti di sé, senza necessariamente mostrare processi di pensiero simili a quelli umani.
In contrasto con l'intelligenza artificiale forte, l'intelligenza artificiale debole si riferisce all'uso di programmi per studiare o risolvere specifici problemi o ragionamenti che non possono essere compresi pienamente nei limiti delle capacità cognitive umane. Diversamente dall'intelligenza artificiale forte, quella debole non realizza un'auto-consapevolezza e non dimostra il largo intervallo di livelli di abilità cognitive proprio dell'uomo, ma è esclusivamente un problem-solver specifico e, parzialmente, intelligente.

\subsection{Riportare almeno due elementi del contraddittorio sulle ipotesi su cui è basata l'ipotesi debole sull'AI [3]}
\begin{enumerate}
  \item Una macchina non può originare nulla di nuovo, esegue dei programmi. Una macchina però può imparare dall’esperienza e quindi costruirsi una cultura in grado di migliorare.
  \item Il comportamento intelligente non può essere completamente replicato.
  \item Il comportamento intelligente non può essere completamente catturato da regole formali (argument for informality).
  \item Anche se un computer si comportasse in modo da superare il test di Turing, non sarebbe comunque classificato come intelligente.
\end{enumerate}

\subsection{Descrivere il "Brain prosthesis thought experiment" di Moravec e commentarlo. [3]}
Ipotizzando di disporre di neuroni artificiali equivalenti a quelli biologici, l’esperimento consisterebbe nel sostituire uno ad uno ogni neurone, fino ad ottenere un cervello completamente artificiale. Secondo Moravec, questo cervello conterrebbe la stessa mente del cervello di partenza, risposta funzionalista. Secondo Searle, la coscienza sparirebbe ma il comportamento visibile all’esterno sarebbe indistinguibile dall’originale, risposta strutturalista.

Vanno quindi prese in esame le varie possibili conclusioni: 1. i meccanismi causali della coscienza stanno ancora operando nel cervello elettronico, 2. gli eventi mentali non sono presenti nel cervello elettronico, quindi non `e conscio, 3. l’esperimento `e inattuabile, pertanto proporre ipotesi al riguardo non ha alcun senso. Escludiamo la terza possibilità perchè siamo interessati alla questione filosofica e non alla realizzabilità dell’esperimento. La seconda opzione proposta ci porta a ritenere la coscienza come un qualcosa che è ininfluente ai fini della determinazione dell’output del soggetto. Avendo, però, riprodotto il funzionamento di un cervello reale, dovremmo anche asserire che gli eventi mentali consci nel cervello umano non hanno un collegamento casuale con il comportamento. Se quindi accettiamo il fatto che l’esperimento della sostituzione del cervello dimostra che quello elettronico è cosciente, dobbiamo concordare che la coscienza è conservata anche quando l’intero cervello è sostituito da un insieme di chip elettronici; ovvero la prima opzione proposta.

\end{document}