\providecommand{\main}{../../}
\documentclass[\main/main.tex]{subfiles}
\begin{document}
\section{Domande su Intelligenza Artificiale}

\subsection{Si descriva il funzionamento della Forward Search.  Perché è considerato un template e non un algoritmo?}
La \textbf{FOrward Search} è un template per algoritmi legate all'esplorazione di un grafo, partendo da uno stato iniziale e cercando di arrivare ad uno stato di goal.

Alla prima iterazione si marca il nodo di start come visitato e lo si inserisce nella coda, quindi si procede iterativamente: a ogni ciclo si estrae (pop) il primo elemento della coda e se è un nodo di \textbf{goal} l'algoritmo termina con successo, altrimenti marca i nodi vicini non visitati come \textbf{alive} e li inserisce in coda. Se un nodo è stato visitato e sono stati visitati tutti i nodi raggiungibili da esso, viene marcato come \textbf{dead}. Se la coda risulta vuota senza avere identificato un nodo di goal, l'algoritmo fallisce.

Si tratta di un \textbf{template} e non un algoritmo perché non è specificato il criterio con cui ordinare la coda di priorità $Q$.

\begin{algorithm}
	\KwData{Un grafo $G$, un nodo iniziale $s \in G$ ed i nodi di goal $X_t \subset G$.}
	\KwResult{Determina se un nodo $t \in X_t$ è raggiungibile dal nodo $s$.}
	\Begin{
		$Q \longleftarrow \crl{s}$\;
		Marca $s$ come visitato\;
		\nl\While{$Q \neq \emptyset$}{
			$x\longleftarrow Q.pop()$\;
			\If{$x\in X_t$}{
				\Return successo\;
			}
			\For{$x' \in \text{Intorno}(x)$}{
				\If{$x'$ non è stato visitato}{
					Marca $x'$ come visitato\;
					$Q \longleftarrow Q \cup \crl{x'}$
				}
			}
		}
		\Return fallimento\;
	}
	\caption{Algoritmo Forward Search}
\end{algorithm}

\subsection{Si elenchino due possibili implementazioni di Forward Search elencandone proprietà, vantaggi e svantaggi.}

\subsubsection{Breadth First Search}
L'algoritmo Breadth First Search, o \textit{visita in ampiezza} implementa la coda di priorità $Q$ come una coda FIFO.

Tutti i nodi alla stessa profondità sono visitati prima di procedere al livello successivo. Se viene trovato il percorso, è garantito che questo avrà la lunghezza minima.

La ricerca risulta di tipo sistematico.

\subsubsection{Deapth First Search}
L'algoritmo Deapth First Search, o \textit{visita in profondità}, implementa la coda $Q$ come una pila LIFO, visitando quindi sempre l'ultimo nodo non visitato identificato, dando quindi priorità ai nodi più lunghi.

Nel caso di grafi infiniti la ricerca risulta non sistematica.
\end{document}