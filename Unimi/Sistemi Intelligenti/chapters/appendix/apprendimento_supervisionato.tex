\providecommand{\main}{../../}
\documentclass[\main/main.tex]{subfiles}
\begin{document}

\section{Definire l'algoritmo di apprendimento di una rete neurale con unità arbitrarie. Definire la funzione obbiettivo utilizzata. [3]}
\section{Come si utilizza la funzione obbiettivo nell'algoritmo di apprendimento. [3]}
\section{Cosa si intende per apprendimento per epoche e per trial? Qual è il vantaggio di ciascuna delle modalità di apprendimento? [3]}
\section{Cosa si intende per training e test set? Perché mai vengono utilizzati? Quali problemi si vogliono evitare? [3]}
\section{Una rete neurale con unità sigmoidali e un modello parametrico? È lineare? Perché? [3]}
\section{Se i dati sono acquisiti senza errori, è una buona scelta aumentare di molto i parametri del modello in modo da garantirsi che l'errore sul training set vada a zero? Perché? [3]}
\section{Cosa si intende per un problema di regressione ed illustrare una possibile soluzione. [3]}
\section{Come funziona l'approssimazione incrementale multi-scala, cosa garantisce e quali vantaggi può avere? [3]}
\section{Determinare la forma analitica dell'aggiornamento dei parametri nel caso di unità lineari e di reti a singolo strato. [2]}
\section{Definire l'algoritmo di apprendimento di una rete neurale con unità lineari e con unità non-lineari. Definire la funzione obbiettivo utilizzata.}

\end{document}