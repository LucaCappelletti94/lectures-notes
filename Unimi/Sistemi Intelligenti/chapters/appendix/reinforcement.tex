\providecommand{\main}{../../}
\documentclass[\main/main.tex]{subfiles}
\begin{document}

\section{Cosa si intende per Apprendimento con Rinforzo?}
\section{Quali sono gli attori?}
\section{Cosa rappresenta la critica?}
\section{Che tipo di architettura si può ipotizzare nell'apprendimento con rinforzo?}
\section{Condizionamento classico e condizionamento operante}
\section{Quale relazione c'è con l'intelligenza?}
\section{Come potreste illustrare: Exploration vs Exploitation?}
\section{Cos'è il credit assignement?}
\section{Cosa si intende per traccia e quale è il suo ruolo? [2]}
\section{Definire l’algoritmo di Q-learning, descrivendo le equazioni opportune.}
\section{Scrivere le equazioni dell'algoritmo Q-learning in cui si consideri anche la traccia. [2]}
\section{Cosa si intende per politica epsilon-greedy? Come entra nell’algoritmo di Q-learning? }
\section{Che differenza c'è tra Q-learning e SARSA?}
\section{Dato un problema a piacere si descriva uno degli algoritmi e mostrare due passaggi di addestramento}
\section{Quale criterio si sceglie per definire i Reward?}
\section{A quali elementi sono associati i reward? Allo stato? All'azione? Allo stato prossimo? Perchè? [2]}
\section{Impostare un problema su griglia (apprendimento del percorso di un agente, con partenza ed arrivo prescelti + ostacoli). La griglia fornisce un reward, diverso da zero, in ogni transizione.}
\subsection{Definire chiaramente il problema, farne un modello definendo le variabili e le funzioni che le legano.}
\subsection{Scrivere un risultato possibile dei primi 2 passi di apprendimento del problema definito al punto precedente.} 

\end{document}