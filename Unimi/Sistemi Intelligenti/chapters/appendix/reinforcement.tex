\providecommand{\main}{../../}
\documentclass[\main/main.tex]{subfiles}
\begin{document}
\section{Domande su Apprendimento con Rinforzo}
\subsection{Cosa si intende per Apprendimento con Rinforzo?}
L'apprendimento per rinforzo è una tecnica di apprendimento automatico che viene utilizzata per realizzare sistemi capaci di apprendere ed adattarsi all'ambiente che li circonda, inizialmente totalmente o parzialmente ignoto, grazie ad una reward, detta rinforzo, che consiste nella valutazione qualitativa delle loro prestazioni. L'apprendimento avviene mediante l'interazione con l'ambiente ed è funzione del raggiungimento di uno o più obiettivi.

\subfile{\main/chapters/appendix/risposte_reinforcement/attori}

\subsection{Cosa rappresenta la critica?}
La \textbf{critica} nell'apprendimento è rappresentata da tutti quei processi che valutano a posteriori le azioni prese dall'agente. Essa fornisce un rinforzo secondario, interno ed a lungo termine interpretato come \textit{cost-to-go} da ogni stato fino al goal che consiste nella \textbf{value function} e permette di valutare se la \textit{policy} stia dando buoni risultati.

\subsection{Che tipo di architettura si può ipotizzare nell'apprendimento con rinforzo?}
La tipica architettura di un agente è costituita da 4 elementi:
\begin{description}
  \item[Memoria interna] Tiene traccia dello stato dell'ambiente.
  \item[Processore] Un componente che, dato uno stato, compie una serie di elaborazioni che hanno lo scopo di trovare la migliore azione in quello stato.
  \item[Sensori] Servono per consentire all'agente di percepire le caratteristiche del mondo esterno.
  \item[Attuatori] Servono all'agente per agire nell'ambiente e provocare quindi dei cambiamenti.
\end{description}

\subsection{Condizionamento classico e condizionamento operante}
\begin{description}
  \item[Condizionamento classico] Il rinforzo viene eseguito istante per istante o azione per azione e permette di ottenere un riscontro \textbf{ad ogni azione eseguita} o ad ogni variazione dell'ambiente o dello stato.
  \item[Condizionamento operante] Il rinforzo avviene "una-tantum", viene quindi valutata \textbf{una catena di azioni}, un comportamento nel suo insieme e non nella singola azione.
\end{description}

\subsubsection{Quale relazione c'è con l'intelligenza?}
L'apprendimento con rinforzo imita, con il processo di esecuzione delle azioni e valutazione successiva, il meccanismo di apprendimento dell'intelligenza umana che, tramite un processo di trial and error, migliora il proprio comportamento.

\subsection{Come potreste illustrare: Exploration vs Exploitation?}
Sono due strategie che consistono nel:
\begin{description}
  \item[Exploration] Provare varie azioni per scoprire nuove possibili azioni ed esplora lo spazio delle azioni per scoprire quelle migliori.
  \item[Exploitation] Scegliere sempre la soluzione che garantisca il miglior reward tra quelle conosciute (politica greedy).
\end{description}
La soluzione ottima viene identificata bilanciando opportunamente queste due strategie.

Il parametro utilizzato tipicamente per indicare la probabilità di scegliere una strategia piuttosto che l'altra è $\epsilon$.

\subsection{Cos'è il problema del credit assignment? È un problema che riguarda la dimensione temporale o spaziale del task?}
I sistemi a singolo agente estesi nel tempo hanno il problema di valutare il contributo di un'azione rispetto alle altre (\textbf{temporal credit assignment}), che viene accentuato se le azioni rilevanti sono temporalmente distanti tra di loro con un intermezzo di azioni poco rilevanti: questo può portare a cattive valutazioni.

I sistemi multi-agente hanno inoltre il problema di determinare il contributo di ogni agente ad un compito comune (\textbf{structural credit assignment}).

\subsection{Cos'è l'eligibility trace (traccia) e quale è il suo ruolo?}
L'eligibility trace è un buffer di memoria contenente tracce di eventi passati ed è utilizzata per gestire i reward ritardati nel tempo.

Gli eventi passati gradualmente perdono di importanza seguendo una legge esponenziale.

Quando si deve valutare il peso di uno stato su un aggiornamento di più step, la traccia dice se esso sia eleggibile.

Essa viene modellata come segue:

\[
  e_t(s,a) = \begin{cases}
    \gamma \lambda e_{t-1} (s,a) + 1 & \text{se } s=s_t \text{ e } a = a_i \\
    \gamma \lambda e_{t-1} (s,a)     & \text{altrimenti}
  \end{cases}
\]

\subsection{Definire l'algoritmo di Q-learning, descrivendo le equazioni opportune.}
L'algoritmo di Q-learning appartiene alla famiglia degli algoritmi di apprendimento da differenze temporali: esso basa l'aggiornamento della \textbf{value function} sulle informazioni relative allo step di esecuzione precedente, utile quando la conoscenza dell'ambiente è parziale.

Lo scopo dell'algoritmo è l'ottimizzazione della funzione valore $Q$ per ogni azione $a$ e stato $s$:

\[
  Q(s_t,a) = (1-\alpha)Q(s_t,a) + \alpha\rnd{r+\gamma\max_{a'}Q(s_{t+1},a')}
\]
I parametri coinvolti sono:
\begin{description}
  \item[$s$] Lo stato attuale.
  \item[$a$] L'azione scelta.
  \item[$s_{t+1}$] Lo stato raggiunto eseguendo l'azione $a$ nello stato attuale $s_t$.
  \item[$\alpha$] Lo step-size, il peso dell'aggiornamento, compreso in $[0,1]$, il parametro che indica quanto il vecchio ed il nuovo passaggio contribuiscono a Q.
  \item[$r$] Reward osservato dopo aver eseguito l'azione $a$ nello stato attuale $s_t$.
  \item[$\gamma$] Il fattore di sconto, compreso in $[0,1]$, che modella l'importanza di reward futuri.
  \item[$\max_{a'}Q$] è la migliore value function dello stato successivo.
\end{description}

\subsection{Scrivere le equazioni dell'algoritmo Q-learning in cui si consideri anche la traccia.}
Considerare anche la traccia va modificare lo step-size, cioè il modo in cui passaggi vecchi e nuovi contribuiscono al nuovo valore, che diviene:

\[
  Q(s_t,a) = (1-\alpha e(s_t,a))Q(s_t,a) + \alpha e(s_t,a)\rnd{r+\gamma\max_{a'}Q(s_{t+1},a')}
\]

\subsection{Cosa si intende per politica epsilon-greedy? Come entra nell'algoritmo di Q-learning? }
Si tratta di un particolare tipo di policy che si basa su di un ulteriore parametro $\epsilon \in [0,1]$. Ad ogni step, un agente che applica una politica $\epsilon$-greedy sceglie l'azione greedy (quella a massimo reward) con probabilità $1-\epsilon$, altrimenti sceglie un'altra azione in modo casuale. Si tratta quindi del parametro che indica la probabilità di scegliere una strategia esplorativa piuttosto che una exploitative.

\subsection{Che differenza c'è tra Q-learning e SARSA?}
La differenza è nel modo in cui viene aggiornato il valore di $Q$: SARSA è on-policy mentre Q-Learning è off-policy, ovvero:
\begin{description}
  \item[Off-policy] Q-Learning aggiorna il valore di $Q$ in funzione della migliore scelta futura possibile.
  \item[On-policy] SARSA aggiorna il valore di $Q$ in funzione della scelta che verrà applicata dalla policy.
\end{description}
Nella pratica Q-­learning converge più lentamente di SARSA, ma è in grado di imparare più rapidamente nel caso l'ambiente si modifichi in quanto SARSA continuerebbe a seguire la propria policy.

\subsection{Quale criterio si sceglie per definire i Reward? A quali elementi sono associati? Allo stato? All'azione? Allo stato prossimo? Perché?}
I comportamenti vengono scelti tenendo a mente quale comportamento si desidera ottenere dall'agente che deve essere ricompensato se non fallisce nel compimento di un'azione e se tale azione lo porta un stato più vicino al goal, penalizzato altrimenti.

Il reward è associato a stato corrente, azione e stato prossimo: nella maggior parte dei problemi di apprendimento reali le transizioni sono non deterministiche e bisogna quindi considerare anche l'eventualità che l'azione scelta non produca l'effetto sperato.

\subfile{\main/chapters/appendix/reinforcement/griglia}
\end{document}