\providecommand{\main}{../../}
\documentclass[\main/main.tex]{subfiles}
\begin{document}
\section{Domande sui Sistemi Fuzzy}
\subsection{Definire i passi per costruire un sistema fuzzy}
\begin{enumerate}
	\item Identifico le variabili di input e output del sistema specificando per ognuna il proprio \textbf{range}.
	\item Identifico le \textbf{classi fuzzy} in cui le variabili sono da suddividere e stabilisco le \textbf{funzioni di membership}.
	\item Definisco le regole logiche della FAM: per ogni combinazione di classi in input deve essere possibile definire una classe di output.
	\item Definisco la \textbf{modalità di defuzzyficazione}, che riconverte le regole attivate in un valore continuo tramite media pesata, massimo o media pesata su aree.
\end{enumerate}

\subsection{Cos'è un insieme fuzzy? Cos'è una membership function? Con quali altri nomi viene anche indicata?}
Un insieme fuzzy è un insieme caratterizzato da una \textbf{funzione di membership} M che, dato un elemento, restituisce il valore di verosimiglianza che esso appartenga al set, compreso tra 0 e 1.

\subsection{Esiste una corrispondenza biunivoca tra insiemi fuzzy e valori numerici?}
Gli insiemi fuzzy \textbf{non} godono di relazioni di univocità e biunivocità tra gli elementi di insiemi diversi. Pertanto gli insiemi fuzzy sono un'estensione ma non una generalizzazione degli insiemi della teoria classica.

\subsection{Distinzione tra fuzzyness e probabilità}
La \textbf{probabilità} riguarda un evento non ancora avvenuto e descrive l'incertezza che l'evento avvenga (Pioverà?). Una volta accaduto l'evento diviene certo, per cui non si parla più di probabilità, ma al più della vaghezza che contraddistingue l'evento (quanto l'evento è ``pioggia'' e quanto è ``sereno'').

La \textbf{fuzzyness} delinea un'incertezza, che è \textbf{deterministica}.

\subsection{La frase seguente: “con la mia preparazione potrei prendere 24 all'esame”, sottintende un processo fuzzy o probabilistico?}
Si tratta di un evento non ancora avvenuto e la frase sottolinea un'incertezza: si tratta quindi di un processo probabilistico.

\subsection{Cosa si intende per FAM? Una FAM memorizza numeri o preposizione logiche? Come?}
Una \textbf{Fuzzy Associative Memory} è il ``motore logico'' di un sistema fuzzy: essa associa allo spazio delle classi discrete in input le classi discrete in output seguendo regole di logica booleana applicate alle classi fuzzy in ingresso. Essa memorizza queste regole e la conoscenza del sistema e le applica in cascata con struttura if-else.

\subsection{Cosa è l'entropia fuzzy?}
Si tratta di una misura di quanto un determinato evento è fuzzy, date due classi. Si calcola come il rapporto tra la distanza dalla classe più vicina e la distanza dalla classe più lontana.

\subfile{\main/chapters/appendix/fuzzy/esempio_fuzzy}
\end{document}