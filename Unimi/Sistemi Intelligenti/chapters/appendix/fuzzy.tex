\providecommand{\main}{../../}
\documentclass[\main/main.tex]{subfiles}
\begin{document}

\section{Definire i passi per costruire un sistema fuzzy. [3]}
\begin{enumerate}
\item Struttura per tutti i modelli.
\begin{enumerate}
\item Identificazione delle variabili di I/O del sistema e del loro range (A,B).
\item Identificazione delle classi fuzzy in cui le variabili sono da suddividere e dei loro boundaries.
\item Definizione della trasformazione I/O come insieme di regole fuzzy: per ogni combinazione di classi fuzzy (con OR e/o AND) di input è possibile definire una classe di output (FAM).
\item Modalità di de-fuzzyfication.
\end{enumerate}
\item Funzionamento
\begin{enumerate}
\item Identificazione delle classi attivate da un certo insieme fuzzy in ingresso.
\item Valutazione del grado di fit delle classi.
\item Identificazione delle regole attivate.
\item Valutazione del grado di fit della regola.
\item Unione degli insiemi fuzzy di output risultanti e calcolo di un singolo valore numerico (defuzzyficazione).
\end{enumerate}
\end{enumerate}
\section{Cosa si intende per FAM? [3]}
\section{Una FAM memorizza numeri o preposizione logiche? Come? [3]}
\section{Definire un problema a piacere che involva almeno due variabili in ingresso e due in uscita. Definire tutti i componenti e calcolare l'uscita passo a passo per un valore di input a piacere. [3]}

\end{document}