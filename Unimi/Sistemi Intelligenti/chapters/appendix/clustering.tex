\providecommand{\main}{../../}
\documentclass[\main/main.tex]{subfiles}
\begin{document}

\subsection{Cosa si intende per clustering? In quali famiglie vengono divisi? [3]}
La classificazione non-supervisionata, più spesso chiamata \textit{clustering}, consiste nel separare un insieme di dati non etichettati in insiemi, i \textit{cluster}, internamente omogenei.
Per effettuare clustering esistono molti tipi di algoritmi, che si dividono principalmente in due classi: algoritmi gerarchici e algoritmi partizionali.

Gli algoritmi gerarchici organizzano il dataset in una struttura ad albero dividendo cluster troppo disomogenei (algoritmi divisivi) o unendo cluster simili tra loro (algoritmi agglomerativi).

Gli algoritmi partizionali impongono una suddivisione dello spazio delle \textit{feature} in più sottoinsiemi, che sono i cluster: se ogni \textit{pattern} può appartenere ad un solo cluster si parla di \textit{hard clustering}, altrimenti, se ogni pattern può appartenere a più cluster con un grado di \textit{membership} si parla di \textit{soft clustering} o \textit{fuzzy clustering}.

Alcuni algoritmi possono essere basati su teorie probabilistiche: si parla di algoritmi statistici.

\subsection{Che relazione c'è tra clustering e classificazione}
Mentre il clustering cerca di raggruppare in cluster a partire dalle feature di un gruppo di oggetti, la classificazione si occupa, date le classi, di assegnare ogni elemento di un gruppo di oggetti ad una classe.

\subsection{Quali sono le criticità? [3]}
Il clustering non è un problema ben posto, perché ci sono diversi gradi di libertà da fissare su come effettuare un clustering, per esempio variando algoritmo, calcolo di feature, misura di similarità il risultato può essere molto diverso. 

\end{document}
