\providecommand{\main}{../../..}
\documentclass[\main/main.tex]{subfiles}
\begin{document}
\chapter{Algoritmi}
\section{Metodo della panalità quadratica}
Questo metodo trasforma il problema vincolato in uno privo di vincoli, sfruttando una \textbf{penalty function}.

Si parte quindi da un problema vincolato:

\begin{align*}
    \min &f(\bmx)\\
    \bmh(\bmx) &= \bm{0}
\end{align*}

Si costruisce una \textbf{penalty function} utilizzando i vincoli:

\begin{figure}
    \[
        p(\bmx) = \sum_{j=1}^h h_j^2(\bmx)
    \]
    \caption{Penalty function}
\end{figure}

Il modello che si ottiene è quindi la somma pesata tra la funzione obbiettivo e la \textbf{penalty function}:

\[
    \min q(\bmx) = f(\bmx) + \alpha\sum_{j=1}^h h_j^2(\bmx)
\]

Aumentando il parametro \(\alpha \) a \(+\infty \), aumentiamo la penalità della violazione dei vincoli, aumentandone la severità.

Possiamo usare su questo problema le tecniche tratte dall'ottimizzazione svincolata.

\section{Metodo delle barriere}

\end{document}