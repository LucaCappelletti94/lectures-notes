\providecommand{\main}{../..}
\documentclass[\main/main.tex]{subfiles}
\begin{document}
\chapter{Alcune definizioni base}
\section{La convessità}
\begin{definition}[Insieme convesso]
	Un insieme \(X \subset \R^n\) è convesso se comunque presi due punti \(\bmx,\bmy \in X\), allora \(\lambda\bmx + (1-\lambda)\bmy \in X\), per ogni \(\lambda \in \sqr{0,1}\).

	La proprietà di convessità è invariante rispetto alle operazioni di moltiplicazione con uno scalare, unione e intersezione con un altro insieme convesso.
\end{definition}
\begin{definition}[Funzione convessa]
	Una funzione \(f: \R^n \rightarrow \R \) è convessa se il suo dominio è un insieme convesso \(X \subseteq \R^n\) e comunque presi due punti \(\bmx, \bmy \in X\) vale la relazione:
	\[
		f(\lambda\bmx + (1-\lambda)\bmy) \leq \lambda f(\bmx) + (1-\lambda)f(\bmy) \qquad \forall \lambda \in \sqr{0,1}
	\]

	La proprietà di convessità è invariante rispetto a moltiplicazione con uno scalare e somma tra funzioni convesse.

	Vale inoltre che la funzione \(\max \) di una o più funzioni convesse e che il luogo dei punti \(\bmx \) per i quali vale che \(f(\bmx) \leq \alpha \) è convesso.
\end{definition}
\begin{definition}[Problema convesso]
	Un problema di ottimizzazione con funzione obiettivo e regione ammissibile entrambe convesse viene detto problema convesso.
\end{definition}
\section{Massimi e minimi locali}
\begin{definition}[Minimo globale]
	Un punto \(\bmx^* \in X\) è un punto di minimo globale di \(f(\bmx)\) se:
	\[
		f(\bmx^*) \leq f(\bmx) \quad \forall \bmx \in X
	\]
\end{definition}
\begin{definition}[Minimo locale]
	Un punto \(\bmx^* \in X\) è un punto di minimo locale di \(f(\bmx)\) se esiste un intorno aperto \(I(\bmx^*, \epsilon)\) di \(\bmx^*\) avente raggio \(\epsilon >0\) tale che:
	\[
		f(\bmx^*) \leq f(\bmx) \quad \forall \bmx \in X \cap I(\bmx^*, \epsilon)
	\]
\end{definition}

\chapter{Ottimizzazione non vincolata}
\begin{definition}[Direzione di discesa]
	Data una funzione \(f:\R^n \rightarrow \R \), un vettore \(\bmd \in \R^n\) si dice direzione di discesa per \(f\) in \(\bmx \) se:
	\[
		\exists \lambda > 0: f(\bmx + \lambda\bmd) < f(\bmx)
	\]
\end{definition}
\begin{definition}[Derivata direzionale]
	Sia data una funzione \(f: \R^n \rightarrow \R \), un vettore \(\bmd \in \R^n\) e un punto dove \(f\) è definita. Se esiste il limite:
	\[
		\lim_{0^+} \frac{f(\bmx + \lambda\bmd) - f(\bmx)}{\lambda}
	\]
	allora tale limite prende il nome di derivata direzionale della funzione \(f\) nel punto \(\bmx \) lungo la direzione \(\bmd \)
\end{definition}

\section{Condizioni necessarie di ottimalità del 1° ordine}
\begin{theorem}[Condizioni necessarie di ottimalità del 1° ordine]
	Data una funzione \(f: \R^n \rightarrow \R \), derivabile in \(\bmx^* \in \R^n\), condizione necessaria affinchè il punto \(\bmx^* \) sia un minimo locale per \(f\) è che il gradiente della funzione calcolato in \(\bmx^*\) sia nullo.
\end{theorem}
\section{Condizioni necessarie di ottimalità del 2° ordine}
\begin{theorem}[Condizioni necessarie di ottimalità del 2° ordine]
	Data una funzione \(f: \R^n \rightarrow \R \) di classe \(\bm{C}^2(\bmx^*)\), condizione necessarie affinchè il punto \(\bmx^* \) sia un minimo locale per \(f\) è che il gradiente della funzione calcolato in \(\bmx^*\) sia nullo e che valga la relazione seguente:
	\[
		\bmd^T H (\bmx^*)\bmd \geq 0 \forall \bmd \in \R^n
	\]
	Cioè l'hessiana è definita come \textbf{semipositiva}.
\end{theorem}
\section{Condizioni necessarie di ottimalità in senso stretto del 2° ordine}
\begin{theorem}[Condizioni necessarie di ottimalità del 2° ordine]
	Data una funzione \(f: \R^n \rightarrow \R \) di classe \(\bm{C}^2(\bmx^*)\), condizione necessarie affinchè il punto \(\bmx^* \) sia un minimo locale in \textbf{senso stretto} per \(f\) è che il gradiente della funzione calcolato in \(\bmx^*\) sia nullo e che valga la relazione seguente:
	\[
		\bmd^T H (\bmx^*)\bmd > 0 \forall \bmd \in \R^n
	\]
	Cioè l'hessiana è definita come \textbf{positiva}.
\end{theorem}

\chapter{Programmazione quadratica}
Nella programmazione quadratica si approssima \(f\) con il seguente modello quadratico:

\[
	\min f(\bmx) = \frac{1}{2}\bmxt Q\bmx  \bmbt \bmx
\]
Esistono pochi casi possibili:

\begin{description}
	\item[\(Q\) non è semidefinita positiva] \(f\) non ha un punto di minimo.
	\item[\(Q\) è definita positiva] \(\bmx^* = Q^{-1}\bmb \) è l'unico minimo globale.
	\item[\(Q\) è definita semi-positiva]
	      \begin{description}
		      \item[\(Q\) non è singolare]  \(\bmx^* = Q^{-1}\bmb \) è l'unico minimo globale.
		      \item[\(Q\) è singolare] non esiste una soluzione o esistono infinite soluzioni.
	      \end{description}
\end{description}

\end{document}