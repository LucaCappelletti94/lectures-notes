\providecommand{\main}{../../..}
\documentclass[\main/main.tex]{subfiles}
\begin{document}
\chapter{SCIP}
\href{http://scip.zib.de/#scipoptsuite}{SCIP} è una libreria che permette di implementare rapidamente problemi lineari interi e di risolverli con \textbf{Branch-and-cut}.

\section{Installare SCIP}
Su sistemi \textbf{macOS} può essere installato in tre passi:

\begin{enumerate}
    \item Installa gmp con \mintinline{shell}{brew install gmp}
    \item Installa gcc 7 con \mintinline{shell}{brew install gcc@7}, rinominando eventualmente in gcc. Purtroppo attualmente SCIP supporta solo fortran 4 disponibile in gcc 7.
    \item Scarica SCIP da \url{http://scip.zib.de/#download} per il tuo sistema operativo.
    \item Estrai l'archivio e sposta i file nella tua \mintinline{shell}{/usr/local}.
\end{enumerate}

\section{Installare PySCIPOpt}
Si tratta di una libreria che consente di usare i metodi di SCIP comodamente da python. Dopo aver installato \textbf{SCIP} è sufficiente eseguire nel terminale per installarla \mintinline{shell}{pip install pyscipopt}.

\section{Problemi di installazione comuni}
\paragraph*{GCC non linkato}
Se quando esegui \mintinline{shell}{brew doctor} ti suggerisce di linkare \mintinline{shell}{gcc} potrebbe essere necessario eseguire i seguenti due comandi:

\begin{minted}{shell}
sudo chown -R $(whoami):admin /usr/local/share/man
brew link --overwrite gcc
\end{minted}

\chapter{Premessa}
Nei script presentati in seguito, si assumerà presente il seguente codice:

\begin{minted}{python}
import random
random.seed(42) # For reproducibility
\end{minted}

\begin{minted}{python}
def print_results(model, EPS=1.e-6):
    x,y = model.data
    edges = [(i,j) for (i,j) in x if model.getVal(x[i,j]) > EPS]
    facilities = [j for j in y if model.getVal(y[j]) > EPS]
    print("Optimal value=", model.getObjVal())
    print("Facilities at nodes:", facilities)
    print("Edges:", edges)
\end{minted}

\end{document}