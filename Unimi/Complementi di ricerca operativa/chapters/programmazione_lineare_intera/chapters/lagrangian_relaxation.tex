\providecommand{\main}{../../..}
\documentclass[\main/main.tex]{subfiles}
\begin{document}
\chapter{Dualità lagrangiana}
\section{Rilassamento lagrangiano}
Viene utilizzata per risolvere problemi di programmazione intera. Dato un problema con vincoli complessi:
\begin{align*}
    z = \max \bmct\bmx \\
    D\bmx &\leq \bmb \\
    \bmx &\in X
\end{align*}

Il seguente è il corrispendente \textbf{rilassamento lagrangiano}:

\begin{definition}{Rilassamento lagrangiano}
    Dato un vettore di moltiplicatori di Lagrange \(\bm{\lambda} \in \R^m, \bm{\lambda}\geq 0\), si definisce \textbf{rilassamento lagrangiano} del problema \(P\) il seguente problema di ottimizzazione:
    \begin{figure}
        \begin{subfigure}{0.49\textwidth}
            \begin{align*}
                z = \max \bmct\bmx \\
                D\bmx &\leq \bmb \\
                \bmx &\in X
            \end{align*}
            \caption{Problema \(P\)}
        \end{subfigure}
        \begin{subfigure}{0.49\textwidth}
            \begin{align*}
                z_{RL} = \max \bmct\bmx - \bm{\lambda}^T(D\bmx-\bmb) \\
                C\bmx &\leq \bmd \\
                \bmx &\in X
            \end{align*}
            \caption{Rilassamento lagrangiano di \(P\)}
        \end{subfigure}
    \end{figure}
\end{definition}

\end{document}