\providecommand{\main}{../../..}
\documentclass[\main/main.tex]{subfiles}
\begin{document}
\chapter{Algoritmi dei piani di taglio}

\begin{definition}[Poliedro]
    Si definisce poliedro un insieme convesso del tipo:
    \[
        \text{conv}(X) = \crl{\bmx: \matr{A}\bmx \leq \bmb, \bmx \geq 0}
    \]\end{definition}
\begin{definition}[Disuguaglianza valida]
    Una disuguaglianza \(\bm{\pi}\bmx \leq \bm{\pi}_0\) è una disuguaglianza valida per \(X \subseteq \R^n\) se \(\bm{\pi}\bmx \leq \bm{\pi}_0 \quad \forall \bmx \in X\).
\end{definition}

\begin{proposition}[Disuguaglianza valida per un problema]
    Una disuguaglianza \(\bm{\pi}\bmx \leq \bm{\pi}_0\) è valida per \(P = \crl{\bmx: \matr{A}\bmx \leq \bmb, \bmx \geq 0} \neq \emptyset \) se e solo se:
    \begin{enumerate}
    \item \(\exists \bmu \geq 0, \bmv \geq 0: \bmu \matr{A} - \bmv = \bm{\pi} \quad \land \quad \bmu\bmb \leq \bm{\pi}_0\), oppure
    \item \(\exists \bmu \geq 0: \bmu \matr{A} \geq \bm{\pi} \quad \land \bmu\bmb \leq \bm{\pi}_0\)
    \end{enumerate}
\end{proposition}

\begin{proposition}[Disuguaglianza valida per un problema intero]
    Sia \(X = \crl{\bmy \in \Z^1: \bmy \leq \bmb}\). Allora la diseguaglianza \(\bmy \leq \floor{\bmb}\) è valida per \(X\).
\end{proposition}

\section{Procedura di Chvatal-Gomory per la costruzione di disuguaglianze valide}
Dato un set \(X = P \cup \Z^n\), dove \(P = \crl{\bmx \in \R^n_+: \matr{A}\bmx \leq \bmb}\), \(\matr{A}\) una matrice \(n\times m\) e \(\bmu \in \R^m_+\), vale che:

La disuguaglianza \(\sum_{j=1}^n ua_j x_j \leq \bmu\bmb \) è valida per \(\matr{P}\) con \(\bmu \geq 0\) e \(\sum_{j=1}^n a_j x_j \leq \bmb \).

La disuguaglianza \(\sum_{j=1}^n \floor{ua_j} x_j \leq \bmu\bmb \) è valida per \(\matr{P}\) con \(\bmx \geq 0\).

La disuguaglianza \(\sum_{j=1}^n \floor{ua_j} x_j \leq \bmu\bmb \) è valida per \(X\) con \(\bmx \in \Z^n_+\), e quindi \(\sum_{j=1}^n \floor{ua_j} x_j\) è un intero.

\begin{theorem}
    Ogni disuguaglianza valida per \(X\) può essere ottenuta applicando la procedura di Chvatal-Gomory per un numero finito di volte.
\end{theorem}
\clearpage
\section{Algoritmo dei piani di taglio}
Un generico algoritmo dei piani di taglio per un problema intero del tipo \(\max\crl{\bmc\bmx: \bmx \in X}\) che generi disuguaglianze utili da una certa famiglia \(\mathcal{F}\) è della forma seguente:

\subsection{Inizializzazione}
Il termine dell'iterazione \(t\) viene inizializzato a zero ed il problema intero elaborato inizialmente coincide con quello originale: \(P^0 = P\).

\subsection{Iterazione}
Si va a risolvere il seguente problema lineare:
\[
    \bar{z}^{(t)} = \max\crl{\bmc\bmx: \bmx \in P^{(t)}}
\]
Sia quindi \(\bmxo^{(t)}\) la soluzione ottima ottenuta:

\begin{enumerate}
    \item Se \(\bmxo^{(t)} \in \Z^n\) allora l'algoritmo si interrompe e \(\bmxo^{(t)}\) è una soluzione ottima della programmazione intera.
    \item Se \(\bmxo^{(t)} \notin \Z^n\) si risolve il problema di separazione per \(\bmxo^{(t)}\) e \(\mathcal{F}\).
    \begin{enumerate}
        \item Se viene identificata una disuguaglianza \(\rnd{\bm{\pi}^{(t)}, \bm{\pi}^{(t)}_0} \in \mathcal{F}: \bm{\pi}^{(t)}\bmxo^{(t)} > \bm{\pi}^{(t)}_0\) tale che viene tagliata la soluzione \(\bmxo^{(t)}\), allora essa viene aggiunta ai vincoli del problema: \(P^{(t+1)} = P^{(t)} \cap \crl{\bmx: \bm{\pi}^{(i)}\bmxo^{(i)} > \bm{\pi}^{(i)}_0} \).
        \item Se non viene identificata, termina.
    \end{enumerate}
\end{enumerate}

\subsection{Considerazioni sul risultato}
Se l'algoritmo termina senza aver identificato una soluzione intera, il problema risultante \(P^{(t)}\) rimane comunque un buon punto di partenza per eseguire un algoritmo branch-and-bound.
\clearpage
\section{Algoritmo dei piani di taglio frazionari di Gomory}
Considerato un problema intero del tipo \(\max\crl{\bmc\bmx: \matr{A}\bmx = \bmb, \bmx \in \N^n}\), si procede ad identificare una base ottima \(\matr{B}\) del rilassamento continuo, si sceglie una variabile di base \(x_{\matr{B}_u}\) che sia frazionaria e quindi si procede a generare una disuguaglianza di Chvatal-Gomory associata con questa variabile in modo tale da tagliare la soluzione dalla regione ammissibile, scegliendo la riga della matrice dei vincoli \(\matr{A}\) tale per cui un coefficiente \(\bar{a}_{uj}\) non sia intero (deve esistere, altrimenti la soluzione ottima sarebbe intera). Se la base fosse priva di variabili frazionarie ovviamente si avrebbe già identificato la soluzione ottima del problema intero.

\begin{figure}
    \begin{subfigure}{0.33\textwidth}
        \[
            x_{\matr{B}_u} + \sum_{i \in \abs{\matr{B}}} \floor{\bar{a}_{uj}}x_j \leq \floor{\bar{a}_{u0}}
        \]        \caption{Taglio di Chvatal-Gomory per la riga \(u\)}
    \end{subfigure}
    \begin{subfigure}{0.33\textwidth}
        \[
            \sum_{i \in \abs{\matr{B}}} (\bar{a}_{uj} - \floor{\bar{a}_{uj}})x_j \leq \bar{a}_{u0} - \floor{\bar{a}_{u0}}
        \]        \caption{Taglio di Chvatal-Gomory per la riga \(u\) eliminando la variabile \(x_{\matr{B}_u}\)}
    \end{subfigure}
    \begin{subfigure}{0.33\textwidth}
        \[
            \sum_{i \in \abs{\matr{B}}} f_{uj}x_j \leq f_{u0}
        \]        \caption{Taglio di Chvatal-Gomory ponendo \(f_{uj} = \bar{a}_{uj} - \floor{\bar{a}_{uj}}\) }
    \end{subfigure}
\caption{Esempi di tagli di Chvatal-Gomory (equivalenti)}
\end{figure}

\subsection{Variabile di slack}
La variabile di slack \(s\) di un taglio di Chvatal-Gomory si ottiene come:

\begin{figure}
    \[
        s = -f_{u0} + \sum_{j \in \abs{\matr{B}}} f_{uj}x_{j}
    \]    \caption{Variabile di slack di un taglio di Chvatal-Gomory}
\end{figure}

\begin{proposition}[Taglio di Gomory e disuguaglianza di Chvatal-Gomory]
    Sia \(\beta \) la riga \(u\)-esima di \(\matr{B}^{-1}\) e \(q_i = \beta_i - \floor{\beta_i} \forall i\). Il taglio di Gomory \(\sum_{i \in \abs{\matr{B}}} f_{uj}x_j \leq f_{u0}\), quando scritto in termini della variabile originale coincide con la disuguaglianza di Chvatal-Gomory:
    \[
        \sum_{j=1}^n \floor{qa_j}x_j \leq \floor{qb}
    \]\end{proposition}
\clearpage
\section{Tagli interi misti}
\begin{proposition}[Disuguaglianza intera mista semplice]
    Sia \(X^{\geq} = \crl{(x,y) \in \R^1_+ \times \Z^1: x+y \geq b}\) e sia \(f = b - \floor{b} > 0\). La disuguaglianza:
    \[
        x \geq f(\ceil{b}-y) \quad \lor \quad \frac{x}{f} + y \geq \ceil{b}
    \]    risulta valida per \(X^{\geq}\)
\end{proposition}

\begin{corollary}
    Se \(X^{\leq} = \crl{(x,y) \in \R^1_+ \times \Z^1: y \leq b + x}\) e sia \(f = b - \floor{b} > 0\), la disuguaglianza
    \[
        y \leq \floor{b} + \frac{x}{1-f}
    \]    È valida per \(X^{\leq}\).
\end{corollary}

\begin{proposition}[Disuguaglianza intera arrotondata mista (MIR)]
    Questa variante della disuguaglianza semplice si ottiene considerando il set:
    \[
        X^{MIR} = \crl{(x,y) \in \R^1_+ \times \Z^2_+: a_1y_1 + a_2y_2 \leq b+x}
    \]    Dove \(a_1, a_2\) e \(b\) sono degli scalari con \(b\) frazionario.

    Sia \(f = b - \floor{b}\) ed \(f_i = a_i - \floor{a_i} \quad i \in \crl{1,2}\). Se \(f_1 \leq f \leq f_2\), allora:
    \[
        \floor{a_1}y_1 + \rnd{\floor{a_2} + \frac{f_2 - f}{1-f}}y_2 \leq \floor{b} + \frac{x}{1-f}
    \]    È valida per \(X^{MIR}\)
\end{proposition}

\begin{proposition}[Taglio di Gomory intero misto]
    Dato un tableau di un problema intero, contenente una \textbf{variabile di base frazionaria} \(y_{\matr{B}_u}\) che quindi genera un set del tipo:
    \[
        X^{(G)} = \crl{(y_{\matr{B}_u}, \bmy, \bmx) \in \Z^1 \times \Z^{n_1}_+ \times \Z^{n_2}_+: y_{\matr{B}_u} + \sum_{j \in N_1} \bar{a}_{uj}y_j + \sum_{j \in N_2} \bar{a}_{uj}x_j = \bar{a}_{u0}}
    \]    Se \(\bar{a}_{u0} \notin \Z^1, f_j = \bar{a}_{uj} - \floor{\bar{a}_{uj}} \forall j \in N_1 \cup N_2\) e \(f_0 = \bar{a}_{u0} - \floor{\bar{a}_{u0}}\) il taglio di Gomory misto:

    \[
        \sum_{f_j\leq f_0} f_j y_j + \sum_{f_j > f_0} \frac{f_0(1-f_j)}{1-f_0}y_j + \sum_{\bar{a}_{uj}>0} \bar{a}_{uj} x_j + \sum_{\bar{a}_{uj}<0} \frac{f_0}{1-f_0}\bar{a}_{uj} x_j \geq f_0
    \]    È valida per \(X^G\). La formula si ottiene calcolando la MRI per il set e sostituire la variabile \(y_{\matr{B}_u}\) con la forma in \(f_j\).
\end{proposition}

\end{document}