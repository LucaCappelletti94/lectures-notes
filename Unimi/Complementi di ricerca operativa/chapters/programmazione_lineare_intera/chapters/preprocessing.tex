\providecommand{\main}{../../..}
\documentclass[\main/main.tex]{subfiles}
\begin{document}
\chapter{Preprocessing}
\section{Qualità della formulazione}
\subsection{Programmazione lineare}
Solitamente in un problema di programmazione lineare una buona metrica per la qualità della formulazione del problema è la dimensione della matrice caratteristica: una matrice più piccola è indice di un problema formulato meglio.

\subsection{Programmazione intera}
Per un problema di programmazione intera o intera mista però, la dimensione della matrice non risulta più essere una metrica valida:
\begin{enumerate}
    \item Variabili intere e continue non possono essere considerate equamente quando si va a misurare la qualità della formulazione: variabili intere tendono ad essere molto più difficili da risolvere che con variabili continue.
    \item Il grado di difficoltà aumenta esponenzialmente con il numero delle variabili.
    \item Una problema intero diventa più facile da risolvere, non più difficile (come viene fatto nell'algoritmo dei piani di taglio).
\end{enumerate}

In questi problemi il criterio per misurare la qualità va a basarsi sul poliedro della regione ammissibile del problema rilassato.

L'idea fondamentale del \textbf{preprocessing} è di riformulare il problema in modo tale da minimizzare la differenza dei valori tra la funzione originale intera e quella rilassata.

\begin{definition}[Poliedro]
    L'insieme di tutti i punti (o soluzioni) che soddisfa un set di vincoli lineari del tipo:
    \[
        P = \crl{\bmx: \matr{A}\bmx \leq \bmb, \bmx \text{ continuo}}
    \]    È detto \textbf{poliedro}.
\end{definition}

\begin{definition}[Formulazione]
    Dato un insieme \(S_y = \crl{\bmy \in \Z^p: \matr{G}\bmy\leq \bmb, \bmy \text{ intero}}\), un poliedro \(P \subseteq E^p\) è una \textbf{formulazione} per \(S_y\) se e solo se \(S_y \subseteq P\).

    Questo significa che una formulazione di un problema intero deve essere un poliedro definito sullo stesso spazio reale \(p\)-dimensionale e deve contenere lo stesso numero di punti ammissibili.
\end{definition}

\begin{definition}[Formulazione di un problema intero]
    Dato \(S_{xy} = \crl{(\bmx, \bmy): \matr{A}\bmx + \matr{G}\bmy \leq \bmb, \bmx \in E^n, \bmy \in \Z^p}\), un poliedro \(P\subseteq E^{n+p}\) sono una \textbf{formulazione} per \(S_{xy}\) se \(S_{xy} \subseteq P\).
\end{definition}

\begin{definition}[Formulazione migliore]
    Date due formulazioni \(P_1, P_2\) per \(S_y\), \(P_1\) è una \textbf{formulazione migliore} di \(P_2\) se \(P_1 \subseteq P_2\).
\end{definition}

\begin{definition}[Formulazione ideale]
    Data una formulazione \(S = \crl{\bmy: \matr{G}\bmy \leq \bmb, \bmy \text{ intero}}\), questa è detta \textbf{ideale} se tutti i vertici del poliedro corrispondente sono \textbf{interi}.
\end{definition}

\section{Migliorare i vincoli in un problema intero}
Supponiamo da iniziare da un vincolo del tipo \(\bm{a}^T\bmy\leq b \). Si procede come segue:

\begin{enumerate}
    \item Definisco \(M = \sum_{a_j>0}a_j\) e \(S = \crl{a_j: \abs{a_j}>M-b}\).
    \item Se il set \(S=\emptyset \) allora il vincolo non è migliorabile.
    \item Altrimenti seleziono un termine \(a_k \in S\):
    \begin{enumerate}
        \item Se \(a_k>0\) allora aggiorno \(\bar{a}_k = M - b, \bar{b} = M-a_k\).
        \item Altrimenti aggiorno \(\bar{a}_k = b - M\).
    \end{enumerate}
    \item Ripeto l'operazione dal punto 1.
\end{enumerate}



\end{document}