\providecommand{\main}{../../}
\documentclass[\main/main.tex]{subfiles}
\begin{document}

\section{Problema di assegnamento}
Dati $n$ lavoratori, $n$ attività e considerando maggiore o uguale di zero il tempo impiegato dal lavoratore $i$ per svolgere l'attività $j$ ($t_{ij}>0$), assegnare a ciascun lavoratore una ed una sola attività in modo che tutte le attività vengano svolte.

\paragraph{Obbiettivo:} minimizzare la somma dei tempi impiegati per svolgere le attività.

\paragraph{Variabili:} utilizzo solo una variabile booleana per indicare se il lavoro $i$-esimo è svolto dal lavoratore $j$-esimo:

\[
	x_{ij} = \begin{cases}
		1 \qquad \text{ se il lavoro } i \text{ è svolto da } j\\
		0 \qquad \text{altrimenti}
	\end{cases}
\]

\subsection{Modello}

\begin{figure}[H]
\[
	\min \sum_{i=1}^n \sum_{j=1}^n t_{ij}x_{ij}
\]
\caption{Funzione di cui calcolare il minimo, pari alla somma dei tempi per eseguire ogni azione}
\end{figure}

\begin{figure}[H]
 \centering
 \begin{minipage}{0.5\textwidth}
 \centering
 \[
	\sum_{i=1}^n x_ij = 1 \forall j = 1...n 
 \]
\caption{Ogni attività viene svolta da un lavoratore.}
 \end{minipage}\hfill
 \begin{minipage}{0.5\textwidth}
 \[
	\sum_{j=1}^n x_ij = 1 \forall j = 1...n 
 \]
\caption{Ogni lavoratore svolge un'attività.}
 \end{minipage}\hfill
\end{figure}


\end{document}