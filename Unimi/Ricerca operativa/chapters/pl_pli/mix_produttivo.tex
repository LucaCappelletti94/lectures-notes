\providecommand{\main}{../../}
\documentclass[\main/main.tex]{subfiles}
\begin{document}

\section{Problema del mix produttivo}
Dato un sistema produttivo caratterizzato da:

\begin{enumerate}
\item $m$ risorse produttive limitate.
\item $b_j$, con $i=1...m$ quantità massima della risorsa $i$. 
\item $n$ diversi prodotti che ottengo dalle risorse.
\item $a_{ij}$ assorbimento unitario di risorsa $i$ per il prodotto $j$ (quantità di risorsa $i$ che utilizzo per produrre un'unità di $j$).
\item $c_j$ profitto unitario per il prodotto $j$.
\end{enumerate}

Sia data inoltre l'ipotesi aggiuntiva che tutta la produzione venga venduta e non sono costretto a produrre tutti i prodotti.
Si chiede di determinare quali prodotti produrre e in quali quantità.

\paragraph{Obbiettivo:} massimizzare il profitto complessivo.
\paragraph{Variabili: } definisco una variabile intera che rappresenta il numero di unità di prodotte di un determinato prodotto.
\[
	X_{j} \geq 0
\]

\subsection{Modello}
\begin{figure}[H]
 \centering
 \begin{minipage}{0.5\textwidth}
 \centering
 \[
	\max \sum_{i=1}^n x_jc_j
 \]
\caption{Funzione da massimizzare.}
 \end{minipage}\hfill
 \begin{minipage}{0.5\textwidth}
 \[
	\sum_{j=1}^n x_j a_{ij} \leq b_i \forall i = 1...m
 \]
\caption{Numero di unità per ogni prodotto.}
 \end{minipage}\hfill
\end{figure}

\subsection{Esempio 1}

\begin{center}
\begin{tabular}{c|c|c|c}
& Modello light & Modello plus & Ore uomo \\
\hline
Profitto unitario & 30 & 20 & \# \\
\hline
Assemblaggio & 8 & 4 & 640 \\ 
\hline
Finitura & 4 & 6 & 540 \\
\hline
Controllo qualità & 1 & 1 & 100
\end{tabular}
\end{center}


\end{document}