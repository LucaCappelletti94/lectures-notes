 \providecommand{\main}{..}
\documentclass[\main/main.tex]{subfiles}
\begin{document}

\section{Libri adottati}
\begin{enumerate}
  \item Lezioni di ricerca operativa (M. Fischetti)
  \item 120 esercizi di ricerca operativa (M. Dell'Amico)
\end{enumerate}

\section{Risorse extra}
È possibile ottenere le video lezione all'indirizzo \url{https://vc.di.unimi.it/?courseid=57}.

\section{Di cosa si occupa la Ricerca Operativa}
Vengono realizzati \textbf{modelli prescrittivi}, cioè modelli di problemi di ottimizzazione che ci suggeriscono cosa fare. La ricerca operativa affronta la risoluzione di processi decisionali complessi tramite modelli matematici ed algoritmi.

\section{Programmazione matematica}
Significa ottimizzare una funzione di più variabili, spesso soggette ad un insieme di vincoli $\text{min} f(x_1,..., x_n) \text{s.t.} \bm{x} \in \mathbb{X}$.

\subsection{Risoluzione di un problema di programmazione matematica}

\begin{enumerate}
  \item Analisi del problema e scrittura di un modello matematico.
  \item Definizione ed applicazione di un metodo di soluzione.
\end{enumerate}

A seconda del tipo di modello si utilizzano tipi di programmazione distinti (in grassetto quelle prese in considerazione in questo corso):

\begin{enumerate}
  \item \textbf{Programmazione lineare continua}
  \item \textbf{Programmazione lineare intera}
  \item \textbf{Programmazione booleana}
  \item Programmazione non lineare
  \item Programmazione stocastica
\end{enumerate}

\end{document}