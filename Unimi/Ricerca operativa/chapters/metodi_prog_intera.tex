 \providecommand{\main}{..}
\documentclass[\main/main.tex]{subfiles}
\begin{document}
\chapter{Metodi di programmazione lineare intera}

\section{Branch \& Bound}
Si tratta del metodo risolutivo per PI più comune, un po' come il simplesso per PL.

Si procede risolvendo molteplici rilassamenti continui di un problema LI usando il metodo del simplesso.

A ogni iterazione, può accadere che:

\begin{enumerate}
  \item L'iterazione $i-esima$ è inammissibile.
  \item Le variabili sono intere e pertanto non è necessario suddividere ulteriormente ma è stata identificata la soluzione ottima intera di questo ramo.
  \item La soluzione continua identificata è inferiore alla soluzione intera identificata in un altro ramo e quindi non ha senso esplorare questa regione ulteriormente.
  \item La soluzione intera ottenuta è abbastanza vicina alla soluzione continua ottima e si decide pertanto di interrompere l'algoritmo (per motivi di tempo).
  \item Le variabili sono continue e quindi suddivideremo ulteriormente la regione in due sotto problemi, per esempio se la variabile intera fosse $x = a$ andremmo a dividere tra $x \leq \floor{a}$ e $x \geq \ceil{a}$, quindi ripeteremo il calcolo della soluzione ottima con l'algoritmo del simplesso avendo aggiunti questi vincoli.
\end{enumerate}

\section{Algoritmo dei piani di taglio}
Banalmente si applicano iterativamente tagli di Chvatal-Gomory alla regione del problema (che sebbene inizialmente molto efficaci lo divengono sempre meno) per eliminare soluzioni non accettabili dal problema discreto ma ottime per il rilassamento continuo.

\begin{definition}[Taglio di Gomory]
  Ogni soluzione ammissibile di $P$ soddisfa la relazione:
  \[
    x_h + \sum_{j \in \mathbb{N}} \floor{a_{ij}}x_j \leq \floor{b_i}
  \]
\end{definition}


\end{document}