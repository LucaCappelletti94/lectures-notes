 \providecommand{\main}{../..}
\documentclass[\main/main.tex]{subfiles}
\begin{document}
\section{Condizioni di ottimalità}
\begin{theorem}[Condizioni di ottimalità]
  I due vettori $\bbmx \in \R^n$,  $\bbmy \in \R^m$ sono ottimi per i problemi primale $P$ e duale $D$ rispettivamente se valgono le seguenti proprietà:
  \begin{align*}
    \matr{A}\bbmx & \geq \bmb,\quad \bbmx \geq \bm{0}            &  & \text{condizioni di ammissibilità primale} \\
    \bmct         & \geq \bbmyt \matr{A},\quad \bbmy \geq \bm{0} &  & \text{condizioni di ammissibilità duale}   \\
    \bmct \bbmx   & = \bbmyt \bmb                                &  & \text{condizioni di ortogonalità}
  \end{align*}
\end{theorem}

\begin{proof}
  Segue dal teorema di dualità debole (teorema \ref{dualita_debole})
\end{proof}

\end{document}