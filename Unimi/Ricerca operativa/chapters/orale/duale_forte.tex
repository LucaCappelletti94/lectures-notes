 \providecommand{\main}{../..}
\documentclass[\main/main.tex]{subfiles}
\begin{document}
\section{Teorema di dualità forte}

\begin{theorem}[Dualità forte]
  \label{dualita_forte}
  Sia $P=\crl{\bmx\geq\bm{0}: \matr{A}\bmx\geq\bmb} \neq \emptyset$ con $\min\crl{\bmct\bmx: \bmx \in P}$ finito. Vale allora:

  \[
    \min\crl{\bmct\bmx: \matr{A}\bmx \geq \bmb, \bmx \geq \bm{0}} = \max\crl{\bmut\bmb: \bmct \geq \bmut \matr{A}, \bmu \geq \bm{0}}
  \]
  \paragraph*{Significato:} Se la soluzione ottima del problema primale esiste ed è limitata allora coincide con la soluzione ottima del problema duale.
\end{theorem}

\begin{proof}
  Segue direttamente dal lemma di Farkas (teorema \ref{lemma_farkas}).
\end{proof}
\end{document}