 \providecommand{\main}{../..}
\documentclass[\main/main.tex]{subfiles}
\begin{document}
\section{Teorema di convergenza del simplesso}

\begin{theorem}[Teorema di convergenza del simplesso]
  Usando la regola di Bland, l'algoritmo del simplesso converge in al più $\binom{n}{m}$ passi, cioè termina senza compiere cicli.
\end{theorem}

\begin{proof}
  Per assurdo supponiamo che, applicando la regola di Bland, l'algoritmo compia dei cicli. Consideriamo il più piccolo tableau che genera una sequenza ciclica pur applicando la regola di Bland, cioè tutte le variabili, a turno, entrano ed escono di base: $B_1 \rightarrow B_2 \rightarrow \ldots \rightarrow B_k \rightarrow B_1$.

  Consideriamo un tableau $T$ e supponiamo che $x_h$ entri in base e $x_n$ esca. Perché questo avvenga deve essere valido che:

  \begin{figure}
    \begin{subfigure}{0.49\textwidth}
      \begin{enumerate}
        \item $\bar{\bm{b}} = \bm{0}$, altrimenti uscirei dal ciclo.
        \item $\bar{c}_h < 0$, coefficiente che entra in base.
        \item $\bar{c}_{B[i]}$, in base.
        \item I rapporti $\crl{\frac{\bar{b}_j}{\bar{a}_{ih}}:\bar{a}_{ih}>0}$ sono tutti nulli, quindi uso la regola di Bland e scelgo il primo indice $i$ tale per cui il coefficiente del vincolo $\bar{a}_{ih}$ sia positivo.
        \item $\bar{a}_{ih} \leq 0$ e  $\bar{a}_{th} > 0$ perché con la regola di Bland di scelga $x_h$.
      \end{enumerate}
      \caption{Le condizioni iniziali}
    \end{subfigure}
    \begin{subfigure}{0.49\textwidth}
      \begin{table}
        \begin{tabular}{c|LcLLLc|L}
                       &  & $x_{B[i]}$ &  & x_h          &  & $x_n$ &   \\
          \hline
          $-z$         &  & 0          &  & \bar{c}_h    &  & 0     &   \\
          \hline

                       &  & 0          &  & \bar{a}_{1h} &  & 0     & 0 \\


          $x_{B[i]}$   &  & 1          &  & \bar{a}_{ih} &  & 0     & 0 \\


          $x_{B[t]=n}$ &  & 0          &  & \bar{a}_{th} &  & 1     & 0 \\


                       &  & 0          &  & \bar{a}_{mh} &  & 0     & 0 \\
        \end{tabular}
        \caption{Tableau iniziale $T$}
      \end{table}
    \end{subfigure}
    \caption{Condizioni iniziali di $T$}
  \end{figure}


  Considero ora $\tilde{T}$ quando, dopo qualche iterazione, rientra in base $x_n$ e quindi, siccome seguo la regola di Bland, $\tilde{c}_n$ deve essere l'unico coefficiente di costo ridotto negativo.

  La riga $0$ di $\tilde{T}$ è ottenibile come la riga $0$ di $T$ sommata a una combinazione lineare di righe di $T$ (figura \ref{T_linear_combination}), quindi è possibile ottenere il coefficiente di costo ridotto $\tilde{c}_n$ come una combinazione lineare dei coefficienti di vincolo (figura \ref{ccr_n_linear_combination}).

  \begin{figure}
    \begin{subfigure}{0.31\textwidth}
      \[
        \tilde{T}_0 = T_0 + \sum_{i=1}^m \mu_i T_i
      \]
      \caption{Comb. lin. della prima riga di $\tilde{T}_0$}
      \label{T_linear_combination}
    \end{subfigure}
    \begin{subfigure}{0.31\textwidth}
      \[
        \tilde{c}_n = \bar{c}_n + \sum_{i=1}^m \mu_i \bar{a}_{in} = \bar{c}_n + \mu_t
      \]
      \caption{Comb. lin. del CCN di base entrante $\tilde{c}_n$}
      \label{ccr_n_linear_combination}
    \end{subfigure}
    \begin{subfigure}{0.31\textwidth}
      \[
        \tilde{c}_{B[i]} = \bar{c}_{B[i]} + \sum_{j=1}^m \mu_j a_{jB[i]} = \bar{c}_{B[i]} + \mu_i  \text{ con } i \neq t
      \]
      \caption{Comb. lin. del CCN di base $\tilde{c}_{B[i]}$}
      \label{ccr_i_linear_combination}
    \end{subfigure}
    \caption{Combinazioni lineari}
  \end{figure}

  Ma $\bar{c}_n$ in $T$ è nullo e l'unico coefficiente di vincolo $\bar{a}_{in}$ non nullo è $\bar{a}_{tn}$ che è pari a $1$. Ne segue che il coefficiente della combinazione $\mu_t$ deve essere negativo se $\tilde{c}_n$ deve essere negativo.

  Anche i coefficienti di costo in base sono ottenibili tramite combinazioni lineari (figura \ref{ccr_n_linear_combination}).

  Tutti i coefficienti $\tilde{c}_{B[i]}$, con $i \neq t$, devono essere non negativi poiché seguo la regola di Bland. $\bar{c}_{B[i]}$ è certamente nullo per la costruzione di $T$, in quanto è in base. Ne segue che $\mu_i$ deve essere non negativo.

  Il coefficiente di costo ridotto della variabile $x_h$ che viene rimossa dalla base deve essere positivo. Anche questo termine può essere ottenuto tramite una combinazione lineare di termini:

  \[
    \tilde{c}_h = \bar{c}_h + \sum_{i=1, i \neq t}^m \mu_i \bar{a}_{ih} + \mu_t\bar{a}_{th}
  \]

  Inizialmente avevamo scelto di inserire la variabile $x_h$ in base poiché per la regola di Bland valeva che $\bar{a}_{ih} \leq 0, i \neq t$, $\bar{a}_{th} > 0$ e $\bar{c}_h<0$.

  In base alle relazioni \ref{ccr_n_linear_combination} e \ref{ccr_i_linear_combination}, risulta che $\tilde{c}_h$ deve essere strettamente negativo, mentre, siccome viene rimossa dalla base la variabile $x_h$, $\tilde{c}_h$ dovrebbe essere non negativo, da cui l'assurdo.
\end{proof}

\end{document}