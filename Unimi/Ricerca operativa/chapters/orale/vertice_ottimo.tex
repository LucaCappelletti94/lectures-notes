 \providecommand{\main}{../..}
\documentclass[\main/main.tex]{subfiles}
\begin{document}
\section{Esistenza di un vertice ottimo}

\begin{theorem}[Esistenza di un vertice ottimo]
  Se l'insieme $P$ delle soluzioni ammissibili del problema di programmazione lineare $\min\crl{\bmct\bmx:\bmx\in P}$ è limitato, allora esiste almeno un vertice di $P$ ottimo.
  \paragraph*{Sinossi:} In ogni \textbf{politopo} con soluzione limitata esiste almeno un vertice ottimo.
\end{theorem}

\begin{proof}
  Siano $\bmy_1, \ldots, \bmy_k$ i vertici del politopo e sia $z^* = \min \crl{\bm{c}^T\bmy_i, i=1,\ldots,k}$, cioè il valore ottimo della funzione obbiettivo tra i vertici del politopo.

  Per il \textbf{teorema di Minkowsky-Weil} (teorema \ref{minkowsky}) posso descrivere $\bmx$ come una combinazione convessa dei vertici $\bm{x} = \sum_{i=1}^k \lambda_i \bmy_i$

  \begin{align*}
    z                                     & = \bmct \bmx                                  &  & \text{Iniziamo dalla funzione obbiettivo}                             \\
    z                                     & = \bm{c}^T\rnd{\sum_{i=1}^k \lambda_i \bmy_i} &  & \text{Sostituisco a $\bmx$ la comb. convessa equivalente}             \\
    z                                     & = \sum_{i=1}^k \lambda_i \bm{c}^T\bmy_i       &  & \text{Sposto il coefficiente dei costi nella serie}                   \\
    \sum_{i=1}^k \lambda_i \bm{c}^T\bmy_i & \geq \sum_{i=1}^k  \lambda_i z^* = z^*        &  & \text{$\bmct \bmy$ è sempre maggiore o uguale alla soluzione ottima.} \\
  \end{align*}
\end{proof}

\end{document}