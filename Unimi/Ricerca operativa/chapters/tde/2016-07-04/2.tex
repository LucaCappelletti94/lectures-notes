 \providecommand{\main}{../../..}
\documentclass[\main/main.tex]{subfiles}
\begin{document}
\subsection{Esercizio 2}
Si consideri il seguente \textbf{tableau} di \textbf{minimo}:

\begin{table}
  \begin{tabular}{|LLLL|L|}
    \hline
    B & E  & D & 4  & -1 \\
    \hline
    0 & -8 & 1 & -1 & 4  \\
    1 & A  & C & 6  & F  \\
    \hline
  \end{tabular}
  \caption{Tableau esercizio 2}
\end{table}

\begin{enumerate}[a)]
  \item Si dica per quali valori dei parametri il tableau è in forma canonica.
  \item Per quali dei parametri rimanenti il tableau corrisponde a:
        \subitem Un problema con ottimo finito.
        \subitem Una soluzione non ottima e degenere.
        \subitem Un problema illimitato.
\end{enumerate}

\subsection{Soluzione esercizio 2}
\subsubsection*{Forma canonica}
Le uniche colonne che possono essere variabili in base son quelle indicate dai parametri $B$ e $D$, che devono quindi essere zero, e le colonne devono formare una matrice identità. La base scelta, quindi, per essere ammissibile, deve essere tale che $\tilde{b} = B^{-1} b \geq 0$, quindi $F\geq 0$.

\begin{table}
  \begin{tabular}{|LLLL|L|}
    \hline
    0 & E  & 0 & 4  & -1 \\
    \hline
    0 & -8 & 1 & -1 & 4  \\
    1 & A  & 0 & 6  & F  \\
    \hline
  \end{tabular}
  \caption{Forma canonica}
\end{table}

\subsubsection*{Problema con ottimo finito}
Un problema è limitato se non esistono variabili fuori base per cui coefficiente di costo e coefficienti sono negativi. Risulta quindi sufficiente dare valore non negativo a $E$ o $A$ od entrambi. In particolare, dando valore non negativo ad $E$ si garantisce per il \textbf{criterio di ottimalità} che la soluzione data è ottima.

\subsubsection*{Soluzione non ottima e degenere}
Una soluzione è degenere quando una dei termini del vettore $\tilde{b}$ è pari a zero, quindi $F = 0$. Non è possibile definire quando una base degenere è ottima.

\subsubsection*{Problema illimitato}
Se, per una variabile fuori base $x_k$ si ha $\tilde{A}_k$ e coefficiente di costo ridotto per $k$ negativo, allora il problema è illimitato inferiormente, per cui:

\[
  E < 0 \quad \land \quad A < 0
\]

\end{document}