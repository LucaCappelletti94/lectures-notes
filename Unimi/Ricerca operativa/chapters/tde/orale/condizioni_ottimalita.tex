 \providecommand{\main}{../../..}
\documentclass[\main/main.tex]{subfiles}
\begin{document}
\section{Condizioni di ottimalità}
\begin{theorem}[Condizioni di ottimalità]
  I due vettori $\bbmx \in \R^n$,  $\bbmy \in \R^m$ sono ottimi per i problemi primale $P$ e duale $D$ rispettivamente se valgono le seguenti proprietà:
  \begin{enumerate}
    \item $\bbmx \in P: A\bbmx \geq \bmb, \bbmx \geq \bm{0}$ (ammissibilità primale)
    \item $\bbmy \in D: \bmct \geq \bbmyt A, \bbmy \geq \bm{0}$ (ammissibilità duale)
    \item $\bmct \bbmx = \bbmyt \bmb$ (condizioni di ortogonalità)
  \end{enumerate}
\end{theorem}

\begin{proof}
  Segue dal teorema di dualità debole (teorema \ref{dualita_debole})
\end{proof}

\end{document}