 \providecommand{\main}{../../..}
\documentclass[\main/main.tex]{subfiles}
\begin{document}
\section{Condizioni di ottimalità}
\begin{minipage}{\textwidth}
  \begin{minipage}{.83\textwidth}
    \flushleft
    \begin{theorem}[Condizioni di ottimalità]
      I due vettori $\bbmx \in \R^n$,  $\bbmy \in \R^m$ sono ottimi per i problemi primale $P$ e duale $D$ rispettivamente se valgono le seguenti proprietà:
      \begin{enumerate}
        \item $\bbmx \in P$ (ammissibilità primale)
        \item $\bbmy \in D$ (ammissibilità duale)
        \item $\bmct \bbmx = \bbmut \bmb$ (condizioni di ortogonalità)
      \end{enumerate}
    \end{theorem}
  \end{minipage}\hfill
  \begin{minipage}{0.15\textwidth}\center
    \includegraphics[width=\textwidth]{random/vinegar}
  \end{minipage}
\end{minipage}

\begin{proof}
  Segue dal teorema di dualità debole (teorema \ref{dualita_debole})
\end{proof}

\end{document}