 \providecommand{\main}{../../..}
\documentclass[\main/main.tex]{subfiles}
\begin{document}
\section{Teorema di dualità forte}
\begin{theorem}[Dualità forte]
  \label{dualita_forte}
  Sia $P = \crl{\bmx \geq \bm{0}: A\bmx \geq \bmb} \neq \emptyset$ con soluzione ottima \textbf{finita}. Allora la soluzione ottima del primale coincide con la soluzione ottima del duale.

  \begin{figure}
    \[
      \min\crl{\bmct\bmx: A\bmx \geq \bmb, \bmx \geq \bm{0}} = \max\crl{\bmut\bmb: \bmct \geq \bmut A, \bmu \geq \bm{0}}
    \]
    \caption{Teorema di dualità forte: la soluzione ottima primale coincide con la soluzione ottima duale}
  \end{figure}
\end{theorem}

\begin{proof}
  Segue direttamente dal lemma di Farkas (\ref{lemma_farkas}).
\end{proof}
\end{document}