 \providecommand{\main}{../../..}
\documentclass[\main/main.tex]{subfiles}
\begin{document}
\section{Criterio di ottimalità della base}
\begin{theorem}[Criterio di ottimalità della base]
  Sia $B$ una base ammissibile. Se $\bbmct = \bmct - \bmct_B B^{-1}A\geq \bm{0}^T$ allora la soluzione di base associata a $B$ è ottima, cioè i coefficienti di costo ridotto dei termini fuori base sono non negativi.

  Inoltre, se $B$ è ottima e non degenere, i coefficienti di costo ridotto sono non negativi: $\bbmct\geq\bm{0}^T$.

  \textbf{N.B.}: nel caso di basi degeneri non è possibile fare affermazioni sull'ottimalità.
\end{theorem}

\begin{proof}
  Riscrivendo la funzione obbiettivo come $\bmct\bmx = \bmct_B B^{-1}\bmb + \bbmct \bmx$, nell'ipotesi che $\bbmct \geq \bm{0}^T$ si ha che:
  \[
    \bmct\bmx \geq \bmct_BB^{-1}\bmb \quad \forall \bmx \geq \bm{0}
  \]
  e quindi vale che, $\forall \bmx \in P$, la funzione obbiettivo in corrispondenza della soluzione di base ammissibile associata a $B$ assome il valore $
    \bmct_BB^{-1}\bmb = \begin{bmatrix}
      \bmct_B & \bmct_F
    \end{bmatrix}
    \begin{bmatrix}
      B^{-1}\bmb \\
      0
    \end{bmatrix}
  $.

\end{proof}

\end{document}