 \providecommand{\main}{../../..}
\documentclass[\main/main.tex]{subfiles}
\begin{document}
\section{Criterio di ottimalità della base}
\begin{minipage}{\textwidth}
  \begin{minipage}{.83\textwidth}
    \flushleft
    \begin{theorem}[Criterio di ottimalità della base]
      Sia $B$ una base ammissibile e sia $\bmct = \begin{bmatrix}
          \bmct_B, \bmct_F
        \end{bmatrix}$. Se $\bbmct = \bmct - \bmct_B B^{-1}A\geq \bm{0}^T$ allora la soluzione di base associata a $B$ è ottima, cioè i coefficienti di costo ridotto dei termini fuori base $\bbmct$ sono non negativi.

      Inoltre, se $B$ è ottima e non degenere, i coefficienti di costo ridotto sono non negativi: $\bbmct\geq\bm{0}^T$.

      \textbf{N.B.}: nel caso di basi degeneri non è possibile fare affermazioni sulla ottimalità.
    \end{theorem}
  \end{minipage}\hfill
  \begin{minipage}{0.15\textwidth}\center
    \includegraphics[width=\textwidth]{random/mickey}
  \end{minipage}
\end{minipage}


\begin{proof}
  Riscriviamo la funzione obbiettivo:
  \begin{align*}
    z & = \bmct \bmx                          &  & \text{Iniziamo dalla funzione obbiettivo}                                           \\
    z & = \rnd{\bbmct + \bmct_B B^{-1}A} \bmx &  & \text{Sostituisco a $\bmct$ la sua definizione: $\bmct = \bbmct + \bmct_B B^{-1}A$} \\
    z & = \bbmct\bmx + \bmct_B B^{-1}A\bmx    &  & \text{Risolvo algebricamente}                                                       \\
    z & = \bbmct\bmx + \bmct_B B^{-1}b        &  & \text{Sostituisco $\bmb = A\bmx$}                                                   \\
  \end{align*}

  La funzione obbiettivo iniziale è pari a $\bbmct\bmx + \bmct_B B^{-1}b$ e sottraendo $\bbmct$, termine non negativo per ipotesi, si ottiene la disequazione $\bmct\bmx \geq \bmct_B B^{-1}b$ e quindi vale che, $\forall \bmx \in P$, la funzione obbiettivo in corrispondenza della soluzione di base ammissibile associata a $B$ assume il valore $
    \bmct_BB^{-1}\bmb = \begin{bmatrix}
      \bmct_B & \bmct_F
    \end{bmatrix}
    \begin{bmatrix}
      B^{-1}\bmb \\
      0
    \end{bmatrix}
  $, dove $\begin{bmatrix}
      B^{-1}\bmb \\
      0
    \end{bmatrix}$ è il vettore dei valori assunti da $\bmx$ quando la soluzione è ottima.

\end{proof}

\end{document}