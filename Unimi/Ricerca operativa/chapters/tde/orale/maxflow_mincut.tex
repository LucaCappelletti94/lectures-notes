 \providecommand{\main}{../../..}
\documentclass[\main/main.tex]{subfiles}
\begin{document}
\section{MAX-FLOW = MIN-CUT}

\begin{theorem}
  \label{serve_qui}
  Per ogni flusso ammissibile $\bmx$ e per ogni sezione $(S, V\ S)$ si ha:
  \[
    \phi(S) \leq k(S)
  \]
\end{theorem}

\begin{proof}
  \[
    \phi(s) = \sum_{(i,j) \in \delta^{+}(S)} x_{ij} - \sum_{(i,j) \in \delta^{-}(S)} x_{ij} \leq \sum_{(i,j) \in  \delta^{*}(S)} k_{ij} = k(S)
  \]
\end{proof}

\begin{theorem}[Il problema di massimo flusso coincide con il taglio minimo]
  Un flusso ammissibile $\bmx$ è ottimo per il problema MAX-FLOW se e solo se esiste una sezione $(S^*, \mathcal{V}\S^*)$ con $\phi(S^*)=k(S^*)$. In questo caso $(S^*, \mathcal{V}\S^*)$ è una sezione di capacità minima nella rete considerata.
  \paragraph*{Significato:} Un flusso ammissibile è ottimo per il problema di flusso massimo se esiste un taglio minimo con capacità pari al flusso. I due problemi sono duali e questa è una relazione di dualità forte.
\end{theorem}

\begin{definition}[Arco carico o saturo]
  Sia $\bmx$ un flusso ammissibile. Un arco $(i,j)$ si dice saturo se $x_{ij}=k_{ij}$, scarico se $x_{ij}=0$.
\end{definition}

\begin{definition}[Rete incrementale]
  Sia $\bmx$ un flusso ammissibile. La \textbf{rete incrementale} $\bar{\mathcal{G}} = (\mathcal{V}, \bar{\mathcal{A}})$ associata $\bmx$ è ottenuta dalla rete originale $\mathcal{G} = (\mathcal{V}, \mathcal{A})$ sostituendo ogni arco $(i,j) \in \mathcal{A}$ con due archi:
  \begin{enumerate}
    \item Un \textbf{arco diretto} $(i,j)$ di capacità residua $\bar{k}_{ij} = k_{ij} - x_{ij} \geq 0$.
    \item Un \textbf{arco inverso} $(j,i)$ di capacità residua $\bar{k}_{ij} \geq 0$
  \end{enumerate}
  ed eliminando alla fine gli archi con capacità residua nulla.
\end{definition}

\begin{theorem}[Soluzione ottima per massimo flusso]
  Un flusso ammissibile $\bmx$ è ottimo per il problema di massimo flusso se e solo se il vertice $t$ pozzo non è raggiungibile dal vertice $s$ sorgente nella rete incrementale $\mathcal{G} = (\mathcal{V}, \mathcal{A})$ associata ad $\bmx$.
\end{theorem}

\begin{proof}
  Sia $\phi_0$ il valore del flusso $\bmx$. Se $t$ è raggiungibile da $s$ in $\bar{\mathcal{G}}$, allora esiste un cammino aumentante $P$ da $s$ a $t$ in $\bar{\mathcal{G}}$. Posto $\delta = \min\crl{\bar{k}_{uv}:(u,v) \in P}>0$, per ogni coppia $(u,v) \in P$ è possibile aggiornare $x_{uv}=x_{uv}+\delta$ se $(u,v)$ è un arco diretto, $x_{uv}=x_{uv}-\delta$ se $(u,v)$ è un arco inverso. È facile verificare che il nuovo vettore $\bmx$ costituisce un flusso ammissibile di valore $\phi_0 + \delta$ nella rete originale, il che dimostra che la soluzione $\bmx$ di partenza non era ottima per il problema di massimo flusso.

  Supponiamo ora che $t$ non sia raggiungibile da $s$ in $\bar{\mathcal{G}}$. Esiste quindi una sezione $(S^*,V\ S^*)$ nella rete incrementale $\bar{\mathcal{G}}$ tale che $\delta_{\bar{\mathcal{G}}}^{+}(S^*) = \emptyset$. Per definizione di rete incrementale si ha allora che, nella rete originale $\mathcal{G}$:
  \begin{enumerate}
    \item Ogni arco $(i,j) \in \delta_{\bar{\mathcal{G}}}^{+}(S^*)$ è saturo.
    \item Ogni arco $(i,j) \in \delta_{\bar{\mathcal{G}}}^{-}(S^*)$ è scarico.
  \end{enumerate}
  Ne consegue che:
  \[
    \phi(S^*) = \sum_{\delta_{(i,j) \in \bar{\mathcal{G}}}^{+}(S^*)} x_{ij} - \sum_{(i,j) \in \delta_{\bar{\mathcal{G}}}^{-}(S^*)} x_{ij} = \sum_{(i,j) \in \delta_{\bar{\mathcal{G}}}^{-}(S^*)} k_{ij} = k(S^*)
  \]
  e quindi l'ottimalità di $\bmx$ deriva dal teorema \ref{serve_qui} che garantisce inoltre che $(S^*, V\ S^*)$ è una sezione di capacità minima nella rete originale.
\end{proof}

\end{document}