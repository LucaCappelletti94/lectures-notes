 \providecommand{\main}{../../..}
\documentclass[\main/main.tex]{subfiles}
\begin{document}
\section{MAX-FLOW = MIN-CUT}

\begin{theorem}[Max-flow = Min-cut]
  Un flusso ammissibile è ottimo per il problema di flusso massimo se esiste un taglio minimo con capacità pari al flusso. I due problemi sono duali.
\end{theorem}


\end{document}