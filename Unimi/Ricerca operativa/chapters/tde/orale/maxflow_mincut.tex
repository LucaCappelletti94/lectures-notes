 \providecommand{\main}{../../..}
\documentclass[\main/main.tex]{subfiles}
\begin{document}
\section{MAX-FLOW = MIN-CUT}

\begin{minipage}{\textwidth}
  \begin{minipage}{.83\textwidth}
    \flushleft
    \begin{theorem}[Max-flow = Min-cut]
      Un flusso ammissibile $\bmx$ è ottimo per il problema di flusso massimo se e solo se esiste un taglio $\rnd{S^*, \mathcal{V} \bs S^*}$ tale che $\phi(S^*)$ è pari alla capacità $u(S^*, \mathcal{V} \bs S^*)$.

      I due problemi sono duali.
    \end{theorem}
  \end{minipage}\hfill
  \begin{minipage}{0.15\textwidth}\center
    \includegraphics[width=\textwidth]{random/scissors}
  \end{minipage}
\end{minipage}


\end{document}