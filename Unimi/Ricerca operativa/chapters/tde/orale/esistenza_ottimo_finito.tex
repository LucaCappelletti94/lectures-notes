 \providecommand{\main}{../../..}
\documentclass[\main/main.tex]{subfiles}
\begin{document}
\section{Esistenza di vertice ottimo}

\begin{minipage}{\textwidth}
  \begin{minipage}{.83\textwidth}
    \flushleft
    \begin{theorem}
      Se $P = \crl{\bm{x} \in \R^n: A\bm{x} \leq \bm{b}, \bm{x} \geq \bm{0}}$ è un politopo, allora esiste almeno un vertice di $P$ ottimo per il problema:
      \[
        \min \crl{\bm{c}^T\bm{x}: \bm{x} \in P}
      \]
    \end{theorem}
  \end{minipage}\hfill
  \begin{minipage}{0.15\textwidth}\center
    \includegraphics[width=\textwidth]{random/vertigo}
  \end{minipage}
\end{minipage}

\begin{proof}
  Siano $\bm{y}_i, \ldots, \bm{y}_k$ i vertici di $P$ e sia $z^* = \min \crl{\bm{c}^T\bm{y}_i, i=1,\ldots,k}$. Sia $\bm{x} \in P$, cioè $z^*$ è soluzione ottima dei vertici.

  Per il teorema di Minkowsky-Weil posso descrivere $\bmx$ come una combinazione convessa dei vertici, cioè: $\exists \bm{\lambda} \geq \bm{0}: \bm{x} = \sum_{i=1}^k \lambda_i \bm{y}_i, \sum_{i=1}^k \lambda_i =1$.

  \begin{align*}
    z                                       & = \bmct \bmx                                    &  & \text{Iniziamo dalla funzione obbiettivo}                             \\
    z                                       & = \bm{c}^T\rnd{\sum_{i=1}^k \lambda_i \bm{y}_i} &  & \text{Sostituisco a $\bmx$ la comb. convessa equivalente}             \\
    z                                       & = \sum_{i=1}^k \lambda_i \bm{c}^T\bm{y}_i       &  & \text{Sposto il coefficiente dei costi nella serie}                   \\
    \sum_{i=1}^k \lambda_i \bm{c}^T\bm{y}_i & \geq \sum_{i=1}^k  \lambda_i z^* = z^*          &  & \text{$\bmct \bmy$ è sempre $\geq$ della soluzione ottima di minimo.} \\
  \end{align*}
\end{proof}

\end{document}