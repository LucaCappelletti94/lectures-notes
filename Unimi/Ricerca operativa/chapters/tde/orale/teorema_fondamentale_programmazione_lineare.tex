 \providecommand{\main}{../../..}
\documentclass[\main/main.tex]{subfiles}
\begin{document}
\section{Teorema fondamentale della programmazione lineare}

\begin{theorem}[Teorema fondamentale della programmazione lineare]
  Un punto è vertice di un \textbf{poliedro} non vuoto \textbf{se e solo se} è anche una soluzione di base ammissibile.
  \[
    \bmx \text{ vertice} \Leftrightarrow B^{-1}\bmb\geq 0
  \]
\end{theorem}

\begin{proof}[C.s.: Se un punto è SBA $\Rightarrow$ è vertice]
  Consideriamo una SBA $\bmx$ con i primi $k \geq 0$ elementi positivi ed i successivi nulli. Queste prime $k$ colonne di $A$, più eventuali altre nel caso di soluzione degenere, devono fare parte di $B$ per l'ipotesi di SBA.

  \[
    \bmx = \begin{bmatrix}
      x_1, \ldots, x_k, 0, \ldots, 0
    \end{bmatrix}^T
  \]

  Supponiamo \textbf{per assurdo} che $\bmx$ non sia un vertice. Allora è possibile costruire una combinazione convessa tra due punti distinti $\bmy' \in P$ e $\bmy'' \in P$, anch'essi con i primi $k$ elementi positivi ed i successivi nulli:

  \[
    \exists \bmy' = \begin{bmatrix}
      y'_1, \ldots, y'_k, 0, \ldots, 0
    \end{bmatrix}^T,
    \bmy'' = \begin{bmatrix}
      y''_1, \ldots, y''_k, 0, \ldots, 0
    \end{bmatrix}^T:
    \bmx = \lambda \bmy' + (1-\lambda)\bmy''
  \]
  È possibile scrivere le equazioni:

  \[
    \begin{cases}
      A\bmy' = \bmb  \\
      A\bmy'' = \bmb \\
    \end{cases}
    \Rightarrow
    \begin{cases}
      A_1y'_1 + \ldots + A_ky'_k = \bmb   \\
      A_1y''_1 + \ldots + A_ky''_k = \bmb \\
    \end{cases}
    \Rightarrow
    (y'_1 - y''_1)A_1 + \ldots + (y'_k -y''_k)A_k = 0
  \]
  Non tutte le coppie $(y'_i, y''_i)$ possono essere uguali, per cui, perché l'equazione sia pari a $0$, le colonne $A_1, \ldots, A_k$ devono essere linearmente dipendenti e non possono fare parte della base $B$, da cui l'assurdo.
\end{proof}

\begin{proof}[C.n.: Se un punto è vertice $\Rightarrow$ è SBA]
  Consideriamo un punto $\bmx \in P$ vertice:

  \[
    \bmx = \begin{bmatrix}
      x_1, \ldots, x_k, 0, \ldots, 0
    \end{bmatrix}^T
  \]
  Supponiamo \textbf{per assurdo} che $\bmx$ sia un vertice ma non una SBA. Allora le prime $k$ colonne di $A$ devono essere linearmente dipendenti. Ne segue che deve esistere una combinazione lineare tale che:

  \[
    \sum_{i=1}^k \a_j A_j = 0
  \]
  Allora è possibile scegliere uno scalare $\epsilon$ che consenta di costruire due punti distinti $\bmy' \neq \bmy''$ e ammissibili:

  \[
    \exists \epsilon: \begin{cases}
      \bmy' = \begin{bmatrix}
        x'_1 + \epsilon\a_1, \ldots, x'_k + \epsilon\a_k, 0, \ldots, 0
      \end{bmatrix}^T \\
      \bmy'' = \begin{bmatrix}
        x'_1 - \epsilon\a_1, \ldots, x'_k - \epsilon\a_k, 0, \ldots, 0
      \end{bmatrix}^T
    \end{cases}
    \bmy', \bmy'' \in P
  \]
  Ma questo significa che è possibile costruire il punto $\bmx$ come combinazione convessa di altri due punti appartenenti al poliedro:

  \[
    \bmx = \frac{1}{2}\bmy' + \frac{1}{2}\bmy''
  \]
  Ma $\bmx$, per ipotesi, è vertice, per cui non può essere ottenuto come combinazione di altri due punti del poliedro.
\end{proof}

\begin{corollary}[Esistenza di soluzione ottima]
  Ogni problema di PL definito su un \textbf{politopo} non vuoto ha almeno una soluzione ottima che è soluzione di base ammissibile.
\end{corollary}


\end{document}