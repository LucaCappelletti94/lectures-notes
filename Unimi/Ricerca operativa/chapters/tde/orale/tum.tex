 \providecommand{\main}{../../..}
\documentclass[\main/main.tex]{subfiles}
\begin{document}
\section{Matrici totalmente unimodulari}

\begin{definition}[Matrice totalmente unimodulare]
  Una matrice $A \in \R^{m \times n}$ si dice \textbf{totalmente unimodulare} (TUM) se per ogni sua sotto-matrice quadrata $Q$ vale che $\det(Q) \in \crl{-1, 0, 1}$.
\end{definition}

\begin{theorem}
  Se $A$ è TUM e $\bmb$ è intero, allora $P$ ha solo vertici interi, cioè $\text{conv}\rnd{\bmx} = P$.
\end{theorem}

\begin{theorem}[Condizioni necessarie per TUM]
  Se $A$ è TUM, allora deve valere che:
  \begin{enumerate}
    \item Ogni termine della matrice $A$ $a_{ij} \in \crl{-1,0,1}$.
    \item $A^T$ è TUM.
    \item $\begin{bmatrix}
              A & I
            \end{bmatrix}$ è TUM.
    \item Moltiplicando una riga o colonna per $-1$ $A$ rimane TUM.
    \item Scambiando righe o colonne $A$ rimane TUM.
    \item L'operazione di pivot lascia $A$ TUM.
  \end{enumerate}
\end{theorem}

\begin{theorem}[Condizioni sufficienti per TUM]
  Sia $A$ una matrice con $a_{ij} \in \crl{-1,0,1}$. Condizione sufficiente perché $A$ sia TUM è che valgano:
  \begin{enumerate}
    \item In ogni colonna vi siano al più due elementi non nulli.
    \item Esiste una partizione $\begin{bmatrix}
              I_1 & I_2
            \end{bmatrix}$ delle righe di $A$ tale che ogni colonna con due elementi non nulli ha questi elementi su righe appartenenti ad insiemi $I_1$ e $I_2$ diversi se e solo se i due elementi sono concordi in segno.
  \end{enumerate}
\end{theorem}

\begin{proof}
  Per dimostrare che $A$ sia TUM è necessario dimostrare che il determinante di ogni sotto-matrice $Q$ è $\det(Q) \in \crl{-1, 0, 1}$, con ordine $k$. Procediamo per induzione su $k$.

  Se $k=1$ allora $Q = a_{ij} \in \crl{-1,0,1}$.

  Supponiamo ora che $\det(Q') \in \crl{-1,0,1}$ per ogni sotto-matrice $Q'$ di ordine $k'\geq1$, ove $k'$ è un valore fissato. Consideriamo quindi una qualsiasi sotto-matrice $Q$ di ordine $k = k'+1$. Possono avvenire solo tre casi:
  \begin{enumerate}
    \item $Q$ ha una colonna di zeri: in questo caso $\det(Q)=0$.
    \item $Q$ ha una colonna con un solo elemento diverso da zero: in questo caso, $\det(Q) = \pm \det(Q') \in \crl{-1,0,1}$.
    \item Ogni colonna di $Q$ ha esattamente due elementi diversi da zero: in questo caso, le righe di $Q$ risultano linearmente dipendenti e $\det(Q)=0$.
  \end{enumerate}
\end{proof}

\end{document}