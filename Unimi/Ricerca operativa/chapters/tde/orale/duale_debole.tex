 \providecommand{\main}{../../..}
\documentclass[\main/main.tex]{subfiles}
\begin{document}
\section{Teorema di dualità debole}
\begin{theorem}[Dualità debole]
  \label{dualita_debole}
  Siano dati un problema $P$ ed il suo duale $D$, entrambi non vuoti:

  \[
    P = \crl{\bmx \geq \bm{0}: A\bmx \geq \bmb} \qquad D = \crl{\bmu \geq \bm{0}: \bmct \geq \bmut A}, \qquad P, D \neq \emptyset
  \]
  Per ogni coppia di punti $\bbmx \in P$, $\bbmu \in D$ si ha che:

  \begin{figure}
    \[
      \bbmut \bmb \leq \bmct\bbmx
    \]
    \caption{Teorema di dualità debole: ogni soluzione del duale è minore o uguale a ogni soluzione del primale}
  \end{figure}
\end{theorem}

\begin{proof}
  Dati $\bbmx \in P$ e $\bbmu \in D$, si ha che $A\bbmx \geq \bmb, \bbmx \geq \bm{0}$ e $\bmct \geq \bbmut A$, da cui segue:
  \[
    \bbmut \bmb \leq \bbmut A \bbmx \leq \bmct \bbmx
  \]
\end{proof}

\begin{corollary}
  Si consideri una coppia di problemi primale e duale. Sono possibili solo 4 casi:
  \begin{enumerate}
    \item Entrambi i problemi hanno ottimo finito, che coincide (da teorema di dualità forte \ref{dualita_forte}).
    \item Il problema primale è illimitato ed il duale è impossibile (da teorema di dualità debole \ref{dualita_debole}).
    \item Il problema duale è illimitato ed il primale è impossibile (da teorema di dualità debole \ref{dualita_debole}).
    \item Entrambi i problemi sono impossibili (banalmente non è escluso da nessuna proprietà).
  \end{enumerate}
\end{corollary}

\end{document}