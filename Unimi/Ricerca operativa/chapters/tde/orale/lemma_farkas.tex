 \providecommand{\main}{../../..}
\documentclass[\main/main.tex]{subfiles}
\begin{document}
\section{Lemma di Farkas}
\begin{theorem}[Lemma di Farkas]
  \label{lemma_farkas}
  La disuguaglianza
  \begin{figure}
    \[
      \bm{c}^T\bmx  \geq c_0
    \]
    \caption{Disuguaglianza di Farkas}
  \end{figure}
  è valida per il poliedro non vuoto $P = \crl{\bmx \geq 0: A\bmx = \bm{b}}$ se e solo se esiste un $\bm{u} \in \R^m$ tale per cui:

  \begin{figure}
    \begin{subfigure}{0.49\textwidth}
      \[
        \bm{c}^T \geq \bm{u}^TA
      \]
      \caption{Prima disequazione}
      \label{farkas_prima}
    \end{subfigure}
    \begin{subfigure}{0.49\textwidth}
      \[
        \bm{u}^T\bm{b}\geq c_0
      \]
      \caption{Seconda disequazione}
      \label{farkas_seconda}
    \end{subfigure}
    \caption{Condizioni del lemma di Farkas}
  \end{figure}
\end{theorem}

\begin{proof}
  Il fatto che la condizione sia sufficiente è vero dato che per ogni $\bmx \geq 0: A\bmx=\bm{b}$ vale che:
  \[
    \bm{c}^T\bmx \geq \bm{u}^TA\bmx = \bm{u}^T\bm{b} \geq c_0
  \]

  Procediamo ora a dimostrare che la condizione sia anche necessaria.

  Dalle ipotesi di validità si ha che:

  \[
    \bm{c}^T\bmx \geq c_0 \forall \bmx \in P \Rightarrow P' = \min_{\bmx}\crl{\bm{c}^T\bmx: \bmx \in P} \text{ non è illimitato inferiormente e } P \neq \emptyset
  \]

  Allora esiste una soluzione ottima finita e per il teorema di convergenza del simplesso esiste una soluzione ottima di base.

  Chiamiamo $B$ la base ottima di $P'$ e sia $\bm{u}^T = \bm{c}^T_B B^{-1}$. I costi ridotti, calcolati in corrispondenza della base ottima devono essere:

  \begin{figure}
    \[
      \bbm{c}^T = \bm{c}^T - \bm{c}_B^TB^{-1}A \geq \bm{0}^T \Rightarrow \bm{c}^T \geq \bm{u}^TA
    \]
    \caption{Dalla regola dei costi ridotti si ottiene la prima disequazione \ref{farkas_prima}}
  \end{figure}

  Infine per ipotesi vale che $\bm{c}^T \bmx \geq 0 \forall \bmx \in P$, da cui:

  \begin{figure}
    \[
      c_0 \leq z^* = \bm{c}^T\bmx^* = \bm{c}_B^T \bmx^*_B + \bm{c}_F^T \bmx^*_F = \bm{c}^TB^{-1}\bm{b} = \bm{u}^T\bm{b}
    \]
    \caption{Dall'ipotesi iniziale si ottiene la seconda disequazione \ref{farkas_seconda}}
  \end{figure}

\end{proof}

\end{document}