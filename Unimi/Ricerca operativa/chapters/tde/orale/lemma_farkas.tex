 \providecommand{\main}{../../..}
\documentclass[\main/main.tex]{subfiles}
\begin{document}
\section{Lemma di Farkas}

\begin{theorem}[Lemma di Farkas]
  \label{lemma_farkas}
  La disuguaglianza $\bmct\bmx \geq c_0$ è valida per il \textbf{poliedro} non vuoto $P=\crl{\bmx\geq\bm{0}:A\bmx=\bmb}$ se e solo se esiste $\bmu \in \R^m$ tale che:
  \[
    \bmct \bmx \geq c_0 \Leftrightarrow \exists \bmu = \bmct_B B^{-1} \in \R^m:\, \bmct \geq \bmut A \land \bmut \bmb \geq c_0
  \]
  \paragraph*{Applicazione:} Se un problema primale definito su un \textbf{poliedro} non vuoto ha soluzione limitata, allora il suo problema duale deve ammettere una soluzione limitata.
\end{theorem}


\begin{proof}[C.s.: Se il duale è limitato $\Rightarrow$ il primale è limitato]
  \begin{align*}
    \bmct     & \geq \bmut A             &  & \text{Iniziamo dalla regione ammissibile del duale}         \\
    \bmct\bmx & \geq \bmut A\bmx         &  & \text{Moltiplichiamo ambo i termini per $\bmx \geq \bm{0}$} \\
    \bmct\bmx & \geq \bmut \bmb \geq c_0 &  & \text{Sostituisco $A\bmx = \bmb$}
  \end{align*}
\end{proof}
\begin{proof}[C.n.: Se il primale è limitato $\Rightarrow$ il duale è limitato]

  Dalle ipotesi il poliedro ha soluzione limitata, allora per il \textbf{teorema di convergenza del simplesso} esiste una SBA ottenibile in tempo finito con l'\textbf{algoritmo del simplesso}.

  Chiamiamo $B$ la base ottima e sia $\bm{u}^T = \bm{c}^T_B B^{-1}$. I costi ridotti, calcolati in corrispondenza della base ottima devono essere:

  \begin{figure}
    \[
      \bbm{c}^T = \bm{c}^T - \bm{c}_B^TB^{-1}A \geq \bm{0}^T \Rightarrow \bm{c}^T \geq \bm{u}^TA
    \]
    \caption{Dalla regola dei costi ridotti si ottiene la regione di ammissibilità del duale}
  \end{figure}

  Infine per ipotesi vale che $\bm{c}^T \bmx \geq c_0 \forall \bmx \in P$, da cui:

  \begin{figure}
    \begin{align*}
      z^* & = \bm{c}^T\bmx^* \geq c_0                   &  & \text{La $z^*$ è data da $x^*$}                                \\
          & = \bm{c}_B^T \bmx^*_B + \bm{c}_F^T \bmx^*_F &  & \text{Spezzo il vettore dei costi in base e fuori base}        \\
          & = \bm{c}_B^TB^{-1}\bm{b}                    &  & \text{La parte fuori base, $\bmx^*_B$ è un vettore di zeri.}   \\
          & = \bm{u}^T\bm{b}  \geq c_0                  &  & \text{Sostituisco la definizione di $\bm{u} = \bmct_B B^{-1}$}
    \end{align*}
    \caption{Dalla funzione obbiettivo del primale si ottiene la funzione obbiettivo del duale, entrambe limitate}
  \end{figure}

\end{proof}

\end{document}