 \providecommand{\main}{../../..}
\documentclass[\main/main.tex]{subfiles}

\newcommand{\defaultWrapperPrepper}[9]{
  \def\GraphArgI{#1}%
  \def\GraphArgII{#2}%
  \def\GraphArgIII{#3}%
  \def\GraphArgIV{#4}%
  \def\GraphArgV{#5}%
  \def\GraphArgVI{#6}%
  \def\GraphArgVII{#7}%
  \def\GraphArgVIII{#8}%
  \def\GraphArgIX{#9}%
  \defaultWrapperPrepperRelay
}

\newcommand{\defaultWrapperPrepperRelay}[9]{
  \def\GraphArgX{#1}%
  \def\GraphArgXI{#2}%
  \def\GraphArgXII{#3}%
  \def\GraphArgXIII{#4}%
  \def\GraphArgXIV{#5}%
  \def\GraphArgXV{#6}%
  \def\GraphArgXVI{#7}%
  \def\GraphArgXVII{#8}%
  \def\GraphArgXVIII{#9}%
  \defaultWrapperPrepperSecondRelay
}

\newcommand{\defaultWrapperPrepperSecondRelay}[9]{
  \def\GraphArgXIX{#1}%
  \def\GraphArgXX{#2}%
  \def\colorI{#3}%
  \def\colorII{#4}%
  \def\colorIII{#5}%
  \def\colorIV{#6}%
  \def\colorV{#7}%
  \def\colorVI{#8}%
  \def\colorVII{#9}%
  \defaultGraph
}

\def\red{red}
\def\blue{blue}
\def\black{black}

\def\redFlag{25500}
\def\blackFlag{0}

\NewDocumentCommand{\doubleArc}{m m m O{midway, left} O{midway, right}}{
  %
  % #1 -> is number (not roman) of variable in game
  % #2 -> first edge
  % #3 -> second edge
  % #4 (optional)-> first label position
  % #5 (optional)-> second label position
  %
  \def\varnum{\rom{#1}}
  \def\firstEdge{#2}
  \def\secondEdge{#3}
  \def\customColor{\black}
  \def\reverseColor{\black}
  \def\defaultWeight{\csname GraphArg\varnum\endcsname}
  \def\deltaWeight{\csname GraphArgX\varnum\endcsname}
  \def\customColorFlag{\csname color\varnum\endcsname}
  \smartif{\customColorFlag}{\redFlag}[
    \FPset{\customColor}{\red}
    \FPset{\reverseColor}{\blue}
  ][
    \FPset{\customColor}{\black}
    \FPset{\reverseColor}{\black}
  ]

  \FPsub{\remainingWeight}{\defaultWeight}{\deltaWeight}
  \FPclip{\remainingWeight}{\remainingWeight}



  \smartif{\deltaWeight}{\zero}[
    \draw[edge,\customColor] (\firstEdge) to node[#4] {\remainingWeight}(\secondEdge);
    ][
    \smartif{\remainingWeight}{\zero}[][
      \draw[edge,\customColor] (\firstEdge) to [bend right=20] node[#4] {\remainingWeight}(\secondEdge);]
    ]

  \smartif{\remainingWeight}{\zero}[
    \draw[edge,\reverseColor] (\secondEdge) to node[#5] {\deltaWeight}(\firstEdge);
    ][\smartif{\deltaWeight}{\zero}[][
      \draw[edge,\reverseColor] (\secondEdge) to [bend right=20] node[#5] {\deltaWeight}(\firstEdge);]]
}

\NewDocumentCommand{\defaultGraph}{m m m m}{
  \def\colorVIII{#1}%
  \def\colorIX{#2}%
  \def\colorX{#3}%
  \def\defaultCuts{#4}%

  \tikzexternaldisable
  \begin{tikzpicture}
    \tikzset{
      vertex/.style={circle,draw,minimum size=2em},
      edge/.style={->,> = latex}
    }

    % vertices
    \node[vertex] (s) at (0,2) {$s$};
    \node[vertex] (t) at (8,2) {$t$};
    \node[vertex] (1) at (2,4) {$1$};
    \node[vertex] (2) at (6,4) {$2$};
    \node[vertex] (3) at (2,0) {$3$};
    \node[vertex] (4) at (6,0) {$4$};

    %edges
    \doubleArc{1}{s}{1}
    \doubleArc{2}{s}{3}[midway, right][midway, left]
    \doubleArc{3}{3}{1}
    \doubleArc{4}{3}{4}[midway, above][midway, below]
    \doubleArc{5}{4}{1}[pos=.3, left][pos=.3, right]
    \doubleArc{6}{4}{2}
    \doubleArc{7}{4}{t}
    \doubleArc{8}{1}{2}[midway, above][midway, below]
    \doubleArc{9}{2}{t}[midway, right][midway, left]
    \doubleArc{10}{2}{3}[pos=.3, left][pos=.3, right]

    \defaultCuts
  \end{tikzpicture}
  \tikzexternalenable
}

\newcommand{\currentGraphPreloader}[9]{
  \def\currentGraphArgI{#1}%
  \def\currentGraphArgII{#2}%
  \def\currentGraphArgIII{#3}%
  \def\currentGraphArgIV{#4}%
  \def\currentGraphArgV{#5}%
  \def\currentGraphArgVI{#6}%
  \def\currentGraphArgVII{#7}%
  \def\currentGraphArgVIII{#8}%
  \def\currentGraphArgIX{#9}%
  \secondCurrentGraphPreloader
}
\newcommand{\secondCurrentGraphPreloader}[9]{
  \def\currentGraphArgX{#1}%
  \def\currentColorI{#2}%
  \def\currentColorII{#3}%
  \def\currentColorIII{#4}%
  \def\currentColorIV{#5}%
  \def\currentColorV{#6}%
  \def\currentColorVI{#7}%
  \def\currentColorVII{#8}%
  \def\currentColorVIII{#9}%
  \currentGraph
}

\NewDocumentCommand{\currentGraph}{m m O{}}{
  \def\currentColorIX{#1}%
  \def\currentColorX{#2}%
  \defaultWrapperPrepper{25}{25}{5}{20}{5}{5}{30}{35}{20}{15}{\currentGraphArgI}{\currentGraphArgII}{\currentGraphArgIII}{\currentGraphArgIV}{\currentGraphArgV}{\currentGraphArgVI}{\currentGraphArgVII}{\currentGraphArgVIII}{\currentGraphArgIX}{\currentGraphArgX}{\currentColorI}{\currentColorII}{\currentColorIII}{\currentColorIV}{\currentColorV}{\currentColorVI}{\currentColorVII}{\currentColorVIII}{\currentColorIX}{\currentColorX}{#3}
}

\begin{document}
\subsection{Esercizio 4}
Si consideri la rete sottostante in cui i valori sugli archi rappresentano le loro capacità:

\begin{figure}
  \currentGraphPreloader
  {0}{0}{0}{0}{0}
  {0}{0}{0}{0}{0}
  {\blackFlag}{\blackFlag}{\blackFlag}{\blackFlag}{\blackFlag}
  {\blackFlag}{\blackFlag}{\blackFlag}{\blackFlag}{\blackFlag}
\end{figure}

\begin{enumerate}[a)]
  \item Determinare il flusso massimo da \textbf{s} e \textbf{t} inviando nelle prime due iterazioni 5 unità lungo i percorsi $s,3,4,2,t$ e $s,3,4,1,2,t$.
  \item Riportare i cammini aumentanti ed il corrispondente aumento di flusso.
  \item SI riporti sulla figura il valore del flusso massimo.
  \item Determinare il taglio minimo.
  \item Scegliere un insieme di archi e gli si associ un costo unitario di flusso in modo che la soluzione trovata non sia a costo minimo.
\end{enumerate}

\subsection{Soluzione esercizio 4}
\subsubsection*{Step preliminare}
Invio 5 unità lungo i due percorsi indicati.

\begin{figure}
  \begin{subfigure}{0.49\textwidth}
    \currentGraphPreloader
    {0}{5}{0}{5}{0}
    {5}{0}{0}{5}{0}
    {\blackFlag}{\redFlag}{\blackFlag}{\redFlag}{\blackFlag}
    {\redFlag}{\blackFlag}{\blackFlag}{\redFlag}{\blackFlag}
    \caption{Invio 5 unità lungo $s,3,4,2,t$}
  \end{subfigure}
  \begin{subfigure}{0.49\textwidth}
    \currentGraphPreloader
    {0}{10}{0}{10}{5}
    {5}{0}{5}{10}{0}
    {\blackFlag}{\redFlag}{\blackFlag}{\redFlag}{\redFlag}
    {\blackFlag}{\blackFlag}{\redFlag}{\redFlag}{\blackFlag}
    \caption{Invio 5 unità lungo $s,3,4,1,2,t$}
  \end{subfigure}
  \caption{Preparazione preliminare richiesta}
\end{figure}

\subsubsection*{Procedo con algoritmo}

Dallo stato corrente, procedo come segue:

\begin{figure}
  \begin{subfigure}{0.49\textwidth}
    \currentGraphPreloader
    {0}{20}{0}{20}{5}
    {5}{10}{5}{10}{0}
    {\blackFlag}{\redFlag}{\blackFlag}{\redFlag}{\blackFlag}
    {\blackFlag}{\redFlag}{\blackFlag}{\blackFlag}{\blackFlag}
    \caption{Invio 10 unità lungo $s,3,4,t$}
  \end{subfigure}
  \begin{subfigure}{0.49\textwidth}
    \currentGraphPreloader
    {5}{20}{0}{20}{0}
    {5}{15}{5}{10}{0}
    {\redFlag}{\blackFlag}{\blackFlag}{\blackFlag}{\redFlag}
    {\blackFlag}{\redFlag}{\blackFlag}{\blackFlag}{\blackFlag}
    \caption{Invio 5 unità lungo $s,1,4,t$}
  \end{subfigure}
  \begin{subfigure}{0.49\textwidth}
    \currentGraphPreloader
    {10}{20}{0}{20}{0}
    {0}{20}{10}{10}{0}
    {\redFlag}{\blackFlag}{\blackFlag}{\blackFlag}{\blackFlag}
    {\redFlag}{\redFlag}{\redFlag}{\blackFlag}{\blackFlag}
    \caption{Invio 5 unità lungo $s,1,2,4,t$}
  \end{subfigure}
  \begin{subfigure}{0.49\textwidth}
    \currentGraphPreloader
    {20}{20}{0}{20}{0}
    {0}{20}{20}{20}{0}
    {\redFlag}{\blackFlag}{\blackFlag}{\blackFlag}{\blackFlag}
    {\blackFlag}{\blackFlag}{\redFlag}{\redFlag}{\blackFlag}
    \caption{Invio 10 unità lungo $s,1,2,t$}
  \end{subfigure}
  \caption{Procedo con algoritmo}
\end{figure}

\subsubsection*{Flusso massimo}
Il flusso massimo è di 40 unità.

\subsubsection*{Taglio minimo}
Il taglio minimo è di 40 unità. I due gruppi che divide sono $(4,t)$ e $(s,1,2,3)$

\begin{figure}
  \currentGraphPreloader
  {20}{20}{0}{20}{0}
  {0}{20}{20}{20}{0}
  {\blackFlag}{\blackFlag}{\blackFlag}{\blackFlag}{\blackFlag}
  {\blackFlag}{\blackFlag}{\blackFlag}{\blackFlag}{\blackFlag}[
  \draw[dashed,\red]
  ([yshift=10pt]$ (3)!0.7!(4) $ ) --
  ([yshift=-10pt]$ (3)!0.7!(4) $ );
  \draw[dashed,\red,rotate=-45]
  ([yshift=10pt]$ (2)!0.7!(t) $ ) --
  ([yshift=-10pt]$ (2)!0.7!(t) $ );
  ]
  \caption{Taglio minimo}
\end{figure}

\subsubsection*{Soluzione non a costo minimo}
È sufficiente scegliere costi in modo tale da creare un ciclo a costo negativo, per esempio assegnando come costi $s\rightarrow3: 1000$, $s\rightarrow1: 1$, $3\rightarrow1: 1$.

\end{document}