 \providecommand{\main}{../../..}
\documentclass[\main/main.tex]{subfiles}
\usetikzlibrary{graphdrawing}
\usetikzlibrary{graphs}
\usegdlibrary{trees}
\begin{document}
\subsection{Esercizio 4}
Si risolva mediante Branch \& Bound il seguente problema dello zaino a capacità massima 9, seguendo le istruzioni seguenti:

\begin{table}
  \begin{tabular}{c|LLLLL}
    Profitti & 21 & 10 & 20 & 4 & 3 \\
    \hline
    Pesi     & 3  & 2  & 5  & 2 & 3
  \end{tabular}
\end{table}

\begin{enumerate}
  \item Si utilizzi come rilassamento quello lineare, risolto mediante un opportuno algoritmo.
  \item Si adotti una strategia ``Depth First''.
  \item Si esplori per primo ad ogni livello il ramo dell'albero di branching associato al vincolo $x_i=0$, dove la variabile di branching $x_i$ è quella che assume un valore frazionario nel rilassamento lineare.
  \item È possibile fissare a zero una variabile qualora la capacità residua dello zaino sia strettamente minore del suo peso.
  \item Come \textbf{lower bound} si usi la miglior soluzione intera data dalla somma dei profitti degli oggetti che è stato possibile inserire nello zaino durante il calcolo dell'\textbf{upper bound}.
  \item Per ogni nodo si riportino:
        \subitem Numero progressivo $i$.
        \subitem Valore di upper bound.
        \subitem Valore del vettore delle variabili.
\end{enumerate}

\subsection{Soluzione esercizio 4}
\begin{figure}
  \begin{tikzpicture}
    \Tree[.\text{$0:\begin{cases}
            \bmx=\sqr{0,0,0,0,0} \\
            U=47
          \end{cases}$}
      [.$x_1=0$
        [.\text{$1:\begin{cases}
                  1:\bmx=\sqr{0,0,0,0,0} \\
                  U=34
                \end{cases}$}
            [.$x_2=0$
              [.\text{$2:\begin{cases}
                        \bmx=\sqr{0,0,0,0,0} \\
                        U=26
                      \end{cases}$}
                  [.$x_3=0$
                    [.\text{$3:\begin{cases}
                              \bmx=\sqr{0,0,0,1,1} \\
                              z=7
                            \end{cases}$} ]
                  ]
                  [.$x_3=1$
                    [.\text{$4:\begin{cases}
                              \bmx=\sqr{0,0,1,0,0} \\
                              U=26
                            \end{cases}$}
                        [.$x_4=0$
                          [.\text{$5:\begin{cases}
                                    \bmx=\sqr{0,0,1,0,1} \\
                                    z=23
                                  \end{cases}$} ]
                        ]
                        [.$x_4=1$
                          [.\text{$6:\begin{cases}
                                    \bmx=\sqr{0,0,1,1,0} \\
                                    z=24
                                  \end{cases}$} ]
                        ]
                      ]
                  ]
                ]
            ]
            [.$x_2=1$
              [.\text{$4:\begin{cases}
                        \bmx=\sqr{0,1,1,0,0} \\
                        U=26
                      \end{cases}$}
                  [.$x_4=0$
                    [.\text{$5:\begin{cases}
                              \bmx=\sqr{0,1,1,0,1} \\
                              z=23
                            \end{cases}$} ]
                  ]
                  [.$x_4=1$
                    [.\text{$6:\begin{cases}
                              \bmx=\sqr{0,0,1,1,0} \\
                              z=24
                            \end{cases}$} ]
                  ]
                ]
            ]
          ]
      ]
    ]
  \end{tikzpicture}
\end{figure}
\end{document}