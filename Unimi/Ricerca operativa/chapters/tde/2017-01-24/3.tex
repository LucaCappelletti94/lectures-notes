 \providecommand{\main}{../../..}
\documentclass[\main/main.tex]{subfiles}
\begin{document}
\subsection{Esercizio 3}
Dato il seguente problema di PI e una base formata dalle variabili $x_2, x_3$, si ricavi il vettore dei coefficienti a costo ridotto delle variabili fuori base e si verifichi che la soluzione corrispondente alla base data è ottima per il rilassamento lineare del problema.

Si ricavi il taglio di Gomory relativo al primo vincolo.

\begin{figure}
  \begin{align*}
    \min z = 7x_1 + x_2 -3x_3 -4x_4 \\
    -3x_1 + 3x_3 +2x_4    & =1      \\
    x_1 + x_2 + 2x_3 -x_4 & =1      \\
    \bm{x}\geq \bm{0}, \bm{x} \in \mathbb{Z}^4
  \end{align*}
  \caption{Esercizio 3}
\end{figure}

\subsection{Soluzione esercizio 3}

\subsubsection*{Calcolo i coefficienti di costo ridotto}
\[
  B = \begin{bmatrix}
    0 & 3 \\
    1 & 2
  \end{bmatrix}
  \qquad
  F = \begin{bmatrix}
    -3 & 2  \\
    1  & -1
  \end{bmatrix}
\]

\[
  B^{-1} = \frac{1}{3} \begin{bmatrix}
    -2 & 3 \\
    1  & 0
  \end{bmatrix}
\]

\begin{definition}[CCR fuori base]
  Il \textbf{vettore dei coefficienti di costo ridotto per le variabili fuori base} si calcola tramite il prodotto \ref{vccr}, dove $\bm{c}_F$ è il vettore dei costi fuori base, $\bm{c}_B$ è il vettore dei costi in base, $B$ è la base, $F$ è la matrice dei vincoli restante tolta la base $B$.
  \begin{figure}
    \[
      CCR = \bm{c}_F^T - \bm{c}_B^T B^{-1} F
    \]
    \caption{Vettore dei coefficienti di costo ridotto fuori base}
    \label{vccr}
  \end{figure}
\end{definition}

\begin{definition}[Base ottima]
  Una base $B$ è \textbf{ottima} quando il vettore dei coefficienti di costo ridotto per le variabili fuori base è $\geq \bm{0}$ nel caso di problemi di minimo o $\leq \bm{0}$ nel caso di problemi di massimo.
\end{definition}

\[
  CCR = \begin{bmatrix}
    7 & -4
  \end{bmatrix}
  -
  \begin{bmatrix}
    1 & -3
  \end{bmatrix}
  \left(
  \frac{1}{3}
  \begin{bmatrix}
    -2 & 3 \\
    1  & 0
  \end{bmatrix}
  \right)
  \begin{bmatrix}
    -3 & 2  \\
    1  & -1
  \end{bmatrix}
  = \begin{bmatrix}
    1 & \frac{1}{3}
  \end{bmatrix}
\]

Si tratta di un problema di minimo, per cui controllo se il vettore dei coefficienti di costo per le variabili fuori base è maggiore del vettore $\bm{0}$.

Tutti i valori sono positivi, pertanto la base è ottima.

\subsubsection*{Taglio di Gomory}

\begin{definition}[Taglio di Gomory]
  Si tratta di un metodo per determinare un piano di taglio per qualsiasi problema di PI a partire dalla soluzione ottima del corrispondente rilassamento lineare.

  \begin{figure}
    \[
      x_h + \sum \floor{a_{ij}}x_j \leq \floor{b_i}
    \]
    \caption{Taglio di Gomory}
  \end{figure}
\end{definition}

TODO

\end{document}