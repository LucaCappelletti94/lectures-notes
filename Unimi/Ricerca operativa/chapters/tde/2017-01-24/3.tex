 \providecommand{\main}{../../..}
\documentclass[\main/main.tex]{subfiles}
\begin{document}
\subsection{Esercizio 3}
Dato il seguente problema di PI e una base formata dalle variabili $x_2, x_3$, si ricavi il vettore dei coefficienti a costo ridotto delle variabili fuori base e si verifichi che la soluzione corrispondente alla base data è ottima per il rilassamento lineare del problema.

Si ricavi il taglio di Gomory relativo al primo vincolo.

\begin{figure}
  \begin{align*}
    \min z = 7x_1 + x_2 -3x_3 -4x_4 \\
    -3x_1 + 3x_3 +2x_4    & =1      \\
    x_1 + x_2 + 2x_3 -x_4 & =1      \\
    \bm{x}\geq \bm{0}, \bm{x} \in \mathbb{Z}^4
  \end{align*}
  \caption{Esercizio 3}
\end{figure}

\subsection{Soluzione esercizio 3}

\subsubsection*{Calcolo i coefficienti di costo ridotto}
\[
  B = \begin{bmatrix}
    0 & 3 \\
    1 & 2
  \end{bmatrix}
  \qquad
  F = \begin{bmatrix}
    -3 & 2  \\
    1  & -1
  \end{bmatrix}
\]

\[
  B^{-1} = \frac{1}{3} \begin{bmatrix}
    -2 & 3 \\
    1  & 0
  \end{bmatrix}
\]

\begin{definition}[CCR fuori base]
  Il \textbf{vettore dei coefficienti di costo ridotto per le variabili fuori base} si calcola tramite il prodotto \ref{vccr}, dove $\bm{c}_F$ è il vettore dei costi fuori base, $\bm{c}_B$ è il vettore dei costi in base, $B$ è la base, $F$ è la matrice dei vincoli restante tolta la base $B$.
  \begin{figure}
    \[
      CCR = \bm{c}_F^T - \bm{c}_B^T B^{-1} F
    \]
    \caption{Vettore dei coefficienti di costo ridotto fuori base}
    \label{vccr}
  \end{figure}
\end{definition}

\begin{definition}[Base ottima]
  Una base $B$ è \textbf{ottima} quando il vettore dei coefficienti di costo ridotto per le variabili fuori base è $\geq \bm{0}$ nel caso di problemi di minimo o $\leq \bm{0}$ nel caso di problemi di massimo.
\end{definition}

\[
  CCR = \begin{bmatrix}
    7 & -4
  \end{bmatrix}
  -
  \begin{bmatrix}
    1 & -3
  \end{bmatrix}
  \left(
  \frac{1}{3}
  \begin{bmatrix}
    -2 & 3 \\
    1  & 0
  \end{bmatrix}
  \right)
  \begin{bmatrix}
    -3 & 2  \\
    1  & -1
  \end{bmatrix}
  = \begin{bmatrix}
    1 & \frac{1}{3}
  \end{bmatrix}
\]

Si tratta di un problema di minimo, per cui controllo se il vettore dei coefficienti di costo per le variabili fuori base è maggiore del vettore $\bm{0}$.

Tutti i valori sono positivi, pertanto la base è ottima.

\subsubsection*{Taglio di Gomory}

\begin{definition}[Taglio di Gomory]
  Si tratta di un metodo per determinare un piano di taglio per qualsiasi problema di PI a partire dalla soluzione ottima del corrispondente rilassamento lineare.

  \begin{figure}
    \begin{subfigure}{0.49\textwidth}
      \[
        x_h + \sum \floor{a_{ij}}x_j \leq \floor{b_i}
      \]
      \caption{Taglio di Gomory con resto in variabile $x_h$}
      \label{gomory_cut_formula_general}
    \end{subfigure}
    \begin{subfigure}{0.49\textwidth}
      \[
        \sum (a_{ij} - \floor{a_{ij}})x_j \geq b_i-\floor{b_i}
      \]
      \caption{Taglio di Gomory in forma di differenza (nota $\geq  $) per vincolo $i-esimo$}
      \label{gomory_cut_formula}
    \end{subfigure}
  \end{figure}
\end{definition}

\begin{definition}[Forma canonica]
  Dato un problema di $PL$:
  \begin{figure}
    \begin{subfigure}{0.49\textwidth}
      \begin{align*}
        z = \bm{c}^T \bm{x} \\
        A\bm{x} & = \bm{b}
      \end{align*}
      \caption{Forma a equazioni del problema di PL}
    \end{subfigure}
    \begin{subfigure}{0.49\textwidth}
      \begin{table}
        \begin{tabular}{|L|L|}
          \hline
          \bm{c}^T & 0      \\
          \hline
                   &        \\
          A        & \bm{b} \\
                   &        \\
          \hline
        \end{tabular}
      \end{table}
      \caption{Forma a tableau del problema di PL}
    \end{subfigure}
  \end{figure}
  Considerata una base $B$ e partizionando $A$ in due sottomatrici $A = \begin{bmatrix}
      B & F
    \end{bmatrix}$ viene chiamata forma canonica del problema di PL la forma \ref{equation_canonica} ottenuta sostituendo $\bm{x}_B$ (cioè le variabili in base) tramite l'equazione $\bm{x}_B = B^{-1} \bm{b} - B^{-1}F\bm{x}_F$, cioè premoltiplicando per l'inversa della matrice di base $B$.

  \begin{figure}
    \begin{subfigure}{0.31\textwidth}
      \begin{align*}
        z = \bm{c}_B^T B^{-1} \bm{b} + (\bm{c}_F^T - \bm{C}_B^T B^{-1} F)\bm{x}_F \\
        I\bm{x}_B + B^{-1}F\bm{x}_F & = B^{-1} \bm{b}
      \end{align*}
      \caption{Forma a equazioni del problema di PL}
      \label{equation_canonica}
    \end{subfigure}
    \begin{subfigure}{0.31\textwidth}
      \begin{table}
        \begin{tabular}{|L|L|L|}
          \hline
          \bm{0} & \bm{c}_F^T - \bm{C}_B^T B^{-1} F & -\bm{c}_B^T B^{-1} \bm{b} \\
          \hline
                 &                                  &                           \\
          I      & B^{-1}F                          & B^{-1} \bm{b}             \\
                 &                                  &                           \\
          \hline
        \end{tabular}
      \end{table}
      \caption{Forma a tableau del problema di PL}
    \end{subfigure}
    \begin{subfigure}{0.31\textwidth}
      \begin{table}
        \begin{tabular}{|L|L|L|}
          \hline
          \bm{0} & \text{CCR fuori base} & \tilde{c}_{\bm{0}} \\
          \hline
                 &                       &                    \\
          I      & \tilde{F}             & \tilde{\bm{b}}     \\
                 &                       &                    \\
          \hline
        \end{tabular}
      \end{table}
      \caption{Forma a tableau in termini canonici}
    \end{subfigure}
  \end{figure}
\end{definition}

Costruisco la forma canonica del sistema:

\[
  \tilde{c}_{\bm{0}} = -\bm{c}_B^T B^{-1} \bm{b} = - \begin{bmatrix}
    1 & -3
  \end{bmatrix}
  \left(
  \frac{1}{3}
  \begin{bmatrix}
    -2 & 3 \\
    1  & 0
  \end{bmatrix}
  \right)
  \begin{bmatrix}
    1 \\
    1
  \end{bmatrix}
  = \frac{10}{3}
\]

\[
  \tilde{F} = B^{-1}F = \frac{1}{3}\begin{bmatrix}
    -2 & 3 \\
    1  & 0
  \end{bmatrix}
  \begin{bmatrix}
    -3 & 2  \\
    1  & -1
  \end{bmatrix}
  =
  \frac{1}{3}
  \begin{bmatrix}
    4  & -7 \\
    -3 & 2
  \end{bmatrix}
\]

\[
  \tilde{b} = B^{-1}\bm{b} =  \frac{1}{3}\begin{bmatrix}
    -2 & 3 \\
    1  & 0
  \end{bmatrix}
  \begin{bmatrix}
    1 \\
    1
  \end{bmatrix}
  = \begin{bmatrix}
    \sfrac{1}{3} \\
    \sfrac{1}{3}
  \end{bmatrix}
\]

\begin{figure}
  \begin{table}
    \begin{tabular}{|LLLL|L|}
      \hline
      x_2                   & x_3 & x_1          & x_4           & \tilde{b}     \\
      \hline
      0                     & 0   & 1            & \sfrac{1}{3}  & \sfrac{10}{3} \\
      \hline
      \rowcolor{green!30} 1 & 0   & \sfrac{4}{3} & \sfrac{-7}{3} & \sfrac{1}{3}  \\
      0                     & 1   & -1           & \sfrac{2}{3}  & \sfrac{1}{3}  \\
      \hline
    \end{tabular}
  \end{table}
  \caption{Tableau in forma canonica}
\end{figure}

Applico la formula del taglio di Gomory (formula \ref{gomory_cut_formula}) al primo vincolo, evidenziato in verde:

\begin{align*}
  (1-1)x_2 + (0)x_3 + (\frac{4}{3}-1)x_1 + (-\frac{7}{3} + 3)x_4 & \geq \frac{1}{3} - \floor{\frac{1}{3}} \\
  \frac{1}{3}x_1 + \frac{2}{3}x_4                                & \geq \frac{1}{3}
\end{align*}

Il taglio di Gomory risulta quindi:

\[
  \frac{1}{3}x_1 + \frac{2}{3}x_4 \geq \frac{1}{3}
\]
\end{document}