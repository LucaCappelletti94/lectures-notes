 \providecommand{\main}{../../..}
\documentclass[\main/main.tex]{subfiles}
\begin{document}
\subsection{Esercizio 4}
Un caseificio vuole pianificare la produzione settimanale di burro ($B$), ricotta ($R$), mozzarella ($M$), sapendo che la domanda massima è rispettivamente di $230, 150, 140 Kg$ alla settimana e che il prezzo di vendita è rispettivamente di $3, 5, 8 \sfrac{euro}{Kg}$.

In una settimana sono disponibili fino a $45$ ore-macchina per la produzione e fino a $450$ litri di latte, al costo di $0.5 \sfrac{euro}{litro}$. Per produrre $1 Kg$ di burro, ricotta o mozzarella, servono rispettivamente $5, 2, 4 Kg$ di latte, mentre le “quote di produzione” (il numero di Kg che sarebbe prodotto in un'ora se tutte le macchine fossero utilizzate per un solo tipo di prodotto) sono rispettivamente di $15, 6, 11 Kg$.
Come deve produrre settimanalmente il caseificio per massimizzare i profitti?

Come cambia il modello se il caseificio deve scegliere in alternativa (o l'uno o l'altro) fra due tipi di mozzarella, quella sopra descritta, ($M$), e un tipo ($M_1$) caratterizzata da una domanda pari a $120 Kg$, un prezzo di vendita di $9 \sfrac{euro}{Kg}$, una richiesta di $4,5 Kg$ di latte per ogni $Kg$ di prodotto finito ed una quota di produzione di $12 Kg$?

Si deve formulare (non risolvere) il problema, definendo le variabili, la funzione obiettivo, i vincoli.

\subsection{Soluzione esercizio 4}

\subsubsection*{Identifico i dati}

\begin{figure}
  \begin{subfigure}{0.31\textwidth}
    \[
      T = 45
    \]
    \caption{Totale ore macchina}
  \end{subfigure}
  \begin{subfigure}{0.31\textwidth}
    \[
      L = 450
    \]
    \caption{Totale latte disponibile}
  \end{subfigure}
  \begin{subfigure}{0.31\textwidth}
    \[
      c_l = 0.5
    \]
    \caption{Costo latte per litro}
  \end{subfigure}
  \begin{subfigure}{0.24\textwidth}
    \[
      \bm{d} = \begin{bmatrix}
        230 \\
        150 \\
        140
      \end{bmatrix}
    \]
    \caption{Domanda dei prodotti}
  \end{subfigure}
  \begin{subfigure}{0.24\textwidth}
    \[
      \bm{p} = \begin{bmatrix}
        3 \\
        5 \\
        8
      \end{bmatrix}
    \]
    \caption{Prezzo di vendita}
  \end{subfigure}
  \begin{subfigure}{0.24\textwidth}
    \[
      \bm{l} = \begin{bmatrix}
        3 \\
        2 \\
        4
      \end{bmatrix}
    \]
    \caption{Latte per kg di prodotto}
  \end{subfigure}
  \begin{subfigure}{0.24\textwidth}
    \[
      \bm{q} = \begin{bmatrix}
        15 \\
        6  \\
        11
      \end{bmatrix}
    \]
    \caption{Quote produzione per prodotto}
  \end{subfigure}
  \caption{Dati del problema}
\end{figure}
\subsubsection*{Identifico le variabili}
Le variabili sono le quantità di prodotto per questa settimana, in $Kg$:

\[
  \bm{x} = \begin{bmatrix}
    x_{1 = burro}   \\
    x_{2 = ricotta} \\
    x_{3 = mozzarella}
  \end{bmatrix},
  \quad
  \bm{x} \in \mathbb{R}^3_{>0}
\]

\subsubsection*{Identifico la funzione obbiettivo}
Se si intende massimizzare i profitti, la funzione obbiettivo sarà il ricavato meno i costi:

\begin{figure}
  \[
    \max_{\bm{x}} z = \bm{p}^T\bm{x} - c_l\bm{l}^T\bm{x}
  \]
  \caption{Funzione obbiettivo}
\end{figure}

\subsubsection*{Identifico i vincoli del problema}

\begin{figure}
  \begin{subfigure}{0.31\textwidth}
    \[
      \sum_{i=1}^{\norm{\bm{x}}} \frac{x_i}{q_i} \leq T
    \]
    \caption{Vincolo ore macchina}
  \end{subfigure}
  \begin{subfigure}{0.31\textwidth}
    \[
      \bm{l}^T\bm{x} \leq L
    \]
    \caption{Vincolo su latte totale}
  \end{subfigure}
  \begin{subfigure}{0.31\textwidth}
    \[
      \bm{x} \leq \bm{d}
    \]
    \caption{Vincolo su domanda}
  \end{subfigure}
\end{figure}

\subsubsection*{Modello}

\begin{align*}
  \max_{\bm{x}} z = \bm{p}^T\bm{x} - c_l\bm{l}^T\bm{x}                       \\
  \text{s.v} \qquad \sum_{i=1}^{\norm{\bm{x}}} \frac{x_i}{q_i} & \leq T      \\
  \bm{l}^T\bm{x}                                               & \leq L      \\
  \bm{x}                                                       & \leq \bm{d} \\
\end{align*}

\subsubsection*{Modifiche al modello}
Si aggiunge ad ogni vettore il rispettivo nuovo valore del prodotto $M_1$ e si aggiunge un vincolo di mutua esclusività, aggiungendo un vettore binario definito come:

\[
  y_i \in \bm{y}, \quad y_i: \begin{cases}
    1\quad x_i > 0 \\
    0\quad \text{altr.}
  \end{cases}
\]

\[
  y_3 + y_4 \leq 1
\]

\subsubsection*{Modello modificato}

\begin{align*}
  \max_{\bm{x}} z = \bm{p}^T\bm{x} - c_l\bm{l}^T\bm{x}                       \\
  \text{s.v} \qquad \sum_{i=1}^{\norm{\bm{x}}} \frac{x_i}{q_i} & \leq T      \\
  \bm{l}^T\bm{x}                                               & \leq L      \\
  \bm{x}                                                       & \leq \bm{d} \\
  y_3 + y_4                                                    & \leq 1      \\
\end{align*}

\end{document}