\providecommand{\main}{..}
\documentclass[\main/main.tex]{subfiles}
\begin{document}
\chapter{Modelli di programmazione lineare}

\section{Formulazione di un modello PL}
Un modello di programmazione lineare si ottiene assumendo che \textbf{funzione obbiettivo} e \textbf{vincoli} e viene espresso come:

\begin{align*}
	\min x(x) = c_1x_1 + \ldots + c_nx_n       \\
	a_{11}x_1 +  \ldots + a_{1n}x_n & \geq b_1 \\
	                                & \vdots   \\
	a_{m1}x_1 +  \ldots + a_{mn}x_n & \geq b_m
\end{align*}

Esso include anche i vincoli di non negatività delle variabili di decisione:

\[
	x_j \geq 0 \quad \forall j \in \{1,\ldots,n\}
\]

In forma matriciale (compatta) il modello di PL può essere formulato nel modo seguente:

\begin{align*}
	\min z(x) = c^Tx \\
	Ax & \geq b      \\
	x  & \geq 0
\end{align*}

Dove $c \in \mathbb{R}^n$ è il vettore dei coefficienti della funzione obbiettivo mentre $b \in \mathbb{R}^m$ è il vettore dei termini noti dei vincoli ed $A$ è la matrice dei coefficienti delle variabili di decisioni nei vincoli.

\section{Modelli di pianificazione della produzione}
Dato un numero di risorse $m$ disponibile per la produzione di $n$ prodotti, $a_{ij}$ con $i \in \{1,\ldots,m\}$ e $j \in \{1,\ldots,n\}$ la quantità di risorsa $i$ necessaria per produrre una unità di prodotto $j$, $b_i$ la quantità della risorsa disponibile $i$ e $p_j$ il profitto lordo unitario ricavabile dalla vendita di un prodotto $j$, il modello PL è costituito come segue:

\begin{figure}
	\begin{align*}
		\max z(x) = \sum_{j=1}^n p_jx_j                                      \\
		\sum_{j=1}^n a_{ij}x_j & \leq b_i \quad \forall i \in \{1,\ldots,m\} \\
	\end{align*}
	\caption{Modello di pianificazione della produzione}
\end{figure}

\section{Modelli di miscelazione}
Si supponga di avere a disposizione $n$ ingredienti, ognuno dei quali contenente una certa quantità degli $m$ componenti, $a_{ij}$ la quantità di componente $i$ presente nell'ingrediente $j$ mentre $b_i$ rappresenta la quantità minima di componente $i$ richiesto nella miscela. Il costo unitario dell'ingrediente $j$ è indicato con $c_j$.

\begin{figure}
	\begin{align*}
		\min z(x) = \sum_{j=1}^n c_jx_j                                      \\
		\sum_{j=1}^n a_{ij}x_j & \geq b_i \quad \forall i \in \{1,\ldots,m\} \\
	\end{align*}
	\caption{Modello di miscelazione}
\end{figure}

Ulteriori vincoli tipici potrebbero essere la presenza di un componente $i$ minore di un valore $d_i$:

\[
	\sum_{j=1}^n a_{ij} x_j \leq d_i
\]

\section{Modelli di flusso su rete}

\subsection{Problema di flusso a costo minimo}  \label{min_cost_flux_sect}
Dato un grafo orientato $\mathcal{D} = (\mathcal{N}, \mathcal{A})$ dove $\mathcal{N}$ è l'insieme dei nodi, mentre $\mathcal{A}$ è l'insieme degli archi, si indica con $b_i$ con $i \in \mathcal{N}$ la fornitura (se positivo) o domanda (se negativo) del nodo $i$ e con $c_{ij}$, $l_{ij}$ e $u_{ij}$ rispettivamente il costo, la capacità minima e massima dell'arco $(i,j), \forall (i,j) \in \mathcal{A}$. La sestupla $R = (\mathcal{N}, \mathcal{A}, b, c, l, u)$ si definisce \textbf{rete}.

Il vincolo afferma che la differenza tra  la quantità di flusso entrante e la quantità uscente dal nodo deve essere uguale alla fornitura / domanda.

\begin{figure}
	\begin{align*}
		\min z(x) = \sum_{(i,j) \in \mathcal{A}} c_{ij}x_{ij}                                                             \\
		\sum_{(i,j) \in \mathcal{A}} x_{ij} - \sum_{(j,i) \in \mathcal{A}} x_{ji} & = b_i \quad \forall i \in \mathcal{N} \\
	\end{align*}
	\caption{Problema di flusso a costo minimo}
\end{figure}

\subsection{Problema del cammino orientato di costo minimo}
Basandosi sul caso base visto nel \textbf{problema di flusso a costo minimo \ref{min_cost_flux_sect}} aggiungiamo i nodi $s$ (origine) e $t$ (destinazione), considerando quindi $i\neq s \neq t$ e le forniture $b_i = 0$, $b_s = 1$ e $b_t = -1$:

\begin{figure}
	\begin{align*}
		\min z(x) = \sum_{(i,j) \in \mathcal{A}} c_{ij}x_{ij}                                                                 \\
		\sum_{(s,j) \in \mathcal{A}} x_{sj} - \sum_{(j,s) \in \mathcal{A}} x_{js} & = b_s = 1                                 \\
		\sum_{(i,j) \in \mathcal{A}} x_{ij} - \sum_{(j,i) \in \mathcal{A}} x_{ji} & = b_i = 0 \quad \forall i \in \mathcal{N} \\
		\sum_{(t,j) \in \mathcal{A}} x_{tj} - \sum_{(j,t) \in \mathcal{A}} x_{jt} & = b_t = -1                                \\
	\end{align*}
	\caption{Problema del cammino orientato di costo minimo}
\end{figure}

\subsection{Problema del massimo flusso}
Basandosi sempre sul caso base del \textbf{problema di flusso a costo minimo \ref{min_cost_flux_sect}}, poniamo i costi $c_{ij}$, capacità minime $l_{ij}$ e le forniture $b_i$ a $0$. L'obbiettivo posto è di inviare la massima quantità di flusso possibile da un nodo di ingresso $s$ (detto sorgente) ed uno di uscita $t$ (detto pozzo). Viene indicata con $v$ la fornitura del nodo $s$ (che non è un parametro ma una variabile dipendente dalle $x_{ij}$, rappresentante il flusso netto uscente da $s$)

\begin{figure}
	\begin{align*}
		\min z(x) = v                                                                                                         \\
		\sum_{(s,j) \in \mathcal{A}} x_{sj} - \sum_{(j,s) \in \mathcal{A}} x_{js} & = b_s =v                                  \\
		\sum_{(i,j) \in \mathcal{A}} x_{ij} - \sum_{(j,i) \in \mathcal{A}} x_{ji} & = b_i = 0 \quad \forall i \in \mathcal{N} \\
		\sum_{(t,j) \in \mathcal{A}} x_{tj} - \sum_{(j,t) \in \mathcal{A}} x_{jt} & = b_t = -v                                \\
		x_{ij}                                                                    & \leq u_{ij} \quad (i,j) \in \mathcal{A}
	\end{align*}
	\caption{Problema del massimo flusso}
\end{figure}

\subsection{Problema di trasporto}
Dati $n$ nodi di origine (stabilimenti di produzione) con una produzione di $a_i, i \in \{1, \ldots, n\}$ e $m$ nodi di destinazione (punti vendita), ciascuno caratterizzato da una domanda $b_j, j \in \{1, \ldots, m\}$ ed un costo unitario di trasporto $c_{ij}$. L'obbiettivo del problema è di determinare il quantitativo di prodotto da inviare da ciascuna origine verso ciascuna destinazione in modo tale da minimizzare il costo complessivo di trasporto rispettando i vincoli sulla quantità di prodotto disponibile in ciascuna origine e garantendo il soddisfacimento delle domande di ogni destinazione.

\begin{figure}
	\begin{align*}
		\min z(x) = \sum_{i=1}^n\sum_{j=1}^m c_{ij}x_{ij} \\
		\sum_{j=1}^m x_{ij} & \leq a_i \quad \forall i    \\
		\sum_{i=1}^n x_{ij} & = b_j  \quad \forall j      \\
	\end{align*}
	\caption{Problema di trasporto}
\end{figure}

Per ricondursi al caso in cui vale il vincolo $\sum_{j=1}^m x_{ij} = a_i$ è sempre possibile aggiungere una destinazione fittizia $m+1$ che funge da discarica.

Una variante del problema di trasporto considera la possibilità di includere $p$ nodi intermedi di transito, che possono scambiare il materiale anche tra loro. Questo porta il numero delle origini e destinazioni a divenire $n+p$ e $m+p$ (ogni origine può inviare a $p$ nuovi nodi ed ogni destinazione può ricevere da $p$ nuovi nodi). Diviene necessario aggiungere due vincoli ulteriori per modellare i nodi $p$ come intermedi, cioè che ogni punto di transito abbia un flusso entrante coincidente con il flusso uscente e che non ponga ulteriori limitazioni:


\begin{figure}
	\begin{align*}
		\min z(x) = \sum_{i=1}^{n+p}\sum_{j=1}^{m+p} c_{ij}x_{ij}       \\
		\sum_{j=1}^{m+p} x_{ij} & \leq a_i \quad \forall i              \\
		\sum_{i=1}^{n+p} x_{ij} & = b_j    \quad \forall j              \\
		\sum_{j=1}^{m+p} x_{ij} & = \sum_{j=1}^m b_j    \quad \forall i \\
		\sum_{i=1}^{n+p} x_{ij} & = \sum_{j=1}^m b_j    \quad \forall j \\
	\end{align*}
	\caption{Problema di trasporto con $p$ nodi intermedi}
\end{figure}

\subsection{Problema dell'assegnamento}
Supponiamo di avere $n$ oggetti (per esempio lavoratori) ed altrettanti posti (per esempio postazioni di lavoro) associate da un costo di assegnamento $c_{ij}$. Il problema consiste nel determinare il modo più conveniente di assegnare ogni oggetto $i$ ad uno e un solo posto $j$. Il problema è a variabili di tipo binario ($x_{ij} \in \{0,1\}$).

\begin{figure}
	\begin{align*}
		\min z(x) = \sum_{i=1}^n\sum_{j=1}^n c_{ij}x_{ij} \\
		\sum_{j=1}^n x_{ij} & = 1 \quad \forall i         \\
		\sum_{i=1}^n x_{ij} & = 1  \quad \forall j        \\
	\end{align*}
	\caption{Problema dell'assegnamento}
\end{figure}

Varianti tipiche possono essere sull'assegnare un numero oggetti diverso dal numero di posti, che vanno a modificare i vincoli di uguaglianza a $\sum_{j=1}^n x_{ij} \leq 1$.

\section{Modelli multi periodo}
Modelli in cui viene utilizzata un intervallo di tempo, con $t \in \{1, \ldots, T\}$ la generica frazione di tempo, genericamente utilizzata per la minimizzazione dei costi su un intervallo o massimizzazione di un'utilità.

\end{document}