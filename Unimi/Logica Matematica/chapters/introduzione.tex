% TEX root = ../main.tex
La Logica è una materia che ha una tradizione millenaria e trae le sue 
origini in ambito filosofico: la definizione che vedremo noi è infatti 
presa da quell'ambito, e noi la declineremo in una forma moderna. 
La Logica è lo studio dei meccanismi del ragionamento razionale. 
In altre parole, si intende lo studio della capacità di trarre conseguenze 
(corrette) da date assunzioni o premesse. Le assunzioni sono delle informazioni 
che affermano che il mondo sta in un certo modo e sono gestibili formalizzandole 
in un linguaggio formale. Queste informazioni codificano uno o più mondi possibili: 
vogliamo trarne delle conclusioni a partire di esse in un modo razionale. 

\section{Motivazioni}
La Logica studia come si ragiona in maniera corretta e, per studiare come 
si ragione, si può utilizzare come prima schematizzazione il partire da delle 
assunzioni vere e da quelle discendere a delle conclusioni. 
Un esempio: ogni uomo è mortale. Socrate è un uomo. 
Dunque, Socrate è mortale. Questo è, in linguaggio naturale, un esempio di quanto 
detto prima: a partire da due informazioni date per vere (ogni uomo è mortale e 
Socrate è un uomo) si traggono delle conseguenze. Quanto fatto prima è un 
\textit{sillogismo}, un punto antichissimo nella storia della Logica.

Altro esempio: ogni gatto ha sette zampe. Pluto è un gatto. Dunque, Pluto 
ha sette zampe. Anche questa è una deduzione esatta, nonostante per l'esperienza 
comune la prima assunzione è falsa; tuttavia, la Logica si occupa di \textit{ogni}
universo e pertanto il ragionamento è valido. 
Ogni Blabla è glug. Sbappo è Blabla. Dunque, Sbappo è glug. Questa forma di 
ragionamento è altrettanto corretta. 
Per non farsi distrarre dalle stranezze irrilevanti, si utilizza un linguaggio 
centrale per il discorso della Logica. Si formalizza quindi questo ragionamento: 
in primo luogo si astrae, fornendo un modello matematico per ragionare.
$$
(\forall x P(x) \rightarrow Q(x) \land P(s)) \rightarrow Q(s)
$$
Ogni fiore è profumato. La Rosa è profumata. Dunque, la Rosa è un fiore. 
Per mostrare che questo ragionamento è falso, si può anche utilizzare l'intuizione: 
se Rosa è mia nonna, benché il senso metaforico sia valido, Rosa è un po' vecchia 
e pertanto il ragionamento non è valido. In che modo è cambiato il ragionamento? 
$$
(\forall x P(x) \rightarrow Q(x) \land Q(s)) \nrightarrow P(s)
$$
In questo caso, si sta cercando di verificare la premessa data la conclusione, 
al contrario di quanto accadeva per il sillogismo aristotelico; questo modo 
di ragionare non può funzionare. 

\paragraph{Matematica}
Vi sono almeno due sensi per cui la Logica è matematica. Il primo è quello che abbiamo 
introdotto immediatamente al discorso iniziale: la matematica è utilizzata per 
la necessità di \textit{astrarre} solamente le informazioni rilevanti scartando 
il resto in un contesto con molte informazioni che non ci interessano, come 
accade per il linguaggio naturale. La trasformazione delle assunzioni e delle 
conclusioni da una forma in linguaggio naturale alla forma astratta permette 
di arrivare a delle \textbf{forme}. I concetti che andremo a formalizzare 
avranno una sintassi dettata da un \textbf{linguaggio formale} e un significato 
semantico \textbf{algebrico-insiemistico}, ottenendo un \textbf{formalismo}, un
modo preciso, rigoroso e privo di ambiguità per esprimere ciò che si vuole 
esprimere in Logica. 

In seconda battuta, la Logica si usa per studiare le strutture matematiche, 
ossia si usa \textit{per fare} matematica. Aprendo un testo qualsiasi di 
Logica matematica si vedrà come gli esempi più interessanti siano basati 
sulla matematica: gruppi, campi e teoremi vari. \newline


\noindent
Una terza parola chiave, che conclude la parte motivazionale, è \textbf{Informatica}, 
intesa come \textit{Computer Science}, inteso come capire il processo dei 
sistemi computazionali. \`E stata infatti una grande rivincita della Logica 
durante il secolo scorso, che ha visto nascere i fondamenti della computazione 
partendo da strumenti logici. Se esiste, la differenza tra \textit{Logica} 
e \textit{Informatica} sono i focus diversi: la prima è più vicina ad un 
approccio dichiarativo, concentrandosi su ciò che si può concludere da 
determinate premesse, mentre la seconda è più vicina ad approcci 
procedurali o imperativi.

\paragraph{Esercizio}
Se piove prendo l'ombrello. \`E lo stesso caso di dire che: 
\begin{enumerate}
  \setlength\itemsep{0pt}
  \item Se non piove non prendo l'ombrello
  \item Se non prendo l'ombrello non piove
  \item Se prendo l'ombrello allora piove
  \item O non piove o prendo l'ombrello 
  \item Piove solo se prendo l'ombrello 
  \item Se prendo l'ombrello piove 
  \item Piove se e solo se prendo l'ombrello
  \item Nessuna delle precedenti
\end{enumerate}

Benché non si sappia cosa voglia dire ``lo stesso caso'', tentiamo di dare le soluzioni 
a questo problema. Nel linguaggio naturale non si può fare a meno di sentire 
una dinamica: si vede che piove e allora si prende l'ombrello e si esce. La logica 
proposizionale non vede questa dinamicità: per farlo si devono elaborare formule 
che esplicitano la dinamicità. Una definizione più precisa di cosa voglia dire 
che due ``frasi'' siano ``uguali'': esse sono ``equivalenti'' quando sono 
vere nelle medesime circostanze. Questa è nuovamente una definizione che pecca 
di precisione in quanto non espressa matematicamente. Cosa vuol dire ``medesime'' 
e ``circostanze''? Un'interpretazione intuitiva che mette in luce la 
``circostanza'' della frase ``Se piove prendo l'ombrello'' è la seguente. Vi 
è una dipendenza tra il fatto che \textit{piove} e il fatto di \textit{prendere 
l'ombrello}. 

In tutte le circostanze possibili, vi sono delle situazioni in cui 
è vero che piove e delle situazioni in cui non è vero e analogamente 
accade per il fatto di prendere l'ombrello. 
Di tutte le possibili circostanze ci interessano 
solo quattro: quando non piove e quando non si prende l'ombrello, quando piove 
e si prende l'ombrello, quando piove e non si prende l'ombrello e, infine, 
quando non piove e si prende l'ombrello. 
 
La frase si può dunque rappresentare nella forma 
$$
P \Rightarrow Q
$$
che rappresenta il \textit{se...allora}. L'implicazione è infatti la più 
difficile da accettare a livello intuitivo. Senz'altro ci sono delle situazioni 
in cui non abbiamo dubbi: per esempio, quando sia $P$ e $Q$ sono vere, cioè, 
sapendo che piove e che si prende l'ombrello la relazione causale sembra sussistere 
e quindi $P \Rightarrow Q$ è verificata; quando $P$ è vero e $Q$ è falso,
risulta infine che $P \Rightarrow Q$ è falsa. Ora, potrebbe succedere che 
la risposta non sia esattamente quella che ci aspettiamo: cominciamo assumendo 
che non piova ma si prenda l'ombrello. Risulta che $P \Rightarrow Q$ è vera, 
anche se l'antecedente è falso: già questa cosa può suonare strano, in quanto 
non suona giusto che il fatto che non piova implichi il fatto che si prenda l'ombrello. 
La questione riguarda sostanzialmente il linguaggio naturale di per sé e torneremo 
su questo discorso in futuro: con lo stesso approccio, si arriva a dire che se 
$P$ e $Q$ sono false allora $P \Rightarrow Q$ è vera. 

Allora, per concludere il nostro esempio: si può cominciare dicendo che l'ottava 
frase è falsa, ossia vi sono, tra le prime sette frasi, alcune frasi equivalenti. 
La prima è sbagliata, in quanto vi è la possibilità di prendere l'ombrello 
anche se non piove. Il rapporto di causalità si rivede anche nella terza frase, 
che è falsa. La quarta frase necessità l'analisi della disgiunzione, l'OR, 
che in questa situazione va interpretato come un OR inclusivo, ossia un 
$\lor$. Analizzando la frase ``O non piove o prendo l'ombrello'' ci si accorge 
come si possa tradurre in $\neg P \lor Q$, che ha la stessa ``immagine di 
verità'' dell'implicazione, pertanto anche la quarta è uguale. Questo è importante 
in quanto è una realizzazione materiale dell'implicazione! 
La quinta frase si può nuovamente interpretare come $P\Rightarrow Q$ ed è pertanto 
vera. 
La sesta frase inverte il rapporto, facendo in modo che $Q \Rightarrow P$, che 
è falso; analogamente accade per la settima. 

Un'ulteriore interpretazione dell'implicazione è: Ogni qualvolta $P$ è vera, 
anche $Q$ è vera. 

\section{Programma}
Il corso tratterà inizialmente la Logica Proposizionale, mentre la seconda 
parte tratterà la Logica Predicativa o del Prim'ordine. Della Logica Proposizionale 
si definirà la sintassi, quindi Alfabeto, Connettivi e Formule per poi parlare
di valutazione, tabelle di verità e princìpi come verofunzionalità e bivalenza. 
Si discuteranno le tautologie, le contraddizioni, le formule soddisfacibili e
la nozione centrale di Conseguenza logica. Seguentemente si 
tratteranno decidibilità, correttezza e completezza, ma in realtà il primo Teorema 
che tratteremo sarà quello di Compattezza. Dopodiché si potranno affrontare 
i metodi formali di deduzione per la Logica Proposizionale. Accenneremo ai 
Seguenti e ai Tableau, oltre che ai calcoli alla Hilbert e i metodi assiomatici. 
Ci concentreremo sulla metodologia più adatta alla deduzione automatica, ossia 
i metodi refutazionali basati sul principio di risoluzione. 

La seconda parte del corso tratterà la Logica dei Predicati, che per noi 
sarà un sinonimo di Logica del Prim'ordine. Dal punto di vista sintattico, 
si affronteranno più approfonditamente alfabeto, quantificatori, simboli 
di predicato e i simboli di funzione. Seguirà la semantica: descriveremo la 
semantica di Tarski, con la nozione fondamentale di L-Struttura e modelli; il 
concetto di Conseguenza logica,completezza e correttezza nell'ambito 
della Logica del Prim'ordine. Termineremo con i metodi di deduzione e le forme 
normali. Assieme alla Teoria di Herbrand, quest'ultime ci permetteranno di 
arrivare alle tecniche di deduzione automatica. 
