% TEX root = ../../main.tex

\chapter{Introduzione e Sintassi}
Si può immaginare la Logica del Primo Ordine come ``costruita'' 
sulla base della logica proposizionale. Il sillogismo aristotelico
\begin{prooftree}
        \AxiomC{Ogni uomo è mortale}
        \AxiomC{Socrate è un uomo}
        \BinaryInfC{Quindi Socrate è mortale}
\end{prooftree}
espresso come possibile nella Logica proposizionale diventa 
\begin{prooftree}
        \AxiomC{$p$}
        \AxiomC{$q$}
        \BinaryInfC{$r$}
\end{prooftree}
Ci si chiede se sia vero, quindi: è forse 
$$
p, q \models r
$$
Con DPLL ci si chiede se 
$\{p, q, \neg r\}$ sia insoddisfacibile. 
Il primo passo di DPLL afferma
\begin{prooftree}
        \AxiomC{$\emptyset \vdash \{\{p\}, \{q\}, \{\neg r\}\}$}
        \UnaryInfC{$\mathfrak{v}(p) = 1 \vdash \{\{q\}, \{\neg r\}\}$}
        \UnaryInfC{$\mathfrak{v}(p)=1, \mathfrak{v}(q) = 1 \vdash \{\{\neg r\}\}$}
        \UnaryInfC{$\mathfrak{v}(r) = 0, \mathfrak{v}(p)=1, \mathfrak{v}(q) = 1 \vdash \emptyset$}
\end{prooftree}
e pertanto 
$$
p, q \models r \text{ è falso }
$$
che va ovviamente contro la nostra intuizione. 
Per gestire argomentazioni
razionali come il sillogismo aristotelico è necessario dotarsi di un linguaggio 
più \textit{espressivo} rispetto alla Logica Proposizionale, arricchendone 
la Sintassi con nuovi operatori, variabili, costanti e la Semantica assegnando 
un modo di interpretare i nuovi ``ingredienti'' del linguaggio in modo da poter 
rappresentare situazioni più raffinate che nella Logica Proposizionale. 

Si arriverà a scrivere 
$$
\forall x (U(x)\rightarrow M(x)), U(s) \models M(s)
$$
per indicare il sillogismo aristotelico.
Oltre ad andare a vedere come mettere in piedi, di primo acchito a livello sintattico, 
tutta questa struttura, si studierà anche il modo per \textit{risolvere} 
delle asserzioni, riutilizzando le tecnologie introdotte per la logica proposizionale.
In particolare, per fare un esempio introduttivo, si tornerà a chiedersi se 
il sillogismo aristotelico sia valido ponendosi  il quesito 
$$
\{\{\neg U(x), M(x)\}, \{U(s)\}, \{\neg M(s)\}\} \text{ è soddisfacibile?}
$$

Si consideri nuovamente 
$$
\text{ Ogni uomo è mortale }
$$
I modi di designare direttamente dell'\textit{Universo} sono un ingrediente 
importante della Logica dei Predicati, fornendo per esempio il modo di 
affermare che Socrate sia mortale, riferendosi ad un preciso elemento. 
Si necessita un modo per designare un elemento non preciso, in maniera indiretta: 
$$
\text{ Il padre di ogni uomo è un uomo }
$$
Si può concludere, quindi, che il padre di Socrate sia un mortale, 
nonostante sia una designazione indiretta di un individuo dell'Universo; 
si può inoltre tradurre quanto detto come 
\begin{prooftree}
        \AxiomC{$\forall x (U(x) \rightarrow M(x)$}
        \AxiomC{$\forall (U(x) \rightarrow U(p(x))$}
        \AxiomC{ $U(s)$}
        \TrinaryInfC{$M(p(s))$}
\end{prooftree}

Per decidere questa \textit{deduzione}, ci sarà qualcosa come 
$$
\{\{\neg U(x), M(x)\}, \{\neg U(x), U(p(x))\}, \{U(s)\}, \{\neg M(p(s))\}\}
$$
Quindi, scopriremo che non basterà sostituire al posto di $x$ la ``lettera'' 
$s$, ma sarà anche necessario considerare $p(s)$. A questo punto, sarà immediato 
arrivare alla conclusione che la situazione sia in realtà un po' più complicata: 
come si considera $p(s)$, si dovrebbe considerare anche $p(p(s))$, $p(p(p(s)))$...

Nel nostro caso, una possibile refutazione è la seguente: 
$$
\neg U(p(s)), M(p(s))
$$
istanziando la $x$ su $p(s)$, in questo modo si può risolvere 
rispetto a $\neg M(p(s))$: 
$$
\{\{\neg U(p(s))\}, \{\neg U(x), U(p(x))\}, \{U(s)\}\}
$$
e si prosegue istanziando $U(x)$ a $U(s)$, arrivando all'insieme vuoto. 
Vedremo, quindi, perché e quando si possono utilizzare queste istanzianzioni. 

\section{Sintassi della Logica del Primo Ordine}
Partiamo, dunque, fissando i dettagli sintattici, cioè il Linguaggio 
della Logica del Primo Ordine nella sua natura grammaticale. 

A livello proposizionale, si fissa $L$ come insieme infinito di lettere 
proposizionali, e si conclude il tutto, poiché da questo insieme si arriva 
direttamente a costruire ogni formula ammissibile nella Logica Proposizionale.

Innanzitutto, nella Logica del Primo Ordine vi sono diversi linguaggi detti 
\textbf{linguaggi elementari}; la situazione è simile alla situazione generica 
dei linguaggi formali, anche se tecnicamente $L$ sarebbe un alfabeto per e $F_L$ 
sarebbe il linguaggio formale, mentre in logica $L$ stesso è chiamato Linguaggio. 

Allo stesso modo, nella logica del Prim'Ordine ci sono dei linguaggi elementari, 
che specificano gli ``ingredienti'' per costruire le formule della Logica dei 
Predicati, le quali sarebbero il \textit{vero} linguaggio nel senso formale. 
In altri termini, in Logica dei Predicati 
il termine \textit{linguaggio elementare} indica l'insieme di elementi che andranno 
a costruire l'insieme delle formule. 

\begin{defi}[Linguaggio Elementare]
I linguaggi elementari sono definiti come $L = (\mathcal{P}, F, \alpha, \beta)$  dove $P$ è un insieme di 
simboli, detti \textbf{predicati} (o simboli di predicato), che deve 
essere diverso dall'insieme vuoto $\mathcal{P} \neq \emptyset$; $F$ è l'insieme 
di simboli detti \textbf{di funzione}, disgiunto da $\mathcal{P}$: $\mathcal{P} \cap F = \emptyset$; 
$\alpha$ è una funzione $P \rightarrow \mathbb{N}$ assegna l'arità a ogni 
$\mathcal{P} \in P$ e, infine, $\beta$ è  l'arità di ogni $f \in F$.
\end{defi}

Ogni elemento del linguaggio varia in base al linguaggio stesso, tuttavia 
vi sono ``ingredienti'' intrinsecamente comuni a tutti i linguaggi elementari:  
l'insieme $V$ (o $Var$) infinito di simboli detti \textbf{variabili individuali}, 
chiamate così perché la loro interpretazione sarà quella di un \textit{individuo 
generico} dell'universo del discorso
e l'insieme dei connettivi $\land, \lor, \neg, \rightarrow, \bot, \top, \iff, \cdots$; 
differentemente dalla logica proposizionale, i linguaggi elementari contengono 
anche i \textbf{quantificatori} $\forall, \exists$. 

Il linguaggio $F_L$ delle formule sul linguaggio elementare $L$ sarà definito 
a partire dai simboli in $L$ e dai possibili connettivi. 

\begin{oss}
        Esiste un predicato speciale nella logica del primo ordine 
        che a livello sintattico è uguale agli altri, ma la sua interpretazione 
        è fissata. Se $\mathcal{P}$ contiene il simbolo $'='$, la sua arità 
        sarà fissata $\alpha('=') = 2$ e $L = (\mathcal{P}, F, \alpha, \beta)$ 
        è detto \textit{linguaggio con identità}.
\end{oss}

\begin{defi}[Formule]
La definizione di \textbf{Formula di} $L$ ($F \in F_L$) è data per strati 
e induttivamente: il primo strato è quello dei \textit{termini}, 
i quali \textbf{non sono formule} ma si \textit{usano} per costruire formule.

\begin{defi}[$L$-termini]
        Sia $L = (\mathcal{P}, F, \alpha, \beta)$. 

        Allora l'insieme $T_L$ degli $L$-termini è definito come segue: 

        \paragraph{Base} 
        \begin{itemize}
                \item Ogni $x \in Var(L)$ è  un termine. 
                \item Per ogni $c \in F$ tale che $\beta(c) = 0$, $c$ è un termine, 
        detto \textbf{costante}. 
        \end{itemize}

        \paragraph{Passo Induttivo} Se $f \in F$ è tale che $\beta(f) = n$ e 
        $t_1, t_2, \cdots, t_n \in T_L$ (sono termini già costruiti), 
        allora $f(t_1, t_2, \cdots, t_n)$ è un $L$-termine. 
        
        Nient'altro è un $L$-termine.
\end{defi}

        L'insieme $F_L$ delle $L$-Formule è definito come segue. 

        \paragraph{Base} Se $p \in \mathcal{P}$ e $\alpha(p)=n$ e $t_1, \cdots, t_n \in T_L$
        allora $P(t_1, \cdots, t_n)$ è una $L$-Formula detta 
        \textbf{formula atomica}. Le formule atomiche sono il corrispettivo 

        \paragraph{Passo Induttivo} Se $A,B \in F_L$ allora anche 
        $A \land B$, $A \lor B$, $\neg A$, $A\rightarrow B$ sono $L$-Formule.
        Se $A \in F_L$ e $x \in V$ allora anche $\forall x A$ e $\exists x A$
        sono $L$-Formule. Se in $A$ non occorre $x$, $\forall x A$ 
        e $\exists x A$ sono comunque formule, come anche 
        $\forall x (\forall x A)$ e $\exists x (\forall x A)$. 

        Nient'altro è una $L$-Formula.
\end{defi}

\paragraph{Esercizio} Dare una nozione di \textit{certificato} per 
$L$-termini e per $L$-Formule analoga alla $L$-costruzione in logica proposizionale. 

\subsection{Terminologia}
Introduciamo alcune nozioni terminologiche per potersi riferire alla sintassi 
della Logica dei Predicati.
Si userà liberamente la scrittura semplificata di formule omettendo coppie di parentesi
quando questo non causa ambiguità, ossia invece di $(\forall x A)$ si scriverà $\forall x A$.

\begin{defi}[Termine Ground]
        Un termine è detto \textbf{chiuso} o \textbf{ground} se è costruito 
        senza utilizzare \textit{variabili}. 
\end{defi}

\begin{defi}[Variabile vincolata]
        Una occorrenza di una variabile $x$ in una Formula $A \in F_L$ è 
        detta \textbf{vincolata} se occorre all'interno di una sottoformula 
        di $A$ (ossia una formula che appare in ogni certificato della $L$-formula $A$) 
        del tipo $\forall x B$ o $\exists x B$. 
\end{defi}

\begin{defi}[Variabile libera]
Una variabile $x \in V$ occorre libera in $A \in F_L$ se e solo se qualche  
sua occorrenza è libera. Questa definizione permette di dare definizioni 
anche nelle situazioni come l'osservazione precedente.
\end{defi}

Per esempio, nella formula 
$$
A = (\forall x R(x,y)) \lor P(x)
$$
$x$ è sia vincolata che libera, mentre $y$ è libera. 
Nella formula 
$$
A' = \forall x (\forall y R(x,y)) \lor P(x)
$$

Sia $x$ che $y$ occorrono sempre vincolate. 

Questo ci permette di introdurre il concetto fondamentale sul quale lavoreremo: 
infatti, non ragioneremo più su \textit{formule} quando definiremo la semantica 
della Logica dei Predicati, ma si ragionerà su un particoalre tipo di formula: 
\begin{defi}[$L$-sentence o Enunciato]
        Un $L$-\textbf{enunciato} o $L$-\textbf{sentence}, detto anche formula 
        chiusa, è una $L$-formula senza occorrenze di variabili libere 
        (o senza variabili libere).
\end{defi}

\begin{defi}[Sostituzione]
        Con la notazione $A[t/x]$ con $A$ una formula, $x$ variabile 
        e $t$ un termine, si intende la formula ottenuta rimpiazzando 
        simultaneamente tutte e sole le occorrenze \textbf{libere} di $x$ 
        con $t$. 
\end{defi}
