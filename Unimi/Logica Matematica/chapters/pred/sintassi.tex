% TEX root = ../../main.tex

\chapter{Introduzione e Sintassi}
Si può immaginare la Logica del Primo Ordine come ``costruita'' 
sulla base della logica proposizionale. Il sillogismo aristotelico
\begin{prooftree}
        \AxiomC{Ogni uomo è mortale}
        \AxiomC{Socrate è un uomo}
        \BinaryInfC{Quindi Socrate è mortale}
\end{prooftree}
espresso come possibile nella Logica proposizionale diventa 
\begin{prooftree}
        \AxiomC{$p$}
        \AxiomC{$q$}
        \BinaryInfC{$r$}
\end{prooftree}
Ci si chiede se sia vero, quindi: 
$$
p, q \stackrel ? \models r
$$
Con DPLL ci si chiede se 
$\{p, q, \neg r\}$ sia insoddisfacibile. 
Il primo passo di DPLL afferma
\begin{prooftree}
        \AxiomC{$\emptyset \vdash \{\{p\}, \{q\}, \{\neg r\}\}$}
        \UnaryInfC{$\mathcal{v}(p) = 1 \vdash \{\{q\}, \{\neg r\}\}$}
        \UnaryInfC{$\mathcal{v}(p)=1, \mathcal{v}(q) = 1 \vdash \{\{\neg r\}\}$}
        \UnaryInfC{$\mathcal{v}(r) = 0, \mathcal{v}(p)=1, \mathcal{v}(q) = 1 \vdash \emptyset$}
\end{prooftree}
e pertanto 
$$
p, q \models r \text{ è falso }
$$
che va ovviamente contro la nostra intuizione. La modellizzazione della frase in italiano non è più adeguata a descrivere il mondo su cui si sta lavorando!
Per gestire argomentazioni
razionali come il sillogismo aristotelico è necessario dotarsi di un linguaggio 
più \textit{espressivo} rispetto alla Logica Proposizionale, arricchendone 
la Sintassi con nuovi operatori, variabili, costanti e la Semantica assegnando 
un modo di interpretare i nuovi ``ingredienti'' del linguaggio in modo da poter 
rappresentare situazioni più raffinate che nella Logica Proposizionale. 

Si arriverà a scrivere 
$$
\forall x (U(x)\rightarrow M(x)), U(s) \models M(s)
$$
per indicare il sillogismo aristotelico.
Oltre ad andare a vedere come mettere in piedi, di primo acchito a livello sintattico, 
tutta questa struttura, si studierà anche il modo per \textit{risolvere} 
delle asserzioni, riutilizzando le tecnologie introdotte per la logica proposizionale.
In particolare, per fare un esempio introduttivo, si tornerà a chiedersi se 
il sillogismo aristotelico sia valido ponendosi  il quesito 
$$
\{\{\neg U(x), M(x)\}, \{U(s)\}, \{\neg M(s)\}\} \text{ è soddisfacibile?}
$$

Si consideri nuovamente 
$$
\text{ Ogni uomo è mortale }
$$
I modi di designare direttamente dell'\textit{Universo} sono un ingrediente 
importante della Logica dei Predicati, fornendo per esempio il modo di 
affermare che Socrate sia mortale, riferendosi ad un preciso elemento. 
Si necessita un modo per designare un elemento non preciso, in maniera indiretta: 
$$
\text{ Il padre di ogni uomo è un uomo }
$$
Si può concludere, quindi, che il padre di Socrate sia un mortale, 
nonostante sia una designazione indiretta di un individuo dell'Universo; 
si può inoltre tradurre quanto detto come 
\begin{prooftree}
  \AxiomC{$\forall x (U(x) \rightarrow M(x)$}
  \AxiomC{$\forall x (U(x) \rightarrow U(p(x))$}
  \AxiomC{ $U(s)$}
  \TrinaryInfC{$M(p(s))$}
\end{prooftree}

Per decidere questa \textit{deduzione}, ci sarà qualcosa come 
$$
\{\{\neg U(x), M(x)\}, \{\neg U(x), U(p(x))\}, \{U(s)\}, \{\neg M(p(s))\}\}
$$
Quindi, scopriremo che non basterà sostituire al posto di $x$ la ``lettera'' 
$s$, ma sarà anche necessario considerare $p(s)$. A questo punto, sarà immediato 
arrivare alla conclusione che la situazione sia in realtà un po' più complicata: 
come si considera $p(s)$, si dovrebbe considerare anche $p(p(s))$, $p(p(p(s)))$...

Nel nostro caso, si trova una possibile refutazione istanziando la $x$ su $p(s)$ come segue: 
\begin{prooftree}
  \AxiomC{$\neg U(p(s)), M(p(s))$}
  \AxiomC{$\neg M(p(s))$}
  \BinaryInfC{$\neg U(p(s))$}
  \AxiomC{$\neg U(s), U(p(s))$}
  \BinaryInfC{$\neg U(s)$}
  \AxiomC{$U(s)$}
  \BinaryInfC{$\qedsymbol$}
\end{prooftree}
Vedremo, quindi, perché e quando si possono utilizzare queste istanzianzioni. 

\section{Sintassi della Logica del Primo Ordine}
Partiamo, dunque, fissando i dettagli sintattici, cioè il Linguaggio 
della Logica del Primo Ordine nella sua natura grammaticale. 

Nella Logica Proposizionale bastava fissare $\mathscr{L}$ come insieme infinito di lettere proposizionali per costruire direttamente ogni formula ammissibile in $\mathscr{F}_\mathscr{L}$.

Nella Logica del Primo Ordine, invece, vi sono diversi linguaggi detti \textbf{linguaggi elementari}. Questi sono un insieme degli elementi per costruire il \textit{vero} linguaggio in senso formale, ovvero le formule della Logica dei Predicati.

\begin{defi}[Linguaggio Elementare]
I linguaggi elementari sono definiti come $\mathscr{L} = (\mathscr{P}, \mathscr{F}, \alpha, \beta)$,  dove:
\begin{itemize}
 \item $\mathscr{P}$ è un insieme non vuoto di simboli, detti \textbf{predicati} (o simboli di predicato)
 \item $\mathscr{F}$ è l'insieme di simboli detti \textbf{di funzione}. È disgiunto da $\mathscr{P}$; $\mathscr{P} \cap F = \emptyset$
 \item $\alpha$ è una funzione $\mathscr{P} \rightarrow \mathbb{N}$ che assegna l'arietà a ogni $p \in \mathscr{P}$
 \item $\beta$ è una funzione $\mathscr{F} \rightarrow \mathbb{N}$ che assegna l'arietà a ogni $f \in \mathscr{F}$
\end{itemize}
\end{defi}

Ogni elemento del linguaggio varia in base al linguaggio stesso, tuttavia vi sono ``ingredienti'' intrinsecamente comuni a tutti i linguaggi elementari:
\begin{itemize}
  \item $\mathscr{V}$ (o $Var$): insieme infinito di simboli detti \textbf{variabili individuali}, chiamate così perché la loro interpretazione sarà quella di un \textit{individuo generico} dell'universo del discorso
  \item \textit{Connettivi}: $\land, \lor, \neg, \rightarrow, \bot, \top, \leftrightarrow, \cdots$
  \item \textit{Quantificatori}: $\forall, \exists$.
\end{itemize} 

Il linguaggio $\mathscr{F}_\mathscr{L}$ delle formule sul linguaggio elementare $\mathscr{L}$ sarà definito 
a partire dai simboli in $\mathscr{L}$ e dai possibili connettivi. 

\begin{oss}
        Esiste un predicato speciale nella logica del primo ordine che a livello sintattico è uguale agli altri, ma la sua interpretazione è fissata. Se $\mathscr{P}$ contiene il simbolo $'='$, la sua arietà sarà fissata $\alpha('=') = 2$ e $\mathscr{L} = (\mathscr{P}, \mathscr{F}, \alpha, \beta)$ è detto \textit{linguaggio con identità}.
\end{oss}

\begin{defi}[$\mathscr{L}$-termini]
  Sia $\mathscr{L} = (\mathscr{P}, \mathscr{F}, \alpha, \beta)$. \\
  Allora l'insieme $T_\mathscr{L}$ degli $\mathscr{L}$-termini è definito come segue:
  \begin{itemize}
    \item \textbf{base}:
      \begin{itemize}
        \item ogni $x \in \mathscr{V}_\mathscr{L}$ è un termine. 
        \item ogni $c \in \mathscr{F}$ tale che $\beta(c) = 0$, $c$ è un termine detto \textbf{costante}
      \end{itemize}
    \item \textbf{passo}: siano $f \in \mathscr{F}$ tale che $\beta(f) = n$ e $t_1, t_2, \cdots, t_n \in T_\mathscr{L}$ termini già costruiti, allora $f(t_1, t_2, \cdots, t_n)$ è un $\mathscr{L}$-termine. 
    \item nient'altro è un $\mathscr{L}$-termine.
  \end{itemize}
  I termini sono dei nomi che designano un oggetto nell'universo del discorso.
\end{defi}

\begin{defi}[Formule]
Il ruolo delle \textit{formule di} $\mathscr{L}$ è predicare sopra dei determinati $\mathscr{L}$-termini, affermando se certe relazioni tra di loro sono vere o false.

L'insieme $\mathscr{F}_\mathscr{L}$ delle $\mathscr{L}$-formule è definito induttivamente come segue:
\begin{itemize}
  \item \textbf{base}: siano $p \in \mathscr{P}$ tale che $\alpha(p)=n$ e $t_1, \cdots, t_n \in T_\mathscr{L}$ degli $\mathscr{L}$-termini, allora $P(t_1, \cdots, t_n)$ è una $\mathscr{L}$-formula detta \textbf{formula atomica}. \\
  Le formule atomiche sono il corrispettivo nella Logica del Primo Ordine delle lettere proposizionali.
  \item \textbf{passo}: se $A,B \in \mathscr{F}_\mathscr{L}$, allora sono $\mathscr{L}$-formule anche $(A \land B)$, $(A \lor B)$, $(\neg A)$, $(A\rightarrow B)$
  \item se $A \in \mathscr{F}_\mathscr{L}$ e $x \in \mathscr{V}_\mathscr{L}$, allora sono $\mathscr{L}$-formule anche $(\forall x A)$ e $(\exists x A)$.
  \item nient'altro è una $\mathscr{L}$-formula.
\end{itemize}
\end{defi}
Nota: se in $A$ non occorre $x$, $\forall x A$ e $\exists x A$ sono comunque formule; così come anche $\forall x (\forall x A)$ e $\exists x (\forall x A)$. 

\paragraph{Esercizio} Dare una nozione di \textit{certificato} per 
$\mathscr{L}$-termini e per $\mathscr{L}$-formule analoga alla $\mathscr{L}$-costruzione in logica proposizionale. 

\subsection{Terminologia}
Introduciamo alcune nozioni terminologiche per potersi riferire alla sintassi 
della Logica dei Predicati.
Si userà liberamente la scrittura semplificata di formule omettendo coppie di parentesi
quando questo non causa ambiguità, ossia invece di $(\forall x A)$ si scriverà $\forall x A$.

\begin{defi}[Termine Ground]
Un termine è detto \textbf{chiuso} o \textbf{ground} se è costruito senza utilizzare \textit{variabili}. 
\end{defi}

\begin{defi}[Variabili vincolate e libere]
Una occorrenza di una variabile $x$ in una formula $A \in \mathscr{F}_\mathscr{L}$ è detta \textbf{vincolata} se occorre all'interno di una sottoformula di $A$ (ossia una formula che appare in ogni certificato della $\mathscr{L}$-formula $A$) del tipo $\forall x B$ o $\exists x B$, altrimenti è detta \textbf{libera}.
\end{defi}

Per esempio, nella formula 
$$
A = (\forall x R(x,y)) \lor P(x)
$$
$x$ è sia vincolata che libera, mentre $y$ è libera. 
Nella formula 
$$
A' = \forall x (\forall y R(x,y)) \lor P(x)
$$

Sia $x$ che $y$ occorrono sempre vincolate. 

Questo ci permette di introdurre il concetto fondamentale sul quale lavoreremo: infatti, non ragioneremo più su \textit{formule} quando definiremo la semantica della Logica dei Predicati, ma si ragionerà su un particoalre tipo di formula: 
\begin{defi}[$\mathscr{L}$-enunciato o formula chiusa]
Un $\mathscr{L}$-\textbf{enunciato} ($\mathscr{L}$-sentence) o \textit{formula chiusa} è una $\mathscr{L}$-formula senza variabili libere.
\end{defi}

\begin{defi}[Sostituzione]
Con la notazione $A[t/x]$ con $A \in \mathscr{F}_\mathscr{L}$, $x \in \mathscr{V}_\mathscr{L}$ e $t$ un $\mathscr{L}$-termine, si intende la formula ottenuta rimpiazzando simultaneamente tutte e sole le occorrenze \textbf{libere} di $x$ con $t$. 
\end{defi}
