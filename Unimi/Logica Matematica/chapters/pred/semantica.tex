% TEX root = ../../main.tex
\chapter{Semantica della Logica del Primo Ordine}
\section{Semantica di Tarski}
La semantica esprime \textit{come e quando} un $\mathscr{L}$-enunciato è vero (o falso) in una data circostanza. È necessario fissare e formalizzare la  nozione di \textit{circostanza} o ``mondo possibile''. Una volta fatto ciò, sarà necessario formalizzare l'\textit{interpretazione} degli enunciati (e quindi dei termini) in ogni circostanza formalizzata.

Nella Logica Proposizionale la circostanza è un \textit{assegnamento}. \\
Il logico polacco Alfred Tarski ha proposto di formalizzare la \textit{verità} in una \textbf{circostanza} nella Logica del Primo Ordine come corrispondenza con lo \textit{stato di fatto}. 

Per la \textbf{semantica tarskiana}, nel linguaggio naturale l'enunciato ``La neve è bianca'' è un enunciato vero se e solo se la neve è bianca. 
Pertanto, ci si può immaginare di matematizzare questo concetto 
tramite la teoria degli insiemi, affermando che esista un certo 
insieme chiamato \textit{Universo del discorso}, composto da vari individui,
tra i quali vi è un certo elemento che rappresenta la neve e, 
rispecchiando la nostra esperienza, si vuole che effettivamente sia 
vero che la neve sia bianca. Pertanto, ``si forza'' l'appartenenza 
della neve all'insieme degli oggetti del discorso che sono bianchi. 

In termini matematici, e per il concetto stesso della semantica tarskiana, 
è necessario considerare Universi in cui vi è la possibilità 
che la neve non sia bianca, ossia l'enunciato è falso. 

A priori, le opzioni \textit{esistono} tutte (la neve è bianca, la neve 
non è bianca), ma si può utilizzare la formalizzazione degli enunciati 
anche per modellizzare l'universo, ossia asserendo che il fatto 
che la neve sia bianca sia vera, e questo sappiamo già farlo 
con l'uso delle \textit{teorie}. 

Per esempio, rielaboriamo il solito sillogismo aristotelico: 
L'insieme di enunciati 
$$
\Gamma := \{\{\forall x (U(x) \rightarrow M(x))\}, \{U(s)\}\}
$$
benché non sia in FNC, è un \textit{teorema}, mentre 
$$
A := M(s)
$$
è la deduzione. Per ottenere un'infrastruttura sulla quale arrivare a compiere 
il calcolo deduttivo 
$$
\Gamma, \neg A \text{ è soddisfacibile?}
$$
è necessario ``concentrarsi'' sugli Universi in cui $\Gamma$ è vero. 
Si possono considerare diversi \textbf{Universi del discorso}, composti da vari 
\textit{individui}: un enunciato caratterizza alcuni  universi in cui 
un enunciato è vero a scapito di altri. 

Si consideri, ora,  un universo composto da altri universi: chiamando 
ciò che abbiamo considerato fino ad ora 
un \textit{universo} una $\mathscr{L}$-struttura, si costruisce l'insieme 
di tutte le $\mathscr{L}$-strutture; data una teoria $\Gamma$ si potrà identificare, 
nell'insieme di tutte le $\mathscr{L}$-strutture, tutte quelle che modellano $\Gamma$, ossia tutti quelli in cui le assunzioni $\Gamma$ sono valide. 

Definiamo formalmente le $\mathscr{L}$-strutture:
\begin{defi}[$\mathscr{L}$-struttura]
Sia dato un linguaggio elementare $\mathscr{L}  = (\mathscr{P}, \mathscr{F}, \alpha, \beta)$. \\
Una $\mathscr{L}$-struttura è una coppia $\mathscr{A} = (U, I)$, dove $U$ è un insieme \textit{non vuoto} detto ``universo del discorso'' (o dominio), mentre $I$ è la funzione d'\textit{interpretazione} che interpreta i simboli linguistici dentro all'universo del discorso. \\
Più specificatamente:
\begin{itemize}
  \item per ogni simbolo di predicato $P \in \mathscr{P}$ con arietà $\alpha(P) = n$, $I(P) \subseteq U^{n}$ (e.g, $M(s)$)
  \item per ogni simbolo di funzione $f \in \mathscr{F}$ con arietà $\beta(f) = n$, $I(f) : U^{n} \rightarrow U$ (e.g. $p(s)$)
\end{itemize}
\end{defi}

La definizione di $\mathscr{L}$-struttura data formulizza a livello del Primo Ordine la nozione intuitiva di ``circostanza'' o ``mondo possibile''. A livello proposizionale, 
essa è formalizzata dalla nozione di assegnamento. 


\begin{oss}[Costanti]
        Se $c \in \mathscr{F}$ e $\beta(c) = 0$ 
        $$
        I(c) : U^{0} \rightarrow U
        $$
        e $U^{0} = \{f: \emptyset  \rightarrow U\}$, la cui cardinalità è  
        $$ 
        |U^0| = |\{f: \emptyset \rightarrow U\}| = |\{\emptyset\}| = 1
        $$
        Pertanto 
        $$
        I(c): \{*\} \rightarrow U
        $$ 
(la notazione $\{*\}$
        rappresenta 
        l'insieme che contiene l'insieme vuoto), che equivale a 
        ``scegliere'' un elemento di $U$.
        Identifichiamo una funzione 
        $$
        g : \{*\} \rightarrow U
        $$
        con l'elemento scelto da $g(*)$. In altre parole, le costanti sono funzioni
        zerarie.
\end{oss}

\begin{oss}[Predicati zerari]
Se $p \in \mathscr{P}, \alpha(P) = 0$ è un simbolo di predicato zerario, si ha 
$$
I(P) \subseteq U^0 ~~~~ I(P) \subseteq \{*\}
$$
I cui sottoinsiemi sono sé stesso $\{*\}$ (quindi $1$) e l'insieme vuoto $\emptyset$ (quindi $0$). \\
L'interpretazione di un predicato zerario svolge, quindi, la funzione di lettera proposizionale. Un modo per giustificare il significato dato a $\{*\}$ e $\emptyset$ è interpretando le proposizioni tramite la loro funzione caratteristica $\chi_P: U^n \rightarrow \{0,1\}$, definita come 
$$
\bar{a} \in U^n \in I(P) \iff \chi_P(\bar{a}) = 1
$$
\end{oss}

\begin{oss}[Interpretazione fissa del predicato di Uguaglianza]
Se $=\ \in \mathscr{P}$ con arietà $\alpha(=) = 2$, allora $\mathscr{L}$ è un \textit{linguaggio con identità} (o uguaglianza). È un predicato importante perché l'interpretazione di ``$=$'' è fissata nella $\mathscr{L}$-struttura associata $\mathscr{A}=(U,I)$:
$$
I(=) \subseteq U^2 ~~~ \text{ e } ~~~ I(=):= \{(a,a) : a \in U\}
$$
In altre parole, $x = y$ è vero se $(I(x), I(y)) \in I(=)$. Un elemento è in relazione d'identità solo con sé stesso.
\end{oss}

L'eccezione sull'interpretazione del predicato di uguaglianza, fissato per 
ogni linguaggio elementare che lo contiene, è giustificato dal fatto che con 
la potenza espressiva della Logica dei Predicati, si riesce a dire che una 
relazione binaria gode della proprietà riflessiva, simmetrica e transitiva.  

Ci si può anche spingere a dire che per ogni altro simbolo, sia di predicato 
che di funzione, quella relazione si comporta in maniera \textit{congruenziale}. 
Anche questo non è sufficiente per inchiodare l'identità, in quanto una congruenza 
su un insieme che non è l'identità rispetta gli stessi assiomi. 

In altre parole, o l'identià si codifica a forza come si sta facendo ora nella Logica del Primo Ordine, oppure se ne ha una versione molto indebolita, in quanto non si ha modo di distinguere con formule della Logica dei Predicati l'identità da una congruenza sullo stesso insieme.  

\subsection{Semantica degli enunciati in ogni $\mathscr{L}$-struttura}
Abbiamo dato la nozione di $\mathscr{L}$-struttura ma non si è ancora detto 
come si interpretano gli enunciati, formalmente, cioè quando un 
enunciato è vero o meno in tale $\mathscr{L}$-struttura. Intuitivamente, quello che 
si vuole fare è quanto anticipato, ossia scegliere le $\mathscr{L}$-strutture che 
verificano una certa teoria. Tuttavia, questo deve essere eseguito 
seguendo una definizione rigorosa e sistematicamente.

Si procederà per strati, ossia definendo prima come interpretare 
i termini e successivamente formule ed enunciati, induttivamente. 
Si deve catturare l'idea intuitiva: ``la neve è bianca'' verrà 
formalizzato in qualcosa come $B(n)$, dove $B$ è un predicato 
unario e $n \in A$ è una costante (quindi esiste una funzione zeraria 
il cui risultato è $n$). Si vorrà formalizzare il tutto in modo tale da avere una 
qualche $\mathscr{L}$-struttura tale che $I(n)$ sia contenuta in $I(B)$; quindi si avrà che 
$B(n)$ è vera in $\mathscr{A}$ se e solo se $I(n) \in I(B)$: 
$$
\mathscr{A} \in B(n) \iff I(n) \in I(B)
$$

Il passo più ostico nella definizione di interpretazione di termini 
e formule (ed enunciati) sarà nell'interpretazione dei quantificatori; 
per esempio, si supponga che si vuole stabilire se 
$$
\mathscr{A} \models \exists x A
$$
che significa: se esiste un $a \in U$ tale che $\mathscr{A} \models A[a/x]$
e  analogamente per l'altro quantificatore 
$$
\mathscr{A} \models \forall x A
$$
che è vero se e solo se per ogni $a \in U$ vale $\mathscr{A} \models A[a/x]$. 

Purtroppo $a$ non è un elemento sintattico (i.e., un termine), ma è un elemento \textit{semantico}, e pertanto $A[a/x]$ non è definita.
Non c'è nessuna ragione per cui ogni elemento di una $\mathscr{L}$-struttura debba avere, a priori, un nome: piuttosto, è vero il contrario. Per esempio, se si vuole parlare dei reali $\mathbb{R}$, si vorrebbe farlo con un linguaggio elementare $\mathscr{L}$ che contiene solo finitamente molti simboli (e.g., $0-9$ se si vuole lavorare in base 10). \\
Noi ovvieremo a questo problema dando un nome agli elementi semantici dell'universo derivato dalla semantica stessa.

\begin{defi}[Espansione di un $\mathscr{L}$-struttura]
Dati un linguaggio $\mathscr{L} = (\mathscr{P}, \mathscr{F}, \alpha, \beta)$ e una $\mathscr{L}$-struttura $\mathscr{A}(U,I)$, definiamo \textbf{espansione} di $\mathscr{L}$ in un nuovo linguaggio $\mathscr{L}_{\mathscr{A}}$ come segue:
\begin{itemize}
  \item per ogni $a \in U$ l'insieme delle costanti di $\mathscr{L}$ viene arricchito con un nuovo simbolo $\bar{a}$ (i.e., nome di $a$)
  \item $\bar a$ viene interpretato in $a$
\end{itemize} 
Formalmente:
\begin{align*}
  \mathscr{L}_{\mathscr{A}} := (\mathscr{P}, \mathscr{F}^{\mathscr{A}}, \alpha, \beta^{\mathscr{A}}) &&
  I(\bar{a}) := a \text{ per ogni } a \in U
\end{align*}
con:
\begin{align*}
  \mathscr{F}^{\mathscr{A}} := \mathscr{F} \cup \{\bar{a}: a \in U\} &&
  \begin{cases}
    \beta^{\mathscr{A}}(f) = \beta(f) & \text{per ogni } f \in \mathscr{F} \\
    \beta^{\mathscr{A}}(\bar{a}) = 0 & \text{per ogni } a \in U \\
  \end{cases}
\end{align*}
\end{defi}

\begin{defi}[Interpretazione degli $\mathscr{L}_{\mathscr{A}}$-Termini ground]
Dati $\mathscr{L} = (\mathscr{P}, \mathscr{F}, \alpha, \beta)$ e $\mathscr{A} = (U,I)$ e la $\mathscr{L}_{\mathscr{A}}$-struttura espansa, si definisce induttivamente: 
\begin{itemize}
  \item \textbf{base} ($c \in \mathscr{F}$, con $\beta(c) = 0$): se $c$ è una ``vecchia'' costante già in $\mathscr{L}$, ha interpretazione $I(c)$ già definita da $\mathscr{A}$ per definizione di $\mathscr{L}$-struttura
  \item \textbf{base} ($\bar{a} \in \mathscr{F}^\mathscr{A}$, con $\beta(a) = 0$): se $\bar{a}$ è un nome ``nuovo'' dell'elemento $a \in U$, ha interpretazione $I(\bar{a}) = a$. 
  \item \textbf{passo}: sia $f \in \mathscr{F}$ un simbolo di funzione con arietà $\beta(f) = n$ e siano $t_1, t_2, \cdots, t_n$ $\mathscr{L}_{\mathscr{A}}$-Termini ground che hanno già un'interpretazione per ipotesi induttiva. \\
  Allora $I(f(t_1, \cdots, t_n)):= I(f)(I(t_1), \cdots, I(t_n))$. 
\end{itemize}
\end{defi}
\noindent
Si noti che $I(f)(I(t_1), \cdots, I(t_n))$ è valida (i.e. $\in U$) perché $I(f): U^n \rightarrow U$ per definizione.

\begin{defi}[Interpretazione degli Enunciati]
Sia data $\mathscr{A} = (U,I)$ una $\mathscr{L}$-struttura; allora, si definisce induttivamente la nozione $\mathscr{A} \models A$ (i.e. $A$ è vera in $\mathscr{A}$) per ogni $\mathscr{L}_{\mathscr{A}}$\textit{-enunciato} $A$:
\begin{itemize}
  \item \textbf{base}: sia $A = P(t_1, \cdots, t_n)$ per $P \in \mathscr{P}$ con $\alpha(P) = n$, e $t_1, \cdots, t_n$ $\mathscr{L}_{\mathscr{A}}$-Termini ground. Allora
  $$
  \mathscr{A} \models P(t_1, \cdots, t_n) \iff \underbrace{\overbrace{I(t_1)}^{\in U}, \cdots, \overbrace{I(t_n)}^{\in U}}_{\in U^n} \in \underbrace{I(P)}_{\subseteq U^n}
  $$
  \item \textbf{passo}:
    \begin{itemize}
      \item $\mathscr{A} \models A_1 \land A_2 \iff \mathscr{A} \models A_1$ e $\mathscr{A} \models A_2$
      \item $\mathscr{A} \models A_1 \lor A_2 \iff \mathscr{A} \models A_1$ o $\mathscr{A} \models A_2$
      \item $\mathscr{A} \models \neg A_1  \iff \mathscr{A} \nvDash A_1 $
      \item $\mathscr{A} \models A_1 \rightarrow A_2 \iff \mathscr{A} \nvDash A_1$ o $\mathscr{A} \models A_2$
      \item $\mathscr{A} \models\forall x A \iff \mathscr{A} \models A[\bar{a}/x] \text{ per ogni } a \in U$
      \item $\mathscr{A} \models\exists x A \iff \mathscr{A} \models A[\bar{a}/x] \text{ per almeno un } a \in U$
    \end{itemize}
\end{itemize}
\noindent
Si noti che $\bar{a}$ è un termine, mentre $x$ è una variabile individuale, quindi $A$ diventa un $\mathscr{L}_{\mathscr{A}}$-enunciato a tutti gli effetti.
\end{defi}

Abbiamo quindi definito induttivamente $\mathscr{A}\models A$ per gli 
$\mathscr{L}_{\mathscr{A}}$-enunciati $A$. A questo punto ci si può ``dimenticare'' degli $\mathscr{L}_{\mathscr{A}}$-enunciati che non sono $\mathscr{L}$-enunciati (i.e., le formule in cui vi è almeno una variabile libera).

\begin{defi}[Chiusura universale di una formula con variabili libere]
Se $A$ è una formula ben formata, ossia $A \in \mathscr{F}_\mathscr{L}$ ed eventualmente,
non un enunciato, quindi $FV(A) \neq \emptyset$, dove $FV$ è definito come l'insieme 
di variabili libere che appaiono in $A$, ci si può chiedere se 
$$ 
\mathscr{A} \models A
$$
Definito, quindi, 
$$
FV(A) = \{x_1, \cdots, x_n\} 
$$
si definisce la chiusura universale di $A$ 
$$
\forall[A] := \forall x_1 \forall x_2 \cdots \forall x_n A
$$
e si postula che
$$
\mathscr{A} \models A \iff \mathscr{A} \models \forall[A]
$$
\end{defi}
ossia $\mathscr{A}$ modella una formula ben formata $A$ se e solo se 
$\mathscr{A}$ rende vera la sua chiusura universale, che è un $\mathscr{L}$-enunciato.

\begin{oss}
Se $\mathscr{L}$ è un linguaggio con identità (o con uguaglianza), ossia $\mathscr{L} = (\mathscr{P}, \mathscr{F}, \alpha, \beta)$ e $=\ \in \mathscr{P}$, $\alpha(=) = 2$, allora, ricordando che in ogni $\mathscr{L}$-struttura $\mathscr{A} = (U,I)$, si deve avere per definizione $I(=) = \{(a,a) : a \in U\}$. \\
Pertanto $\mathscr{A} \models t_1 = t_2$ se e solo se $I(t_1) = I(t_2)$, con $t_1$ e $t_2$ $\mathscr{L}$-Termini ground (che, più correttamente, andrebbe scritto in forma postfissa ``$=(t_1, t_2)$'').
\end{oss}

\subsection{Definizione alternativa di $\mathscr{A} \models A$} 
In questa definizione alternativa,  le variabili sono oggetti di interesse; tuttavia le due definizioni—quella basata sull'ampliamento $\mathscr{L} \rightarrow \mathscr{L}_{\mathscr{A}}$ e quella che andiamo a dare—sono equivalenti. 
\begin{defi}[Ambiente]
Un \textbf{ambiente di interpretazione} in una $\mathscr{L}$-struttura $\mathscr{A} = (U,I)$ è una funzione
$$
\mathcal{v} : Var \rightarrow U
$$
ossia ad ogni variabile viene assegnato un elemento di $U$: $\mathcal{v}(x) \in U$.
\end{defi}
\begin{defi}[Variante di ambiente]
La \textbf{variante di ambiente} servirà a dare la semantica alle espressioni quantificate ed è indicato $\mathcal{v}[a/x](y)$ e si definisce 
$$
\mathcal{v}[a/x](y) :=
\begin{cases}
  \mathcal{v}(y) & \text{se } y \neq x \\
  a & \text{se } y = x
\end{cases}
$$
\end{defi}
\begin{defi}[Interpretazione degli $\mathscr{L}$-Termini]
L'interpretazione degli $\mathscr{L}$-Termini (anche aperti!)
\begin{itemize}
  \item \textit{variabili}: per ogni $x \in Var$ e $\mathcal{v} : Var \rightarrow U$,
  $$
  I_{\mathcal{v}}(x) := \mathcal{v}(x)
  $$
  \item \textit{costanti}: per ogni $c \in \mathscr{F}$ con $\beta(c) = 0$,
  $$
  I_{\mathcal{v}}(c) := I(c)
  $$
  \item \textit{termini composti}: per ogni $f \in \mathscr{F}$ con $\beta(f) > 0$
  $$
  I_{\mathcal{v}}(f(t_1, \cdots, t_n)) := (I(f))(I_{\mathcal{v}}(t_1), \cdots, I_{\mathcal{v}}(t_n))
  $$
\end{itemize}
\end{defi}
\begin{defi}[Interpretazione delle formule]
Si indica con $\mathscr{A}\models_{\mathcal{v}} A$ se $\mathscr{A}$ rende vera $A$ nell'ambiente $\mathcal{v}: Var \rightarrow U$, e si definisce
\begin{itemize}
  \item \textit{atomica}:
  $$
  \mathscr{A} \models_{\mathcal{v}} P(t_1, \cdots, t_2) \iff (I_{\mathcal{v}}(t_1), \cdots, I_{\mathcal{v}}(t_n)) \in I(P)
  $$
  \item \textit{proposizioni}:
    \begin{itemize}
      \item  $\mathscr{A} \models_{\mathcal{v}} A_1 \land A_2 \iff \mathscr{A} \models A_1$ e $\mathscr{A} \models A_2$
      \item $\mathscr{A} \models_{\mathcal{v}}A_1 \lor A_2 \iff \mathscr{A} \models A_1$ o $\mathscr{A} \models A_2$
      \item $\mathscr{A} \models_{\mathcal{v}}  \neg A_1  \iff \mathscr{A} \nvDash A_1 $
      \item $\mathscr{A} \models_{\mathcal{v}} A_1 \rightarrow A_2 \iff \mathscr{A} \nvDash A_1$ o $\mathscr{A} \models A_2$
      \item $\mathscr{A} \models_{\mathcal{v}} \forall x A \iff \mathscr{A} \models_{\mathcal{v}} A[a/x] \text{ per ogni } a \in U$
      \item $\mathscr{A} \models_{\mathcal{v}} \exists x A \iff \mathscr{A} \models_{\mathcal{v}} A[a/x] \text{ per almeno una  } a \in U$
    \end{itemize}
\end{itemize}
Concludiamo dicendo che $\mathscr{A} \models A$ sse \textit{per ogni}  $\mathcal{v}: Var \rightarrow A$ si ha $\mathscr{A} \models_\mathcal{v} A$.
\end{defi}

\subsection{Intuizione: differenze delle due semantiche}
Si esprime, ora, un'idea intuitiva di come avvenga il processo di interpretazione 
di un $\mathscr{L}$-enunciato. Nella prima definizione si ha il linguaggio elementare $\mathscr{L}$ e delle 
$\mathscr{L}$-strutture che vengono associate. 
$$
\mathscr{L} \leadsto
\begin{cases}
        \mathscr{A}  \leadsto \mathscr{L}_{\mathscr{A}} \\
        \mathscr{B}  \leadsto \mathscr{L}_{\mathscr{B}} \\
        \mathscr{C}  \leadsto \mathscr{L}_{\mathscr{C}} \\
        \vdots 
\end{cases}
$$
Per ogni $\mathscr{L}$-struttura si crea l'espansione associata. 

Nella seconda definizione si ha nuovamente il linguaggio $\mathscr{L}$ e delle $\mathscr{L}$-strutture, 
tuttavia esse non si espandono ma vi sono un numero di ambienti da considerare: 

$$
\mathscr{L} \leadsto 
\begin{cases}
        \mathscr{A} \begin{cases}
                \mathscr{A}, \mathcal{v}_1 \\
                \mathscr{A}, \mathcal{v}_2 \\
                \mathscr{A}, \mathcal{v}_3 \\
                \cdots \\
        \end{cases} \\
        \mathscr{B} \begin{cases}
                \mathscr{B}, \mathcal{v}_1\\
                \mathscr{B}, \mathcal{v}_2\\
                \mathscr{B}, \mathcal{v}_3\\
                \cdots                     
        \end{cases} \\
        \vdots
\end{cases}
$$

\section{Terminologia e Nozioni}
\begin{defi}[$\mathscr{L}$-teoria]
Una $\mathscr{L}$-teoria è un insieme di $\mathscr{L}$-enunciati.
\end{defi}

Se $\Gamma$ è una $\mathscr{L}$-teoria e $\mathscr{A}$ è una $\mathscr{L}$-struttura, si dice che $\mathscr{A}$ è modello di $\Gamma$ ($\mathscr{A} \models \Gamma$) sse $\mathscr{A} \models \gamma$ per ogni $\gamma \in \Gamma$. 

La $\mathscr{L}$-teoria $\Gamma$ è soddisfacibile sse esiste almeno un $\mathscr{A}$ tale che $\mathscr{A} \models \Gamma$, altrimenti $\Gamma$ è insoddisfacibile.

Sia $\Gamma$ una $\mathscr{L}$-teoria e $A$ un $\mathscr{L}$-enunciato. Allora $A$ è conseguenza logica di $\Gamma$ e si scrive $\Gamma \models A$ sse $A$ è vera in ogni modello di $\Gamma$, ossia per ogni $\mathscr{L}$-struttura $\mathscr{A}$, $\mathscr{A} \models \Gamma \implies  \mathscr{A} \models A$.

\begin{defi}[Verità Logica]
Un $\mathscr{L}$-enunciato $A$ è detto \textbf{verità logica} sse per ogni $\mathscr{L}$-struttura $\mathscr{A}$, $\mathscr{A} \models A$. 
\end{defi}
Il concetto di Verità Logica è analogo al concetto di tautologia nella Logica Proposizionale.
Un esempio di verità logica è $\forall x (x = x)$. \\
Lo scopo della nostra indagine sarà, d'ora in poi, cercare di stabilire se
$$
\Gamma \stackrel ? \models A
$$
dati $\Gamma$  una $\mathscr{L}$-teoria e $A$ un $\mathscr{L}$-enunciato. Se $\Gamma$ è finita, ossia $\Gamma = \{\gamma_1, \cdots, \gamma_n\}$, varrà nuovamente il fatto 
$$
\Gamma \models A \iff \gamma_1 \land \cdots \land \gamma_n \land  \{\neg A\} \text{ insodd.}
$$
ossia ci si riduce all'analisi di una singola formula, 
mentre invece se $\Gamma$ è infinito, possiamo solo chiederci se
$$
\Gamma \models A \iff \Gamma \cup \{\neg A\} \text{ insodd.}
$$
Anche nel caso in cui $\Gamma$ è finito, ci saranno casi in cui il processo di deduzione sarà solo \textit{semidecidibile}, quando nella Logica Proposizionale il problema di soddisfacibilità almeno era decidibile. 
\subsection{Esempi di $\mathscr{L}$-teorie} 

\subsubsection{Congruenze su un $\mathscr{L}$-Linguaggio $\mathscr{L}$}
Data una relazione $R$, si vorrebbe modellarla come congruenza.
$$
\Gamma := 
\begin{cases}
  \forall x ~~ R(x,x) & \text{ riflessiva } \\
  \forall x \forall y ~~ R(x,y) \rightarrow R(y,x) & \text{ simmetrica } \\
  \forall x \forall y \forall z ~~ R(x,y) \land R(y,z) \rightarrow R(x,z) & \text{ transitiva } \\

  \forall x_1 \cdots \forall x_n \forall y_1 \cdots \forall y_n \\ ~~ (R(x_1,y_1) \land \cdots \land R(x_n, y_n)) \rightarrow R(f(x_1, \cdots, x_n), f(y_1, \cdots y_n)) \\
  \forall x_1 \cdots \forall x_n \forall y_1 \cdots \forall y_n \\ ~~ (R(x_1, y_1) \land \cdots \land R(x_n, y_n)) \rightarrow (P(x_1, \cdots x_n) \rightarrow P(y_1, \cdots, y_n))
\end{cases}
$$
I primi tre $\mathscr{L}$-enunciati—o \textit{assiomi} di $\Gamma$—assiomatizzano una relazione d'equivalenza, ossia
$$
\text{ se }\mathscr{A} \models \Gamma \text{ allora } I(R) \text{ è una relazione d'equivalenza}
$$
Il quarto $\mathscr{L}$-enunciato assiomatizza la congruenza $R$ rispetto ad
una generica $f \in \mathscr{F}$ con $\beta(f) = n$, mentre il quinto fa 
la stessa cosa rispetto un generico predicato $P \in \mathscr{P}$. \\
Si noti che la relazione di uguaglianza ($=$) è una relazione di congruenza, ma non è per forza vero il contrario! Inoltre, per un limite espressivo della Logica del Primo ordine, non si riesce a distinguere queste due relazioni. Per questo noi abbiamo fissato una specifica semantica all'$=$. 

\subsubsection{Relazione d'ordine (parziale)}
$$
\Gamma_{PO} := 
\begin{cases}
  \forall x ~~ R(x,x) & \text{ riflessiva } \\
  \forall x,y ~~ (R(x,y) \land R(y,x)) \rightarrow =(x,y) & \text{ antisimmetrica } \\
  \forall x \forall y \forall z ~~ (R(x,y) \land R(y,z)) \rightarrow R(x,z) & \text{ transitiva } \\    
\end{cases}
$$
Si noti l'utilizzo del predicato d'uguaglianza ( $=$), pertanto il Linguaggio è dotato di tale predicato. Un esempio è l'insieme dei naturali con la relazione di minore-o-uguale che modella $\Gamma$: 
$$
(\mathbb{N}, \leq) \models \Gamma_{PO}
$$
(la notazione utilizzata significa che $I(R) =\ \leq$) così come i naturali con l'ordine di divisibilità: 
$$
(\mathbb{N}, |) \models \Gamma_{PO}
$$


\paragraph{Relazione d'ordine (totale)}
Si può costruire una $\mathscr{L}$-teoria $\Gamma_{TO}$ che, ``allargando'' $\Gamma_{PO}$ con un solo $\mathscr{L}$-enunciato, modella solo uno dei due esempi:
\begin{align*}
  \Gamma_{TO} := \Gamma_{PO} \cup \{\forall x \forall y ~~ (R(x,y) \lor R(y,x))\}
\end{align*}
Effettivamente, $(\mathbb{N}, \leq) \models \Gamma_{PO}$, mentre $(\mathbb{N}, |) \nvDash \Gamma_{PO}$ perché vi sono elementi $m$, $n$ tali che né $m | n$ né $n |m $.

Aggiungendo nuovi assiomi si possono caratterizzare altri tipi di ordine: ad esempio, si può caratterizzare un ordine che abbia un elemento minimo:
\begin{align*}
  \Gamma_{min} := \Gamma_{PO} \cup \{\exists m \forall x ~~ (x \neq m \land \neg R(x,m))\}
\end{align*}

\subsubsection{Teoria dei gruppi (1)}
Un gruppo è una struttura algebrica. 
Un gruppo è un modello dell'insieme di assiomi $\Gamma_G$ sul linguaggio $\mathscr{L} = (\mathscr{P}, \mathscr{F}, \alpha, \beta)$. 
$$
\Gamma_{G_1} := 
  \begin{cases}
    \forall x \forall y \forall z  ~~ ((x * y)*z) = (x*(y*z))  & \text{ associatività}\\
    \forall x ~~ (x * e) = x \land (e * x) = x  & \text{ elemento neutro}\\
    \forall x ~~ (x * x^{-1}) = e \land (x^{-1} * x = e) & \text{ invertibilità}
  \end{cases}
$$
con $\mathscr{P} = \{ = \}$, $\mathscr{F} = \{*, e, \_^{-1}\}$ e relative arietà $\alpha$ e $\beta$. \\
$(G, \cdot, ^{-1}, 0) \models \Gamma_{G_1}$ se e solo se $G$ è un gruppo; in altri termini un gruppo è una struttura con un'operazione associativa in cui ogni elemento è invertibile. \\
Per esempio, $(\mathbb{Z}, +, -_{_1}, 0) \models \Gamma_{G_1}$ e $(\mathbb{Q}\setminus{\{0\}}, \cdot, ^{-1}, 1) \models \Gamma_{G_1}$. 

\paragraph{Teoria dei gruppi (2)}
Una seconda formulazione è la seguente
$$
\Gamma_{G_2} := 
  \begin{cases}
    \forall x \forall y \forall z  ~~ ((x * y)*z) = (x*(y*z)) & \text{ associatività}\\
    \forall x ~~ (x * e) = x \land (e * x) = x & \text{ elemento neutro } \\
    \forall x \exists y ~~ (x * y) = e \land (y * x) = e & \text{ esistenza inverso} \\
  \end{cases}
$$
Questa $\mathscr{L}$-teoria rimuove la specifica dell'inverso. Se si ottiene un gruppo che modella l'altra formulazione, esso andrà bene anche per questo, ma non è detto il viceversa. 
Ora, infatti, non è più necessario specificare l'operazione inverso, 
e si può quindi affermare $(\mathbb{Z}, +, 0) \models \Gamma_{G_2}$, $(Q, +, 0) \models \Gamma_{G_2}$, 
eccetera. Come anticipato, questo vale anche per i gruppi esplicitati prima: 
$(\mathbb{Q}\setminus{0}, \cdot, ^{-1}, 1) \models \Gamma_{G_2}$ e così via, 
anche se l'inverso non viene utilizzato come simbolo esplicito. 
Potremmo quindi utilizzare, al posto del simbolo dell'inverso, qualsiasi cosa e 
rimarrà comunque un modello di $\Gamma_{G_2}$: 
$$
(\mathbb{Z}, +, +1, 0) \models \Gamma_{G_2}
$$
Ma ovviamente non è vero il contrario, ossia 
$$
(\mathbb{Z}, +, +1, 0) \nvDash \Gamma_{G_1}
$$
in questo senso, abbastanza sottile, le due formulazioni non sono equivalenti. 

Possiamo quindi dire che la Teoria dei Gruppi (1) è la Teoria dei Gruppi (2) \textit{skolemizzata} perché priva dell'esistenziale).

\subsubsection{Tipi di dato: Stack}
Definiamo il linguaggio $\mathscr{L}= (\mathscr{P}, \mathscr{F}, \alpha, \beta)$ 
con 
\begin{align*}
\mathscr{P} = \{Stack, Elem, =\} &&
\mathscr{F} = \{push, pop, top, nil\}
\end{align*}
$$
\Gamma_{Stack} := 
  \begin{cases}
    \forall x ~~ (Stack(x) \lor Elem(x)) \\
    \forall x ~~ (\neg Stack(x) \lor \neg Elem(x)) \\
    \forall x \forall y ~~ ((Stack(x) \land Elem(y)) \rightarrow top(push(x,y)) = y) \\
    \forall x \forall y ~~ ((Stack(x) \land Elem(y)) \rightarrow pop(push(x,y)) = x) \\
    \forall x ~~ (Stack(x) \rightarrow push(pop(x),top(x)) =x) \\
  \end{cases}
$$

\subsubsection{Aritmetica di Peano}
Sia $\mathscr{L}_{PA} = ( \mathscr{P}, \mathscr{F}, \alpha, \beta)$, con 
$\mathscr{P} = \{=\}$, $\alpha(=) = 2$, $\mathscr{F} =  \{ + , \times, s, 0\}$ 
$\beta(+) = \beta(\times) = 2$, $\beta(s) = 1$, $\beta(0) =0$. Inoltre sia $P$ una formula con una sola variabile libera ($FV(P) = \{v\}$). \\
L'aritmetica di Peano afferma:
$$
\Gamma_{PA} := 
  \begin{cases}
    \forall x ~~ \neg (0 = s(x)) \\
    \forall x, \forall y ~~ (s(x) = s(y)) \rightarrow (x = y) \\
    \forall x ~~ (x + 0 = x) \\
    \forall x \forall y ~~ (x + s(y)) = s(x+y) \\
    \forall x ~~ (x \times 0 = 0) \\
    \forall x \forall y ~~ (x \times s(y)) = (x + (x \times y)) \\
    (P[0/v] \land \forall x ~~ (P[x/v] \rightarrow P[s(x)/v])) \rightarrow \forall x ~~ P[x/v]
\end{cases}
$$
L'ultimo postulato è una descrizione di infiniti molti assiomi (uno per ogni $P$) che codificano il principio di induzione—principio alla base dei numeri naturali. 
