% TEX root = ../../main.tex
\chapter{Semantica della Logica del Primo Ordine}
\section{Semantica di Tarski}
La semantica esprime \textit{come e quando} un $L$-enunciato è vero (o falso) in una data 
circostanza. \`E necessario fissare e formalizzare la  nozione di \textit{circostanza}
o ``mondo possibile''. Nel livello Proposizionale, la circostanza è un 
\textit{assegnamento}. Una volta fatto ciò, sarà necessario formalizzare 
l'\textit{interpretazione} degli enunciati (e quindi dei termini) in ogni circostanza 
formalizzata. 

Un'idea per la formalizzazione del concetto di 
\textbf{circostanza}, dovuta al logico polacco Alfred Tarski, 
è che la \textit{verità} è una corrispondenza con lo stato di \textit{Fatto}. 

Per la \textbf{semantica tarskiana}, nel linguaggio naturale l'enunciato
``La neve è bianca'' è un enunciato vero se e solo se la neve è bianca. 
Pertanto, ci si può immaginare di matematizzare questo concetto 
tramite la teoria degli insiemi, affermando che esista un certo 
insieme chiamato \textit{Universo del discorso}, composto da vari individui,
tra i quali vi è un certo elemento che rappresenta la neve e, 
rispecchiando la nostra esperienza, si vuole che effettivamente sia 
vero che la neve sia bianca. Pertanto, ``si forza'' l'appartenenza 
della neve all'insieme degli oggetti del discorso che sono bianchi. 

In termini matematici, e per il concetto stesso della semantica tarskiana, 
è necessario considerare Universi in cui vi è la possibilità 
che la neve non sia bianca, ossia l'enunciato è falso. 

A priori, le opzioni \textit{esistono} tutte (la neve è bianca, la neve 
non è bianca), ma si può utilizzare la formalizzazione degli enunciati 
anche per modellizzare l'universo, ossia asserendo che il fatto 
che la neve sia bianca sia vera, e questo sappiamo già farlo 
con l'uso delle \textit{teorie}. 

Per esempio, rielaboriamo il solito sillogismo aristotelico: 
L'insieme di enunciati 
$$
\Gamma := \{\{\forall x (U(x) \rightarrow M(x))\}, \{U(s)\}\}
$$
benché non sia in FNC, è un \textit{teorema}, mentre 
$$
A := M(s)
$$
è la deduzione. Per ottenere un'infrastruttura sulla quale arrivare a compiere 
il calcolo deduttivo 
$$
\Gamma, \neg A \text{ è soddisfacibile?}
$$
è necessario ``concentrarsi'' sugli Universi in cui $\Gamma$ è vero. 
Si possono considerare diversi \textbf{Universi del discorso}, composti da vari 
\textit{individui}: un enunciato caratterizza alcuni  universi in cui 
un enunciato è vero a scapito di altri. 

Si consideri, ora,  un universo composto da altri universi: chiamando 
ciò che abbiamo considerato fino ad ora 
un \textit{universo} una $L$-Struttura, si costruisce l'insieme 
di tutte le $L$-Strutture; data una teoria $\Gamma$ si potrà identificare, 
nell'insieme di tutte le $L$-Strutture, tutte quelle che modellano $\Gamma$. 

Definiamo formalmente le $L$-Strutture:
\begin{defi}[$L$-Struttura]
        Sia dato un linguaggio elementare $L  = (\mathcal{P}, F, \alpha, \beta)$. 
        Una $L$-struttura è una coppia $\mathcal{A} = (A, I)$, dove 
        $A$ è un insieme detto ``universo del discorso'' o, semplicemente, 
        \textit{universo} (o dominio), che deve rispettare 
        $$
        A \neq \emptyset
        $$
        mentre $I$ è una funzione detta \textit{interpretazione}, definita 
        tale che per ogni simbolo di 
        predicato $P \in \mathcal{P}$ con arità $\alpha(P) = n$ si ha 
        $$
        I(P) \subseteq A^{n} 
        $$
        ossia $I(P)$ è un insieme di $n$-ple di elementi di $A$; inoltre, 
        per ogni $f \in F$, $\beta(f) = n$ si ha 
        $$
        I(f) : A^{n} \rightarrow A
        $$
        ossia $I(F)$ è una funzione dall'insieme delle $n$-ple di $A$ verso elementi 
        di $A$.
\end{defi}

La definizione di $L$-struttura data formulizza a livello del Primo ordine la 
nozione intuitiva di ``circostanza'' o ``mondo possibile''. A livello proposizionale, 
essa è formalizzata dalla nozione di assegnamento. 


\begin{oss}[Costanti]
        Se $c \in F$ e $\beta(c) = 0$ 
        $$
        I(c) : A^{0} \rightarrow A
        $$
        e $A^{0} = \{f: \emptyset  \rightarrow A\}$, la cui cardinalità è  
        $$ 
        |A^0| = |\{f:\emptyset\rightarrow A\}| = |\{\emptyset\}| = 1
        $$
        Pertanto 
        $$
        I(c): \{*\} \rightarrow A
        $$ 
(la notazione $\{*\}$
        rappresenta 
        l'insieme che contiene l'insieme vuoto), che equivale a 
        ``scegliere'' un elemento di $A$.
        Identifichiamo una funzione 
        $$
        g : \{*\} \rightarrow A 
        $$
        con l'elemento scelto da $g(*)$. In altre parole, le costanti sono funzioni
        zerarie.
\end{oss}

\begin{oss}[Predicati zerari]
        Se $p \in \mathcal{P}, \alpha(P) = 0$ è un simbolo di 
        predicato zerario, si ha 
        $$
        I(P) \subseteq A^0 ~~~~ I(P) \subseteq \{*\}
        $$
        I cui sottoinsiemi sono sé stesso $\{*\}$ e l'insieme vuoto $\emptyset$. 
        E quindi l'interpretazione di un predicato zerario svolge le funzioni 
        di una vecchia lettera proposizionale, in quanto l'interpretazione 
        di una proposizione può essere identificata tramite la sua funzione 
        caratteristica 
        $$
        \chi_P: A^n \rightarrow \{0,1\}
        $$
        definita come 
        $$
        \bar{a} \in A^n \in I(P) \iff \chi_P(\bar{a}) = 1
        $$
\end{oss}

\begin{oss}[Interpretazione fissa del predicato di Uguaglianza]
        Se $ = \in \mathcal{P}$ allora $L$ è un linguaggio con identità e 
        $\alpha(=) = 2$. L'interpretazione di ``='' è fissata: 
        se $\mathcal{A}=(A,I)$ è una $L$-struttura, allora 
        $$
        I(=) \subseteq A^2 
        $$
        è fissato e definito come 
        $$
        I(=):= \{(a,a) : a \in A\}
        $$
        e, in altre parole, $x = y$ è vero se $(I(x), I(y)) \in I(=)$.
\end{oss}

L'eccezione sull'interpretazione del predicato di uguaglianza, fissato per 
ogni linguaggio elementare che lo contiene, è giustificato dal fatto che con 
la potenza espressiva della Logica dei Predicati, si riesce a dire che una 
relazione binaria gode della proprietà riflessiva, simmetrica e transitiva.  

Ci si può anche spingere a dire che per ogni altro simbolo, sia di predicato 
che di funzione, quella relazione si comporta in maniera \textit{congruenziale}. 
Anche questo non è sufficiente per inchiodare l'identità, in quanto una congruenza 
su un insieme che non è l'identità rispetta gli stessi assiomi. 

In altre parole, o l'identià si codifica a forza, come accade ora nella Logica 
del Prim'Ordine, oppure se ne ha una versione molto indebolita, in quanto non si 
ha modo di distinguere con formule della Logica dei Predicati l'identità 
da una congruenza sullo stesso insieme.  

\subsection{Semantica degli enunciati in ogni $L$-struttura}
Abbiamo dato la nozione di $L$-Struttura ma non si è ancora detto 
come si interpretano gli enunciati, formalmente, cioè quando un 
enunciato è vero o meno in tale $L$-Struttura. Intuitivamente, quello che 
si vuole fare è quanto anticipato, ossia scegliere le $L$-Strutture che 
verificano una certa teoria. Tuttavia, questo deve essere eseguito 
seguendo una definizione rigorosa e sistematicamente.

Si procederà per strati, ossia definendo prima come interpretare 
i termini e successivamente formule ed enunciati, induttivamente. 
Si deve catturare l'idea intuitiva: ``la neve è bianca'' verrà 
formalizzato in qualcosa come $B(n)$, dove $B$ è un predicato 
unario e $n \in A$ è una costante (quindi esiste una funzione zeraria 
il cui risultato è $n$). Si vorrà formalizzare il tutto in modo tale da avere una 
qualche $L$ struttura tale che $I(n)$ sia contenuta in $I(B)$; quindi si avrà che 
$B(n)$ è vera in $\mathcal{A}$ se e solo se $I(n) \in I(B)$: 
$$
\mathcal{A} \in B(n) \iff I(n) \in I(B)
$$

Il passo più ostico nella definizione di interpretazione di termini 
e formule (ed enunciati) sarà nell'interpretazione dei quantificatori; 
per esempio, si supponga che si vuole stabilire se 
$$
\mathcal{A} \models \exists x A
$$
che significa: se esiste un $a \in A$ tale che $\mathcal{A} \models A[a/x]$
e  analogamente per l'altro quantificatore 
$$
\mathcal{A} \models \forall x A
$$
che è vero se e solo se per ogni $a \in A$ vale $\mathcal{A} \models A[a/x]$. 

Purtroppo $a$ non è un elemento sintattico, ma è un elemento semantico, 
ossia non è un termine e pertanto $A[a/x]$ non è definita.
Non c'è nessuna ragione per cui ogni elemento di una $L$-struttura 
debba avere, a priori, un nome: piuttosto, è vero il contrario. Se si vuole 
parlare ad esempio dei reali, si vorrebbe farlo con un linguaggio 
elementare $L$ che contiene solo finitamente molti simboli. 

\begin{defi}[Espansione di un $L$-Struttura]
        Dato un linguaggio $L$ e una $L$-struttura $\mathcal{A}(A,I)$, si definisce 
        una \textbf{espansione} di $L$ in un nuovo linguaggio $L_{\mathcal{A}}$
        come segue: 
        per ogni $a \in A$ arricchisci l'insieme delle costanti di $L$ con un 
        nuovo simbolo $\bar{a}$ (nome di $a$) che viene interpretato in $a$. 

        $$
        L_{\mathcal{A}} := (\mathcal{P}, F^{\mathcal{A}}, \alpha, \beta^{\mathcal{A}})
        $$
        con
        $$
        F^{\mathcal{A}} := F\cup \{\bar{a}: a \in A\}
        $$
        e 
        $$
        \begin{cases}
                \beta^{\mathcal{A}} = B & f \in F\\
                \beta^{\mathcal{A}}(\bar{a}) = 0 & a \in A\\
        \end{cases}
        $$
        E inoltre $I(\bar{a}) := a $ per ogni $a \in A$. 
\end{defi}

\begin{defi}[Interpretazione degli $L_{\mathcal{A}}$-termini ground]
        Dati $L = (\mathcal{P}, F, \alpha, \beta)$ e $\mathcal{A} = (A,I)$ 
        e la $L_{\mathcal{A}}$-struttura espansa, si definisce induttivamente: 
        \begin{itemize}
                \item{\textbf{Base}} ogni $c \in F$ tale che $\beta(c) = 0$ ha un'interpretazione $I(c)$
                        già definita, dalla definizione di $L$-struttura. Invece, 
                        $\bar{a} \in F^{A} = F \cup \{\bar{a} : a \in A\}$ 
                        ha come interpretazione $I(\bar{a}) = a \in A$. 
                \item{\textbf{Passo Induttivo}}
                        Sia $f \in F$ un simbolo di funzione con arità $\beta(f) = n$ e 
                        siano $t_1, t_2, \cdots, t_n$ $L_{\mathcal{A}}$-termini ground. 
                        Allora $I(F(t_1, \cdots, t_n)):= I(f)(I(t_1), \cdots, I(t_n))$. 
        \end{itemize}
\end{defi}

\begin{defi}[Interpretazione degli Enunciati]
        Sia data $\mathcal{A} = (A,I)$ una $L$-struttura; allora, si definisce 
        induttivamente la nozione $\mathcal{A} \models A$ per ogni 
        $L_{\mathcal{A}}$-enunciato $A$, ossia $A$ è vera in $\mathcal{A}$. 

        \paragraph{Base}
        Sia 
        $A = P(t_1, \cdots, t_n)$ per $P \in \mathcal{P}$ con $\alpha(P) = n$ 
        e $t_1, \cdots, t_n$ $L_{\mathcal{A}}$-termini ground. Allora 
        $$
        \mathcal{A} \models P(t_1, \cdots, t_n) \iff I(t_1), \cdots, I(t_n) \in I(P) \subseteq A^n
        $$
        \paragraph{Passo induttivo}
                \begin{itemize}
                \item $\mathcal{A} \models A_1 \land A_2 \iff \mathcal{A} \models A_1 \land \mathcal{A} \models A_2$
                \item $\mathcal{A} \models A_1 \lor A_2 \iff \mathcal{A} \models A_1 \lor \mathcal{A} \models A_2$
                \item $\mathcal{A} \models \neg A_1  \iff \mathcal{A} \nvdash A_1 $
                \item $\mathcal{A} \models A_1 \rightarrow A_2 \iff \mathcal{A} \models A_1 \land \mathcal{A} \models A_2$
                \item $\mathcal{A} \models\forall x A \iff \mathcal{A} \models A[\bar{a}/x] \text{ per ogni } a$
                \item $\mathcal{A} \models\exists x A \iff \mathcal{A} \models A[\bar{a}/x] \text{ per almeno un } a$
                \end{itemize}
                Si noti che $\bar{a}$ è un termine, mentre $x$ è una variabile 
                individuale, quindi $A$ diventa un $L_{\mathcal{A}}$-enunciato.
\end{defi}

Abbiamo quindi definito induttivamente $\mathcal{A}\models A$ per gli 
$L_{\mathcal{A}}$-enunciati $A$. A questo punto ci si può ``dimenticare'' degli 
$L_{\mathcal{A}}$-enunciati che non sono $L$-enunciati, 
ossia quelle formule in cui vi è almeno una variabile libera.  

\begin{defi}[Chiusura universale di una formula con variabili libere]
Se $A$ è una formula ben formata, ossia $A \in F_L$ ed eventualmente,
non un enunciato, quindi $FV(A) \neq \emptyset$, dove $FV$ è definito come l'insieme 
di variabili libere che appaiono in $A$, ci si può chiedere se 
$$ 
\mathcal{A} \models A
$$
Definito, quindi, 
$$
FV(A) = \{x_1, \cdots, x_n\} 
$$
si definisce la chiusura universale di $A$ 
$$
\forall[A] := \forall x_1 \forall x_2 \cdots \forall x_n A
$$
e si postula che
$$
\mathcal{A} \models A \iff \mathcal{A} \models \forall[A]
$$
\end{defi}
ossia $\mathcal{A}$ modella una formula ben formata $A$ se e solo se 
$\mathcal{A}$ rende vera la sua chiusura universale, che è un $L$-enunciato.

\begin{oss}
        Se $L$ è un linguaggio con identità (o con uguaglianza), ossia 
        $L = (\mathcal{P}, F, \alpha, \beta)$ e $= \in \mathcal{P}$, 
        $\alpha(=) = 2$, 
        allora, ricordando che in ogni $L$-struttura $\mathcal{A} = (A,I)$, 
        si deve avere per definizione $I(=) = \{(a,a) : a \in A\}$. 
        Pertanto $\mathcal{A} \models t_1 = t_2$ (che, più correttamente, 
        andrebbe scritto in forma postfissa ``$=(t_1, t_2)$'' se e solo se 
        $I(t_1) = I(t_2)$. 
\end{oss}

\subsection{Definizione alternativa di $\mathcal{A} \models A$} 
In questa definizione alternativa,  le variabili sono oggetti di interesse; 
tuttavia le due definizioni, quella su $L\rightarrow L_{\mathcal{A}}$ e quella
che andiamo a dare, sono equivalenti. 
\begin{defi}[Ambiente]
        Un \textbf{ambiente di interpretazione} in una $L$-struttura $\mathcal{A} = (A,I)$ è 
        una funzione 
        $$
        \mathfrak{v} : Var \rightarrow A
        $$
        ossia $\mathfrak{v}(x) \in A$ è un elemento di $A$.         
\end{defi}
\begin{defi}[Variante di ambiente]
        La \textbf{variante di ambiente} servirà a dare la semantica alle espressioni 
        quantificate ed è indicato $\mathfrak{v}[a/x](y)$ e si definisce 
        $$
        \mathfrak{v}[a/x](y) :=
        \begin{cases}
                \mathfrak{v}(y) & y \neq x \\
                a & y = x
        \end{cases}
        $$
\end{defi}
\begin{defi}[Interpretazione degli $L$-termini]
        L'interpretazione degli $L$-termini, anche aperti, per ogni 
        $x \in Var$ e $\mathfrak{v} : Var \rightarrow A$ è 
        definita come 
        $$
        I_{\mathfrak{v}}(x) := \mathfrak{v}(x)
        $$
        e si può ora passare a definire 
        $$
        I_{\mathfrak{v}}(F(t_1, \cdots, t_n)) := (I(F))(I_{\mathfrak{v}}(t_1), \cdots, I_{\mathfrak{v}}(t_n))
        $$
        Si dirà
        $$
        \mathcal{A}\models_{\mathfrak{v}} A 
        $$
        se $\mathcal{A}$ rende vera $A$ nell'ambiente $\mathfrak{v}: Var \rightarrow A$, 
        e si dirà 
        $$
        \mathcal{A} \models_{\mathfrak{v}} P(t_1, \cdots, t_2) \iff (I_{\mathfrak{v}}(t_1), \cdots, I_{\mathfrak{v}}(t_n)) \in I(P)
        $$
         e quindi 

         \begin{itemize}
                 \item  $\mathcal{A} \models_{\mathfrak{v}} A_1 \land A_2 \iff \mathcal{A} \models A_1 \land \mathcal{A} \models A_2$
                 \item $\mathcal{A} \models_{\mathfrak{v}}A_1 \lor A_2 \iff \mathcal{A} \models A_1 \lor \mathcal{A} \models A_2$
                 \item $\mathcal{A} \models_{\mathfrak{v}}  \neg A_1  \iff \mathcal{A} \nvDash A_1 $
                 \item $\mathcal{A} \models_{\mathfrak{v}} A_1 \rightarrow A_2 \iff \mathcal{A} \models A_1 \land \mathcal{A} \models A_2$
                 \item $\mathcal{A} \models_{\mathfrak{v}} \forall x A \iff \mathcal{A} \models_{\mathfrak{v}} A[a/x] \text{ per ogni } a$
                 \item $\mathcal{A} \models_{\mathfrak{v}} \exists x A \iff \mathcal{A} \models_{\mathfrak{v}} A[a/x] \text{ per almeno una  } a$
        \end{itemize}
        A questo punto si  può dire che $\mathcal{A} \models A$ se e solo se 
        per ogni $\mathfrak{v}: Var \rightarrow A$ si ha $\mathcal{A} \models A$. 
\end{defi}

\noindent
Si esprime, ora, un'idea intuitiva di come avvenga il processo di interpretazione 
di un $L$-enunciato. Nella prima definzione si ha il linguaggio $L$ e delle 
$L$-strutture che vengono associate. 
$$
L 
\leadsto
\begin{cases}
        \mathcal{A}  \leadsto L_{\mathcal{A}} \\
        \mathcal{B}  \leadsto L_{\mathcal{B}} \\
        \mathcal{C}  \leadsto L_{\mathcal{C}} \\
        \cdots
\end{cases}
$$
Per ogni $L$-struttura si crea l'espansione associata. 
Nella seconda definizione si ha nuovamente il linguaggio $L$ e delle $L$-strutture, 
tuttavia esse non si espandono ma vi sono un numero di ambienti da considerare: 

$$
L \leadsto 
\begin{cases}
        \mathcal{A} \begin{cases}
                \mathcal{A}, \mathfrak{v}_1 \\
                \mathcal{A}, \mathfrak{v}_2 \\
                \mathcal{A}, \mathfrak{v}_3 \\
                \cdots \\
        \end{cases} \\
        \mathcal{B} \begin{cases}
                \mathcal{B}, \mathfrak{v}_1\\
                \mathcal{B}, \mathfrak{v}_2\\
                \mathcal{B}, \mathfrak{v}_3\\
                \cdots                     
        \end{cases} \\
        \cdots
\end{cases}
$$

\section{Terminologia e Nozioni}

\begin{defi}[$L$-Teoria]
        Una $L$-Teoria è un insieme di $L$-enunciati. 
        Se $\Gamma$ è una $L$-Teoria e $\mathcal{A}$ è una $L$-struttura, 
        si dice che $\mathcal{A}$ è modello di $\Gamma$ e si scrive 
        $\mathcal{A} \models \Gamma$ se e solo se $\mathcal{A} \models \gamma$ 
        per ogni $\gamma \in \Gamma$. 

        La $L$-teoria $\Gamma$ è soddisfacibile e solo se esiste almeno un 
        $\mathcal{A}$ tale che $\mathcal{A} \models \Gamma$, 
        altrimenti $\Gamma$ è insoddisfacibile.

        Sia $\Gamma$ una $L$-teoria e $A$ un $L$-enunciato. 
        Allora $A$ è conseguenza logica di $\Gamma$ e si scrive $\Gamma \models A$ 
        se e solo se $A$ è vera in ogni modello di $\Gamma$, ossia 
        per ogni $L$-struttura $\mathcal{A}$, se $\mathcal{A} \models \Gamma$, 
        allora $\mathcal{A} \models A$. 
\end{defi}

\begin{defi}[Verità Logica]
        Un $L$-enunciato $A$ è detto \textbf{verità logica} se e solo se 
        per ogni $L$-struttura $\mathcal{A}$, $\mathcal{A} \models A$. 
\end{defi}
Il concetto di Verità Logica è analogo al concetto di tautologia nella 
Logica Proposizionale. 
Un esempio di verità logica è $\forall x (x = x)$. 

\noindent
Lo scopo della nostra indagine sarà, d'ora in poi, cercare di stabilire se
$$
\Gamma \models A
$$
dati $\Gamma$  una $L$-teoria e $A$ un $L$-enunciato. 
Se $\Gamma$ è finito, ossia $\Gamma = \{\gamma_1, \cdots, \gamma_n\}$, 
varrà nuovamente il fatto 
$$
\Gamma \models A \iff \gamma_1 \land \cdots \land \gamma_n \land  \{\neg A\} \text{ insodd.}
$$
ossia ci si riduce all'analisi di una singola formula, 
mentre invece se $\Gamma$ è infinito, possiamo solo chiederci se
$$
\Gamma \models A \iff \Gamma \cup \{\neg A\} \text{ insodd.}
$$
Anche nel caso in cui $\Gamma$ è finito, ci saranno casi in cui il processo 
di deduzione sarà solo \textit{semidecidibile}, mentre nella Logica Proposizionale 
il problema di soddisfacibilità era decidibile. 
\subsection{Esempi di $L$-Teorie} 

\subsubsection{Congruenze su un $L$-Linguaggio $L$}
Data una relazione $R$, si vorrebbe modellarla come congruenza.
$$
\Gamma := 
\begin{cases}
        \forall x ~~ R(x,x) \text{ riflessiva } \\
        \forall x \forall y ~~ R(x,y) \rightarrow R(y,x)  \text{ simmetrica } \\
        \forall x \forall y \forall z ~~ R(x,y) \land R(y,z) \rightarrow R(x,z)  \text{ transitiva }\\

        \forall x_1 \cdots \forall x_n \forall y_1 \cdots \forall y_n 
        \begin{cases}
        (R(x_1,y_1) \land \cdots \land R(x_n,y_n)) \rightarrow R(f(x_1, \cdots, x_n), f(y_1, \cdots y_n)) \\
        (R(x_1, y_1) \land \cdots R(x_n, y_n)) \rightarrow (P(x_1, \cdots x_n) \rightarrow P(y_1, \cdots, y_n))
        \end{cases}

\end{cases}
$$
I primi tre $L$-enunciati (assiomi di $\Gamma$) assiomatizzano una 
relazione d'equivalenza, ossia
$$
\text{ se }\mathcal{A} \models \Gamma \text{ allora } I(R) \text{ è una relazione d'equivalenza}
$$
Il quarto $L$-enunciato assiomatizza la congruenza $R$ rispetto ad
una generica $f \in F$ con $\beta(f) = n$, mentre il quinto fa 
la stessa cosa rispetto un generico predicato $P \in \mathcal{P}$. 

\subsubsection{Relazione d'ordine (parziale)}
$$
        \Gamma_{po} := 
        \begin{cases}
                \forall x R(x,x) & \text{ riflessiva }\\
                \forall xy (R(x,y) \land R(y,x)) \rightarrow =(x,y) & \text{ simmetrica }\\
                \forall x \forall y \forall z (R(x,y) \land R(y,z)) \rightarrow R(x,z) & \text{ transitiva }\\
      
        \end{cases}
$$
Si noti l'utilizzo del predicato $=$ d'uguaglianza, pertanto il Linguaggio 
è dotato di tale predicato. 
Un esempio, l'insieme dei naturali con la relazione di minore o uguale 
modellano $\Gamma$: 
$$
(\mathbb{N}, \leq) \models \Gamma
$$
(la notazione utilizzata significa che $I(R) = \leq$)
così come i naturali con l'ordine di divisibilità: 
$$
(\mathbb{N}, |) \models \Gamma
$$


\paragraph{Relazione d'ordine (totale)}
Esiste un modo per costruire una $L$-Teoria che ``allarga'' $\Gamma$ che è modello 
di uno e non dell'altra? Effettivamente, il primo è un ordine totale, mentre 
il secondo non lo è, ossia vi sono elementi tali che né $m | n$ né $n |m $; se 
si riesce ad esprimere la totalità con un enunciato, si riesce a discernere 
uno dall'altro: 
\begin{align*}
        \Gamma_{to} := \Gamma_{po} \cup \{\forall x \forall y (R(x,y) \lor R(y,x))\}
\end{align*}

Aggiungendo nuovi assiomi si possono caratterizzare altri tipi di ordine: 
ad esempio (esercizio), si può caratterizzare un ordine che abbia un elemento 
minimo. 

\subsubsection{Teoria dei gruppi}
Un gruppo è una struttura algebrica. 
Un gruppo è un modello dell'insieme di assiomi $\Gamma_G$ sul linguaggio 
$L = (\mathcal{P}, F, \alpha, \beta)$. 
$$
\Gamma_G := 
        \begin{cases}
                \forall x \forall y \forall z  ~~ ((x * y)*z) = (x*(y*z))  & \text{ associatività}\\
                \forall x (x * e) = x \land (e * x) = x  & \text{ elemento neutro}\\
                \forall x (x * x^{-1}) = e \land (x^{-1} * x = e) & \text{ invertibilità}
        \end{cases}
$$
con $\mathcal{P} = \{ = \}$, $F = \{*, e, ()^{-1}\}$ e relative 
arità. 
$(G, \cdot, ^{-1}, 0) \models \Gamma_G$ se e solo se $G$ è un gruppo; in 
altri termini un gruppo è una struttura con un'operazione associativa in cui 
ogni elemento è invertibile. 
Per esempio, $(\mathbb{Z}, +, -, 0) \models \Gamma_G$ e 
$(\mathbb{Q}\setminus{0}, \cdot, ^{-1}, 1) \models \Gamma_G$. 

\paragraph{Formulazione alternativa}
Una seconda formulazione è la seguente
$$
        \Gamma_{G_2} := 
                \begin{cases}
                        \forall x \forall y \forall z  ~~ ((x * y)*z) = (x*(y*z)) & \text{ associatività}\\
                        \forall x (x * e) = x \land (e * x) = x & \text{ elemento neutro } \\
                        \forall x \exists y (x * y) = e \land (y * x) = e & \text{ esistenza inverso} \\
                \end{cases}
$$
Questa $L$-Teoria rimuove la specifica dell'inverso. Se si ottiene un gruppo 
che modella l'altra formulazione, esso andrà bene anche per questo, ma non è detto 
il viceversa. 
Ora, infatti, non è più necessario specificare l'operazione inverso, 
e si può quindi affermare $(\mathbb{Z}, +, 0) \models \Gamma_{G_2}$, $(Q, +, 0) \models \Gamma_{G_2}$, 
eccetera. Come anticipato, questo vale anche per i gruppi esplicitati prima: 
$(\mathbb{Q}\setminus{0}, \cdot, ^{-1}, 1) \models \Gamma_{G_2}$ e così via, 
anche se l'inverso non viene utilizzato come simbolo esplicito. 
Potremmo quindi utilizzare, al posto del simbolo dell'inverso, qualsiasi cosa e 
rimarrà comunque un modello di $\Gamma_{G_2}$: 
$$
(\mathbb{Z}, +, +1, 0) \models \Gamma_{G_2}
$$
Ma ovviamente non è vero il contrario, ossia 
$$
(\mathbb{Z}, +, +1, 0) \nvDash \Gamma_G
$$
in questo senso, abbastanza sottile, le due formulazioni non sono equivalenti. 
\subsubsection{Tipi di dato: stack}
Definiamo il linguaggio $L= (\mathcal{P}, F, \alpha, \beta)$ 
con 
$$
\mathcal{P} = \{Stack, Elem, =\}
$$
e 
$$
F = \{push, pop, top, nil\}
$$
$$
        \Gamma_{Stack} := 
        \begin{cases}
                \forall x (Stack(x) \lor Elem(x)) \\
                \forall x (\neg Stack(x) \lor \neg Elem(x)) \\
                \forall x \forall y ((Stack(x) \land Elem(y)) \rightarrow top(push(x,y)) = y) \\
                \forall x \forall y ((Stack(x) \land Elem(y)) \rightarrow pop(push(x,y)) = x) \\
                \forall x (Stack(x) \rightarrow push(pop(x),top(x)) =x) \\
        \end{cases}
$$

\subsubsection{Aritmetica di Peano}

Sia $L_{P_A} = ( \mathcal{P}, F, \alpha, \beta)$, con 
$\mathcal{P} = \{=\}$, $\alpha(=) = 2$, $F =  \{ + , *, s, 0,\}$ 
$\beta(+) = \beta(*) = 2$, $\beta(s) = 1$, $\beta(0) =0$. 

L'aritmetica di Peano afferma: 
\begin{itemize}
        \item $\forall x \neg (=(0,s(x)))$, ossia nessun numero ha come successore $0$. 
        \item $\forall x, \forall y (s(x) = s(y)) \rightarrow (x = y)$
        \item $\forall x (x + 0 = x)$
        \item $\forall x \forall y (x + s(y)) = s(x+y)$
        \item $\forall x (x * 0 = 0)$
        \item $\forall x \forall y (x * s(y)) = (x + (x*y))$
        \item $(P[0/x] \land \forall x (P[x/x] \rightarrow P[s(x)/x])) \rightarrow \forall x P[x/x]$ 
\end{itemize}
L'ultimo postulato è una codifica dell'induzione sui numeri naturali. 
