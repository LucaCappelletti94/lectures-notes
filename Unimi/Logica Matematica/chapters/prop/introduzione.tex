\chapter{Introduzione alla Logica Proposizionale}
Si comincia ora l'introduzione alla Logica Proposizionale, 
partendo tuttavia da un concetto espresso senza formalizzarlo immediatamente: 
\begin{defi}[Enunciato]
Con il termine \textbf{enunciato} si intende una frase o un'espressione per la 
quale sia sensato chiedersi se sia vera o se sia falsa in ogni data 
circostanza, ossia ha un \textbf{valore di verità} relativo ad una 
certa circostanza. 
\end{defi}
``Piove'' è un enunciato, così come ``prendo l'ombrello'' 
e ``se piove prendo l'ombrello''. Sapremmo già dire che quest'ultimo ha 
qualcosa di diverso dai primi: quest'ultimo infatti è un \textbf{enunciato 
composto}, mentre i primi sono \textbf{enunciati atomici}. Ci sono frasi 
che non sono enunciati e possiamo anche limitarci all'italiano per trovarne 
alcuni: ``Paolo corre?'' e ``Piove?'' non sono enunciati. Oltre al linguaggio naturale 
vi sono anche altre frasi che non sono enunciati, per esempio ``2''. 

\section{Senso, denotazione e connotazione di un enunciato}
Il senso filosofico dei concetti di \textbf{denotazione} e \textbf{connotazione} verrà tralasciato e, per questo, verrà visto quanto basta per distinguerli.

Si considerino le seguenti espressioni del linguaggio dell'aritmetica: 
\begin{itemize}
  \setlength\itemsep{0pt}
  \item $4$
  \item $2^2$
  \item il predecessore di $5$ 
  \item $3+1$
\end{itemize}
Nessuno di questi è un enunciato, ma non è necessario che lo siano. 
Sappiamo dire cosa significhino, in quanto matematicamente sono sempre 
modi per esprimere il \textit{numero naturale} ``quattro''. \\
Questo esempio inquadra a livello intuitivo cosa sia la \textbf{denotazione} (il numero naturale quattro) e la \textbf{connotazione} (quattro diversi modi per ottenere quattro). 

Anche a questo livello stiamo dicendo una cosa interessante, in quanto questo implica che dobbiamo essere molto precisi riguardo cosa dovrebbe essere la \textit{denotazione} di qualcosa. Un'espressione ha, quindi, una denotazione che è qualcosa di diverso dalla sua connotazione.

In un modo astratto, una espressione $E$ è un \textbf{nome} (e.g., $4$, $2^2$, ``il predecessore di $5$'' o $3+1$) di qualcosa, il quale si riferisce in modo univoco a qualche \textbf{entità}. L'entità alla quale si riferisce l'espressione è essa stessa la denotazione. La connotazione, in questo senso, è quanto l'espressione effettivamente esprime, ossia tutto il resto dell'informazione contenuta nell'espressione stessa. 

La logica studia la denotazione degli enunciati (chiamati anche \textit{sentences}) e non le connotazioni, in quanto esse sono troppo difficili da gestire a livello iniziale. Le denotazioni godono infatti dell'importante proprietà dell'\textbf{invarianza per sostituzione}, ossia se ad un'espressione si cambiano delle parti sostituendole con parti denotazionalmente uguali, la denotazione globale non cambia, mentre la connotazione può potenzialmente cambiare totalmente, come dimostrano le quattro frasi iniziali.

\subsection{Denotazione di un Enunciato}
Si può definire ora, più formalmente, cosa sia la \textbf{denotazione} di 
un enunciato. Si prenda, per esempio, l'enunciato
$$
4 = \text{pred}(5)
$$
e si applichi una sostituzione con espressioni denotazionalmente equivalenti: 
$$
4 = 4
$$
Il principio d'invarianza dice che questi due enunciati sono \textbf{denotazionalmente}
equivalenti in quanto entrambe sono enunciati ``veri''. Questo benché l'ultimo enunciato contenga meno informazioni rispetto all'altro. \\
Questo ci dice che la denotazione di un enunciato è il suo \textit{valore di verità}, e perciò gli enunciati non sono altro che una forma connotazionale di una costante: \texttt{vero} o \texttt{falso}.

L'oggetto della Logica Proposizionale sono le \textbf{proposizioni}: il contenuto denotazionale degli enunciati, che può essere \texttt{vero} o \texttt{falso}. 

\section{Enunciati e Connettivi}
Alcuni enunciati semplici come ``Piove'' o ``Paolo corre'' sono definiti 
\textbf{atomici} in quanto la loro denotazione è solamente un valore di 
verità. Altri enunciati, definiti \textbf{composti}, sono enunciati che si possono
``smontare'', come ``Piove e c'è vento'', che è chiaramente composto 
dagli enunciati atomici ``Piove'' e ``c'è vento''. Il connettivo ``e'' è 
ciò che li unisce, come potrebbe accadere anche per ``o'', ``non'' 
e ``se...allora''. 

\begin{defi}[Enunciato semplice]
  Un enunciato semplice (o atomico) rappresenta un fatto \textit{denotato} da un valore di verità, detto \textbf{valore semantico}. È rappresentato da un \textit{simbolo} chiamato \textbf{lettera proposizionale} scelto dall'insieme infinito $\mathscr{L}$: il \textbf{linguaggio proposizionale}.
  
  Per esempio: $p, q, r, p_1, \cdots \in \mathscr{L}$
\end{defi}
\begin{defi}[Enunciato composto]
  Un enunciato composto è una rappresentazione di un fatto composto da enunciati semplici tramite i seguenti \textbf{connettivi}: $\land, \lor, \neg$ e $\rightarrow$. \\ 
  Il valore denotazionale di ogni enunciato composto dipende dai valori denotazionali degli enunciati atomici e dal valore semantico dei connettivi che lo compongono. 
  
  Per esempio: $p \land q$, $p \lor q$, $\neg p$
\end{defi}
