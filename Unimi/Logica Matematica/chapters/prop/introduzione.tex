\chapter{Introduzione alla Logica Proposizionale}
Si comincia ora l'introduzione alla Logica Proposizionale, 
partendo tuttavia da un concetto espresso senza formalizzarlo immediatamente: 
\begin{defi}[Enunciato]
Con il termine \textbf{enunciato} si intende una frase o un'espressione per la 
quale sia sensato chiedersi se sia vera o se sia falsa in ogni data 
circostanza, ossia ha un \textbf{valore di verità} relativo ad una 
certa circostanza. 
\end{defi}
``Piove'' è un enunciato, così come ``prendo l'ombrello'' 
e ``se piove prendo l'ombrello''. Sapremmo già dire che quest'ultimo ha 
qualcosa di diverso dai primi: quest'ultimo infatti è un \textbf{enunciato 
composto}, mentre i primi sono \textbf{enunciati atomici}. Ci sono frasi 
che non sono enunciati e possiamo anche limitarci all'italiano per trovarne 
alcuni: ``Paolo corre?'' e ``Piove?'' non sono enunciati. Oltre al linguaggio naturale 
vi sono anche altre frasi che non sono enunciati, per esempio ``2''. 

\section{Senso, denotazione e connotazione di un enunciato}
Il senso filosofico dei concetti di \textbf{denotazione} 
e \textbf{connotazione} verrà tralasciato e verranno infatti 
trattati in una maniera poco profonda, soprattutto per capire la distinzione
tra denotazione e connotazione. Ecco alcune espressioni del linguaggio 
dell'aritmetica: 
\begin{itemize}
  \setlength\itemsep{0pt}
  \item $4$
  \item $2^2$
  \item il predecessore di $5$ 
  \item $3+1$
\end{itemize}
Nessuno di questi è un enunciato, ma non è necessario che lo siano. 
Sappiamo dire cosa significhino, in quanto matematicamente sono sempre 
modi per esprimere \textit{il numero naturale quattro}. Questo esempio inquadra 
a livello intuitivo cosa sia la \textbf{denotazione} (il numero naturale 
quattro) e la \textbf{connotazione} (quattro diversi modi per ottenere quattro). 
Anche a questo livello stiamo dicendo una cosa interessante, in quanto 
questo implica che dobbiamo essere molto precisi riguardo cosa dovrebbe
essere la \textit{denotazione} di qualcosa. 
Un'espressione ha, quindi, una denotazione che è qualcosa di diverso dalla 
sua connotazione. In un modo astratto, una espressione $E$ è un \textit{nome}
di qualcosa, come per esempio $4$, $2^2$, il predecessore di $5$ e $3+1$, 
il quale si riferisce in modo univoco a qualche entità. L'\textit{entità} alla 
quale si riferisce l'espressione è 
essa stessa la denotazione. La connotazione, in questo senso, è quanto 
l'espressione effettivamente esprime, ossia tutto il resto dell'informazione 
contenuta nell'espressione stessa. 

La logica studia la denotazione degli enunciati (chiamati anche \textit{sentences}) e 
non le connotazioni, in quanto esse sono troppo difficili da gestire a livello 
iniziale. Le denotazioni godono infatti dell'importante proprietà dell'\textbf{invarianza 
per sostituzione}, ossia se ad un'espressione si cambiano delle parti 
sostituendole con parti denotazionalmente uguali, la denotazione globale 
non cambia, mentre la connotazione può potenzialmente cambiare totalmente, 
come dimostrano le quattro frasi iniziali.

\subsection{Denotazione di un Enunciato}
Si può definire ora, più formalmente, cosa sia la \textbf{denotazione} di 
un enunciato. Si prenda, per esempio, l'enunciato
$$
4 = \text{pred}(5)
$$
e si applichi una sostituzione con espressioni denotazionalmente equivalenti: 
$$
4 = 4
$$
Il principio d'invarianza dice che questi due enunciati sono \textbf{denotazionalmente}
equivalenti, benché l'ultimo enunciato non contiene nessuna informazione 
ulteriore rispetto alla denotazione stessa (circa). La denotazione 
di un enunciato è, quindi, il loro valore di verità, ossia il fatto che sono 
una forma connotazionale di una costante \texttt{vero} o \texttt{falso}. 
Quindi, l'oggetto della Logica Proposizionale sono le \textbf{proposizioni}, 
ossia il contenuto denotazionale degli enunciati, che può essere 
\texttt{vero} o \texttt{falso}. 

\section{Enunciati e Connettivi}
Alcuni enunciati semplici come ``Piove'' o ``Paolo corre'' sono definiti 
\textbf{atomici} in quanto la loro denotazione è solamente un valore di 
verità. Altri enunciati, definiti \textbf{composti}, sono enunciati che si possono
``smontare'', come ``Piove e c'è vento'', che è chiaramente composto 
dagli enunciati atomici ``Piove'' e ``c'è vento''. Il connettivo ``e'' è 
ciò che li unisce, come potrebbe accadere anche per ``o'', ``non'' 
e ``se...allora''. 

In una maniera più formale, gli enunciati semplici 
devono solamente rappresentare il fatto che denotano un valore di verità e 
saranno quindi rappresentati da \textbf{simboli} appartenente all'insieme infinito
$L$ chiamato \textbf{linguaggio proposizionale}. I simboli 
$p, q, r, p_1, \cdots \in L$ sono chiamati \textbf{lettere proposizionali}. 
Oltre alla sintassi, formalmente il valore semantico (quindi denotazionale) di ogni 
lettera proposizionale è un valore di verità. 

Gli enunciati composti sono formalizzabili con una simbologia che rispetta 
i simboli per gli enunciati atomici: $p \land q$, $p \lor q$, $\neg p$ e 
$p \rightarrow q$ sono enunciati composti. I simboli $\land, \lor, \neg$ 
e $\rightarrow$ sono chiamati \textbf{connettivi}. Il valore denotazionale 
di ogni enunciato composto dipenderà dai 
valori denotazionali degli enunciati atomici e dal valore semantico dei connettivi 
che lo compongono. 
