% TEX root = ../../main.tex
\chapter{Sintassi della Logica Proposizionale}

Dopo aver introdotto la Logica Proposizionale, si può 
ora formalizzare la struttura sintattica degli enunciati: 
l'insieme $F_{L}$ degli enunciati costruibili 
sul linguaggio $L$ rispettando la \textbf{sintassi 
degli enunciati} è definito come segue: 

\begin{defi}{(Sintassi degli Enunciati)}
\begin{itemize}
  \setlength\itemsep{0pt}
 \item $F_L$ è il più piccolo insieme tale che 
    \begin{itemize}
      \item per ogni $p \in L$ si ha $p \in F_L$
      \item se $A,B \in F_L$ allora anche $(A\land B) \in F_L$, $(A\lor B) \in F_L$, 
        $(A \rightarrow B) \in F_L$ e $(\neg A) \in F_L$, dove $A$ e $B$ possono 
        essere a loro volta enunciati complessi. 
      \end{itemize}
  \item $F_L$ è l'intersezione di tutti gli insiemi $X$ tali che 
    \begin{itemize}
      \item $L \subseteq X$
      \item se $A,B \in X$ allora $(A\land B)$, $(A\lor B)$, $(\neg A)$ e 
        $(A \rightarrow B)$ sono contenuti in $X$. 
    \end{itemize}
  \item (induttiva) $F_L$ è l'insieme che rispetta le condizioni seguenti: 
    \begin{itemize}
      \item $L \subseteq F_L$: se $p \in L$, allora $p \in F_L$
      \item Se $A,B \in F_L$, allora $(A\land B)$, $(A\lor B)$, $(\neg A)$ e 
        $(A \rightarrow B)$ sono contenuti in $F_L$
      \item Nient'altro appartiene a $F_L$.
    \end{itemize}
\end{itemize}
\end{defi}
\paragraph{Esercizio}
Scrivere qualcosa che non sia un enunciato utilizzando solamente la sintassi della 
logica proposizionale. Un esempio è $p q \neg$. Un altro è $\rightarrow q$.

\section{L-Costruzioni}
Per stabilire se una stringa $w \in (\{\land, \lor, \neg, \rightarrow\}
\cup L)^*$\footnote{L'utilizzo dell'operatore $^*$ è paragonabile all'operatore 
  Stella di Kleene nella Teoria dei Linguaggi e denota l'insieme di tutte le
  stringhe di lunghezza finita componibili utilizzando le lettere 
  dell'alfabeto indicato, in questo caso i connettivi e le lettere 
proposizionali.}
è anche $w \in F_L$ si utilizza una \textbf{L-costruzione} o 
\textbf{certificato}, ossia una sequenza finita di formule $w_1, w_2, \cdots, w_n$ tale che 
per ogni $i = 1,\cdots, n$ si ha $w_i \in L$, ossia 
$w$ è una lettera proposizionale, oppure esiste $j < i$ tale 
che $w_i = \neg w_j$, ossia $w_i$ è la forma negata 
di una formula più semplice che appare prima nell'elenco di 
formule; oppure esistono $k,j < i$ tali per cui 
$w_i = w_k \land w_j$ o $w_i = w_k \lor w_j$ o $w_i = w_k \rightarrow w_j$.
Sia, per esempio $w = (p\land q)\rightarrow r$. Per dimostrare che 
è $w \in F_L$ si possono introdurre come L-costruzione 
$$
w_1 = p, w_2 = q, w_3 = (p\land q), w_4 = r, w_5 = (p \land q) \rightarrow r
$$
questa L-costruzione \textit{certifica} che $w \in F_L$.
In termini computazionali, si può controllare velocemente che quanto fatto 
sopra sia una costruzione corretta e che pertanto $w \in F_L$. Si noti, inoltre, 
che questa L-costruzione non è unica. 
I certificati portano con sé una proprietà importante riguardo le formule, 
ossia 
\begin{pro}[di unica leggibilità dei Certificati]
Se $w \in F_L$, allora vale uno e uno solo dei seguenti casi: o $w \in L$, 
o esiste $v$ tale che $w = \neg v$, o esistono $v_1, v_2$ tali che 
$w = (v_1 \land v_2)$, 
$(v_1 \lor v_2)$ o $(v_1 \rightarrow v_2)$, dove i $v_i$ sono determinati 
univocamente. 
\end{pro}
Questa proprietà garantisce l'esistenza di una sorta di operazione 
inversa della costruzione di certificati: mentre quest'ultimo ``monta'' 
una formula, questa proprietà garantisce il fatto che sia possibile 
``smontarla'' in un unico modo.
Questa proprietà non è condivisa tra tutti i linguaggi formali: per 
esempio nelle grammatiche una stringa $w = ciao$ può essere composta da $v_1 = \epsilon$ 
e $v_2 = ciao$ eccetera. 
Spesso, il concetto di leggibilità può essere espresso anche tramite il concetto 
di \textbf{albero di parsing} di una formula, che è unico e mostra come 
essa sia costruita. 

\section{Osservazioni e convenzioni riguardo alla sintassi}
Esistono ulteriori connettivi (o connettivi derivati) oltre a quelli utilizzati fino ad ora, 
ossia $\land$, $\lor$, $\neg$ e $\rightarrow$: 
uno di questi è $\bot$, che è un connettivo di arità zero che denota il falso; un 
altro è $\top$, che è un connettivo zerario che denota il
sempre vero e chiaramente $\bot = \neg \top$. Altro connettivo è $\iff$, 
definito come $(A \rightarrow B) \land (B \rightarrow A)$.

Una seconda osservazione riguarda l'uso delle parentesi: per come abbiamo definito 
le formule, l'oggetto $p \land q \notin F_L$ in quando mancano le parentesi. 
In formule complesse, le parentesi possono aggiungere complicatezza e portare l'errore: 
useremo il buonsenso per ``dimenticarci'' delle parentesi laddove non ci sia 
pericolo di confusione. Tuttavia non bisogna farsi trasportare troppo, in quanto 
esistono delle parentesi necessarie, come per esempio $(p \land q ) \lor r$ 
oppure $p \land (q \lor r)$. 

