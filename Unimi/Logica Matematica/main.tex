\documentclass[12pt, a4paper,titlepage]{book}
\usepackage[a4paper,left=3cm,right=2cm,top=2.5cm,bottom=2.5cm]{geometry}
\usepackage{
  mathptmx, % use ~Times New Roman
  wrapfig,
  graphicx, 
  titlesec, 
  fancyhdr,
  tikz, 
  pgfplots, 
  amsmath,
  amssymb,
  subcaption, 
  multirow,
  amsthm, 
  bussproofs
}
\usepackage[italian]{babel}

\newtheorem{lemn}{Lemma}
\newtheorem*{lem}{Lemma}
\newtheorem{ossn}{Osservazione}
\newtheorem*{oss}{Osservazione}
\newtheorem{defin}{Definizione}
\newtheorem*{defi}{Definizione}
\newtheorem{pron}{Proprietà}
\newtheorem*{pro}{Proprietà}
\newtheorem{prin}{Principio}
\newtheorem*{pri}{Principio}
\newtheorem{teon}{Teorema}
\newtheorem*{teo}{Teorema}
\newtheorem{corn}{Corollario}
\newtheorem*{cor}{Corollario}





\pgfplotsset{compat=1.17}
\newcommand{\pgfenv}{
\pgfplotsset{
        standard/.style={%Axis format configuration
        axis x line=middle,
        axis y line=middle,
        enlarge x limits=0.15,
        enlarge y limits=0.15,
        every axis x label/.style={at={(current axis.right of origin)},anchor=north west},
        every axis y label/.style={at={(current axis.above origin)},anchor=north east},
        every axis plot post/.style={mark options={fill=white}}
        }
    }
  }
\setcounter{tocdepth}{2}
\setcounter{secnumdepth}{2}

% Chapter Titling: Chapter [0-9] LEFT, Chapter Title RIGHT
\newcommand*{\justifyheading}{\raggedleft}
\titleformat{\chapter}[display]
{\normalfont\large}{\MakeUppercase\chaptertitlename \ \ \thechapter}
{20pt}{\Huge\bfseries\justifyheading}


\begin{document}
% Forewords + TOC Page header Style
% pageNumber -- Chapter Title -------- | ------- Chapter Title -- pageNumber
\pagestyle{fancy}
\renewcommand{\headrulewidth}{0pt} % to remove line on header
\renewcommand{\footrulewidth}{0pt} % to remove line on footer
\renewcommand{\chaptermark}[1]{\markboth{#1}{}}
\fancyhead[LE]{\thepage \ \ }
\fancyhead[RO]{\MakeUppercase\leftmark \ \ \thepage}
\fancyfoot[C] {\thepage}



\frontmatter
\begin{titlepage}
  \centering
              \vspace*{6cm}
              \Huge{Logica Matematica}
  
              \vspace*{0.5cm}

           \Large{Lecture Notes} 

              \vspace*{0.5cm}

              \Large{\textbf{Corso del Prof. Stefano Aguzzoli}}
        
              \vspace*{2cm}

         \Large{Edoardo Marangoni}

              \vspace*{2cm}

              \small{University of Milan}

              \small{Department of Computer Science}

              \small{\today}
              
              \vspace*{2cm}
\includegraphics[width=0.2\textwidth]{images/unimi.png} 

\end{titlepage}

\tableofcontents    


\mainmatter
% Corpus Header Style
% pageNumber -- ChapterTitle ----- Chapter | Chapter ------ Section -- pageNumber
\pagestyle{fancy}
\renewcommand{\headrulewidth}{0pt}
\renewcommand{\chaptermark}[1]{\markboth{#1}{}}
\fancyhf{}
\fancyhead[LE]{\thepage \ \ \MakeUppercase\leftmark}
\fancyhead[RE, LO]{\MakeUppercase\chaptertitlename \ \ \thechapter}
\fancyhead[RO]{\rightmark \ \ \thepage}
\fancyfoot[C]{\thepage}



\chapter*{Disclaimer}
Questo documento contiene degli appunti presi durante il corso tenutosi 
nell'A.A. 2020/2021. Benché i materiali si attengono a ciò che è stato 
spiegato nel corso, queste note \textbf{non sono in alcun modo approvate 
dal docente}, con tutte le conseguenze implicate.

\chapter{Introduzione}
\chapter{Introduzione alla Logica Proposizionale}
Si comincia ora l'introduzione alla Logica Proposizionale, 
partendo tuttavia da un concetto espresso senza formalizzarlo immediatamente: 
\begin{defi}[Enunciato]
Con il termine \textbf{enunciato} si intende una frase o un'espressione per la 
quale sia sensato chiedersi se sia vera o se sia falsa in ogni data 
circostanza, ossia ha un \textbf{valore di verità} relativo ad una 
certa circostanza. 
\end{defi}
``Piove'' è un enunciato, così come ``prendo l'ombrello'' 
e ``se piove prendo l'ombrello''. Sapremmo già dire che quest'ultimo ha 
qualcosa di diverso dai primi: quest'ultimo infatti è un \textbf{enunciato 
composto}, mentre i primi sono \textbf{enunciati atomici}. Ci sono frasi 
che non sono enunciati e possiamo anche limitarci all'italiano per trovarne 
alcuni: ``Paolo corre?'' e ``Piove?'' non sono enunciati. Oltre al linguaggio naturale 
vi sono anche altre frasi che non sono enunciati, per esempio ``2''. 

\section{Senso, denotazione e connotazione di un enunciato}
Il senso filosofico dei concetti di \textbf{denotazione} 
e \textbf{connotazione} verrà tralasciato e verranno infatti 
trattati in una maniera poco profonda, soprattutto per capire la distinzione
tra denotazione e connotazione. Ecco alcune espressioni del linguaggio 
dell'aritmetica: 
\begin{itemize}
  \setlength\itemsep{0pt}
  \item $4$
  \item $2^2$
  \item il predecessore di $5$ 
  \item $3+1$
\end{itemize}
Nessuno di questi è un enunciato, ma non è necessario che lo siano. 
Sappiamo dire cosa significhino, in quanto matematicamente sono sempre 
modi per esprimere \textit{il numero naturale quattro}. Questo esempio inquadra 
a livello intuitivo cosa sia la \textbf{denotazione} (il numero naturale 
quattro) e la \textbf{connotazione} (quattro diversi modi per ottenere quattro). 
Anche a questo livello stiamo dicendo una cosa interessante, in quanto 
questo implica che dobbiamo essere molto precisi riguardo cosa dovrebbe
essere la \textit{denotazione} di qualcosa. 
Un'espressione ha, quindi, una denotazione che è qualcosa di diverso dalla 
sua connotazione. In un modo astratto, una espressione $E$ è un \textit{nome}
di qualcosa, come per esempio $4$, $2^2$, il predecessore di $5$ e $3+1$, 
il quale si riferisce in modo univoco a qualche entità. L'\textit{entità} alla 
quale si riferisce l'espressione è 
essa stessa la denotazione. La connotazione, in questo senso, è quanto 
l'espressione effettivamente esprime, ossia tutto il resto dell'informazione 
contenuta nell'espressione stessa. 

La logica studia la denotazione degli enunciati (chiamati anche \textit{sentences}) e 
non le connotazioni, in quanto esse sono troppo difficili da gestire a livello 
iniziale. Le denotazioni godono infatti dell'importante proprietà dell'\textbf{invarianza 
per sostituzione}, ossia se ad un'espressione si cambiano delle parti 
sostituendole con parti denotazionalmente uguali, la denotazione globale 
non cambia, mentre la connotazione può potenzialmente cambiare totalmente, 
come dimostrano le quattro frasi iniziali.

\subsection{Denotazione di un Enunciato}
Si può definire ora, più formalmente, cosa sia la \textbf{denotazione} di 
un enunciato. Si prenda, per esempio, l'enunciato
$$
4 = \text{pred}(5)
$$
e si applichi una sostituzione con espressioni denotazionalmente equivalenti: 
$$
4 = 4
$$
Il principio d'invarianza dice che questi due enunciati sono \textbf{denotazionalmente}
equivalenti, benché l'ultimo enunciato non contiene nessuna informazione 
ulteriore rispetto alla denotazione stessa (circa). La denotazione 
di un enunciato è, quindi, il loro valore di verità, ossia il fatto che sono 
una forma connotazionale di una costante \texttt{vero} o \texttt{falso}. 
Quindi, l'oggetto della Logica Proposizionale sono le \textbf{proposizioni}, 
ossia il contenuto denotazionale degli enunciati, che può essere 
\texttt{vero} o \texttt{falso}. 

\section{Enunciati e Connettivi}
Alcuni enunciati semplici come ``Piove'' o ``Paolo corre'' sono definiti 
\textbf{atomici} in quanto la loro denotazione è solamente un valore di 
verità. Altri enunciati, definiti \textbf{composti}, sono enunciati che si possono
``smontare'', come ``Piove e c'è vento'', che è chiaramente composto 
dagli enunciati atomici ``Piove'' e ``c'è vento''. Il connettivo ``e'' è 
ciò che li unisce, come potrebbe accadere anche per ``o'', ``non'' 
e ``se...allora''. 

In una maniera più formale, gli enunciati semplici 
devono solamente rappresentare il fatto che denotano un valore di verità e 
saranno quindi rappresentati da \textbf{simboli} appartenente all'insieme infinito
$L$ chiamato \textbf{linguaggio proposizionale}. I simboli 
$p, q, r, p_1, \cdots \in L$ sono chiamati \textbf{lettere proposizionali}. 
Oltre alla sintassi, formalmente il valore semantico (quindi denotazionale) di ogni 
lettera proposizionale è un valore di verità. 

Gli enunciati composti sono formalizzabili con una simbologia che rispetta 
i simboli per gli enunciati atomici: $p \land q$, $p \lor q$, $\neg p$ e 
$p \rightarrow q$ sono enunciati composti. I simboli $\land, \lor, \neg$ 
e $\rightarrow$ sono chiamati \textbf{connettivi}. Il valore denotazionale 
di ogni enunciato composto dipenderà dai 
valori denotazionali degli enunciati atomici e dal valore semantico dei connettivi 
che lo compongono. 

\cleardoublepage

\part{Logica Proposizionale}
\chapter{Introduzione alla Logica Proposizionale}
Si comincia ora l'introduzione alla Logica Proposizionale, 
partendo tuttavia da un concetto espresso senza formalizzarlo immediatamente: 
\begin{defi}[Enunciato]
Con il termine \textbf{enunciato} si intende una frase o un'espressione per la 
quale sia sensato chiedersi se sia vera o se sia falsa in ogni data 
circostanza, ossia ha un \textbf{valore di verità} relativo ad una 
certa circostanza. 
\end{defi}
``Piove'' è un enunciato, così come ``prendo l'ombrello'' 
e ``se piove prendo l'ombrello''. Sapremmo già dire che quest'ultimo ha 
qualcosa di diverso dai primi: quest'ultimo infatti è un \textbf{enunciato 
composto}, mentre i primi sono \textbf{enunciati atomici}. Ci sono frasi 
che non sono enunciati e possiamo anche limitarci all'italiano per trovarne 
alcuni: ``Paolo corre?'' e ``Piove?'' non sono enunciati. Oltre al linguaggio naturale 
vi sono anche altre frasi che non sono enunciati, per esempio ``2''. 

\section{Senso, denotazione e connotazione di un enunciato}
Il senso filosofico dei concetti di \textbf{denotazione} 
e \textbf{connotazione} verrà tralasciato e verranno infatti 
trattati in una maniera poco profonda, soprattutto per capire la distinzione
tra denotazione e connotazione. Ecco alcune espressioni del linguaggio 
dell'aritmetica: 
\begin{itemize}
  \setlength\itemsep{0pt}
  \item $4$
  \item $2^2$
  \item il predecessore di $5$ 
  \item $3+1$
\end{itemize}
Nessuno di questi è un enunciato, ma non è necessario che lo siano. 
Sappiamo dire cosa significhino, in quanto matematicamente sono sempre 
modi per esprimere \textit{il numero naturale quattro}. Questo esempio inquadra 
a livello intuitivo cosa sia la \textbf{denotazione} (il numero naturale 
quattro) e la \textbf{connotazione} (quattro diversi modi per ottenere quattro). 
Anche a questo livello stiamo dicendo una cosa interessante, in quanto 
questo implica che dobbiamo essere molto precisi riguardo cosa dovrebbe
essere la \textit{denotazione} di qualcosa. 
Un'espressione ha, quindi, una denotazione che è qualcosa di diverso dalla 
sua connotazione. In un modo astratto, una espressione $E$ è un \textit{nome}
di qualcosa, come per esempio $4$, $2^2$, il predecessore di $5$ e $3+1$, 
il quale si riferisce in modo univoco a qualche entità. L'\textit{entità} alla 
quale si riferisce l'espressione è 
essa stessa la denotazione. La connotazione, in questo senso, è quanto 
l'espressione effettivamente esprime, ossia tutto il resto dell'informazione 
contenuta nell'espressione stessa. 

La logica studia la denotazione degli enunciati (chiamati anche \textit{sentences}) e 
non le connotazioni, in quanto esse sono troppo difficili da gestire a livello 
iniziale. Le denotazioni godono infatti dell'importante proprietà dell'\textbf{invarianza 
per sostituzione}, ossia se ad un'espressione si cambiano delle parti 
sostituendole con parti denotazionalmente uguali, la denotazione globale 
non cambia, mentre la connotazione può potenzialmente cambiare totalmente, 
come dimostrano le quattro frasi iniziali.

\subsection{Denotazione di un Enunciato}
Si può definire ora, più formalmente, cosa sia la \textbf{denotazione} di 
un enunciato. Si prenda, per esempio, l'enunciato
$$
4 = \text{pred}(5)
$$
e si applichi una sostituzione con espressioni denotazionalmente equivalenti: 
$$
4 = 4
$$
Il principio d'invarianza dice che questi due enunciati sono \textbf{denotazionalmente}
equivalenti, benché l'ultimo enunciato non contiene nessuna informazione 
ulteriore rispetto alla denotazione stessa (circa). La denotazione 
di un enunciato è, quindi, il loro valore di verità, ossia il fatto che sono 
una forma connotazionale di una costante \texttt{vero} o \texttt{falso}. 
Quindi, l'oggetto della Logica Proposizionale sono le \textbf{proposizioni}, 
ossia il contenuto denotazionale degli enunciati, che può essere 
\texttt{vero} o \texttt{falso}. 

\section{Enunciati e Connettivi}
Alcuni enunciati semplici come ``Piove'' o ``Paolo corre'' sono definiti 
\textbf{atomici} in quanto la loro denotazione è solamente un valore di 
verità. Altri enunciati, definiti \textbf{composti}, sono enunciati che si possono
``smontare'', come ``Piove e c'è vento'', che è chiaramente composto 
dagli enunciati atomici ``Piove'' e ``c'è vento''. Il connettivo ``e'' è 
ciò che li unisce, come potrebbe accadere anche per ``o'', ``non'' 
e ``se...allora''. 

In una maniera più formale, gli enunciati semplici 
devono solamente rappresentare il fatto che denotano un valore di verità e 
saranno quindi rappresentati da \textbf{simboli} appartenente all'insieme infinito
$L$ chiamato \textbf{linguaggio proposizionale}. I simboli 
$p, q, r, p_1, \cdots \in L$ sono chiamati \textbf{lettere proposizionali}. 
Oltre alla sintassi, formalmente il valore semantico (quindi denotazionale) di ogni 
lettera proposizionale è un valore di verità. 

Gli enunciati composti sono formalizzabili con una simbologia che rispetta 
i simboli per gli enunciati atomici: $p \land q$, $p \lor q$, $\neg p$ e 
$p \rightarrow q$ sono enunciati composti. I simboli $\land, \lor, \neg$ 
e $\rightarrow$ sono chiamati \textbf{connettivi}. Il valore denotazionale 
di ogni enunciato composto dipenderà dai 
valori denotazionali degli enunciati atomici e dal valore semantico dei connettivi 
che lo compongono. 

% TEX root = ../../main.tex

\chapter{Introduzione e Sintassi}
Si può immaginare la Logica del Primo Ordine come ``costruita'' 
sulla base della logica proposizionale. Il sillogismo aristotelico
\begin{prooftree}
        \AxiomC{Ogni uomo è mortale}
        \AxiomC{Socrate è un uomo}
        \BinaryInfC{Quindi Socrate è mortale}
\end{prooftree}
espresso come possibile nella Logica proposizionale diventa 
\begin{prooftree}
        \AxiomC{$p$}
        \AxiomC{$q$}
        \BinaryInfC{$r$}
\end{prooftree}
Ci si chiede se sia vero, quindi: è forse 
$$
p, q \models r
$$
Con DPLL ci si chiede se 
$\{p, q, \neg r\}$ sia insoddisfacibile. 
Il primo passo di DPLL afferma
\begin{prooftree}
        \AxiomC{$\emptyset \vdash \{\{p\}, \{q\}, \{\neg r\}\}$}
        \UnaryInfC{$\mathfrak{v}(p) = 1 \vdash \{\{q\}, \{\neg r\}\}$}
        \UnaryInfC{$\mathfrak{v}(p)=1, \mathfrak{v}(q) = 1 \vdash \{\{\neg r\}\}$}
        \UnaryInfC{$\mathfrak{v}(r) = 0, \mathfrak{v}(p)=1, \mathfrak{v}(q) = 1 \vdash \emptyset$}
\end{prooftree}
e pertanto 
$$
p, q \models r \text{ è falso }
$$
che va ovviamente contro la nostra intuizione. 
Per gestire argomentazioni
razionali come il sillogismo aristotelico è necessario dotarsi di un linguaggio 
più \textit{espressivo} rispetto alla Logica Proposizionale, arricchendone 
la Sintassi con nuovi operatori, variabili, costanti e la Semantica assegnando 
un modo di interpretare i nuovi ``ingredienti'' del linguaggio in modo da poter 
rappresentare situazioni più raffinate che nella Logica Proposizionale. 

Si arriverà a scrivere 
$$
\forall x (U(x)\rightarrow M(x)), U(s) \models M(s)
$$
per indicare il sillogismo aristotelico.
Oltre ad andare a vedere come mettere in piedi, di primo acchito a livello sintattico, 
tutta questa struttura, si studierà anche il modo per \textit{risolvere} 
delle asserzioni, riutilizzando le tecnologie introdotte per la logica proposizionale.
In particolare, per fare un esempio introduttivo, si tornerà a chiedersi se 
il sillogismo aristotelico sia valido ponendosi  il quesito 
$$
\{\{\neg U(x), M(x)\}, \{U(s)\}, \{\neg M(s)\}\} \text{ è soddisfacibile?}
$$

Si consideri nuovamente 
$$
\text{ Ogni uomo è mortale }
$$
I modi di designare direttamente dell'\textit{Universo} sono un ingrediente 
importante della Logica dei Predicati, fornendo per esempio il modo di 
affermare che Socrate sia mortale, riferendosi ad un preciso elemento. 
Si necessita un modo per designare un elemento non preciso, in maniera indiretta: 
$$
\text{ Il padre di ogni uomo è un uomo }
$$
Si può concludere, quindi, che il padre di Socrate sia un mortale, 
nonostante sia una designazione indiretta di un individuo dell'Universo; 
si può inoltre tradurre quanto detto come 
\begin{prooftree}
        \AxiomC{$\forall x (U(x) \rightarrow M(x)$}
        \AxiomC{$\forall (U(x) \rightarrow U(p(x))$}
        \AxiomC{ $U(s)$}
        \TrinaryInfC{$M(p(s))$}
\end{prooftree}

Per decidere questa \textit{deduzione}, ci sarà qualcosa come 
$$
\{\{\neg U(x), M(x)\}, \{\neg U(x), U(p(x))\}, \{U(s)\}, \{\neg M(p(s))\}\}
$$
Quindi, scopriremo che non basterà sostituire al posto di $x$ la ``lettera'' 
$s$, ma sarà anche necessario considerare $p(s)$. A questo punto, sarà immediato 
arrivare alla conclusione che la situazione sia in realtà un po' più complicata: 
come si considera $p(s)$, si dovrebbe considerare anche $p(p(s))$, $p(p(p(s)))$...

Nel nostro caso, una possibile refutazione è la seguente: 
$$
\neg U(p(s)), M(p(s))
$$
istanziando la $x$ su $p(s)$, in questo modo si può risolvere 
rispetto a $\neg M(p(s))$: 
$$
\{\{\neg U(p(s))\}, \{\neg U(x), U(p(x))\}, \{U(s)\}\}
$$
e si prosegue istanziando $U(x)$ a $U(s)$, arrivando all'insieme vuoto. 
Vedremo, quindi, perché e quando si possono utilizzare queste istanzianzioni. 

\section{Sintassi della Logica del Primo Ordine}
Partiamo, dunque, fissando i dettagli sintattici, cioè il Linguaggio 
della Logica del Primo Ordine nella sua natura grammaticale. 

A livello proposizionale, si fissa $L$ come insieme infinito di lettere 
proposizionali, e si conclude il tutto, poiché da questo insieme si arriva 
direttamente a costruire ogni formula ammissibile nella Logica Proposizionale.

Innanzitutto, nella Logica del Primo Ordine vi sono diversi linguaggi detti 
\textbf{linguaggi elementari}; la situazione è simile alla situazione generica 
dei linguaggi formali, anche se tecnicamente $L$ sarebbe un alfabeto per e $F_L$ 
sarebbe il linguaggio formale, mentre in logica $L$ stesso è chiamato Linguaggio. 

Allo stesso modo, nella logica del Prim'Ordine ci sono dei linguaggi elementari, 
che specificano gli ``ingredienti'' per costruire le formule della Logica dei 
Predicati, le quali sarebbero il \textit{vero} linguaggio nel senso formale. 
In altri termini, in Logica dei Predicati 
il termine \textit{linguaggio elementare} indica l'insieme di elementi che andranno 
a costruire l'insieme delle formule. 

\begin{defi}[Linguaggio Elementare]
I linguaggi elementari sono definiti come $L = (\mathcal{P}, F, \alpha, \beta)$  dove $P$ è un insieme di 
simboli, detti \textbf{predicati} (o simboli di predicato), che deve 
essere diverso dall'insieme vuoto $\mathcal{P} \neq \emptyset$; $F$ è l'insieme 
di simboli detti \textbf{di funzione}, disgiunto da $\mathcal{P}$: $\mathcal{P} \cap F = \emptyset$; 
$\alpha$ è una funzione $P \rightarrow \mathbb{N}$ assegna l'arità a ogni 
$\mathcal{P} \in P$ e, infine, $\beta$ è  l'arità di ogni $f \in F$.
\end{defi}

Ogni elemento del linguaggio varia in base al linguaggio stesso, tuttavia 
vi sono ``ingredienti'' intrinsecamente comuni a tutti i linguaggi elementari:  
l'insieme $V$ (o $Var$) infinito di simboli detti \textbf{variabili individuali}, 
chiamate così perché la loro interpretazione sarà quella di un \textit{individuo 
generico} dell'universo del discorso
e l'insieme dei connettivi $\land, \lor, \neg, \rightarrow, \bot, \top, \iff, \cdots$; 
differentemente dalla logica proposizionale, i linguaggi elementari contengono 
anche i \textbf{quantificatori} $\forall, \exists$. 

Il linguaggio $F_L$ delle formule sul linguaggio elementare $L$ sarà definito 
a partire dai simboli in $L$ e dai possibili connettivi. 

\begin{oss}
        Esiste un predicato speciale nella logica del primo ordine 
        che a livello sintattico è uguale agli altri, ma la sua interpretazione 
        è fissata. Se $\mathcal{P}$ contiene il simbolo $'='$, la sua arità 
        sarà fissata $\alpha('=') = 2$ e $L = (\mathcal{P}, F, \alpha, \beta)$ 
        è detto \textit{linguaggio con identità}.
\end{oss}

\begin{defi}[Formule]
La definizione di \textbf{Formula di} $L$ ($F \in F_L$) è data per strati 
e induttivamente: il primo strato è quello dei \textit{termini}, 
i quali \textbf{non sono formule} ma si \textit{usano} per costruire formule.

\begin{defi}[$L$-termini]
        Sia $L = (\mathcal{P}, F, \alpha, \beta)$. 

        Allora l'insieme $T_L$ degli $L$-termini è definito come segue: 

        \paragraph{Base} 
        \begin{itemize}
                \item Ogni $x \in Var(L)$ è  un termine. 
                \item Per ogni $c \in F$ tale che $\beta(c) = 0$, $c$ è un termine, 
        detto \textbf{costante}. 
        \end{itemize}

        \paragraph{Passo Induttivo} Se $f \in F$ è tale che $\beta(f) = n$ e 
        $t_1, t_2, \cdots, t_n \in T_L$ (sono termini già costruiti), 
        allora $f(t_1, t_2, \cdots, t_n)$ è un $L$-termine. 
        
        Nient'altro è un $L$-termine.
\end{defi}

        L'insieme $F_L$ delle $L$-Formule è definito come segue. 

        \paragraph{Base} Se $p \in \mathcal{P}$ e $\alpha(p)=n$ e $t_1, \cdots, t_n \in T_L$
        allora $P(t_1, \cdots, t_n)$ è una $L$-Formula detta 
        \textbf{formula atomica}. Le formule atomiche sono il corrispettivo 

        \paragraph{Passo Induttivo} Se $A,B \in F_L$ allora anche 
        $A \land B$, $A \lor B$, $\neg A$, $A\rightarrow B$ sono $L$-Formule.
        Se $A \in F_L$ e $x \in V$ allora anche $\forall x A$ e $\exists x A$
        sono $L$-Formule. Se in $A$ non occorre $x$, $\forall x A$ 
        e $\exists x A$ sono comunque formule, come anche 
        $\forall x (\forall x A)$ e $\exists x (\forall x A)$. 

        Nient'altro è una $L$-Formula.
\end{defi}

\paragraph{Esercizio} Dare una nozione di \textit{certificato} per 
$L$-termini e per $L$-Formule analoga alla $L$-costruzione in logica proposizionale. 

\subsection{Terminologia}
Introduciamo alcune nozioni terminologiche per potersi riferire alla sintassi 
della Logica dei Predicati.
Si userà liberamente la scrittura semplificata di formule omettendo coppie di parentesi
quando questo non causa ambiguità, ossia invece di $(\forall x A)$ si scriverà $\forall x A$.

\begin{defi}[Termine Ground]
        Un termine è detto \textbf{chiuso} o \textbf{ground} se è costruito 
        senza utilizzare \textit{variabili}. 
\end{defi}

\begin{defi}[Variabile vincolata]
        Una occorrenza di una variabile $x$ in una Formula $A \in F_L$ è 
        detta \textbf{vincolata} se occorre all'interno di una sottoformula 
        di $A$ (ossia una formula che appare in ogni certificato della $L$-formula $A$) 
        del tipo $\forall x B$ o $\exists x B$. 
\end{defi}

\begin{defi}[Variabile libera]
Una variabile $x \in V$ occorre libera in $A \in F_L$ se e solo se qualche  
sua occorrenza è libera. Questa definizione permette di dare definizioni 
anche nelle situazioni come l'osservazione precedente.
\end{defi}

Per esempio, nella formula 
$$
A = (\forall x R(x,y)) \lor P(x)
$$
$x$ è sia vincolata che libera, mentre $y$ è libera. 
Nella formula 
$$
A' = \forall x (\forall y R(x,y)) \lor P(x)
$$

Sia $x$ che $y$ occorrono sempre vincolate. 

Questo ci permette di introdurre il concetto fondamentale sul quale lavoreremo: 
infatti, non ragioneremo più su \textit{formule} quando definiremo la semantica 
della Logica dei Predicati, ma si ragionerà su un particoalre tipo di formula: 
\begin{defi}[$L$-sentence o Enunciato]
        Un $L$-\textbf{enunciato} o $L$-\textbf{sentence}, detto anche formula 
        chiusa, è una $L$-formula senza occorrenze di variabili libere 
        (o senza variabili libere).
\end{defi}

\begin{defi}[Sostituzione]
        Con la notazione $A[t/x]$ con $A$ una formula, $x$ variabile 
        e $t$ un termine, si intende la formula ottenuta rimpiazzando 
        simultaneamente tutte e sole le occorrenze \textbf{libere} di $x$ 
        con $t$. 
\end{defi}

% TEX root = ../../main.tex
\chapter{Semantica della Logica del Primo Ordine}
\section{Semantica di Tarski}
La semantica esprime \textit{come e quando} un $L$-enunciato è vero (o falso) in una data 
circostanza. \`E necessario fissare e formalizzare la  nozione di \textit{circostanza}
o ``mondo possibile''. Nel livello Proposizionale, la circostanza è un 
\textit{assegnamento}. Una volta fatto ciò, sarà necessario formalizzare 
l'\textit{interpretazione} degli enunciati (e quindi dei termini) in ogni circostanza 
formalizzata. 

Un'idea per la formalizzazione del concetto di 
\textbf{circostanza}, dovuta al logico polacco Alfred Tarski, 
è che la \textit{verità} è una corrispondenza con lo stato di \textit{Fatto}. 

Per la \textbf{semantica tarskiana}, nel linguaggio naturale l'enunciato
``La neve è bianca'' è un enunciato vero se e solo se la neve è bianca. 
Pertanto, ci si può immaginare di matematizzare questo concetto 
tramite la teoria degli insiemi, affermando che esista un certo 
insieme chiamato \textit{Universo del discorso}, composto da vari individui,
tra i quali vi è un certo elemento che rappresenta la neve e, 
rispecchiando la nostra esperienza, si vuole che effettivamente sia 
vero che la neve sia bianca. Pertanto, ``si forza'' l'appartenenza 
della neve all'insieme degli oggetti del discorso che sono bianchi. 

In termini matematici, e per il concetto stesso della semantica tarskiana, 
è necessario considerare Universi in cui vi è la possibilità 
che la neve non sia bianca, ossia l'enunciato è falso. 

A priori, le opzioni \textit{esistono} tutte (la neve è bianca, la neve 
non è bianca), ma si può utilizzare la formalizzazione degli enunciati 
anche per modellizzare l'universo, ossia asserendo che il fatto 
che la neve sia bianca sia vera, e questo sappiamo già farlo 
con l'uso delle \textit{teorie}. 

Per esempio, rielaboriamo il solito sillogismo aristotelico: 
L'insieme di enunciati 
$$
\Gamma := \{\{\forall x (U(x) \rightarrow M(x))\}, \{U(s)\}\}
$$
benché non sia in FNC, è un \textit{teorema}, mentre 
$$
A := M(s)
$$
è la deduzione. Per ottenere un'infrastruttura sulla quale arrivare a compiere 
il calcolo deduttivo 
$$
\Gamma, \neg A \text{ è soddisfacibile?}
$$
è necessario ``concentrarsi'' sugli Universi in cui $\Gamma$ è vero. 
Si possono considerare diversi \textbf{Universi del discorso}, composti da vari 
\textit{individui}: un enunciato caratterizza alcuni  universi in cui 
un enunciato è vero a scapito di altri. 

Si consideri, ora,  un universo composto da altri universi: chiamando 
ciò che abbiamo considerato fino ad ora 
un \textit{universo} una $L$-Struttura, si costruisce l'insieme 
di tutte le $L$-Strutture; data una teoria $\Gamma$ si potrà identificare, 
nell'insieme di tutte le $L$-Strutture, tutte quelle che modellano $\Gamma$. 

Definiamo formalmente le $L$-Strutture:
\begin{defi}[$L$-Struttura]
        Sia dato un linguaggio elementare $L  = (\mathcal{P}, F, \alpha, \beta)$. 
        Una $L$-struttura è una coppia $\mathcal{A} = (A, I)$, dove 
        $A$ è un insieme detto ``universo del discorso'' o, semplicemente, 
        \textit{universo} (o dominio), che deve rispettare 
        $$
        A \neq \emptyset
        $$
        mentre $I$ è una funzione detta \textit{interpretazione}, definita 
        tale che per ogni simbolo di 
        predicato $P \in \mathcal{P}$ con arità $\alpha(P) = n$ si ha 
        $$
        I(P) \subseteq A^{n} 
        $$
        ossia $I(P)$ è un insieme di $n$-ple di elementi di $A$; inoltre, 
        per ogni $f \in F$, $\beta(f) = n$ si ha 
        $$
        I(f) : A^{n} \rightarrow A
        $$
        ossia $I(F)$ è una funzione dall'insieme delle $n$-ple di $A$ verso elementi 
        di $A$.
\end{defi}

La definizione di $L$-struttura data formulizza a livello del Primo ordine la 
nozione intuitiva di ``circostanza'' o ``mondo possibile''. A livello proposizionale, 
essa è formalizzata dalla nozione di assegnamento. 


\begin{oss}[Costanti]
        Se $c \in F$ e $\beta(c) = 0$ 
        $$
        I(c) : A^{0} \rightarrow A
        $$
        e $A^{0} = \{f: \emptyset  \rightarrow A\}$, la cui cardinalità è  
        $$ 
        |A^0| = |\{f:\emptyset\rightarrow A\}| = |\{\emptyset\}| = 1
        $$
        Pertanto 
        $$
        I(c): \{*\} \rightarrow A
        $$ 
(la notazione $\{*\}$
        rappresenta 
        l'insieme che contiene l'insieme vuoto), che equivale a 
        ``scegliere'' un elemento di $A$.
        Identifichiamo una funzione 
        $$
        g : \{*\} \rightarrow A 
        $$
        con l'elemento scelto da $g(*)$. In altre parole, le costanti sono funzioni
        zerarie.
\end{oss}

\begin{oss}[Predicati zerari]
        Se $p \in \mathcal{P}, \alpha(P) = 0$ è un simbolo di 
        predicato zerario, si ha 
        $$
        I(P) \subseteq A^0 ~~~~ I(P) \subseteq \{*\}
        $$
        I cui sottoinsiemi sono sé stesso $\{*\}$ e l'insieme vuoto $\emptyset$. 
        E quindi l'interpretazione di un predicato zerario svolge le funzioni 
        di una vecchia lettera proposizionale, in quanto l'interpretazione 
        di una proposizione può essere identificata tramite la sua funzione 
        caratteristica 
        $$
        \chi_P: A^n \rightarrow \{0,1\}
        $$
        definita come 
        $$
        \bar{a} \in A^n \in I(P) \iff \chi_P(\bar{a}) = 1
        $$
\end{oss}

\begin{oss}[Interpretazione fissa del predicato di Uguaglianza]
        Se $ = \in \mathcal{P}$ allora $L$ è un linguaggio con identità e 
        $\alpha(=) = 2$. L'interpretazione di ``='' è fissata: 
        se $\mathcal{A}=(A,I)$ è una $L$-struttura, allora 
        $$
        I(=) \subseteq A^2 
        $$
        è fissato e definito come 
        $$
        I(=):= \{(a,a) : a \in A\}
        $$
        e, in altre parole, $x = y$ è vero se $(I(x), I(y)) \in I(=)$.
\end{oss}

L'eccezione sull'interpretazione del predicato di uguaglianza, fissato per 
ogni linguaggio elementare che lo contiene, è giustificato dal fatto che con 
la potenza espressiva della Logica dei Predicati, si riesce a dire che una 
relazione binaria gode della proprietà riflessiva, simmetrica e transitiva.  

Ci si può anche spingere a dire che per ogni altro simbolo, sia di predicato 
che di funzione, quella relazione si comporta in maniera \textit{congruenziale}. 
Anche questo non è sufficiente per inchiodare l'identità, in quanto una congruenza 
su un insieme che non è l'identità rispetta gli stessi assiomi. 

In altre parole, o l'identià si codifica a forza, come accade ora nella Logica 
del Prim'Ordine, oppure se ne ha una versione molto indebolita, in quanto non si 
ha modo di distinguere con formule della Logica dei Predicati l'identità 
da una congruenza sullo stesso insieme.  

\subsection{Semantica degli enunciati in ogni $L$-struttura}
Abbiamo dato la nozione di $L$-Struttura ma non si è ancora detto 
come si interpretano gli enunciati, formalmente, cioè quando un 
enunciato è vero o meno in tale $L$-Struttura. Intuitivamente, quello che 
si vuole fare è quanto anticipato, ossia scegliere le $L$-Strutture che 
verificano una certa teoria. Tuttavia, questo deve essere eseguito 
seguendo una definizione rigorosa e sistematicamente.

Si procederà per strati, ossia definendo prima come interpretare 
i termini e successivamente formule ed enunciati, induttivamente. 
Si deve catturare l'idea intuitiva: ``la neve è bianca'' verrà 
formalizzato in qualcosa come $B(n)$, dove $B$ è un predicato 
unario e $n \in A$ è una costante (quindi esiste una funzione zeraria 
il cui risultato è $n$). Si vorrà formalizzare il tutto in modo tale da avere una 
qualche $L$ struttura tale che $I(n)$ sia contenuta in $I(B)$; quindi si avrà che 
$B(n)$ è vera in $\mathcal{A}$ se e solo se $I(n) \in I(B)$: 
$$
\mathcal{A} \in B(n) \iff I(n) \in I(B)
$$

Il passo più ostico nella definizione di interpretazione di termini 
e formule (ed enunciati) sarà nell'interpretazione dei quantificatori; 
per esempio, si supponga che si vuole stabilire se 
$$
\mathcal{A} \models \exists x A
$$
che significa: se esiste un $a \in A$ tale che $\mathcal{A} \models A[a/x]$
e  analogamente per l'altro quantificatore 
$$
\mathcal{A} \models \forall x A
$$
che è vero se e solo se per ogni $a \in A$ vale $\mathcal{A} \models A[a/x]$. 

Purtroppo $a$ non è un elemento sintattico, ma è un elemento semantico, 
ossia non è un termine e pertanto $A[a/x]$ non è definita.
Non c'è nessuna ragione per cui ogni elemento di una $L$-struttura 
debba avere, a priori, un nome: piuttosto, è vero il contrario. Se si vuole 
parlare ad esempio dei reali, si vorrebbe farlo con un linguaggio 
elementare $L$ che contiene solo finitamente molti simboli. 

\begin{defi}[Espansione di un $L$-Struttura]
        Dato un linguaggio $L$ e una $L$-struttura $\mathcal{A}(A,I)$, si definisce 
        una \textbf{espansione} di $L$ in un nuovo linguaggio $L_{\mathcal{A}}$
        come segue: 
        per ogni $a \in A$ arricchisci l'insieme delle costanti di $L$ con un 
        nuovo simbolo $\bar{a}$ (nome di $a$) che viene interpretato in $a$. 

        $$
        L_{\mathcal{A}} := (\mathcal{P}, F^{\mathcal{A}}, \alpha, \beta^{\mathcal{A}})
        $$
        con
        $$
        F^{\mathcal{A}} := F\cup \{\bar{a}: a \in A\}
        $$
        e 
        $$
        \begin{cases}
                \beta^{\mathcal{A}} = B & f \in F\\
                \beta^{\mathcal{A}}(\bar{a}) = 0 & a \in A\\
        \end{cases}
        $$
        E inoltre $I(\bar{a}) := a $ per ogni $a \in A$. 
\end{defi}

\begin{defi}[Interpretazione degli $L_{\mathcal{A}}$-termini ground]
        Dati $L = (\mathcal{P}, F, \alpha, \beta)$ e $\mathcal{A} = (A,I)$ 
        e la $L_{\mathcal{A}}$-struttura espansa, si definisce induttivamente: 
        \begin{itemize}
                \item{\textbf{Base}} ogni $c \in F$ tale che $\beta(c) = 0$ ha un'interpretazione $I(c)$
                        già definita, dalla definizione di $L$-struttura. Invece, 
                        $\bar{a} \in F^{A} = F \cup \{\bar{a} : a \in A\}$ 
                        ha come interpretazione $I(\bar{a}) = a \in A$. 
                \item{\textbf{Passo Induttivo}}
                        Sia $f \in F$ un simbolo di funzione con arità $\beta(f) = n$ e 
                        siano $t_1, t_2, \cdots, t_n$ $L_{\mathcal{A}}$-termini ground. 
                        Allora $I(F(t_1, \cdots, t_n)):= I(f)(I(t_1), \cdots, I(t_n))$. 
        \end{itemize}
\end{defi}

\begin{defi}[Interpretazione degli Enunciati]
        Sia data $\mathcal{A} = (A,I)$ una $L$-struttura; allora, si definisce 
        induttivamente la nozione $\mathcal{A} \models A$ per ogni 
        $L_{\mathcal{A}}$-enunciato $A$, ossia $A$ è vera in $\mathcal{A}$. 

        \paragraph{Base}
        Sia 
        $A = P(t_1, \cdots, t_n)$ per $P \in \mathcal{P}$ con $\alpha(P) = n$ 
        e $t_1, \cdots, t_n$ $L_{\mathcal{A}}$-termini ground. Allora 
        $$
        \mathcal{A} \models P(t_1, \cdots, t_n) \iff I(t_1), \cdots, I(t_n) \in I(P) \subseteq A^n
        $$
        \paragraph{Passo induttivo}
                \begin{itemize}
                \item $\mathcal{A} \models A_1 \land A_2 \iff \mathcal{A} \models A_1 \land \mathcal{A} \models A_2$
                \item $\mathcal{A} \models A_1 \lor A_2 \iff \mathcal{A} \models A_1 \lor \mathcal{A} \models A_2$
                \item $\mathcal{A} \models \neg A_1  \iff \mathcal{A} \nvdash A_1 $
                \item $\mathcal{A} \models A_1 \rightarrow A_2 \iff \mathcal{A} \models A_1 \land \mathcal{A} \models A_2$
                \item $\mathcal{A} \models\forall x A \iff \mathcal{A} \models A[\bar{a}/x] \text{ per ogni } a$
                \item $\mathcal{A} \models\exists x A \iff \mathcal{A} \models A[\bar{a}/x] \text{ per almeno un } a$
                \end{itemize}
                Si noti che $\bar{a}$ è un termine, mentre $x$ è una variabile 
                individuale, quindi $A$ diventa un $L_{\mathcal{A}}$-enunciato.
\end{defi}

Abbiamo quindi definito induttivamente $\mathcal{A}\models A$ per gli 
$L_{\mathcal{A}}$-enunciati $A$. A questo punto ci si può ``dimenticare'' degli 
$L_{\mathcal{A}}$-enunciati che non sono $L$-enunciati, 
ossia quelle formule in cui vi è almeno una variabile libera.  

\begin{defi}[Chiusura universale di una formula con variabili libere]
Se $A$ è una formula ben formata, ossia $A \in F_L$ ed eventualmente,
non un enunciato, quindi $FV(A) \neq \emptyset$, dove $FV$ è definito come l'insieme 
di variabili libere che appaiono in $A$, ci si può chiedere se 
$$ 
\mathcal{A} \models A
$$
Definito, quindi, 
$$
FV(A) = \{x_1, \cdots, x_n\} 
$$
si definisce la chiusura universale di $A$ 
$$
\forall[A] := \forall x_1 \forall x_2 \cdots \forall x_n A
$$
e si postula che
$$
\mathcal{A} \models A \iff \mathcal{A} \models \forall[A]
$$
\end{defi}
ossia $\mathcal{A}$ modella una formula ben formata $A$ se e solo se 
$\mathcal{A}$ rende vera la sua chiusura universale, che è un $L$-enunciato.

\begin{oss}
        Se $L$ è un linguaggio con identità (o con uguaglianza), ossia 
        $L = (\mathcal{P}, F, \alpha, \beta)$ e $= \in \mathcal{P}$, 
        $\alpha(=) = 2$, 
        allora, ricordando che in ogni $L$-struttura $\mathcal{A} = (A,I)$, 
        si deve avere per definizione $I(=) = \{(a,a) : a \in A\}$. 
        Pertanto $\mathcal{A} \models t_1 = t_2$ (che, più correttamente, 
        andrebbe scritto in forma postfissa ``$=(t_1, t_2)$'' se e solo se 
        $I(t_1) = I(t_2)$. 
\end{oss}

\subsection{Definizione alternativa di $\mathcal{A} \models A$} 
In questa definizione alternativa,  le variabili sono oggetti di interesse; 
tuttavia le due definizioni, quella su $L\rightarrow L_{\mathcal{A}}$ e quella
che andiamo a dare, sono equivalenti. 
\begin{defi}[Ambiente]
        Un \textbf{ambiente di interpretazione} in una $L$-struttura $\mathcal{A} = (A,I)$ è 
        una funzione 
        $$
        \mathfrak{v} : Var \rightarrow A
        $$
        ossia $\mathfrak{v}(x) \in A$ è un elemento di $A$.         
\end{defi}
\begin{defi}[Variante di ambiente]
        La \textbf{variante di ambiente} servirà a dare la semantica alle espressioni 
        quantificate ed è indicato $\mathfrak{v}[a/x](y)$ e si definisce 
        $$
        \mathfrak{v}[a/x](y) :=
        \begin{cases}
                \mathfrak{v}(y) & y \neq x \\
                a & y = x
        \end{cases}
        $$
\end{defi}
\begin{defi}[Interpretazione degli $L$-termini]
        L'interpretazione degli $L$-termini, anche aperti, per ogni 
        $x \in Var$ e $\mathfrak{v} : Var \rightarrow A$ è 
        definita come 
        $$
        I_{\mathfrak{v}}(x) := \mathfrak{v}(x)
        $$
        e si può ora passare a definire 
        $$
        I_{\mathfrak{v}}(F(t_1, \cdots, t_n)) := (I(F))(I_{\mathfrak{v}}(t_1), \cdots, I_{\mathfrak{v}}(t_n))
        $$
        Si dirà
        $$
        \mathcal{A}\models_{\mathfrak{v}} A 
        $$
        se $\mathcal{A}$ rende vera $A$ nell'ambiente $\mathfrak{v}: Var \rightarrow A$, 
        e si dirà 
        $$
        \mathcal{A} \models_{\mathfrak{v}} P(t_1, \cdots, t_2) \iff (I_{\mathfrak{v}}(t_1), \cdots, I_{\mathfrak{v}}(t_n)) \in I(P)
        $$
         e quindi 

         \begin{itemize}
                 \item  $\mathcal{A} \models_{\mathfrak{v}} A_1 \land A_2 \iff \mathcal{A} \models A_1 \land \mathcal{A} \models A_2$
                 \item $\mathcal{A} \models_{\mathfrak{v}}A_1 \lor A_2 \iff \mathcal{A} \models A_1 \lor \mathcal{A} \models A_2$
                 \item $\mathcal{A} \models_{\mathfrak{v}}  \neg A_1  \iff \mathcal{A} \nvDash A_1 $
                 \item $\mathcal{A} \models_{\mathfrak{v}} A_1 \rightarrow A_2 \iff \mathcal{A} \models A_1 \land \mathcal{A} \models A_2$
                 \item $\mathcal{A} \models_{\mathfrak{v}} \forall x A \iff \mathcal{A} \models_{\mathfrak{v}} A[a/x] \text{ per ogni } a$
                 \item $\mathcal{A} \models_{\mathfrak{v}} \exists x A \iff \mathcal{A} \models_{\mathfrak{v}} A[a/x] \text{ per almeno una  } a$
        \end{itemize}
        A questo punto si  può dire che $\mathcal{A} \models A$ se e solo se 
        per ogni $\mathfrak{v}: Var \rightarrow A$ si ha $\mathcal{A} \models A$. 
\end{defi}

\noindent
Si esprime, ora, un'idea intuitiva di come avvenga il processo di interpretazione 
di un $L$-enunciato. Nella prima definzione si ha il linguaggio $L$ e delle 
$L$-strutture che vengono associate. 
$$
L 
\leadsto
\begin{cases}
        \mathcal{A}  \leadsto L_{\mathcal{A}} \\
        \mathcal{B}  \leadsto L_{\mathcal{B}} \\
        \mathcal{C}  \leadsto L_{\mathcal{C}} \\
        \cdots
\end{cases}
$$
Per ogni $L$-struttura si crea l'espansione associata. 
Nella seconda definizione si ha nuovamente il linguaggio $L$ e delle $L$-strutture, 
tuttavia esse non si espandono ma vi sono un numero di ambienti da considerare: 

$$
L \leadsto 
\begin{cases}
        \mathcal{A} \begin{cases}
                \mathcal{A}, \mathfrak{v}_1 \\
                \mathcal{A}, \mathfrak{v}_2 \\
                \mathcal{A}, \mathfrak{v}_3 \\
                \cdots \\
        \end{cases} \\
        \mathcal{B} \begin{cases}
                \mathcal{B}, \mathfrak{v}_1\\
                \mathcal{B}, \mathfrak{v}_2\\
                \mathcal{B}, \mathfrak{v}_3\\
                \cdots                     
        \end{cases} \\
        \cdots
\end{cases}
$$

\section{Terminologia e Nozioni}

\begin{defi}[$L$-Teoria]
        Una $L$-Teoria è un insieme di $L$-enunciati. 
        Se $\Gamma$ è una $L$-Teoria e $\mathcal{A}$ è una $L$-struttura, 
        si dice che $\mathcal{A}$ è modello di $\Gamma$ e si scrive 
        $\mathcal{A} \models \Gamma$ se e solo se $\mathcal{A} \models \gamma$ 
        per ogni $\gamma \in \Gamma$. 

        La $L$-teoria $\Gamma$ è soddisfacibile e solo se esiste almeno un 
        $\mathcal{A}$ tale che $\mathcal{A} \models \Gamma$, 
        altrimenti $\Gamma$ è insoddisfacibile.

        Sia $\Gamma$ una $L$-teoria e $A$ un $L$-enunciato. 
        Allora $A$ è conseguenza logica di $\Gamma$ e si scrive $\Gamma \models A$ 
        se e solo se $A$ è vera in ogni modello di $\Gamma$, ossia 
        per ogni $L$-struttura $\mathcal{A}$, se $\mathcal{A} \models \Gamma$, 
        allora $\mathcal{A} \models A$. 
\end{defi}

\begin{defi}[Verità Logica]
        Un $L$-enunciato $A$ è detto \textbf{verità logica} se e solo se 
        per ogni $L$-struttura $\mathcal{A}$, $\mathcal{A} \models A$. 
\end{defi}
Il concetto di Verità Logica è analogo al concetto di tautologia nella 
Logica Proposizionale. 
Un esempio di verità logica è $\forall x (x = x)$. 

\noindent
Lo scopo della nostra indagine sarà, d'ora in poi, cercare di stabilire se
$$
\Gamma \models A
$$
dati $\Gamma$  una $L$-teoria e $A$ un $L$-enunciato. 
Se $\Gamma$ è finito, ossia $\Gamma = \{\gamma_1, \cdots, \gamma_n\}$, 
varrà nuovamente il fatto 
$$
\Gamma \models A \iff \gamma_1 \land \cdots \land \gamma_n \land  \{\neg A\} \text{ insodd.}
$$
ossia ci si riduce all'analisi di una singola formula, 
mentre invece se $\Gamma$ è infinito, possiamo solo chiederci se
$$
\Gamma \models A \iff \Gamma \cup \{\neg A\} \text{ insodd.}
$$
Anche nel caso in cui $\Gamma$ è finito, ci saranno casi in cui il processo 
di deduzione sarà solo \textit{semidecidibile}, mentre nella Logica Proposizionale 
il problema di soddisfacibilità era decidibile. 
\subsection{Esempi di $L$-Teorie} 

\subsubsection{Congruenze su un $L$-Linguaggio $L$}
Data una relazione $R$, si vorrebbe modellarla come congruenza.
$$
\Gamma := 
\begin{cases}
        \forall x ~~ R(x,x) \text{ riflessiva } \\
        \forall x \forall y ~~ R(x,y) \rightarrow R(y,x)  \text{ simmetrica } \\
        \forall x \forall y \forall z ~~ R(x,y) \land R(y,z) \rightarrow R(x,z)  \text{ transitiva }\\

        \forall x_1 \cdots \forall x_n \forall y_1 \cdots \forall y_n 
        \begin{cases}
        (R(x_1,y_1) \land \cdots \land R(x_n,y_n)) \rightarrow R(f(x_1, \cdots, x_n), f(y_1, \cdots y_n)) \\
        (R(x_1, y_1) \land \cdots R(x_n, y_n)) \rightarrow (P(x_1, \cdots x_n) \rightarrow P(y_1, \cdots, y_n))
        \end{cases}

\end{cases}
$$
I primi tre $L$-enunciati (assiomi di $\Gamma$) assiomatizzano una 
relazione d'equivalenza, ossia
$$
\text{ se }\mathcal{A} \models \Gamma \text{ allora } I(R) \text{ è una relazione d'equivalenza}
$$
Il quarto $L$-enunciato assiomatizza la congruenza $R$ rispetto ad
una generica $f \in F$ con $\beta(f) = n$, mentre il quinto fa 
la stessa cosa rispetto un generico predicato $P \in \mathcal{P}$. 

\subsubsection{Relazione d'ordine (parziale)}
$$
        \Gamma_{po} := 
        \begin{cases}
                \forall x R(x,x) & \text{ riflessiva }\\
                \forall xy (R(x,y) \land R(y,x)) \rightarrow =(x,y) & \text{ simmetrica }\\
                \forall x \forall y \forall z (R(x,y) \land R(y,z)) \rightarrow R(x,z) & \text{ transitiva }\\
      
        \end{cases}
$$
Si noti l'utilizzo del predicato $=$ d'uguaglianza, pertanto il Linguaggio 
è dotato di tale predicato. 
Un esempio, l'insieme dei naturali con la relazione di minore o uguale 
modellano $\Gamma$: 
$$
(\mathbb{N}, \leq) \models \Gamma
$$
(la notazione utilizzata significa che $I(R) = \leq$)
così come i naturali con l'ordine di divisibilità: 
$$
(\mathbb{N}, |) \models \Gamma
$$


\paragraph{Relazione d'ordine (totale)}
Esiste un modo per costruire una $L$-Teoria che ``allarga'' $\Gamma$ che è modello 
di uno e non dell'altra? Effettivamente, il primo è un ordine totale, mentre 
il secondo non lo è, ossia vi sono elementi tali che né $m | n$ né $n |m $; se 
si riesce ad esprimere la totalità con un enunciato, si riesce a discernere 
uno dall'altro: 
\begin{align*}
        \Gamma_{to} := \Gamma_{po} \cup \{\forall x \forall y (R(x,y) \lor R(y,x))\}
\end{align*}

Aggiungendo nuovi assiomi si possono caratterizzare altri tipi di ordine: 
ad esempio (esercizio), si può caratterizzare un ordine che abbia un elemento 
minimo. 

\subsubsection{Teoria dei gruppi}
Un gruppo è una struttura algebrica. 
Un gruppo è un modello dell'insieme di assiomi $\Gamma_G$ sul linguaggio 
$L = (\mathcal{P}, F, \alpha, \beta)$. 
$$
\Gamma_G := 
        \begin{cases}
                \forall x \forall y \forall z  ~~ ((x * y)*z) = (x*(y*z))  & \text{ associatività}\\
                \forall x (x * e) = x \land (e * x) = x  & \text{ elemento neutro}\\
                \forall x (x * x^{-1}) = e \land (x^{-1} * x = e) & \text{ invertibilità}
        \end{cases}
$$
con $\mathcal{P} = \{ = \}$, $F = \{*, e, ()^{-1}\}$ e relative 
arità. 
$(G, \cdot, ^{-1}, 0) \models \Gamma_G$ se e solo se $G$ è un gruppo; in 
altri termini un gruppo è una struttura con un'operazione associativa in cui 
ogni elemento è invertibile. 
Per esempio, $(\mathbb{Z}, +, -, 0) \models \Gamma_G$ e 
$(\mathbb{Q}\setminus{0}, \cdot, ^{-1}, 1) \models \Gamma_G$. 

\paragraph{Formulazione alternativa}
Una seconda formulazione è la seguente
$$
        \Gamma_{G_2} := 
                \begin{cases}
                        \forall x \forall y \forall z  ~~ ((x * y)*z) = (x*(y*z)) & \text{ associatività}\\
                        \forall x (x * e) = x \land (e * x) = x & \text{ elemento neutro } \\
                        \forall x \exists y (x * y) = e \land (y * x) = e & \text{ esistenza inverso} \\
                \end{cases}
$$
Questa $L$-Teoria rimuove la specifica dell'inverso. Se si ottiene un gruppo 
che modella l'altra formulazione, esso andrà bene anche per questo, ma non è detto 
il viceversa. 
Ora, infatti, non è più necessario specificare l'operazione inverso, 
e si può quindi affermare $(\mathbb{Z}, +, 0) \models \Gamma_{G_2}$, $(Q, +, 0) \models \Gamma_{G_2}$, 
eccetera. Come anticipato, questo vale anche per i gruppi esplicitati prima: 
$(\mathbb{Q}\setminus{0}, \cdot, ^{-1}, 1) \models \Gamma_{G_2}$ e così via, 
anche se l'inverso non viene utilizzato come simbolo esplicito. 
Potremmo quindi utilizzare, al posto del simbolo dell'inverso, qualsiasi cosa e 
rimarrà comunque un modello di $\Gamma_{G_2}$: 
$$
(\mathbb{Z}, +, +1, 0) \models \Gamma_{G_2}
$$
Ma ovviamente non è vero il contrario, ossia 
$$
(\mathbb{Z}, +, +1, 0) \nvDash \Gamma_G
$$
in questo senso, abbastanza sottile, le due formulazioni non sono equivalenti. 
\subsubsection{Tipi di dato: stack}
Definiamo il linguaggio $L= (\mathcal{P}, F, \alpha, \beta)$ 
con 
$$
\mathcal{P} = \{Stack, Elem, =\}
$$
e 
$$
F = \{push, pop, top, nil\}
$$
$$
        \Gamma_{Stack} := 
        \begin{cases}
                \forall x (Stack(x) \lor Elem(x)) \\
                \forall x (\neg Stack(x) \lor \neg Elem(x)) \\
                \forall x \forall y ((Stack(x) \land Elem(y)) \rightarrow top(push(x,y)) = y) \\
                \forall x \forall y ((Stack(x) \land Elem(y)) \rightarrow pop(push(x,y)) = x) \\
                \forall x (Stack(x) \rightarrow push(pop(x),top(x)) =x) \\
        \end{cases}
$$

\subsubsection{Aritmetica di Peano}

Sia $L_{P_A} = ( \mathcal{P}, F, \alpha, \beta)$, con 
$\mathcal{P} = \{=\}$, $\alpha(=) = 2$, $F =  \{ + , *, s, 0,\}$ 
$\beta(+) = \beta(*) = 2$, $\beta(s) = 1$, $\beta(0) =0$. 

L'aritmetica di Peano afferma: 
\begin{itemize}
        \item $\forall x \neg (=(0,s(x)))$, ossia nessun numero ha come successore $0$. 
        \item $\forall x, \forall y (s(x) = s(y)) \rightarrow (x = y)$
        \item $\forall x (x + 0 = x)$
        \item $\forall x \forall y (x + s(y)) = s(x+y)$
        \item $\forall x (x * 0 = 0)$
        \item $\forall x \forall y (x * s(y)) = (x + (x*y))$
        \item $(P[0/x] \land \forall x (P[x/x] \rightarrow P[s(x)/x])) \rightarrow \forall x P[x/x]$ 
\end{itemize}
L'ultimo postulato è una codifica dell'induzione sui numeri naturali. 

\chapter{Complessità Computazionale e Deduzione Automatica}

In questo momento, per decidere se una formula $F \in F_L$ che menziona $n$ lettere 
proposizionali diverse sia soddisfacibile o meno, l'algoritmo che conosciamo
computa la sua tabella di verità, analizzando quindi $2^n$ possibilità, ossia 
un numero esponenziale. 

Cominciamo definendo cosa sia un problema di 
decisione:
\begin{defi}[Problema di Decisione]
 Dato un alfabeto finito $\Sigma$ e un sottoinsieme non necessariemente 
 finito di stringhe finite  
  $$
 L \subseteq \Sigma^* 
 $$
 il problema di decisione consiste nel affermare se una precisa 
 stringa finita appartenga o meno a $L$, che viene chiamato Linguaggio, 
dove $\Sigma^*$ è l'insieme di tutte le stringhe di lunghezza finita, 
chiamate anche \textit{parole}, costruite 
coi simboli di $\Sigma$. 
\end{defi}

Ad esempio, fissato 
$$
\Sigma = \{\land, \lor, \neg, \rightarrow, (, ), p, |\}
$$
si possono definire alcuni problemi, per esempio
$$
F_L \subseteq \Sigma ^*
$$
dove il problema risiede nel capire se una certa formula di lunghezza finita
risiede in $F_L$ oppure no. 
Vi sono dei problemi principalmente di natura sintattica, come appunto $F_L$ 
ma anche $NNF$, $CNF$ e $DNF$. Vi sono problemi in cui la natura semantica 
è decisiva, per esempio $SAT$, dove 
$$
w \in \Sigma^* : w\in SAT \iff w \in F_L \land w \text{ è soddisfacibile}
$$
oppure $TAUTO$, dove una stringa finita appartiene a $TAUTO$ se e solo se 
è una formula ed è una tautologia, ossia
$$
w \in \Sigma^* : w \in TAUTO \iff w \in F_L \land \models w
$$
e analogamente $UNSAT$, il problema di decidere se una certa formula è
insoddisfacibile.

\section{Complessità Computazionale}
\subsection{Efficienza}
Alcuni dei problemi elencati precedentemente possono essere risolti (decisi) 
in maniera efficiente, ossia data una \textit{parola} $w$ decidere se 
appartiene ad un certo linguaggio: esempi di soluzioni efficienti sono 
quelle che riguardano tutti i problemi sintattici, risolvibili in tempo 
al massimo quadratico e in realta anche lineare con alcune tecniche di parsing. 

I problemi che riguardano la semantica, invece, sono tipicamente più complessi
da risolvere e infatti, per quanto conosciamo fino ad ora, si possono risolvere 
solo (nel caso peggiore) con un analisi dell'intera tabella di verità della 
formula, quindi con un numero esponenziale di calcoli. \`E importante rimarcare 
nuovamente che benché sussista questa difficoltà, verificare che 
un singolo assegnamento verifichi la formula è invece un calcolo semplice 
che richiede un tempo decisamente più contenuto rispetto al problema di 
decidibilità.

Quindi, vi sono dei problemi decidibili efficientemente, 
dei problemi decidibili molto difficilmente e verificabili facilmente 
e dei problemi decidibili molto difficilmente e verificabili difficilmente; 
da qui parte una gerarchia di classi di complessità infinita, ma a noi interesseranno 
solo le prime due. 
 
Per essere più precisi, si fissa un modello astratto di computazione, che fornisca 
la nozione di \textit{passo elementare}, in modo da poter ragionare precisamente 
sull'efficienza dei problemi. 
Qualunque modello di computazione che costituisca un modello realistico 
di calcolatore, con associata una nozione di passo elementare per costruire 
algoritmi, classifica i problemi nello stesso modo. 

\begin{oss}[Tesi di Church-Turing]
  Ogni modello ragionevole di computazione è equivalente.
\end{oss}

Uno dei modelli di computazione è quello delle Macchine di Turing (MdT): un'idea 
della Macchina di Turing è immaginarla come una macchina che lavora su un 
nastro infinito in lettura e scrittura, mantenendo uno stato e operando su 
una singola porzione di nasto in lettura utilizzando un programma per 
muoversi tra gli stati e scrivere sul nastro. 

Si può passare ora alla definizione formale delle classi dei problemi: 
\begin{defi}[Classe $\mathbb{P}$]
  La classe dei problemi ``efficientemente decidibili'' è definita, con una 
  certa ideologia sottesa, come tutti quei problemi che possono essere 
  risolti in tempo polinomiale, ossia esiste una certa Macchina di Turing T 
  e un polinomio $p: \mathbb{N}\rightarrow \mathbb{N}$ per i quali per ogni 
  $w \in \Sigma^*$  la computazione della MdT sull'input $w$, 
  denotato $T(w)$ termina entro $p(||w||)$ passi e per ogni $w \in L$ si 
  ha che $T(w)$ accetta il problema e per ogni $w \notin L$  si ha 
  che $T(w)$ non accetta, ossia risolve il problema.
\end{defi}

\begin{defi}[Classe $\mathbb{NP}$]
  La classe dei problemi ``verificabili efficientemente'' è definita come 
  tutti quei problemi tali per cui esiste una Macchina di Turing deterministica 
  e due polinomi $p,q:\mathbb{N}\rightarrow \mathbb{N}$ tali che per ogni 
  $w \in \Sigma^*$ e ogni $z \in \Gamma^*$, ossia un \textbf{certificato}, 
  che si può immaginare prodotto oracolarmente, si ha 
  che $T(w,z)$ termina entro $p(||w||)$ passi e si ha che per ogni 
  $w \in L$ esiste $z \in \Gamma^*$ tale che $||z|| \leq q(||w||)$ (ossia 
  il certificato è sufficientemente corto) e $T(w,z)$ accetta e per ogni 
  $w \notin L$ si ha che per ogni possibile $z \in \Gamma^*$ sufficientemente corto
  $T(w,z)$ rifiuta, ossia ``valida'' il certificato.
\end{defi}

\begin{defi}[Riducibilità]
        Un problema $L_1 \subseteq \Sigma^*$ è \textbf{riducibile in tempo polinomiale} 
        a un altro problema $L_2 \subseteq \Gamma^*$ se e solo se esistono
        una Macchina di Turing $T_{{L_1}, {L_2}}$ e un polinomio
        $p: \mathbb{N} \rightarrow \mathbb{N}$ tale che per ogni $w \in L_1$
        $T(w)$ trasfroma $w$ in $w' \in \Gamma^*$ in un numero di passi 
        minore o uguale a $p(||w||)$. 
\end{defi}

Per indicare la relazione di riducibilità tra due problemi si indica la notazione 
$L_1 \preceq_p L_2$ per indicare che il primo problema è riducibile polinomialmente 
al secondo; la nozione di riducibilità è utile poiché rende possibile risolvere 
istanze del problema $L_1$ ``riscrivendole'' come se fossero istanze di $L_2$
(chiaramente questa utilità si verifica quando è più facile risolvere $L_2$ di 
$L_1$). 
Un esempio di riducibilità polinomiale è 
$$
TAUTO \preceq_p UNSAT
$$
poiché la trasformazione $w \rightarrow \neg w$ è semplice e si sa che 
$w$ è tautologica se e solo se $\neg w$ è insoddisfacibile.

\begin{defi}[Problemi $\mathbb{NP}$-completi]
        Un problema appartenente alla classe $\mathbb{NP}_c$ è un problema 
        $L \subseteq \Sigma^*$ se e solo se 
        \begin{itemize}
                \item $L \in \mathbb{NP}$
                \item ogni $L' \in \mathbb{NP}$ è tale che $L' \preceq_p L$ 
        \end{itemize}
        La seconda proprietà si chiama $\mathbb{NP}$-hardness. 
\end{defi}
Come corollario della definizione dei problemi $\mathbb{NP}_c$, si ha che risolvendo 
polinomialmente un problema di tale classe si dimostra che 
$$
\mathbb{P} = \mathbb{NP}
$$

\begin{teo}[di Cook-Levin]
        $SAT \in \mathbb{NP}$-completo. $CNFSAT \in \mathbb{NP}$-completo. 
\end{teo}

Per dimostrare che $CNFSAT \preceq SAT$, basta realizzare che una formula 
in CNF è comunque ancora una formula e pertanto la trasformazione è ovviamente 
polinomiale, essendo la funzione identità. Dimostrare che $SAT \preceq CNFSAT$ 
è tutt'altra questione benché sia ovvio dal Teorema di Cook; tuttavia, questa 
proprietà che invece non vale per le DNF, ossia $SAT \npreceq DNFSAT$, è una 
buona motivazione per concentrarsi sulle CNF. 
Si mostra ora, esplicitamente, che SAT si riduce polinomialmente a CNFSAT, 
conservando non l'equivalenza logica ma la relazione di equisoddisfacibilità. 

\subsection{Equisoddisfacibilità}
\begin{defi}[Equisoddisfacibilità]
        Siano $A, B \in F_L$. Le due formule sono equisoddisfacibili se e solo 
        se $A$ è soddisfacibile se e solo se $B$ è soddisfacibile, ossia 
        se $A$ e $B$ sono entrambe soddisfsacibili o entrambe insoddisfacibili. 
\end{defi}

\subsubsection{Esempi}
\paragraph{1}
Date le due formule $\neg (A \land B)$ e $\neg A \lor \neg B$, è noto 
che sono equivalenti grazie alle leggi di De Morgan e sono, di conseguenza, 
anche equisoddisfacibili.
\paragraph{2} $\neg (A \land B)$ e $(\neg A \land \neg B)$ non sono equivalenti, 
infatti l'assegnamento $A=1$ e $B=1$ soddisfa solo una delle due, tuttavia 
sono equisoddisfacibili.
\paragraph{3} $A \land \neg A$ e $\neg A \neg B$ non sono né equivalenti né
equisoddisfacibili.

Da questi esempi si può chiaramente notare che l'equivalenza implica l'equisoddisfacibilità 
mentre il contrario non è affatto verificato.

\paragraph{4}
L'equisoddisfacibilità è un tipo di relazione d'equivalenza, in quanto 
valgono le proprietà di simmetria, transitività e riflessività.
\paragraph{5} L'equisoddisfacibilità è una congruenza rispetto ai 
connettivi, pensati come operazioni? Non lo è. Sia, per esempio 
$\models A$, quindi $A \in [\top]$ e $B$ soddisfacibile ma non tautologica; 
sono equisoddisfacibili, in quanto sono entrambi chiaramente soddisfacibili. 
Se fosse una congruenza, rispetto alle varie operazioni l'equisoddisfacibilità 
dovrebbe essere mantenuta, invece $\neg A$ è $\bot$, mentre $\neg B$ è 
ancora soddisfacibile, pertanto non sono equisoddisfacibili.

\paragraph{6} Le classi di equivalenza delle formule, per esempio su 
due variabili, sono costruite nelle forme come $F_L^{(2)}/\equiv$, ossia 
$16$ diverse classi ($2^{2^2}$). Per l'equisoddisfacibilità vi saranno unicamente 
due classi, ossia l'insieme delle formule insoddisfacibili ($[\bot]$) e 
tutte le rimanenti, ossia tutte quelle almeno soddisfacibili.  

\subsection{Riduzione $SAT \preceq CNFSAT$}
Sia $A \in F_L$ tale che $A$ sia in Negation Normal Form, ossia $A \in NNF$. 
Se $A \notin CNF$, allora contiene almeno una sottoformula del tipo 
$C \lor (D_1 \land D_2)$ oppure $(D_1 \land D_2) \lor C$. Questo si dimostra 
semplicemente confermando che $A \in NNF$ ma non $A \in CNF$, quindi la parte 
che non rispetta la CNF deve essere una disgiunzione di congiunzioni. Trattiamo, 
d'ora in poi, solo il primo dei due casi, ossia quello in cui nella formula 
appare la sottoformula $C \lor (D_1 \land D_2)$, senza perdita di generalità.
Sia data $A \in NNF$ e sia $B = C \lor (D_1 \land D_2)$ una sua violazione; 
per ogni violazione si introduce una nuova lettera proposizionale $a \in L$ che 
ancora non è presente in $A$. 
Si definisce ora 
$$
B' := B[a/D_1 \land D_2] \land (\neg a \lor D_1) \land (\neg a \lor D_2)
$$
dove 
$$
B'' := B[a/D_1\land D_2]
$$
è la formula ottenuta rimpiazzando ogni occorrenza di $D_1 \land D_2$ con $a$ 
in $B$. 
Come prima osservazione, si ha che $(\neg a \lor D_1) \land (\neg a \lor D_2)$ 
è equivalente a $a \rightarrow (D_1 \land D_2)$. 
Si mostra che $B'$ e $B$ sono equisoddisfacibili, ma in genere non 
sono equivalenti: 
\begin{proof}[$B \in SAT \rightarrow B' \in SAT$]
        Per ipotesi, dato che $B$ è soddisfacibile, sia $\mathfrak{v}:L \rightarrow \{0,1\}$
        tale che $\mathfrak{v}(B) = 1$, ossia $\mathfrak{v} \models B$. 
        Si definisce 
        $$
        \mathfrak{v}_a: L \rightarrow \{0,1\} = 
        \begin{cases}
                \mathfrak{v}_a(p) = \mathfrak{v}(p) & \forall p \in L, p \neq a \\
                \mathfrak{v}_a(a) = \mathfrak{v}(D_1 \land D_2) 
        \end{cases} 
        $$
        L'assegnamento $\mathfrak{v}(a)$ è ben definito, nel senso che non 
        da alla stessa lettera proposizionale due assegnamenti diversi.
        Si nota che $\mathfrak{v}_a \models a \rightarrow (D_1 \land D_2)$, 
        dato che per definizione $\mathfrak{v}_a(a) = \mathfrak{v}(D_1 \land D_2)$
        e per interpretazione dell'implicazione $0 \rightarrow 0 = 1$ e 
        $1 \rightarrow 1 = 1$; dunque, $\mathfrak{v}_a(B') = \mathfrak{v}_a(B'')$, 
        in quanto la ``coda'' di $B'$, ossia $(\neg a \lor D_1) \land (\neg a \lor D_2)$, 
        ha un assegnamento uguale a $1$, ossia 
        $\mathfrak{v}_a(B') = \mathfrak{v}_a(B'') \land 1$.

        $B$ si può riscrivere come 
        $$
        B = B''[D_1 \land D_2/a]
        $$
        dato che $a$ non appare in $B$, e quindi 
        $$
        B'' = (B[a/D_1 \land D_2])[a/a]
        $$
        e grazie al Lemma di Sostituzione si può affermare che i due valori 
        di verità di $B$ e $B''$ sono uguali in quanto i valori di verità di $a$ 
        e $D_1 \land D_2$ sono uguali. Ora si può scrivere 
        \begin{align*}
                1 &= \mathfrak{v}(B) \text{ per ipotesi } \\
                  &= \mathfrak{v}_a(B) \text { poiché } a \text{ non occore in } B \\
                  &= \mathfrak{v}_a(B'') \text{ per il lemma di sostituzione}  \\
                  &= \mathfrak{v}_a(B') 
        \end{align*}
        ossia l'assegnamento che soddisfa $B$ soddisfa anche $B'$, ossia 
        sono equisoddisfacibili. 
\end{proof}
\begin{proof}[$B' \in SAT \rightarrow B \in SAT$]
        La sequenza di derivazioni utilizzata per la dimostrazione precedente 
        è ancora buona, tuttavia c'è un vincolo all'inizio, 
        ossia l'assunzione che se $\mathfrak{v}(B) = 1$ allora $\mathfrak{v}_a(B) = 1$; 
        per definizione 
        $$
        \mathfrak{v}_a: L \rightarrow \{0,1\} = 
        \begin{cases}
                \mathfrak{v}_a(p) = \mathfrak{v}(p) & \forall p \in L, p \neq a \\
                \mathfrak{v}_a(a) = \mathfrak{v}(D_1 \land D_2) 
        \end{cases} 
        $$
        che non è la stessa cosa di dire che $B'$ è soddisfacibile, 
        ossia $\mathfrak{v}_a \models B'$, poiché $B'$ potrebbe essere soddisfatto 
        da assegnamenti che non sono nella forma $\mathfrak{v}_a$. Il ragionamento 
        è un po' più complicato. 

        Si supponga che $B'$ sia soddisfacibile da un assegnamento della forma 
        $\mathfrak{v}_a$, ossia tale che $\mathfrak{v}_a(a) = \mathfrak{v}(D_1 \land D_2)$; 
        allora, in questo caso si può tranquillamente ribaltare la catena di uguaglianze 
        precedente: 
        \begin{align*}
                1 &= \mathfrak{v}_a(B') \\
                  &= \mathfrak{v}_a(B'') \text{ per il lemma di sostituzione}  \\
                  &= \mathfrak{v}_a(B) \text { poiché } a \text{ non occore in } B 
        \end{align*}
        dunque $\mathfrak{v}_a \models B$. 
        Si supponga, ora, che $B'$ sia soddisfatto solo da assegnamenti 
        $\mathfrak{w}$ che non sono della forma $\mathfrak{v}_a$, 
        ossia $\mathfrak{w}(a) \neq \mathfrak{v}(D_1 \land D_2)$. 
        Vi sono allora due casi: nel primo $\mathfrak{w}(a)= 0 \neq \mathfrak{v}(D_1 \land D_2) =  1$, 
        nel secondo accade il contrario, ossia 
        $\mathfrak{w}(a) = 1 \neq \mathfrak{v}(D_1 \land D_2) = 0$ tuttavia 
        quest'ultimo caso è impossibile, poiché 
        $\mathfrak{w}(a \rightarrow D_1 \land D_2) = \mathfrak{w}(1 \rightarrow 0) = 0$
        e quindi non può soddisfare $B'$. Quindi rimane 
        il primo caso e bisogna mostrare che anche 
        tale assegnamento effettivamente soddisfa $B'$ e non è un assurdo; come 
        informazione di carattere generale utile per risolvere 
        questo problema, sia E un'espressione formata solo da $0, 1, \land, \lor$ 
        interpretati come $\min$ e $\max$. Il valore di $E$ è $0$ o $1$. 
        Sia $E'$ ottenuta da $E$ rimpiazzando nessuna o più occorrenze 
        del simbolo $0$ con $1$. Allora $E \leq E'$. Questo si dimostra 
        affermando che $0, 1, \min, \max$ sono tutte funzioni 
        non decrescenti e $E$ ed $E'$ sono composizioni di funzioni 
        non decrescenti, pertanto sono non decrescenti (un'altra prova è 
        per induzione strutturale su $E$).

        Ora, tornando al problema principale, si ha che  $\mathfrak{w}(B') = 1$, 
        $\mathfrak{w}(a) = 0$ e $\mathfrak{w}(D_1 \land D_2) = 1$. Dato 
        che la formula iniziale era in $NNF$, anche $B'$ e $B$ sono in 
        NNF. Dunque, considerando $\mathfrak{w}(B)$ e $\mathfrak{w}(B'')$ 
        si possono esprimere entrambe come funzioni termine 
        di $B''$, ossia  
        $$
        \hat{B}''(\mathfrak{w}(p_1), \cdots, \mathfrak{w}(p_n), \mathfrak{w}(D_1 \land D_2))
        $$
        e
        $$
        \hat{B}''(\mathfrak{w}(p_1), \cdots, \mathfrak{w}(p_n), \mathfrak{w}(a))
        $$
        rispettivamente. Come prima osservazione, $\mathfrak{w}(B'')$ e $\mathfrak{w}(B)$
        sono considerabili come espressioni costruite su $0,1,\land,\lor$, poiché 
        sono entrambi in $NNF$. Si può concludere, quindi, che 
        $$
        \mathfrak{w}(B'') \leq \mathfrak{w}(B) \rightarrow 1 \leq \mathfrak{w}(B) \rightarrow \mathfrak{w}(B) = 1
        $$
        applicando le sostituzioni sui valori di verità di $a$ e $D_1 \land D_2$. 
\end{proof}

\begin{oss}[Nota finale]
        Si sarebbe potuto definire $B'$, alternativamente, come segue: 
        \begin{align*}
                B' &:= B[a/D_1 \land D_2] \land (a \iff (D_1 \land D_2))^c \\
                   &=  B[a/D_1 \land D_2] \land (\neg a \lor D_1) \land (\neg A \lor D_2) \land ((D_1 \land D_2) \rightarrow a)^c \\
                   &=  B[a/D_1 \land D_2] \land (\neg a \lor D_1) \land (\neg A \lor D_2) \land (a \lor \neg D_1 \lor \neg D_2)
        \end{align*}
        Adottando tale definizione per $B'$, allora nella prova che $B'$ soddisfacibile 
        implica $B$ soddisfacibile si sarebbe potuto mostrare che $B'$ è 
        soddisfatto solo da assegnamenti nella forma $\mathfrak{v}_a$, 
        ossia $\mathfrak{v}_a(a) = \mathfrak{v}(D_1\land D_2)$.
\end{oss}

Nel passaggio da $B$ a $B'$ si osserva una dilatazione nella lunghezza della 
formula, ossia $B'$ è \textit{più lungo} di $B$; data la formula 
generale 
$$
B' \rightarrow B[a/D_1 \land D_2] \land (\neg a \lor D_1) \land (\neg a \lor D_2)
$$
la parte della formula $B$ viene al massimo accorciata, però sia aggiunge una 
decina di simboli: si può affermare che la lunghezza di 
$B'$ sia $|B'| = |B| + k$, con $k$ una costante dipendente dal 
modo in cui si conta la lunghezza (si contano le parentesi come simboli eccetera). 
Per ogni ``violazione'' nella formula originale, la formula risultante equisoddisfacibile
senza tale violazione è più lunga di $k$ caratteri e, se vi sono $v$ violazioni, 
dopo $v$ passi la formula risultante sarà CNF e avrà lunghezza $|B| + v * k$, 
non esponenziale rispetto alla lunghezza iniziale. 

La tecnica usata per ridurre $SAT \preceq_p CNFSAT$ che consiste nel rimpiazzare 
$B$ con $B'$ equisoddisfacibile è ispirata alla riduzione 
$$
SAT \preceq 3CNFSAT
$$
dovuta a Karp. Il passaggio $B \rightarrow B'$ è un esempio del cosiddetto 
``Tseytin's Trick'', che si usa per ridurre $F \in F_L$ a una 
in $3CNFSAT$ ad essa equisoddisfacibile. 

\begin{defi}[Problemi $3CNFSAT$]
        $3CNFSAT \subseteq CNFSAT \subseteq F_L$ è costituito da tutte 
        e sole le CNF dove ogni clausola contiene o esattamente $3$ 
        letterali. $3CNFSAT \in \mathbb{NP}$ ed è addirittura $\mathbb{NP}$-completo.
\end{defi}
\begin{oss}[Problemi $2CNFSAT$]
        $2CNFSAT \in \mathbb{P}$.
\end{oss}
\subsubsection{Algoritmo di riduzione a CNF equisoddisfacibili}
Sia data $F \in F_L$ e sia $A \in NNF$ $A \equiv F$, che si 
ottiene in tempo polinomiale rispetto alla lunghezza di $F$. 
Se $A \in CNF$, allora l'algoritmo termina, altrimenti è necessario 
individuare la violazione $B$ sottoformula di $A$ e si sostituisce con 
$B'$ e si ritorna al check su $A \in CNF$. 
\subsubsection{Esempi}
\paragraph{1} Sia $A := p \lor (q \land r)$. Si trasforma ora in una 
formula equisoddisfacibile in CNF: 
$$
A' := (p \lor a) \land (\neg a \lor q) \land (\neg a \lor r)
$$
Queste due formule non sono equivalenti, infatti vi sono assegnamenti 
che portano a risultati diversi; tuttavia sono equisoddisfacibili.

\paragraph{2} 
Sia $A := (p_1 \land p_2) \lor (p_2 \land q_2) \lor (p_3 \land q_3)$. Si 
ha $A'$ uguale a 
\[
        ((p_1 \land q_1) \lor (p_2 \land q_2) \lor a_3) \land (\neg a_2 \lor p_3) \land (\neg a_3 \lor q_3)
\]
e, applicando di nuovo la trasformazione, si ottiene 
\[
        ((p_1 \land q_1) \lor a_2 \lor a_3) \land (\neg a_3 \lor p_3) \land (\neg a_3 \lor q_3) \land (\neg a_2 \lor p_2) \land (\neg a_2 \lor p_2) 
\]
e si conclude definendo $A'''$
\[
(a_1 \lor a_2 \lor a_3) \land (\neg a_3 \lor p_3) \land (\neg a_3 \lor q_3) \land (\neg a_2 \lor p_2) \land (\neg a_2 \lor q_2) \land (\neg a_1 \lor p_1) \land (\neg a_1 \lor q_1)
\]
Data una formula con $n$ letterali si generano $2*n +1 $ clausole, 
di cui una con $n$ letterali e $2n$ con due letterali, laddove utilizzando la 
distributività se ne generavano $2^n$ ognuna con $n$ letterali. 


\section{Deduzione Automatica}
Vogliamo studiare i problemi $CNFSAT$, che è $\mathbb{NP}$-completo, e 
il suo complemento $CNFUNSAT$, che è un problema $\mathbb{NP}$-completo, 
in quanto la riduzione polinomiale $SAT \preceq_p CNFSAT$ è valida e pertanto 
studiare $CNFSAT$ è uguale a studiare $SAT$. Analogamente si può studiare 
$TAUTO \preceq_p SAT^c \preceq_p CNFUNSAT$. In realtà siamo nella posizione 
di studiare $\Gamma \models A$ tramite la risoluzione di $CNFSAT$ e $CNFUNSAT$.
Il motivo per cui non studiare invece $DNFSAT$ e $DNFUNSAT$ è che non 
c'è un algoritmo per ``tradurre'' una formula arbitraria equisoddisfacibile 
in DNF in modo che sia sufficientemente corta: i metodi conosciuti allungano
esponenzialmente la formula. 

\begin{defi}[Variazione notazionale]
        Date $F_i \in F_L$, le formule 
        $$
        F_1 \land F_2 \land \cdots \land F_k
        $$
        e 
        $$
        F_1 \lor F_2 \lor \cdots \lor F_k
        $$
        si possono scrivere senza operatori grazie all'associatività, 
        e inoltre le formule interne nelle formule esterne si possono 
        scambiare ($F_1 \land F_2 \equiv F_2 \land F_1$) grazie 
        alla commutatività e si possono espandere le singole formule 
        ($F_1 \land F_1  \equiv F_1$) grazie all'idempotenza. 

        In questo frangente si possono anche ``tralasciare'' gli operatori 
        $\land$ e $\lor$, chiaramente nel caso in cui ci sia un utilizzo uniforme. 
        La notazione può diventare puramente insiemistica, ossia 
        $$
        \{F_1, F_2, \cdots, F_k\}
        $$
        per indicare entrambe le formule, specificando il connettivo che le 
        unisce. D'ora in poi una CNF sarò un \textbf{insieme} di clausole, 
        dove ogni clausola sarà un insieme di letterali. 
        Per esempio 
        $$
        (p \lor q \lor \neg r) \land (q \lor r \lor a) \land p
        $$
        diventa 
        $$
        \{ \{p, q, \neg r\}, \{q, r, a\}, \{p\}\}
        $$
        ossia una CNF in forma insiemistica, che contiene clausole in 
        forma insiemistica. 
        Una teoria costruita da CNF è l'insieme di tutte le clausole appartenenti 
        a qualche CNF della teoria. 
        La CNF vuota è $\emptyset$, mentre la clausola vuota $\qedsymbol$ 
        e la CNF che contiene la clausola vuota avrà la forma 
        $\{\cdots, \qedsymbol, \cdots \}$.
\end{defi}

Il problema della conseguenza logica di una teoria 
$$
\Gamma \models A
$$
si può ridurre a calcolare la soddisfacibilità di 
$$
S := \{\Gamma^c \cup \{\neg A\}^c\} \text{ dove } \Gamma^c := \{\gamma^c | \gamma \in \Gamma\}
$$
unione di CNF. 
Si è quindi ridotto il problema $\Gamma \models A$ al problema $S \in CNFSAT$,
dove $S$ è un insieme di clausole considerate come insiemi di letterali, 
eventualmente infinito. 


\subsection{Metodi refutazionali}
$S$ è soddisfacibile se esiste $\mathfrak{v}:L \rightarrow \{0,1\}$ tale 
che $\mathfrak{v} \models C$ per ogni clausola $C \in S$, in altre parole 
in ogni $C \in S$ esiste un letterale $l \in C$ tale che $\mathfrak{v} \models l$. 

Se l'obiettivo è dimostrare che $S$ è insoddisfacibile, una strategia 
può essere ampliare $S$ in un nuovo insieme $S \subseteq S'$ in modo 
tale che $S'$ è logicamente equivalente o almeno equisoddisfacibile a $S$.
Ad esempio, se ampliando iterativamente $S$ in $S'$, $S''$ fino a $S^{(u)}$ 
e alla fine la clausola vuota $\qedsymbol$ appartiene a $S^{(u)}$ allora quest'ultimo 
è insoddisfacibile, e quindi anche $S$ lo è. Questa idea può essere sfruttata 
disegnando \textbf{metodi refutazionali}, ossia metodi che hanno l'obiettivo di 
provare l'insoddisfacibilità di un insieme di clausole, in questo frangente, 
ma più in generale anche di una formula o una teoria, i quali sono basati 
sulla regola di inferenza chiamata \textbf{principio di risoluzione}, 
il quale è utile per un tipo di calcolo particolarmente adatto a essere automatizzato, 
ossia SAT solver e Theorem Prover. 

\begin{defi}[Principio di Risoluzione]
        Date due clausole $C_1$ e $C_2$ si dice che $D$ è la \textbf{risolvente}
        di $C_1$ e $C_2$ sul pivot $\ell$ se e solo se
        \begin{itemize}
                \item $\ell \in C_1$
                \item $\bar{\ell} \in C_2$
                \item $D := (C_1 \setminus \{\ell\}) \cup (C_2 \setminus \{\bar{\ell}\})$
        \end{itemize}
       e si scriverà $D = \mathbb{R}(C_1,C_2; \ell, \bar{\ell})$. 
\end{defi}

\subsubsection{Esercizio}
Mostrare che $(C_1 \setminus \{\ell\}) \cup (C_2 \setminus \{\bar{\ell}\})$ può essere 
diverso da $(C_1 \cup C_2) \setminus \{\ell, \bar{\ell}\}$. 

\subsubsection{Esempio}
Sia $C_1 := \{x, y, \neg t\}$ e $C_2 := \{u, \neg y, t\}$; 
si ha $D_1 = \mathbb{R}(C_1, C_2; y, \neg y) = \{x, \neg t, u\}$ e 
$D_2 = \mathbb{R}(C_1, C_2; \neg t, t) = \{x, y, y\}$. 

\begin{lem}[di correttezza della risoluzione]
        Sia $D = \mathbb{R}(C_1, C_2; \ell, \bar{\ell})$, allora 
        $$
        \{C_1, C_2\} \models D
        $$
\end{lem}

\begin{proof}
        Bisogna dimostrare che ogni $\mathfrak{v}$ tale che $\mathfrak{v} \models C_1$ 
        e $\mathfrak{v} \models C_2$ implica $\mathfrak{v} \models D$. Sia allora 
        $\mathfrak{v}$ tale che $\mathfrak{v} \models C_1$ 
        e $\mathfrak{v} \models C_2$, allora $\mathfrak{v} \models m$ e 
        $\mathfrak{v} \models n$ per due letterali $m \in C_1$ e $n \in C_2$; 
        se fosse il caso che $m = \ell$ e  $n = \bar{\ell}$ allora 
        $\mathfrak{v}(\ell) = 1$ e $\mathfrak{v}(\bar\ell) = 1$, impossibile. 
        Allora,almeno uno tra $m$ e $n$ è tale che $m \neq \ell$ o $n \neq \ell$ 
        e assumendo senza perdita di generalità $m \neq l$ si ha 
        $m \in D$, dunque $\mathfrak{v}(D) = 1$. 
\end{proof}

\begin{cor}
        Se $\mathbb{R}(C_1, C_2; \ell, \bar{\ell}) = \qedsymbol$ allora 
        $\{C_1, C_2\}$ è insoddisfacibile, in quanto $\{C_1, C_2\} \models \qedsymbol$.
\end{cor}
\begin{cor}
        Sia $S$ un insieme di clausole e sia $S := S_0, S_1, \cdots, S_k$ una 
        successione di insiemi tale che per ogni $i = 0, \cdots, k-1$ 
        si ha che $S_{i+1}$ si ottiene unendo a $S_i$ una o più risolventi 
        di clausole in $S_i$ e che $\qedsymbol \in S_k$. $S$ è 
        insoddisfacibile.
\end{cor}
L'obiettivo è mostare che i calcoli refutazionali basati su applicazioni 
ripetute della risoluzione sono corretti e completi (refutazionalmente).
In genere, per un calcolo logico si studiano infatti le due
seguenti proprietà:
\begin{defi}[Correttezza]
        Un calcolo si definisce \textbf{corretto} se i certificati che produce 
        testimoniano il vero, ossia sono corretti, in altre parole 
        non produce certificati fasulli. 
\end{defi}

Per esempio, nel frangente attuale 
$S_0, S_1, \cdots, S_k \ni \qedsymbol$ è un certificato 
corretto dell'insoddisfacibilità di $S = S_0$. 

\begin{defi}[Completezza]
        Un calcolo è completo se non omette alcun certificato. 
\end{defi}
Nella situazione attuale vuol dire che se $S$ è insoddisfacibile, allora 
esiste $S_0, S_1, \cdots, S_k$ tale che si produce $\qedsymbol \in S_k$.


Prima di mostrare che il calcolo refutazionale è sia corretto (anche se 
il lemma di correttezza è già stato esplicitato) e completo, è necessario 
prepararsi la strada, poiché non sarà banale. 

\begin{teo}[Teorema di Completezza del principio di risoluzione (o di Robinson)]
       Un insieme $S$ di clausole è insoddisfacibile se e solo se 
       $\qedsymbol \in R^*(S)$, dove $R^*(S)$ è definito come segue: 
       \begin{align*}
               R(S) &:= S \cup \{D: D = \mathbb{R}(C_1, C_2; \ell, \bar{\ell}) C_1, C_2 \in S \land \ell \in C_1 \land \bar{\ell} \in C_2\} \\
               R^2(S) &= R(R(S)) \\
               \cdots \\
               R^{t+1}(S) &:= R(R^{t}(S)) \\
               \cdots \\
               R^*(S) &= \cup_{i \in \omega} R^i(S)
       \end{align*}
       dato $R^0(S) := S$. 
\end{teo}

Il calcolo $R^*$ è un metodo \textit{a forza bruta}, applica il 
Principio di Risoluzione calcolando ciecamente 
tutte le risoluzioni possibili e non c'è da sperare che faccia molto meglio 
delle tavole di verità. 

\begin{proof}[$\qedsymbol \in R^*(S) \rightarrow S$ insoddisfacibile (Correttezza del calcolo R)]
        Si supponga $\qedsymbol \in R^*(S)$, allora c'è un $i \in \omega$ tale 
        per cui, per definizione, $\qedsymbol \in R^i(S)$ e pertanto 
        $R^i(S)$ è insoddisfacibile, per definizione di clausola vuota; 
        $R^i(S)$ è equivalente a $R^{i-1}(S)$ per il lemma di correttezza, 
        arrivando fino a $R^0(S) = S$, che è pertanto insoddisfacibile.
\end{proof}
\begin{proof}[$S \text{ insoddisfacibile } \rightarrow \qedsymbol \in R^*(S)$ (Completezza Refutazionale del calcolo R)]
        Dato che $S$ è insoddisfacibile, per il Teorema di Compattezza esiste 
        un $S_{fin} \subseteq_{\omega} S$ finito e $S_{fin}$ è insoddisfacibile; 
        bisogna capire come sia fatto $S_{fin}$: l'insieme finito $S_{fin}$ contiene 
        un numero finito di lettere proposizionali e si definisce, come è stato fatto 
        precedentemente,
        $Var(S_{fin}) \subseteq \{p_1, p_2, \cdots, p_n\}$ per qualche $n \in  \omega$, 
        $p_i \in L$.
        La notazione $C^{n}_L$ indica l'insieme di tutte le clausole scrivibili 
        sulle prime $n$ lettere proposizionali, ossia $\{p_1, p_2, \cdots, p_n\}$  
        e si ha $S_{fin} \subseteq C^{n}_L$. 
        Si osserva che $C^{0}_L = \{ \qedsymbol \}$. A fortiori, dato che 
        $S_{fin} \subseteq C^{n}_L$ si ha $S_{fin} \subseteq (C^{n}_L \cap S) \subseteq (C_L^{n} \cap R^*(S))$
        e pertanto $C^{n}_L \cap R^*(S)$ è insoddisfacibile, dato che $S_{fin}$ è 
        insoddisfacibile. 

        Se si riesce a dimostrare che per ogni $ k = n, \cdots, 1,0$ si ha 
        $$
        C^k_L \cap R^*(S) \text{ è insoddisfacibile}
        $$
        allora si ha che per $k = 0$
        $$
        C^{0}_L \cap R^*(S) \text{ è insoddisfacibile }
        $$
        poichè se si dimostra che $C^0_L \cap R^*(S)$ 
        insoddisfacibile, allora $\qedsymbol \in R^*(S)$: 
        $$
        C^{0}_L \cap R^*(S) = 
        \begin{cases}
        \emptyset  & \rightarrow C^0_L \cap R^*(S) \text{soddisfacibile, assurdo.} \\
        \qedsymbol & \text{ unico caso possibile }
        \end{cases}
        $$
        dunque $\qed \in R^*(S)$. Per dimostrare che 
        per ogni $k = n, \cdots, 1, 0$ 
        $$
        C^k_L \cap R^*(S) \text{ è insoddisfacibile} 
        $$
        si usa l'induzione decrescente su $k$, ossia l'induzione aritmetica 
        su $t = n - k$; se $k = n \rightarrow t = 0$ e se $k = 0 \rightarrow t = n$. 
        La base dell'induzione è ``regalata'' dal Teorema di Compattezza, 
        ossia che $C^k_L \cap R^*(S)$ sia insoddisfacibile è dovuto all'esistenza $S_{fin}$ 
        insoddisfacibile. Si assume l'asserto vero per $n-0, n-1, \cdots,n-k-1 $ e 
        si prova per $t=n-k$, ossia 
        $$
        C^n_L \cap R^*(S), C^{n-1}_L \cap R^*(S), \cdots, C^{k+1}_L \cap R^*(S)
        $$
        sono insoddisfacibili e si vuole dimostrare per $C^k_L \cap R^*(S)$ sia 
        insoddisfacibile. 

        Per assurdo, si assume $C^k_L \cap R^*(S)$ 
        soddisfacibile, dunque esiste $\mathfrak{v} \models C$ per ogni 
        $C \in C^k_L \cap R^*(S)$; si definiscono ora 
        $$
        \mathfrak{v}^+ : L \rightarrow \{0,1\} ~ ~ ~ \mathfrak{v}^+(p_{k+1}) = 1 
        $$
        $$
        \mathfrak{v}^- : L \rightarrow \{0,1\} ~ ~ ~ \mathfrak{v}^-(p_{k+1}) = 0
        $$
        mentre per ogni altra $i$ $\mathfrak{v}(i) = \mathfrak{v}^+(i) = \mathfrak{v}^-(i)$;
        si noti che $p_{k+1} \notin C^{(k)}_L$. 
        Per ipotesi induttiva esiste $C_1 \in C^{k+1}_K \cap R^*(S)$ tale che 
        $\mathfrak{v}^+(C_1) = 0$; analogamente esiste $C_2 \in C^{k+1}_L \cap R^*(S)$ 
        tale che $\mathfrak{v}^-(C_2) = 0$, in quanto è insoddisfacibile.
       
        La lettera proposizionale $p_{k+1}$ occorre in $C_1$ e più precisamente come 
        letterale $\neg p_{k+1}$, infatti $p_{k+1}$ non può occorrere come letterale 
        positivo poiché $\mathfrak{v}^+(p_{k+1}) =1$ per definizione e dunque $ \mathfrak{v}^+(C_1) = 1$, 
        assurdo. Deve, però, apparire nel letterale $\neg p_{k+1}$, poiché altrimenti 
        si ottiene che $p_{k+1}, \neg p_{k+1} \notin C_1$ dunque $C_1 \in C^k_L \cap R^*(S)$ 
        e dunque $\mathfrak{v}(C_1) = 1$. Analogamente, 
        $p_{k+1}$ occorre in $C_2$ e più precisamente come letterale $p_{k+1}$ 
        e la prova induttiva è identica, mutandis mutandis. 


        Dunque, esiste $D = R(C_1, C_2; \neg p_{k+1}, p_{k+1})$, ossia 
        $$
        D = (C_1 \setminus \{ \neg p_{k+1}\}) \cup (C_2 \setminus \{p_{k+1}\})
        $$
        e $p_{k+1}, \neg p_{k+1} \notin D$ e $D \in C^{k}_L \cap R^*(S)$ 
        e $\mathfrak{v}(D) = 1$. Vi sono due casi: il primo è che 
        $\mathfrak{v}$ soddisfi qualche letterale in $C_1 \setminus \{p_{k+1}\}$, 
        ma allora $\mathfrak{v}(C_1)  = 1$ e $\mathfrak{v}^+(C_1) = 1$, 
        assurdo.
        Il secondo caso è che $\mathfrak{v}$ soddisfi qualche letterale 
        in $C_2 \setminus \{p_{k+1}\}$, e allora $\mathfrak{v}^-(c_2) = 1$, 
        assurdo.
        Non essendoci altri casi, si è raggiunta la contraddizione 
        che conclude la dimostrazione per assurdo, dunque 
        $C^k_L \cap R^*(S)$ è
        insoddisfacibile per ogni $k$, che chiude la prova per induzione; dunque, 
        per ogni $k = n, n-1, \cdots, 1, 0$, $C^k_L \cap R^*(S)$ è 
        insoddisfacibile e $C^0_L \cap R^*(S)$ anche, pertanto 
        $C^0\cap R^*(S) = \{\qedsymbol\}$ e $\qedsymbol \in R^*(S)$.
\end{proof}
\begin{oss}
        Se $S$ è un insieme finito di clausole, la costruzione di $R^*(S)$ 
        costituisce una procedura di decisione, cioè sia che $S$ sia insodd. 
        sia che $S$ sia sodd. termina in tempo finito, dando la risposta 
        corretta. Infatti, se $S$ è finito menziona solo un numero finito $n = |Var(S)|$ di 
        lettere proposizionali diverse e i letterali scrivibili su $\{p_1, \cdots, p_n\}$ 
        sono $2*n$, dunque in $S$ occorrono al più $2*n$ letterali diversi; le 
        clasuole scrivibili su $2*n$ letterali sono al più $2^{2*n}$ dunque 
        in $S$ occorrono al più $2^{2*n}$ clausole. 
        Si nota, ora, che la risoluzione non introduce mai nuovi letterali, 
        dunque la sequenza $R^{(0)}(S) \subseteq R(S) \subseteq \cdots R^{k}(S) \subseteq \cdots$ 
        è tale che  che ogni $R^{i}(S)$ è un sottoinsieme delle $2^{2*n}$ clausole 
        scrivibili su $\{p_1, \cdots, p_n\}$, dunque esiste un $t$ tale che 
        $R^{t}(S) = R^{t+1}(S)$, poiché la sequenza non può crescere all'infinito; 
        pertanto
        $$
        R^{t}(S) = R^{t+1}(S) = \cdots = R^*(S)
        $$
        Se si trova $\qedsymbol \in R^i(S)$, si può concludere che $S$ sia 
        insoddisfacibile. Se, al contrario, $R^t(S) = R^{t+1}(S)$ e $\qedsymbol \notin R^t(S)$, 
        si può concludere che $S$ sia soddisfacibile. 
\end{oss}
\begin{oss}
        Dato che $S_{fin}$ è finito, esiste $t \in \omega$ tale 
        che $R^*(S_{fin})=R^t(S_{fin}) = R^{t+1}(S_{fin})$, 
        tuttavia questo non fornisce una procedura di decisione per 
        gli $S$ infiniti, ma esclusivamente di semidicesione, in quanto 
        in genere non si sa \textit{scegliere} $S_{fin}$ e 
        il meglio che si può fare è descrivere una successione infinita 
        $$
        S_1 \subseteq S_2 \subseteq \cdots \subseteq S_k  \subseteq \cdots 
        $$
        di sottoinsiemi finiti di $S$ tali che $\bigcup S_i = S$ e 
        per ogni $i$ si calcola $R^*(S_i)$: se $\qedsymbol \in S_i$ allora 
        $S$ insoddisfacibile, 
        altrimenti si aumenta $i$ e si procede al passo successivo.
\end{oss}

La Completezza Refutazionale non è la Completezza \textit{tout-court}. 
\`E qualcosa che è più debole, e in realtà proprio per questo è una 
proprietà desiderabile dal punto di vista computazionale. 

\begin{defi}[Deduzione per Risoluzione]
        Una deduzione per risoluzione di una clausola $C$ da un insieme di clausole 
        $S$, indicato 
        $$
                S \vdash_R C
        $$
        è una sequenza finita di clausole 
        $$
        C_1,C_2, \cdots, C_n
        $$ 
        tale che 
        \begin{itemize}
                \item $C_n = C$
                \item $\forall C_i ~~~ C_i \in S \text{ oppure } C_i = R(C_j, C_k, \ell, \bar{\ell})$
        \end{itemize}
\end{defi}

In particolare, una deduzione per risoluzione della clausola vuota 
($S \vdash_R \qedsymbol$) è detta \textbf{refutazione} di $S$. 
\begin{teo}[Teorema di Completezza Refutazionale]
        Un insieme di clausole $S$ è insoddisfacibile se e solo se $S \vdash_R \qedsymbol$.
\end{teo}
Al momento, la refutazione al momento la sappiamo costruire solo tramite 
il metodo $R^*(S)$. 

\paragraph{Esempio}
$$
\{\{a, b, \neg c\}, \{a, b, c\}, \{a, \neg b\}\} \models \{a\}?
$$
Si costruisce una refutazione: 
inizialmente, si trasforma il problema in un problema di insoddisfacibilità, 
ossia 
$$
\{\{a, b, \neg c\}, \{a, b, c\}, \{a, \neg b\}, \{\neg a\}\} \text{ è soddisfacibile?}
$$
Non è soddisfacibile, poiché 
\begin{align*}
        (\{a,b,\neg c\}, \{a, b, c\}) &\vdash_R \{a,b\} \\
        (\{a,b\}, \{a, \neg b\}) &\vdash_R \{a\}\\
        (\{a\}, \{\neg a\}) &\vdash_R \qedsymbol
\end{align*}

\noindent
Genericamente, chiedersi se $\Gamma \models A$ è uguale a chiedersi 
se $\Gamma, \neg A$ sia insoddisfacibile, che è 
 uguale a $\Gamma ^c, (\neg A)^c \vdash_R \qedsymbol$ per Completezza 
 Refutazionale, ma che è diverso da 
$\Gamma^c \vdash_R A^c$; è questa la debolezza 
della completezza refutazionale rispetto alla Completezza tout court
($\Gamma \models A \iff \Gamma^c \vdash_c A$). Quindi, per un calcolo 
refutazionale $R$ per il quale vale la Completezza Refutazionale si ha 
$$
\Gamma \models A \iff \Gamma, \neg A \text{ è insoddisfacibile } \iff \Gamma^C, (\neg A)^C \vdash_R \qedsymbol \rightarrow \nleftarrow \Gamma^C \vdash_R A^C
$$
Si consideri, per esempio, l'insieme di clausole 
$$
\{\{b\}, \{\neg b\}, \{\neg a\}\} \vdash_R \qedsymbol
$$
ma per il fatto che la Risoluzione non introduce mai nuovi letterali, 
non si potrà mai a provare 
$$
\{\{b\},\{\neg b\}\} \vdash_R \{a\}
$$
\subsection{Sistemi Assiomatici (Calcoli alla Hilbert)}
I Sistemi Assiomatici sono un tipo di calcolo tradizionale completo 
tout court. Hanno una formalizzazione tra le più semplici, ma non 
sono in particolar modo adatti alla ricerca \textit{human-oriented} e nemmeno 
alla ricerca di prove tramite algoritmi. 

\begin{defi}[Calcoli alla Hilbert]
        Dato un insieme di assiomi (che sono tautologie della Logica), come per esempio 
il seguente, che è corretto e completo per la Logica Proposizionale classica
$$
\begin{cases}
        A \rightarrow (B \rightarrow A) \\
        (A \rightarrow (B \rightarrow C)) \rightarrow ((A \rightarrow B) \rightarrow (A \rightarrow C)) \\
        ( \neg B \rightarrow \neg A) \rightarrow (A \rightarrow B) 
\end{cases}
$$
con $A, B, C \in F_L$ e avendo regole di inferenza come il \textit{modus ponens} 

\begin{prooftree}
        \AxiomC{$A$}
        \AxiomC{$A\rightarrow B$}
        \BinaryInfC{$B$}
\end{prooftree}

una prova di $A$ da una teoria $\Gamma$ nel calcolo definito Calcolo H o 
Calcolo alla Hilbert, ossia 
$$
\Gamma \vdash_H A
$$
è una successione finita di formule 
$$
A_1, A_2, \cdots, A_u
$$
tale che: 
\begin{enumerate}
        \item $A_u = A$ 
        \item ogni $A_i$ per $i = 1, \cdots, u$ è tale che: 
                \begin{enumerate}
                        \item $A_i \in \Gamma$
                        \item $A_i$ è un'istanza di assioma 
                        \item esistono $j,k < i$ tali che $A_j= A_k \rightarrow A_i$, quindi 
                                \begin{prooftree}
                                        \AxiomC{$A_k$}
                                        \AxiomC{$A_k \rightarrow A_i$}
                                        \BinaryInfC{$A_i$}
                                \end{prooftree}
                \end{enumerate}
\end{enumerate}
\end{defi}

\begin{teo}[Completezza Forte del Calcolo H]
$\Gamma \models A \iff \Gamma \vdash_H A$, 
anche per teorie $\Gamma$ infinite. 
\end{teo}

Il Calcolo H non è particolarmente adatto alla deduzione automatica, come 
sottolineato precedentemente. I vari tipi di calcolo corrispondono ad esigenze 
diverse: quando la necessità è la deduzione automatica, vi sono ottime 
ragioni per preferire il Calcolo Refutazionale. Diamo ora qualche evidenza 
del perché non sia così saggio utilizzare il Calcolo H. Quest'ultimo è 
semplice dal punto di vista concettuale, ma tirar fuori prove è più complicato. 

\subsubsection{Differenze tra i due Calcoli}
Verificare se  $\neg \neg A \models A$ è vera. 

\paragraph{Calcolo R}
Bisogna inizialmente trasportare $A \in F_L$ in lettere proposizionali: 
$$
\neg \neg p \models p 
$$
con $p \in L$; nel Calcolo H questo passaggio non è necessario, si potrà 
tranquillamente ragionare sulle metavariabili.
Si studia l'insoddisfacibilità di 
$$
\{ \neg \neg p, \neg p\} \models \bot
$$
che si traduce, in refutazione, in 
$$
\{\{p\}, \{\neg p \}\} \vdash_R \qedsymbol ?
$$
si ha: $\mathbb{R}(\{p\}, \{\neg p\}; p, \neg p) =  \qedsymbol$, pertanto la prima 
formula è vera. 

\paragraph{Calcolo H}
Per il Calcolo di Hilbert bisogna ``inventarsi'' una dimostrazione; si 
comincia a scrivere 
\begin{align*}
\neg \neg A \rightarrow (\neg \neg \neg \neg A \rightarrow \neg \neg A), \neg \neg A ( \in \Gamma)\\
\therefore \neg \neg \neg \neg A \rightarrow \neg \neg A, ( \neg \neg \neg \neg A \rightarrow \neg \neg A) \rightarrow (\neg A \rightarrow \neg \neg \neg \neg A)\\
\therefore \neg A \rightarrow \neg \neg \neg \neg A, (\neg A \rightarrow \neg \neg \neg \neg A) \rightarrow ( \neg \neg A \rightarrow A) \\
\therefore  \neg \neg A \rightarrow A, \neg \neg A \\
\therefore A
\end{align*}

Il calcolo H non presenta un calcolo estremamente meccanico. 
\paragraph{Esercizio}
Dimostrare che $\models A \rightarrow A$

Si prenda una formula $B$ soddisfacibile, già provata, ossia $\models_H B$. 
\begin{prooftree}
        \AxiomC{$B$}
        \AxiomC{$B \rightarrow ( A \rightarrow B)$}
        \BinaryInfC{$A \rightarrow B$}
\end{prooftree}
e 
\begin{prooftree}
        \AxiomC{$A \rightarrow (B \rightarrow A)$}
        \AxiomC{$(A \rightarrow (B \rightarrow A)) \rightarrow (A \rightarrow A)$}
        \BinaryInfC{$(A \rightarrow B) \rightarrow (A \rightarrow A)$}
\end{prooftree}
da cui si può concludere 
$$
(A \rightarrow A)
$$
\subsection{Procedura refutazionale di Davis-Putnam}
Si supponga che sia stato definito un insieme di clausole
$$
S = \{\{p,q\}, \{p, \neg q\}, \{\neg p, q\}, \{\neg p, \neg q\}\} 
$$
e ci si chiede se sia refutabile.
Si ha 
\begin{prooftree}
        \AxiomC{$p,q$}
        \AxiomC{$p,\neg q$}
        \BinaryInfC{$p$}
        \AxiomC{$\neg p,q$}
        \AxiomC{$\neg p,\neg q$}
        \BinaryInfC{$\neg p$}
        \BinaryInfC{$\qedsymbol$}
\end{prooftree}

Utilizzando il calcolo refutazionale $R^*(S)$ si utilizzano molti più passaggi: 
il problema che si vuole affronare e risolvere, per quanto possibile, è 
designare una tecnica che permetta di mantenere la completezza 
refutazionale selezionando, in qualche modo, i risolventi da calcolare. 

\begin{defi}[Sussunzione]
        Una clausola $C_1$ \textbf{sussume} $C_2$ se e solo se 
        $C_1 \subset C_2$, ossia è un insieme proprio di $C_2$. 
        In termini logici, si ha $\models C_1 \rightarrow C_2$, 
        ossia $\{C_1, C_2\} \equiv \{C_1\}$.
\end{defi}        

\begin{defi}[Regola di Sussunzione]
        Sia $S$ un insieme di clausole e sia $S'$ ottenuto da 
        $S$ eliminando tutte le clausole sussunte. Allora si può 
        concludere che 
        $S \equiv S'$. 
\end{defi}

\begin{oss}
        $S'$ non contiene sussunte, ossia non è ulteriormente riducibile 
        per sussunzione.
\end{oss}

\begin{defi}[Clausola Banale]
        Una clausola $C$ è detta \textbf{banale} (trivial) se contiene 
        sia $L$ che  il sua opposto $\bar{L}$ per qualche letterale 
        $l$, ossia $C \equiv \top$ o $\models C$. 
\end{defi}

\begin{defi}[Regola di rimozione banali]
        Rimuovendo da $S$ tutte le banali costruendo $S'$, 
        si ha $S \equiv S'$, e $S'$ non contiene banali.
\end{defi}

\begin{oss}
        Risolvendo tra loro le clausole non banali $C_1$ e $C_2$, 
        allora in $D = R(C_1, C_2; \ell, \bar{\ell})$ si ha 
        che $\ell \notin D$ e $\bar{\ell} \notin D$, 
        cioè $(C_1 \setminus \{\ell\}) \cup (C_2 \setminus \{ \bar{\ell}\}) = (C_1 \cup C_2) \setminus \{\ell, \bar{\ell}\}$
\end{oss}

\begin{oss}
        $R(\{a, \neg a\}, \{a, \neg a\}; a, \bar{a}) = ( \{a, \neg a\} \setminus \{a\}) \cup (\{a, \neg a\} \setminus \{\neg a\}) = \{\neg a\} \cup \{a\} = \{a, \neg a\}$
\end{oss}


\begin{defi}[Passo di DPP]
        Dato un  input $S$ finito, già ripulito di banali e sussunte, 
        un passo di David-Putnam procedure 
        si articola nei seguenti \textit{micropassi}: 
        \begin{enumerate}
                \item Scegliere una lettera proposizionale $p \in L$ 
                        tra quelle che appaiono in $S$. $p$ sarà detto 
                        il \textit{pivot} del passo.
                \item $S'$ è l'insieme delle $p$-esonerate, ossia 
                        l'insieme delle clausole in $S$ in cui non 
                        occorre $p$ come lettera proposizionale, quindi 
                        né $p$ né $\neg p$.
                \item $S''$ è l'insieme delle $p$-risolventi ed è l'insieme 
                        delle clausole ottenute per risoluzione sul pivot 
                        $p$ in $S\setminus S'$. 
                \item $S'''$ è l'insieme  $S' \cup S''$ ripulito, che è 
                        l'output del passo di DPP.
        \end{enumerate}
\end{defi}

\subsubsection{Esempi di DPP}
\paragraph{1}
Sia $S_0 = \{\{a,b,\neg c\}, \{a, \neg b, \neg d\}, \{a, \neg b, \neg c, \neg d\}, 
\{\neg a, d\}, \{\neg a, \neg c, \neg d\}, \{c\}\}$. 
\begin{itemize}
        \item $S_0$ è ripulito. Si sceglie, come pivot, $c$. 
        Le $c$-esonerate sono la seconda e la terza clausola. 
Le $c$-risolventi sono, per esempio, $\{a,b\}, \{a, \neg b, \neg d\}, \{\neg a, \neg d\}$. 
L'output ripulito è 
$$
\{\{a, \neg b, d\}, \{\neg a, d\}, \{a, b\}, \{a, \neg b, \{d\}, \{\neg a, \neg d\}\}
        $$
\item Per il secondo passo, si sceglie come pivot $a$. 
Le $a$-esonerate sono $\emptyset$, mentre le $a$-risolventi sono 
$$
\{\neg b, d\}, \{\neg b, d, \neg d\}, \{b, d\}, \{\neg b, d\}, \{b, \neg d\}, \{\neg b, \neg d\}
$$
L'output ripulito è: 
$$
\{\{\neg b, d\}, \{b, d\}, \{b, \neg d\}, \{\neg b, \neg d\}\}
$$
\item Come terzo passo si prende come pivot $b$. Le $b$-esonerate sono 
$$
\emptyset
$$
e le $b$-risolventi sono 
$$
\{d\},\{d, \neg d\}, \{\neg d, d\} \{\neg d\}
$$
mentre l'output ripulito è 
$$
\{\{d\}, \{\neg d\}\}
$$
\item Per l'ultimo passo si è obbligati a scegliere come pivot $d$. 
Si ha che le $d$-esonerate è nuovamente l'insieme vuoto $\emptyset$, 
mentre le $d$-risolventi sono 
$$
\qedsymbol
$$
e l'output è 
$$
\{\qedsymbol\}
$$
\end{itemize}
Dato che è stata ottenuta la clausola vuota, si dichiara la clausola 
iniziale $S_0$ insoddisfacibile. 
\paragraph{2}
Sia $S_0 = \{\{a, \neg b, c\}, \{b, \neg c\}, \{a,c,\neg d\}, \{\neg b, \neg d\}, \{a,b,d\}, \{\neg a, d,b\} \{b, \neg c, d\}\{b, c, d\}\}$. 
\begin{itemize}
        \item Ripulito, si trovano le clausole 
$$ 
\{\{a, \neg b, c\}, \{b, \neg c\}, \{a,c,\neg d\}, \{\neg b, \neg d\}, \{a,b,d\}, \{\neg a, d,b\}\}
$$
Si sceglie, come pivot, $b$. 
le $b$- esonerate sono
$$
\{a, c, \neg d\}
$$
e le $b$-risolventi sono: 
$$
\{a, c, \neg c\}, \{a,c,d\}, \{a, \neg a, c, d\}, \{\neg c, \neg d\}, \{a, \neg d, d\}, \{\neg a, \neg d, d\}
$$
e l'output è 
$$
S_1 := \{a, c,d\}, \{\neg c, \neg d\}, \{a, c, \neg d\}
$$
\item Al secondo passo si sceglie, come pivot, $c$. 
Non vi sono $c$-esonerate, mentre 
le $c$-risolventi sono 
$$
\{a, d, \neg d\}, \{a, \neg d, \neg d\}
$$
e l'output è 
$$
S_2 := \{a, \neg d\}
$$
\item Si sceglie, come pivot, $a$. 
Non vi sono $a$-esonerate e non vi sono $a$-risolventi; 
L'output di questo passaggio è l'insieme vuoto $\emptyset$ e si dimostra 
che questo implica che $S_0$ è soddisfacibile.
La lettera proposizionale $d$ non è mai stata scelta come lettera pivot, quindi è 
una \textit{fuoriuscita}. 
\end{itemize}

\begin{defi}
        In un'esecuzione di DPP le lettere che non sono mai pivot si 
        definiscono \textit{fuoriuscite}.
\end{defi}

Delineamo, prima di affrontare la dimostrazione, la strategia che con questi 
dati costruisce un assegnamento che soddisfa $S$.
Sia $\mathfrak{v}: L \rightarrow \{0,1\}$ tale che $\mathfrak{v} \models S_0$. 
Si definisce
$$
\mathfrak{v} := 
\begin{cases}
        a & \mathfrak{v}(a) = 1\\
        b & \mathfrak{v}(b) = 1\\
        c & \mathfrak{v}(c) = 1\\
        d & \mathfrak{v}(d) = 0 \text{ per convenzione per le fuoriuscite, arbitrario}\\
\end{cases}
$$
Si definisce $\mathfrak{v}(a)$ come un valore che soddisfi $S_2$, ossia $\{a, \neg d\}$.
Si definisce $\mathfrak{v}(c)$ come un valore che soddisfi $S_1$ (si è liberi, in 
quanto tutte le clausole sono già soddisfatte dagli altri assegnamenti). 
Si definisce $\mathfrak{v}(b)$ come un valore che soddisfi $S_0$.

\subsubsection{Teorema di completezza (e correttezza) refutazionale di DPP}
\begin{teo}
        Un insieme \textbf{finito} $S$ di clausole nelle lettere proposizionali 
        $p_1, p_2, \cdots, p_n$ ($:= Var(S)$) è insoddisfacibile se e solo se 
        entro al più $n$ passi si ottiene la clausola vuota $\qedsymbol$. 
        $S$ è soddisfacibile se e solo se entro al più $n$ passi si ottiene 
        l'insieme vuoto di clausole $\emptyset$.
        Non vi sono altri modi di terminare. 
\end{teo}

\begin{proof}[Terminazione]
        Sia $DPP(S) := S_0, S_1, \cdots, S_k$ una sequenza di clausole, dove
        $S_0$ è la ripulitura di $S$ e $S_i$ è ottenuto da $S_{i-1}$ attraverso un 
        passo di DPP sul pivot $q_i \in \{p_1, \cdots, p_n\}$.
        Si osserva che $S_i$ non conterrà più alcuna occorrenza di $q_i$ e, quindi, 
        dopo al più $k \leq n$ passi si ottiene $Var(S_k) = \emptyset$, dunque 
        o $S_k = \{ \qedsymbol \}$ oppure $S_k = \emptyset$. 
\end{proof}

\begin{proof}[Correttezza e Completezza Refutazionale di DPP]
        La correttezza di DPP si può specificare in due modi: 
        $$
        \begin{cases}
                S_k = \{\qedsymbol\} \implies S_0 \text{ insoddisfacibile } \\
                S_0 \text{ soddisfacibile } \implies S_k = \emptyset
        \end{cases}
        $$
        La prima delle due versioni è ``gratuitamente'' provata dal Lemma di Correttezza, 
        ancora valido, poiché il calcolo dei risolventi non è cambiato
        (pertanto la correttezza è provata.)

        La completezza di DPP si può specificare nei due modi rimanenti: 
        $$
        \begin{cases}
                S_0 \text{ insoddisfacibile } \implies S_k = \{\qedsymbol\} \\
                S_k = \emptyset \implies S_0 \text{ soddisfacibile } 

        \end{cases}
        $$
        Ci basta dimostrare quest'ultima parte, ossia che se 
        $S_k = \emptyset$ allora $S_0$ è soddisfacibile: 
        questo fatto è chiamato anche \textit{Lemma di Model Building}, poiché 
        la dimostrazione si occupa di mostrare come costruire l'assegnamento 
        che soddisfa $S$. 
        
        Si dimostra per induzione decrescente su $k$, ossia induzione 
        crescente su $t=k -i$:
        \paragraph{base}
        $$
        S_k = \emptyset \rightarrow S_k \text{ soddisfacibile.}
        $$
        \paragraph{passo induttivo} si assume che $S_{i+1}$ soddisfacibile e si mostra $S_i$ 
        soddisfacibile, ossia esiste $\mathfrak{v}: L \rightarrow \{0,1\}$ tale 
        che $\mathfrak{v} \models S_i$. Dato $\mathfrak{v} : L \rightarrow \{0,1\}$ 
        tale che $\mathfrak{v} \models S_{i+1}$ si definiscono le due varianti 
        $$
        \mathfrak{v}^+ : L \rightarrow \{0,1\} ~ ~ ~ \mathfrak{v}^+(p) = 1 
        $$
        $$
        \mathfrak{v}^- : L \rightarrow \{0,1\} ~ ~ ~ \mathfrak{v}^-(p) = 0
        $$
        dove $p$ è il pivot di $S_i \leadsto S_{i+1}$, 
        mentre 
        $$
        \mathfrak{v}(q) = \mathfrak{v}^+(q) = \mathfrak{v}^-(q) \forall q \in L q \neq p
        $$
        Si mostrerà che una delle due varianti soddisfa $S_i$. 
        Per assurdo, si assume $\mathfrak{v}^+, \mathfrak{v}^-$ non verifichino 
        $S_{i}$. 

        Allora esistono $C_1, C_2 \in S_i$ tali che $\mathfrak{v}^+(C_1) = 0$ 
        e $\mathfrak{v}^-(C_2) = 0$. Si assume che
        $\neg p \in C_1$ (senza perdita di generalità), poiché altrimenti se $p \in C_1$ allora 
        $\mathfrak{v}^+(C_1) = 1$, assurdo, e se 
        $p \notin C_1 e \neg p_1 \notin C_1$, allora $C_1$ sarebbe $p$-esonerata, 
        dunque $C_1 \in S_{i+1}$ e $\mathfrak{v} \models C_1$ e $\mathfrak{v}^+ \models C_1$, 
        assurdo. Analogamente per $p \in C_2$. 

        Si conclude che 
        $D = R(C_1, C_2; \neg p, p)$ e $D \in p$-risolventi implica $D \in S_{i+1}$, 
        che implica $\mathfrak{v} \models D$ e allora 
        $$
        \mathfrak{v} \models D \rightarrow \mathfrak{v}^+ \models D \text{ oppure } \mathfrak{v}^- \models D \rightarrow \mathfrak{v}^+ \models C_1 \text{ oppure } \mathfrak{v}^- \models C_2
        $$
        che è assurdo, dunque $S_i$ è soddisfacibile, e pertanto $S_0$ è soddisfacibile.
\end{proof}

\begin{oss}
Sia $S$ un insieme finito di clausole tale che $Var(S) \subseteq \{p_1, \cdots, p_n\}$. 
Allora $DPP(S)$ termina in $k \leq n$ passi. Se $k \leq n \leq ||S||$, 
si potrebbe concludere che si ha una procedura che funziona in tempo 
polinomiale e decide se $S$ è soddisfacibile o meno, e quindi $P = NP$. 
\end{oss}

Se si potesse garantire che $||S_i||$ sia sempre polinomiale rispetto 
a $||S||$ allora davvero $k \leq n$ passi di DPP porterebbero a 
un tempo di esecuzione complessivo che è polinomiale rispetto a $||S||$ 
e dunque $SAT \in P$, e dunque $P = NP$, ma purtroppo questo non si 
può garantire.

Un risultato, dovuto ad Haken, afferma che ogni prova per refutazione del 
\textit{Principio della Piccionaia} su $n$ ``piccioni''
richiede tempo almeno $T(n) = \Omega(2^n)$. 


\begin{defi}[Principio della Piccionaia]
        Il Principio della Piccionaia, notazionalmente espresso su un 
        numero naturale $n$ come 
        $$
        PHP(n)
        $$
        è esresso affermando che $n+1$ piccioni non possono occupare 
        $n$ posti nella piccionaia in modo che ogni posto abbia al più un 
        piccione. 
\end{defi}

Come formalizzare in clausole il $PHP(n)$? Definendo $P$ i ``piccioni'' e 
$H$ i ``posti'', si può definire
$$
\bigwedge_{i \in P} \bigvee_{j \in H} p_{ij} \text{ (ogni piccione ha trovato posto }
$$
In altre parole, per ogni piccione, il piccione ha trovato almeno un posto, 
ossia il piccione $i$ sta nel posto $j$. 
Questo, però va unito al fatto che ogni piccione ha al più un posto: 
$$
\bigwedge_{i \neq j ~ i,j \in P}\bigwedge_{k \in H} (\neq p_{ik} \lor \neq p_{jk}) \text{ nessuna coppia di piccioni 
sta nel posto }k 
$$
in altre parole non c'è nessun posto che contenga contemporaneamente due piccioni. 
Quindi, la formalizzazione finale sarà 
$$
PHP(n) := \bigwedge_{i \in P} \bigvee_{j \in H} p_{ij} ~~ \land ~~ \bigwedge_{i \neq j ~ i,j \in P}\bigwedge_{k \in H} (\neq p_{ik} \lor \neq p_{jk})
$$
Mostrare che il Principio della Piccionaia è vero consiste nel mostrare che 
$PHP(n)$ è insoddisfacibile, ossia che $PHP(n) \in CNFUNSAT \forall n \in \mathbb{N}$. 
Haken ha dimostrato che la risoluzione di questo problema impiega, per ogni $n$, 
tempo esponenziale. 


\subsubsection{Complessità di DPP}
Dato che i passi di DPP sono ``pochi'', $k \leq |Var(S)|$, anche se nel caso 
peggiore DPP richiede tempo esponenziale, nella pratica è un 
algoritmo molto più efficiente di $R^*$ e delle tavole di verità. 
Alcuni frammenti, ossia sottolinguaggi di CNFSAT, sono noti avere tempo di 
decisione, attraverso la DPP, polinomiale, come KROMSAT e HORNSAT; 
una CNF è HORN se e solo se ogni sua clausola è una clausola di Horn, ossia 
una clausola che ha la forma della clausola vuota o è un'unità $\{p\}$ o 
$\{p, \neg p_1, \neg p_2, \cdots\}$ o più genericamente contiene al più un 
letterale positivo; una clausola è di Krom se ha al più due letterali 
(quindi $KROMSAT = 2CNFSAT$). 

\subsubsection{Davis Putnam Logemann Loveland Procedure}
La DPLL è un metodo di ``reingegnerizzazione'' della DPLL. Su un insieme $S$ di 
clausole finito, DPLL cerca di costruire un assegnamento che soddisfi $S$. 
L'idea è di costruire un assegnamento parziale che viene esteso ad ogni passo 
ad una nuova lettera proposizionale $p$, chiamata pivot. Se si giunge ad assegnare 
tutte le lettere in $Var(S)$ si è ottenuto un assegnamento che mostra $S$ soddisfacibile: 
se non si giunge a tal punto, il teorema di completezza e correttezza di DPLL 
dimostra che $S$ è insoddisfacibile. Si propaga, ad ogni passo, tutta l'informazione 
che si guadagna estendendo l'assegnamento al pivot. Sostanzialmente, quello che era 
il passo finale della DPP si utilizza come azione fondamentale nella DPLL. 

Può accadere che non vi siano informazioni sfruttabili per quanto riguarda 
la scelta del pivot ad un certo passo codificata nelle regole che verranno 
date, allora si ``spezza'' la procedura in due parti: in una si assegna al pivot 
il valore di verità $1$ e alla seconda si assegna al pivot il valore di verità 
$0$ e la prova risulta dunque in una 
struttura ad albero i cui rami si visitano in ``backtracking'', la cui implementazione
(anche nel caso di backtracking non cronologico) è soggetto di ampie ricerche; 
se in un ramo si raggiunge $\emptyset$ si è costruito un assegnamento che soddisfa 
l'insieme iniziale e se su \textit{tutti} i rami si raggiunge la clausola 
vuota, allora $S$ è insoddisfacibile.

Le regole utilizzate dalla DPPL sono quelle della DPP adattate a questo contesto; 
il punto iniziale è l'assegnamento vuoto. 

\begin{defi}[Assegnamento Parziale]
        Un assegnamento parziale è una funzione parziale 
        $$
        \mathfrak{v}: L \rightarrow \{0,1,?\}
        $$
        Un assegnamento vuoto è un assegnamento parziale tale che 
        $\mathfrak{v}(p_i) = ? \forall p_i \in \{p_1, \cdots, p_n\}$.
        Un assegnamento completo o totale sulle prime $n$ lettere proposizionali 
        è una mappa $\mathfrak{v}: \{p_1, \cdots, p_n\} \rightarrow \{0,1,?\}$ 
        tale che $\mathfrak{v}(p_i) \neq ? \forall p_i \in \{p_1, \cdots, p_n\}$.
\end{defi}

\paragraph{Regole di DPLL}
\begin{itemize}
        \item Regola iniziale: $\emptyset \vdash S$ è la radice della prova. 
        \item Sussunzione: se $\mathfrak{v}(p_i) = 1$ allora si possono cancellare 
                da $S$ tutte le clausole che contengono  il letterale $p_i$
                \begin{prooftree}
                        \AxiomC{$\mathfrak{v}, \mathfrak{v}(p_i) = 1 \vdash S \cup \{\{p_i\} \cup C\}$}
                                \UnaryInfC{$\mathfrak{v}, \mathfrak{v}(p_i) = 1 \vdash S$}
                \end{prooftree}
                analogamente, se $\mathfrak{v}(p_i) = 0$, allora si possono cancellare 
                da $S$ tutte le clausole che contengono il letterale $\neg p_i$. 
        \item Risoluzione unitaria: se $\mathfrak{v}(p_i) = 0$ si cancella 
                da ogni clausola di $S$ il letterale $p_i$
                \begin{prooftree}
                        \AxiomC{$\mathfrak{v}(p_i) = 0$}
                        \AxiomC{$(\{\neg p_i, \{p_i,\cdots\})$}
                        \BinaryInfC{$C$}
                \end{prooftree}
                Ossia, in termini di ``nodi'' DPLL
                \begin{prooftree}
                        \AxiomC{$\mathfrak{v}, \mathfrak{v}(p_i) = 0 \vdash S \cup \{\{p_i\}\cup C\}$}
                                \UnaryInfC{$\mathfrak{v},\mathfrak{v}(p_i) = 0 \vdash S \cup \{C\}$}
                \end{prooftree}
               mentre se $\mathfrak{v}(p_i) = 1$, 
        allora si elimina da ogni clausola di $S$ il letterale $\neg p_i$.  
        \item Asserzione: se $S$ contiene la clausola $\{p_i\}$, allora 
                si estende $\mathfrak{v}$ ponendo $\mathfrak{v}(p_i) = 1$, 
                cancellando $\{p_i\}$ da $S$
                \begin{prooftree}
                        \AxiomC{$\mathfrak{v}\vdash S \cup \{\{p_i\}\}$}
                        \UnaryInfC{$\mathfrak{v}, \mathfrak{v}(p_i) = 1 \vdash S$}
                \end{prooftree}
                mentre si fa al contrario se $S$ contiene $\{\neg p_i\}$.
        \item Letterale puro: Se il letterale $p_i$ occorre in $S$ e $\neg p_i$ non 
                occorre, allora si estende $\mathfrak{v}$ ponendo $\mathfrak{v}(p_i) = 1$, 
                \begin{prooftree}
                        \AxiomC{$\mathfrak{v}\models S$}
                        \UnaryInfC{$\mathfrak{v}, \mathfrak{v}(p_i)=1 \vdash S$}
                \end{prooftree}
                (chiaramente se $p_i \in C \in S, \neg p_i \notin C \forall C \in S$),
                mentre si fa contrario se $\neg p_i$ occorre in $S$ e $p_i$ 
                non occorre in $S$, ponendo $\mathfrak{v}(p_i) = 0$. 
        \item Spezzamento: in ogni momento, si può biforcare la prova in due 
                sottoprove, dando origine ad una struttura ad albero, in cui 
                in una si pone per il pivot scelto $\mathfrak{v}(p) = 1$ 
                e nell'altra si pone $\mathfrak{v}(p)= 0$. 
                \begin{prooftree}
                        \AxiomC{$\mathfrak{v} \vdash S$} 
                        \UnaryInfC{$\mathfrak{v}, \mathfrak{v}(p_i) = 0 \vdash S ~~~~ ||  ~~~~ \mathfrak{v}, \mathfrak{v}(p_i) = 1 \vdash S$}
                \end{prooftree}
        \item Terminazione: se in un ramo si ottiene $\mathfrak{v} \vdash \emptyset$ 
                allora si prova che $\mathfrak{v}$ è completo su 
                $Var(S)$ e $\mathfrak{v} \models S$. Al contrario, se su tutti 
                i rami si ottiene $\mathfrak{v} \vdash \qedsymbol$, 
                allora $S$ è insoddisfacibile. 
        \item Ogni ramo o ha come foglia $\mathfrak{v} \vdash \emptyset$ 
                oppure $\mathfrak{v} \vdash \qedsymbol$.
\end{itemize}

\cleardoublepage

\part{Logica dei Predicati}
% TEX root = ../../main.tex

\chapter{Introduzione e Sintassi}
Si può immaginare la Logica del Primo Ordine come ``costruita'' 
sulla base della logica proposizionale. Il sillogismo aristotelico
\begin{prooftree}
        \AxiomC{Ogni uomo è mortale}
        \AxiomC{Socrate è un uomo}
        \BinaryInfC{Quindi Socrate è mortale}
\end{prooftree}
espresso come possibile nella Logica proposizionale diventa 
\begin{prooftree}
        \AxiomC{$p$}
        \AxiomC{$q$}
        \BinaryInfC{$r$}
\end{prooftree}
Ci si chiede se sia vero, quindi: è forse 
$$
p, q \models r
$$
Con DPLL ci si chiede se 
$\{p, q, \neg r\}$ sia insoddisfacibile. 
Il primo passo di DPLL afferma
\begin{prooftree}
        \AxiomC{$\emptyset \vdash \{\{p\}, \{q\}, \{\neg r\}\}$}
        \UnaryInfC{$\mathfrak{v}(p) = 1 \vdash \{\{q\}, \{\neg r\}\}$}
        \UnaryInfC{$\mathfrak{v}(p)=1, \mathfrak{v}(q) = 1 \vdash \{\{\neg r\}\}$}
        \UnaryInfC{$\mathfrak{v}(r) = 0, \mathfrak{v}(p)=1, \mathfrak{v}(q) = 1 \vdash \emptyset$}
\end{prooftree}
e pertanto 
$$
p, q \models r \text{ è falso }
$$
che va ovviamente contro la nostra intuizione. 
Per gestire argomentazioni
razionali come il sillogismo aristotelico è necessario dotarsi di un linguaggio 
più \textit{espressivo} rispetto alla Logica Proposizionale, arricchendone 
la Sintassi con nuovi operatori, variabili, costanti e la Semantica assegnando 
un modo di interpretare i nuovi ``ingredienti'' del linguaggio in modo da poter 
rappresentare situazioni più raffinate che nella Logica Proposizionale. 

Si arriverà a scrivere 
$$
\forall x (U(x)\rightarrow M(x)), U(s) \models M(s)
$$
per indicare il sillogismo aristotelico.
Oltre ad andare a vedere come mettere in piedi, di primo acchito a livello sintattico, 
tutta questa struttura, si studierà anche il modo per \textit{risolvere} 
delle asserzioni, riutilizzando le tecnologie introdotte per la logica proposizionale.
In particolare, per fare un esempio introduttivo, si tornerà a chiedersi se 
il sillogismo aristotelico sia valido ponendosi  il quesito 
$$
\{\{\neg U(x), M(x)\}, \{U(s)\}, \{\neg M(s)\}\} \text{ è soddisfacibile?}
$$

Si consideri nuovamente 
$$
\text{ Ogni uomo è mortale }
$$
I modi di designare direttamente dell'\textit{Universo} sono un ingrediente 
importante della Logica dei Predicati, fornendo per esempio il modo di 
affermare che Socrate sia mortale, riferendosi ad un preciso elemento. 
Si necessita un modo per designare un elemento non preciso, in maniera indiretta: 
$$
\text{ Il padre di ogni uomo è un uomo }
$$
Si può concludere, quindi, che il padre di Socrate sia un mortale, 
nonostante sia una designazione indiretta di un individuo dell'Universo; 
si può inoltre tradurre quanto detto come 
\begin{prooftree}
        \AxiomC{$\forall x (U(x) \rightarrow M(x)$}
        \AxiomC{$\forall (U(x) \rightarrow U(p(x))$}
        \AxiomC{ $U(s)$}
        \TrinaryInfC{$M(p(s))$}
\end{prooftree}

Per decidere questa \textit{deduzione}, ci sarà qualcosa come 
$$
\{\{\neg U(x), M(x)\}, \{\neg U(x), U(p(x))\}, \{U(s)\}, \{\neg M(p(s))\}\}
$$
Quindi, scopriremo che non basterà sostituire al posto di $x$ la ``lettera'' 
$s$, ma sarà anche necessario considerare $p(s)$. A questo punto, sarà immediato 
arrivare alla conclusione che la situazione sia in realtà un po' più complicata: 
come si considera $p(s)$, si dovrebbe considerare anche $p(p(s))$, $p(p(p(s)))$...

Nel nostro caso, una possibile refutazione è la seguente: 
$$
\neg U(p(s)), M(p(s))
$$
istanziando la $x$ su $p(s)$, in questo modo si può risolvere 
rispetto a $\neg M(p(s))$: 
$$
\{\{\neg U(p(s))\}, \{\neg U(x), U(p(x))\}, \{U(s)\}\}
$$
e si prosegue istanziando $U(x)$ a $U(s)$, arrivando all'insieme vuoto. 
Vedremo, quindi, perché e quando si possono utilizzare queste istanzianzioni. 

\section{Sintassi della Logica del Primo Ordine}
Partiamo, dunque, fissando i dettagli sintattici, cioè il Linguaggio 
della Logica del Primo Ordine nella sua natura grammaticale. 

A livello proposizionale, si fissa $L$ come insieme infinito di lettere 
proposizionali, e si conclude il tutto, poiché da questo insieme si arriva 
direttamente a costruire ogni formula ammissibile nella Logica Proposizionale.

Innanzitutto, nella Logica del Primo Ordine vi sono diversi linguaggi detti 
\textbf{linguaggi elementari}; la situazione è simile alla situazione generica 
dei linguaggi formali, anche se tecnicamente $L$ sarebbe un alfabeto per e $F_L$ 
sarebbe il linguaggio formale, mentre in logica $L$ stesso è chiamato Linguaggio. 

Allo stesso modo, nella logica del Prim'Ordine ci sono dei linguaggi elementari, 
che specificano gli ``ingredienti'' per costruire le formule della Logica dei 
Predicati, le quali sarebbero il \textit{vero} linguaggio nel senso formale. 
In altri termini, in Logica dei Predicati 
il termine \textit{linguaggio elementare} indica l'insieme di elementi che andranno 
a costruire l'insieme delle formule. 

\begin{defi}[Linguaggio Elementare]
I linguaggi elementari sono definiti come $L = (\mathcal{P}, F, \alpha, \beta)$  dove $P$ è un insieme di 
simboli, detti \textbf{predicati} (o simboli di predicato), che deve 
essere diverso dall'insieme vuoto $\mathcal{P} \neq \emptyset$; $F$ è l'insieme 
di simboli detti \textbf{di funzione}, disgiunto da $\mathcal{P}$: $\mathcal{P} \cap F = \emptyset$; 
$\alpha$ è una funzione $P \rightarrow \mathbb{N}$ assegna l'arità a ogni 
$\mathcal{P} \in P$ e, infine, $\beta$ è  l'arità di ogni $f \in F$.
\end{defi}

Ogni elemento del linguaggio varia in base al linguaggio stesso, tuttavia 
vi sono ``ingredienti'' intrinsecamente comuni a tutti i linguaggi elementari:  
l'insieme $V$ (o $Var$) infinito di simboli detti \textbf{variabili individuali}, 
chiamate così perché la loro interpretazione sarà quella di un \textit{individuo 
generico} dell'universo del discorso
e l'insieme dei connettivi $\land, \lor, \neg, \rightarrow, \bot, \top, \iff, \cdots$; 
differentemente dalla logica proposizionale, i linguaggi elementari contengono 
anche i \textbf{quantificatori} $\forall, \exists$. 

Il linguaggio $F_L$ delle formule sul linguaggio elementare $L$ sarà definito 
a partire dai simboli in $L$ e dai possibili connettivi. 

\begin{oss}
        Esiste un predicato speciale nella logica del primo ordine 
        che a livello sintattico è uguale agli altri, ma la sua interpretazione 
        è fissata. Se $\mathcal{P}$ contiene il simbolo $'='$, la sua arità 
        sarà fissata $\alpha('=') = 2$ e $L = (\mathcal{P}, F, \alpha, \beta)$ 
        è detto \textit{linguaggio con identità}.
\end{oss}

\begin{defi}[Formule]
La definizione di \textbf{Formula di} $L$ ($F \in F_L$) è data per strati 
e induttivamente: il primo strato è quello dei \textit{termini}, 
i quali \textbf{non sono formule} ma si \textit{usano} per costruire formule.

\begin{defi}[$L$-termini]
        Sia $L = (\mathcal{P}, F, \alpha, \beta)$. 

        Allora l'insieme $T_L$ degli $L$-termini è definito come segue: 

        \paragraph{Base} 
        \begin{itemize}
                \item Ogni $x \in Var(L)$ è  un termine. 
                \item Per ogni $c \in F$ tale che $\beta(c) = 0$, $c$ è un termine, 
        detto \textbf{costante}. 
        \end{itemize}

        \paragraph{Passo Induttivo} Se $f \in F$ è tale che $\beta(f) = n$ e 
        $t_1, t_2, \cdots, t_n \in T_L$ (sono termini già costruiti), 
        allora $f(t_1, t_2, \cdots, t_n)$ è un $L$-termine. 
        
        Nient'altro è un $L$-termine.
\end{defi}

        L'insieme $F_L$ delle $L$-Formule è definito come segue. 

        \paragraph{Base} Se $p \in \mathcal{P}$ e $\alpha(p)=n$ e $t_1, \cdots, t_n \in T_L$
        allora $P(t_1, \cdots, t_n)$ è una $L$-Formula detta 
        \textbf{formula atomica}. Le formule atomiche sono il corrispettivo 

        \paragraph{Passo Induttivo} Se $A,B \in F_L$ allora anche 
        $A \land B$, $A \lor B$, $\neg A$, $A\rightarrow B$ sono $L$-Formule.
        Se $A \in F_L$ e $x \in V$ allora anche $\forall x A$ e $\exists x A$
        sono $L$-Formule. Se in $A$ non occorre $x$, $\forall x A$ 
        e $\exists x A$ sono comunque formule, come anche 
        $\forall x (\forall x A)$ e $\exists x (\forall x A)$. 

        Nient'altro è una $L$-Formula.
\end{defi}

\paragraph{Esercizio} Dare una nozione di \textit{certificato} per 
$L$-termini e per $L$-Formule analoga alla $L$-costruzione in logica proposizionale. 

\subsection{Terminologia}
Introduciamo alcune nozioni terminologiche per potersi riferire alla sintassi 
della Logica dei Predicati.
Si userà liberamente la scrittura semplificata di formule omettendo coppie di parentesi
quando questo non causa ambiguità, ossia invece di $(\forall x A)$ si scriverà $\forall x A$.

\begin{defi}[Termine Ground]
        Un termine è detto \textbf{chiuso} o \textbf{ground} se è costruito 
        senza utilizzare \textit{variabili}. 
\end{defi}

\begin{defi}[Variabile vincolata]
        Una occorrenza di una variabile $x$ in una Formula $A \in F_L$ è 
        detta \textbf{vincolata} se occorre all'interno di una sottoformula 
        di $A$ (ossia una formula che appare in ogni certificato della $L$-formula $A$) 
        del tipo $\forall x B$ o $\exists x B$. 
\end{defi}

\begin{defi}[Variabile libera]
Una variabile $x \in V$ occorre libera in $A \in F_L$ se e solo se qualche  
sua occorrenza è libera. Questa definizione permette di dare definizioni 
anche nelle situazioni come l'osservazione precedente.
\end{defi}

Per esempio, nella formula 
$$
A = (\forall x R(x,y)) \lor P(x)
$$
$x$ è sia vincolata che libera, mentre $y$ è libera. 
Nella formula 
$$
A' = \forall x (\forall y R(x,y)) \lor P(x)
$$

Sia $x$ che $y$ occorrono sempre vincolate. 

Questo ci permette di introdurre il concetto fondamentale sul quale lavoreremo: 
infatti, non ragioneremo più su \textit{formule} quando definiremo la semantica 
della Logica dei Predicati, ma si ragionerà su un particoalre tipo di formula: 
\begin{defi}[$L$-sentence o Enunciato]
        Un $L$-\textbf{enunciato} o $L$-\textbf{sentence}, detto anche formula 
        chiusa, è una $L$-formula senza occorrenze di variabili libere 
        (o senza variabili libere).
\end{defi}

\begin{defi}[Sostituzione]
        Con la notazione $A[t/x]$ con $A$ una formula, $x$ variabile 
        e $t$ un termine, si intende la formula ottenuta rimpiazzando 
        simultaneamente tutte e sole le occorrenze \textbf{libere} di $x$ 
        con $t$. 
\end{defi}

% TEX root = ../../main.tex
\chapter{Semantica della Logica del Primo Ordine}
\section{Semantica di Tarski}
La semantica esprime \textit{come e quando} un $L$-enunciato è vero (o falso) in una data 
circostanza. \`E necessario fissare e formalizzare la  nozione di \textit{circostanza}
o ``mondo possibile''. Nel livello Proposizionale, la circostanza è un 
\textit{assegnamento}. Una volta fatto ciò, sarà necessario formalizzare 
l'\textit{interpretazione} degli enunciati (e quindi dei termini) in ogni circostanza 
formalizzata. 

Un'idea per la formalizzazione del concetto di 
\textbf{circostanza}, dovuta al logico polacco Alfred Tarski, 
è che la \textit{verità} è una corrispondenza con lo stato di \textit{Fatto}. 

Per la \textbf{semantica tarskiana}, nel linguaggio naturale l'enunciato
``La neve è bianca'' è un enunciato vero se e solo se la neve è bianca. 
Pertanto, ci si può immaginare di matematizzare questo concetto 
tramite la teoria degli insiemi, affermando che esista un certo 
insieme chiamato \textit{Universo del discorso}, composto da vari individui,
tra i quali vi è un certo elemento che rappresenta la neve e, 
rispecchiando la nostra esperienza, si vuole che effettivamente sia 
vero che la neve sia bianca. Pertanto, ``si forza'' l'appartenenza 
della neve all'insieme degli oggetti del discorso che sono bianchi. 

In termini matematici, e per il concetto stesso della semantica tarskiana, 
è necessario considerare Universi in cui vi è la possibilità 
che la neve non sia bianca, ossia l'enunciato è falso. 

A priori, le opzioni \textit{esistono} tutte (la neve è bianca, la neve 
non è bianca), ma si può utilizzare la formalizzazione degli enunciati 
anche per modellizzare l'universo, ossia asserendo che il fatto 
che la neve sia bianca sia vera, e questo sappiamo già farlo 
con l'uso delle \textit{teorie}. 

Per esempio, rielaboriamo il solito sillogismo aristotelico: 
L'insieme di enunciati 
$$
\Gamma := \{\{\forall x (U(x) \rightarrow M(x))\}, \{U(s)\}\}
$$
benché non sia in FNC, è un \textit{teorema}, mentre 
$$
A := M(s)
$$
è la deduzione. Per ottenere un'infrastruttura sulla quale arrivare a compiere 
il calcolo deduttivo 
$$
\Gamma, \neg A \text{ è soddisfacibile?}
$$
è necessario ``concentrarsi'' sugli Universi in cui $\Gamma$ è vero. 
Si possono considerare diversi \textbf{Universi del discorso}, composti da vari 
\textit{individui}: un enunciato caratterizza alcuni  universi in cui 
un enunciato è vero a scapito di altri. 

Si consideri, ora,  un universo composto da altri universi: chiamando 
ciò che abbiamo considerato fino ad ora 
un \textit{universo} una $L$-Struttura, si costruisce l'insieme 
di tutte le $L$-Strutture; data una teoria $\Gamma$ si potrà identificare, 
nell'insieme di tutte le $L$-Strutture, tutte quelle che modellano $\Gamma$. 

Definiamo formalmente le $L$-Strutture:
\begin{defi}[$L$-Struttura]
        Sia dato un linguaggio elementare $L  = (\mathcal{P}, F, \alpha, \beta)$. 
        Una $L$-struttura è una coppia $\mathcal{A} = (A, I)$, dove 
        $A$ è un insieme detto ``universo del discorso'' o, semplicemente, 
        \textit{universo} (o dominio), che deve rispettare 
        $$
        A \neq \emptyset
        $$
        mentre $I$ è una funzione detta \textit{interpretazione}, definita 
        tale che per ogni simbolo di 
        predicato $P \in \mathcal{P}$ con arità $\alpha(P) = n$ si ha 
        $$
        I(P) \subseteq A^{n} 
        $$
        ossia $I(P)$ è un insieme di $n$-ple di elementi di $A$; inoltre, 
        per ogni $f \in F$, $\beta(f) = n$ si ha 
        $$
        I(f) : A^{n} \rightarrow A
        $$
        ossia $I(F)$ è una funzione dall'insieme delle $n$-ple di $A$ verso elementi 
        di $A$.
\end{defi}

La definizione di $L$-struttura data formulizza a livello del Primo ordine la 
nozione intuitiva di ``circostanza'' o ``mondo possibile''. A livello proposizionale, 
essa è formalizzata dalla nozione di assegnamento. 


\begin{oss}[Costanti]
        Se $c \in F$ e $\beta(c) = 0$ 
        $$
        I(c) : A^{0} \rightarrow A
        $$
        e $A^{0} = \{f: \emptyset  \rightarrow A\}$, la cui cardinalità è  
        $$ 
        |A^0| = |\{f:\emptyset\rightarrow A\}| = |\{\emptyset\}| = 1
        $$
        Pertanto 
        $$
        I(c): \{*\} \rightarrow A
        $$ 
(la notazione $\{*\}$
        rappresenta 
        l'insieme che contiene l'insieme vuoto), che equivale a 
        ``scegliere'' un elemento di $A$.
        Identifichiamo una funzione 
        $$
        g : \{*\} \rightarrow A 
        $$
        con l'elemento scelto da $g(*)$. In altre parole, le costanti sono funzioni
        zerarie.
\end{oss}

\begin{oss}[Predicati zerari]
        Se $p \in \mathcal{P}, \alpha(P) = 0$ è un simbolo di 
        predicato zerario, si ha 
        $$
        I(P) \subseteq A^0 ~~~~ I(P) \subseteq \{*\}
        $$
        I cui sottoinsiemi sono sé stesso $\{*\}$ e l'insieme vuoto $\emptyset$. 
        E quindi l'interpretazione di un predicato zerario svolge le funzioni 
        di una vecchia lettera proposizionale, in quanto l'interpretazione 
        di una proposizione può essere identificata tramite la sua funzione 
        caratteristica 
        $$
        \chi_P: A^n \rightarrow \{0,1\}
        $$
        definita come 
        $$
        \bar{a} \in A^n \in I(P) \iff \chi_P(\bar{a}) = 1
        $$
\end{oss}

\begin{oss}[Interpretazione fissa del predicato di Uguaglianza]
        Se $ = \in \mathcal{P}$ allora $L$ è un linguaggio con identità e 
        $\alpha(=) = 2$. L'interpretazione di ``='' è fissata: 
        se $\mathcal{A}=(A,I)$ è una $L$-struttura, allora 
        $$
        I(=) \subseteq A^2 
        $$
        è fissato e definito come 
        $$
        I(=):= \{(a,a) : a \in A\}
        $$
        e, in altre parole, $x = y$ è vero se $(I(x), I(y)) \in I(=)$.
\end{oss}

L'eccezione sull'interpretazione del predicato di uguaglianza, fissato per 
ogni linguaggio elementare che lo contiene, è giustificato dal fatto che con 
la potenza espressiva della Logica dei Predicati, si riesce a dire che una 
relazione binaria gode della proprietà riflessiva, simmetrica e transitiva.  

Ci si può anche spingere a dire che per ogni altro simbolo, sia di predicato 
che di funzione, quella relazione si comporta in maniera \textit{congruenziale}. 
Anche questo non è sufficiente per inchiodare l'identità, in quanto una congruenza 
su un insieme che non è l'identità rispetta gli stessi assiomi. 

In altre parole, o l'identià si codifica a forza, come accade ora nella Logica 
del Prim'Ordine, oppure se ne ha una versione molto indebolita, in quanto non si 
ha modo di distinguere con formule della Logica dei Predicati l'identità 
da una congruenza sullo stesso insieme.  

\subsection{Semantica degli enunciati in ogni $L$-struttura}
Abbiamo dato la nozione di $L$-Struttura ma non si è ancora detto 
come si interpretano gli enunciati, formalmente, cioè quando un 
enunciato è vero o meno in tale $L$-Struttura. Intuitivamente, quello che 
si vuole fare è quanto anticipato, ossia scegliere le $L$-Strutture che 
verificano una certa teoria. Tuttavia, questo deve essere eseguito 
seguendo una definizione rigorosa e sistematicamente.

Si procederà per strati, ossia definendo prima come interpretare 
i termini e successivamente formule ed enunciati, induttivamente. 
Si deve catturare l'idea intuitiva: ``la neve è bianca'' verrà 
formalizzato in qualcosa come $B(n)$, dove $B$ è un predicato 
unario e $n \in A$ è una costante (quindi esiste una funzione zeraria 
il cui risultato è $n$). Si vorrà formalizzare il tutto in modo tale da avere una 
qualche $L$ struttura tale che $I(n)$ sia contenuta in $I(B)$; quindi si avrà che 
$B(n)$ è vera in $\mathcal{A}$ se e solo se $I(n) \in I(B)$: 
$$
\mathcal{A} \in B(n) \iff I(n) \in I(B)
$$

Il passo più ostico nella definizione di interpretazione di termini 
e formule (ed enunciati) sarà nell'interpretazione dei quantificatori; 
per esempio, si supponga che si vuole stabilire se 
$$
\mathcal{A} \models \exists x A
$$
che significa: se esiste un $a \in A$ tale che $\mathcal{A} \models A[a/x]$
e  analogamente per l'altro quantificatore 
$$
\mathcal{A} \models \forall x A
$$
che è vero se e solo se per ogni $a \in A$ vale $\mathcal{A} \models A[a/x]$. 

Purtroppo $a$ non è un elemento sintattico, ma è un elemento semantico, 
ossia non è un termine e pertanto $A[a/x]$ non è definita.
Non c'è nessuna ragione per cui ogni elemento di una $L$-struttura 
debba avere, a priori, un nome: piuttosto, è vero il contrario. Se si vuole 
parlare ad esempio dei reali, si vorrebbe farlo con un linguaggio 
elementare $L$ che contiene solo finitamente molti simboli. 

\begin{defi}[Espansione di un $L$-Struttura]
        Dato un linguaggio $L$ e una $L$-struttura $\mathcal{A}(A,I)$, si definisce 
        una \textbf{espansione} di $L$ in un nuovo linguaggio $L_{\mathcal{A}}$
        come segue: 
        per ogni $a \in A$ arricchisci l'insieme delle costanti di $L$ con un 
        nuovo simbolo $\bar{a}$ (nome di $a$) che viene interpretato in $a$. 

        $$
        L_{\mathcal{A}} := (\mathcal{P}, F^{\mathcal{A}}, \alpha, \beta^{\mathcal{A}})
        $$
        con
        $$
        F^{\mathcal{A}} := F\cup \{\bar{a}: a \in A\}
        $$
        e 
        $$
        \begin{cases}
                \beta^{\mathcal{A}} = B & f \in F\\
                \beta^{\mathcal{A}}(\bar{a}) = 0 & a \in A\\
        \end{cases}
        $$
        E inoltre $I(\bar{a}) := a $ per ogni $a \in A$. 
\end{defi}

\begin{defi}[Interpretazione degli $L_{\mathcal{A}}$-termini ground]
        Dati $L = (\mathcal{P}, F, \alpha, \beta)$ e $\mathcal{A} = (A,I)$ 
        e la $L_{\mathcal{A}}$-struttura espansa, si definisce induttivamente: 
        \begin{itemize}
                \item{\textbf{Base}} ogni $c \in F$ tale che $\beta(c) = 0$ ha un'interpretazione $I(c)$
                        già definita, dalla definizione di $L$-struttura. Invece, 
                        $\bar{a} \in F^{A} = F \cup \{\bar{a} : a \in A\}$ 
                        ha come interpretazione $I(\bar{a}) = a \in A$. 
                \item{\textbf{Passo Induttivo}}
                        Sia $f \in F$ un simbolo di funzione con arità $\beta(f) = n$ e 
                        siano $t_1, t_2, \cdots, t_n$ $L_{\mathcal{A}}$-termini ground. 
                        Allora $I(F(t_1, \cdots, t_n)):= I(f)(I(t_1), \cdots, I(t_n))$. 
        \end{itemize}
\end{defi}

\begin{defi}[Interpretazione degli Enunciati]
        Sia data $\mathcal{A} = (A,I)$ una $L$-struttura; allora, si definisce 
        induttivamente la nozione $\mathcal{A} \models A$ per ogni 
        $L_{\mathcal{A}}$-enunciato $A$, ossia $A$ è vera in $\mathcal{A}$. 

        \paragraph{Base}
        Sia 
        $A = P(t_1, \cdots, t_n)$ per $P \in \mathcal{P}$ con $\alpha(P) = n$ 
        e $t_1, \cdots, t_n$ $L_{\mathcal{A}}$-termini ground. Allora 
        $$
        \mathcal{A} \models P(t_1, \cdots, t_n) \iff I(t_1), \cdots, I(t_n) \in I(P) \subseteq A^n
        $$
        \paragraph{Passo induttivo}
                \begin{itemize}
                \item $\mathcal{A} \models A_1 \land A_2 \iff \mathcal{A} \models A_1 \land \mathcal{A} \models A_2$
                \item $\mathcal{A} \models A_1 \lor A_2 \iff \mathcal{A} \models A_1 \lor \mathcal{A} \models A_2$
                \item $\mathcal{A} \models \neg A_1  \iff \mathcal{A} \nvdash A_1 $
                \item $\mathcal{A} \models A_1 \rightarrow A_2 \iff \mathcal{A} \models A_1 \land \mathcal{A} \models A_2$
                \item $\mathcal{A} \models\forall x A \iff \mathcal{A} \models A[\bar{a}/x] \text{ per ogni } a$
                \item $\mathcal{A} \models\exists x A \iff \mathcal{A} \models A[\bar{a}/x] \text{ per almeno un } a$
                \end{itemize}
                Si noti che $\bar{a}$ è un termine, mentre $x$ è una variabile 
                individuale, quindi $A$ diventa un $L_{\mathcal{A}}$-enunciato.
\end{defi}

Abbiamo quindi definito induttivamente $\mathcal{A}\models A$ per gli 
$L_{\mathcal{A}}$-enunciati $A$. A questo punto ci si può ``dimenticare'' degli 
$L_{\mathcal{A}}$-enunciati che non sono $L$-enunciati, 
ossia quelle formule in cui vi è almeno una variabile libera.  

\begin{defi}[Chiusura universale di una formula con variabili libere]
Se $A$ è una formula ben formata, ossia $A \in F_L$ ed eventualmente,
non un enunciato, quindi $FV(A) \neq \emptyset$, dove $FV$ è definito come l'insieme 
di variabili libere che appaiono in $A$, ci si può chiedere se 
$$ 
\mathcal{A} \models A
$$
Definito, quindi, 
$$
FV(A) = \{x_1, \cdots, x_n\} 
$$
si definisce la chiusura universale di $A$ 
$$
\forall[A] := \forall x_1 \forall x_2 \cdots \forall x_n A
$$
e si postula che
$$
\mathcal{A} \models A \iff \mathcal{A} \models \forall[A]
$$
\end{defi}
ossia $\mathcal{A}$ modella una formula ben formata $A$ se e solo se 
$\mathcal{A}$ rende vera la sua chiusura universale, che è un $L$-enunciato.

\begin{oss}
        Se $L$ è un linguaggio con identità (o con uguaglianza), ossia 
        $L = (\mathcal{P}, F, \alpha, \beta)$ e $= \in \mathcal{P}$, 
        $\alpha(=) = 2$, 
        allora, ricordando che in ogni $L$-struttura $\mathcal{A} = (A,I)$, 
        si deve avere per definizione $I(=) = \{(a,a) : a \in A\}$. 
        Pertanto $\mathcal{A} \models t_1 = t_2$ (che, più correttamente, 
        andrebbe scritto in forma postfissa ``$=(t_1, t_2)$'' se e solo se 
        $I(t_1) = I(t_2)$. 
\end{oss}

\subsection{Definizione alternativa di $\mathcal{A} \models A$} 
In questa definizione alternativa,  le variabili sono oggetti di interesse; 
tuttavia le due definizioni, quella su $L\rightarrow L_{\mathcal{A}}$ e quella
che andiamo a dare, sono equivalenti. 
\begin{defi}[Ambiente]
        Un \textbf{ambiente di interpretazione} in una $L$-struttura $\mathcal{A} = (A,I)$ è 
        una funzione 
        $$
        \mathfrak{v} : Var \rightarrow A
        $$
        ossia $\mathfrak{v}(x) \in A$ è un elemento di $A$.         
\end{defi}
\begin{defi}[Variante di ambiente]
        La \textbf{variante di ambiente} servirà a dare la semantica alle espressioni 
        quantificate ed è indicato $\mathfrak{v}[a/x](y)$ e si definisce 
        $$
        \mathfrak{v}[a/x](y) :=
        \begin{cases}
                \mathfrak{v}(y) & y \neq x \\
                a & y = x
        \end{cases}
        $$
\end{defi}
\begin{defi}[Interpretazione degli $L$-termini]
        L'interpretazione degli $L$-termini, anche aperti, per ogni 
        $x \in Var$ e $\mathfrak{v} : Var \rightarrow A$ è 
        definita come 
        $$
        I_{\mathfrak{v}}(x) := \mathfrak{v}(x)
        $$
        e si può ora passare a definire 
        $$
        I_{\mathfrak{v}}(F(t_1, \cdots, t_n)) := (I(F))(I_{\mathfrak{v}}(t_1), \cdots, I_{\mathfrak{v}}(t_n))
        $$
        Si dirà
        $$
        \mathcal{A}\models_{\mathfrak{v}} A 
        $$
        se $\mathcal{A}$ rende vera $A$ nell'ambiente $\mathfrak{v}: Var \rightarrow A$, 
        e si dirà 
        $$
        \mathcal{A} \models_{\mathfrak{v}} P(t_1, \cdots, t_2) \iff (I_{\mathfrak{v}}(t_1), \cdots, I_{\mathfrak{v}}(t_n)) \in I(P)
        $$
         e quindi 

         \begin{itemize}
                 \item  $\mathcal{A} \models_{\mathfrak{v}} A_1 \land A_2 \iff \mathcal{A} \models A_1 \land \mathcal{A} \models A_2$
                 \item $\mathcal{A} \models_{\mathfrak{v}}A_1 \lor A_2 \iff \mathcal{A} \models A_1 \lor \mathcal{A} \models A_2$
                 \item $\mathcal{A} \models_{\mathfrak{v}}  \neg A_1  \iff \mathcal{A} \nvDash A_1 $
                 \item $\mathcal{A} \models_{\mathfrak{v}} A_1 \rightarrow A_2 \iff \mathcal{A} \models A_1 \land \mathcal{A} \models A_2$
                 \item $\mathcal{A} \models_{\mathfrak{v}} \forall x A \iff \mathcal{A} \models_{\mathfrak{v}} A[a/x] \text{ per ogni } a$
                 \item $\mathcal{A} \models_{\mathfrak{v}} \exists x A \iff \mathcal{A} \models_{\mathfrak{v}} A[a/x] \text{ per almeno una  } a$
        \end{itemize}
        A questo punto si  può dire che $\mathcal{A} \models A$ se e solo se 
        per ogni $\mathfrak{v}: Var \rightarrow A$ si ha $\mathcal{A} \models A$. 
\end{defi}

\noindent
Si esprime, ora, un'idea intuitiva di come avvenga il processo di interpretazione 
di un $L$-enunciato. Nella prima definzione si ha il linguaggio $L$ e delle 
$L$-strutture che vengono associate. 
$$
L 
\leadsto
\begin{cases}
        \mathcal{A}  \leadsto L_{\mathcal{A}} \\
        \mathcal{B}  \leadsto L_{\mathcal{B}} \\
        \mathcal{C}  \leadsto L_{\mathcal{C}} \\
        \cdots
\end{cases}
$$
Per ogni $L$-struttura si crea l'espansione associata. 
Nella seconda definizione si ha nuovamente il linguaggio $L$ e delle $L$-strutture, 
tuttavia esse non si espandono ma vi sono un numero di ambienti da considerare: 

$$
L \leadsto 
\begin{cases}
        \mathcal{A} \begin{cases}
                \mathcal{A}, \mathfrak{v}_1 \\
                \mathcal{A}, \mathfrak{v}_2 \\
                \mathcal{A}, \mathfrak{v}_3 \\
                \cdots \\
        \end{cases} \\
        \mathcal{B} \begin{cases}
                \mathcal{B}, \mathfrak{v}_1\\
                \mathcal{B}, \mathfrak{v}_2\\
                \mathcal{B}, \mathfrak{v}_3\\
                \cdots                     
        \end{cases} \\
        \cdots
\end{cases}
$$

\section{Terminologia e Nozioni}

\begin{defi}[$L$-Teoria]
        Una $L$-Teoria è un insieme di $L$-enunciati. 
        Se $\Gamma$ è una $L$-Teoria e $\mathcal{A}$ è una $L$-struttura, 
        si dice che $\mathcal{A}$ è modello di $\Gamma$ e si scrive 
        $\mathcal{A} \models \Gamma$ se e solo se $\mathcal{A} \models \gamma$ 
        per ogni $\gamma \in \Gamma$. 

        La $L$-teoria $\Gamma$ è soddisfacibile e solo se esiste almeno un 
        $\mathcal{A}$ tale che $\mathcal{A} \models \Gamma$, 
        altrimenti $\Gamma$ è insoddisfacibile.

        Sia $\Gamma$ una $L$-teoria e $A$ un $L$-enunciato. 
        Allora $A$ è conseguenza logica di $\Gamma$ e si scrive $\Gamma \models A$ 
        se e solo se $A$ è vera in ogni modello di $\Gamma$, ossia 
        per ogni $L$-struttura $\mathcal{A}$, se $\mathcal{A} \models \Gamma$, 
        allora $\mathcal{A} \models A$. 
\end{defi}

\begin{defi}[Verità Logica]
        Un $L$-enunciato $A$ è detto \textbf{verità logica} se e solo se 
        per ogni $L$-struttura $\mathcal{A}$, $\mathcal{A} \models A$. 
\end{defi}
Il concetto di Verità Logica è analogo al concetto di tautologia nella 
Logica Proposizionale. 
Un esempio di verità logica è $\forall x (x = x)$. 

\noindent
Lo scopo della nostra indagine sarà, d'ora in poi, cercare di stabilire se
$$
\Gamma \models A
$$
dati $\Gamma$  una $L$-teoria e $A$ un $L$-enunciato. 
Se $\Gamma$ è finito, ossia $\Gamma = \{\gamma_1, \cdots, \gamma_n\}$, 
varrà nuovamente il fatto 
$$
\Gamma \models A \iff \gamma_1 \land \cdots \land \gamma_n \land  \{\neg A\} \text{ insodd.}
$$
ossia ci si riduce all'analisi di una singola formula, 
mentre invece se $\Gamma$ è infinito, possiamo solo chiederci se
$$
\Gamma \models A \iff \Gamma \cup \{\neg A\} \text{ insodd.}
$$
Anche nel caso in cui $\Gamma$ è finito, ci saranno casi in cui il processo 
di deduzione sarà solo \textit{semidecidibile}, mentre nella Logica Proposizionale 
il problema di soddisfacibilità era decidibile. 
\subsection{Esempi di $L$-Teorie} 

\subsubsection{Congruenze su un $L$-Linguaggio $L$}
Data una relazione $R$, si vorrebbe modellarla come congruenza.
$$
\Gamma := 
\begin{cases}
        \forall x ~~ R(x,x) \text{ riflessiva } \\
        \forall x \forall y ~~ R(x,y) \rightarrow R(y,x)  \text{ simmetrica } \\
        \forall x \forall y \forall z ~~ R(x,y) \land R(y,z) \rightarrow R(x,z)  \text{ transitiva }\\

        \forall x_1 \cdots \forall x_n \forall y_1 \cdots \forall y_n 
        \begin{cases}
        (R(x_1,y_1) \land \cdots \land R(x_n,y_n)) \rightarrow R(f(x_1, \cdots, x_n), f(y_1, \cdots y_n)) \\
        (R(x_1, y_1) \land \cdots R(x_n, y_n)) \rightarrow (P(x_1, \cdots x_n) \rightarrow P(y_1, \cdots, y_n))
        \end{cases}

\end{cases}
$$
I primi tre $L$-enunciati (assiomi di $\Gamma$) assiomatizzano una 
relazione d'equivalenza, ossia
$$
\text{ se }\mathcal{A} \models \Gamma \text{ allora } I(R) \text{ è una relazione d'equivalenza}
$$
Il quarto $L$-enunciato assiomatizza la congruenza $R$ rispetto ad
una generica $f \in F$ con $\beta(f) = n$, mentre il quinto fa 
la stessa cosa rispetto un generico predicato $P \in \mathcal{P}$. 

\subsubsection{Relazione d'ordine (parziale)}
$$
        \Gamma_{po} := 
        \begin{cases}
                \forall x R(x,x) & \text{ riflessiva }\\
                \forall xy (R(x,y) \land R(y,x)) \rightarrow =(x,y) & \text{ simmetrica }\\
                \forall x \forall y \forall z (R(x,y) \land R(y,z)) \rightarrow R(x,z) & \text{ transitiva }\\
      
        \end{cases}
$$
Si noti l'utilizzo del predicato $=$ d'uguaglianza, pertanto il Linguaggio 
è dotato di tale predicato. 
Un esempio, l'insieme dei naturali con la relazione di minore o uguale 
modellano $\Gamma$: 
$$
(\mathbb{N}, \leq) \models \Gamma
$$
(la notazione utilizzata significa che $I(R) = \leq$)
così come i naturali con l'ordine di divisibilità: 
$$
(\mathbb{N}, |) \models \Gamma
$$


\paragraph{Relazione d'ordine (totale)}
Esiste un modo per costruire una $L$-Teoria che ``allarga'' $\Gamma$ che è modello 
di uno e non dell'altra? Effettivamente, il primo è un ordine totale, mentre 
il secondo non lo è, ossia vi sono elementi tali che né $m | n$ né $n |m $; se 
si riesce ad esprimere la totalità con un enunciato, si riesce a discernere 
uno dall'altro: 
\begin{align*}
        \Gamma_{to} := \Gamma_{po} \cup \{\forall x \forall y (R(x,y) \lor R(y,x))\}
\end{align*}

Aggiungendo nuovi assiomi si possono caratterizzare altri tipi di ordine: 
ad esempio (esercizio), si può caratterizzare un ordine che abbia un elemento 
minimo. 

\subsubsection{Teoria dei gruppi}
Un gruppo è una struttura algebrica. 
Un gruppo è un modello dell'insieme di assiomi $\Gamma_G$ sul linguaggio 
$L = (\mathcal{P}, F, \alpha, \beta)$. 
$$
\Gamma_G := 
        \begin{cases}
                \forall x \forall y \forall z  ~~ ((x * y)*z) = (x*(y*z))  & \text{ associatività}\\
                \forall x (x * e) = x \land (e * x) = x  & \text{ elemento neutro}\\
                \forall x (x * x^{-1}) = e \land (x^{-1} * x = e) & \text{ invertibilità}
        \end{cases}
$$
con $\mathcal{P} = \{ = \}$, $F = \{*, e, ()^{-1}\}$ e relative 
arità. 
$(G, \cdot, ^{-1}, 0) \models \Gamma_G$ se e solo se $G$ è un gruppo; in 
altri termini un gruppo è una struttura con un'operazione associativa in cui 
ogni elemento è invertibile. 
Per esempio, $(\mathbb{Z}, +, -, 0) \models \Gamma_G$ e 
$(\mathbb{Q}\setminus{0}, \cdot, ^{-1}, 1) \models \Gamma_G$. 

\paragraph{Formulazione alternativa}
Una seconda formulazione è la seguente
$$
        \Gamma_{G_2} := 
                \begin{cases}
                        \forall x \forall y \forall z  ~~ ((x * y)*z) = (x*(y*z)) & \text{ associatività}\\
                        \forall x (x * e) = x \land (e * x) = x & \text{ elemento neutro } \\
                        \forall x \exists y (x * y) = e \land (y * x) = e & \text{ esistenza inverso} \\
                \end{cases}
$$
Questa $L$-Teoria rimuove la specifica dell'inverso. Se si ottiene un gruppo 
che modella l'altra formulazione, esso andrà bene anche per questo, ma non è detto 
il viceversa. 
Ora, infatti, non è più necessario specificare l'operazione inverso, 
e si può quindi affermare $(\mathbb{Z}, +, 0) \models \Gamma_{G_2}$, $(Q, +, 0) \models \Gamma_{G_2}$, 
eccetera. Come anticipato, questo vale anche per i gruppi esplicitati prima: 
$(\mathbb{Q}\setminus{0}, \cdot, ^{-1}, 1) \models \Gamma_{G_2}$ e così via, 
anche se l'inverso non viene utilizzato come simbolo esplicito. 
Potremmo quindi utilizzare, al posto del simbolo dell'inverso, qualsiasi cosa e 
rimarrà comunque un modello di $\Gamma_{G_2}$: 
$$
(\mathbb{Z}, +, +1, 0) \models \Gamma_{G_2}
$$
Ma ovviamente non è vero il contrario, ossia 
$$
(\mathbb{Z}, +, +1, 0) \nvDash \Gamma_G
$$
in questo senso, abbastanza sottile, le due formulazioni non sono equivalenti. 
\subsubsection{Tipi di dato: stack}
Definiamo il linguaggio $L= (\mathcal{P}, F, \alpha, \beta)$ 
con 
$$
\mathcal{P} = \{Stack, Elem, =\}
$$
e 
$$
F = \{push, pop, top, nil\}
$$
$$
        \Gamma_{Stack} := 
        \begin{cases}
                \forall x (Stack(x) \lor Elem(x)) \\
                \forall x (\neg Stack(x) \lor \neg Elem(x)) \\
                \forall x \forall y ((Stack(x) \land Elem(y)) \rightarrow top(push(x,y)) = y) \\
                \forall x \forall y ((Stack(x) \land Elem(y)) \rightarrow pop(push(x,y)) = x) \\
                \forall x (Stack(x) \rightarrow push(pop(x),top(x)) =x) \\
        \end{cases}
$$

\subsubsection{Aritmetica di Peano}

Sia $L_{P_A} = ( \mathcal{P}, F, \alpha, \beta)$, con 
$\mathcal{P} = \{=\}$, $\alpha(=) = 2$, $F =  \{ + , *, s, 0,\}$ 
$\beta(+) = \beta(*) = 2$, $\beta(s) = 1$, $\beta(0) =0$. 

L'aritmetica di Peano afferma: 
\begin{itemize}
        \item $\forall x \neg (=(0,s(x)))$, ossia nessun numero ha come successore $0$. 
        \item $\forall x, \forall y (s(x) = s(y)) \rightarrow (x = y)$
        \item $\forall x (x + 0 = x)$
        \item $\forall x \forall y (x + s(y)) = s(x+y)$
        \item $\forall x (x * 0 = 0)$
        \item $\forall x \forall y (x * s(y)) = (x + (x*y))$
        \item $(P[0/x] \land \forall x (P[x/x] \rightarrow P[s(x)/x])) \rightarrow \forall x P[x/x]$ 
\end{itemize}
L'ultimo postulato è una codifica dell'induzione sui numeri naturali. 

\chapter{Complessità Computazionale e Deduzione Automatica}

In questo momento, per decidere se una formula $F \in F_L$ che menziona $n$ lettere 
proposizionali diverse sia soddisfacibile o meno, l'algoritmo che conosciamo
computa la sua tabella di verità, analizzando quindi $2^n$ possibilità, ossia 
un numero esponenziale. 

Cominciamo definendo cosa sia un problema di 
decisione:
\begin{defi}[Problema di Decisione]
 Dato un alfabeto finito $\Sigma$ e un sottoinsieme non necessariemente 
 finito di stringhe finite  
  $$
 L \subseteq \Sigma^* 
 $$
 il problema di decisione consiste nel affermare se una precisa 
 stringa finita appartenga o meno a $L$, che viene chiamato Linguaggio, 
dove $\Sigma^*$ è l'insieme di tutte le stringhe di lunghezza finita, 
chiamate anche \textit{parole}, costruite 
coi simboli di $\Sigma$. 
\end{defi}

Ad esempio, fissato 
$$
\Sigma = \{\land, \lor, \neg, \rightarrow, (, ), p, |\}
$$
si possono definire alcuni problemi, per esempio
$$
F_L \subseteq \Sigma ^*
$$
dove il problema risiede nel capire se una certa formula di lunghezza finita
risiede in $F_L$ oppure no. 
Vi sono dei problemi principalmente di natura sintattica, come appunto $F_L$ 
ma anche $NNF$, $CNF$ e $DNF$. Vi sono problemi in cui la natura semantica 
è decisiva, per esempio $SAT$, dove 
$$
w \in \Sigma^* : w\in SAT \iff w \in F_L \land w \text{ è soddisfacibile}
$$
oppure $TAUTO$, dove una stringa finita appartiene a $TAUTO$ se e solo se 
è una formula ed è una tautologia, ossia
$$
w \in \Sigma^* : w \in TAUTO \iff w \in F_L \land \models w
$$
e analogamente $UNSAT$, il problema di decidere se una certa formula è
insoddisfacibile.

\section{Complessità Computazionale}
\subsection{Efficienza}
Alcuni dei problemi elencati precedentemente possono essere risolti (decisi) 
in maniera efficiente, ossia data una \textit{parola} $w$ decidere se 
appartiene ad un certo linguaggio: esempi di soluzioni efficienti sono 
quelle che riguardano tutti i problemi sintattici, risolvibili in tempo 
al massimo quadratico e in realta anche lineare con alcune tecniche di parsing. 

I problemi che riguardano la semantica, invece, sono tipicamente più complessi
da risolvere e infatti, per quanto conosciamo fino ad ora, si possono risolvere 
solo (nel caso peggiore) con un analisi dell'intera tabella di verità della 
formula, quindi con un numero esponenziale di calcoli. \`E importante rimarcare 
nuovamente che benché sussista questa difficoltà, verificare che 
un singolo assegnamento verifichi la formula è invece un calcolo semplice 
che richiede un tempo decisamente più contenuto rispetto al problema di 
decidibilità.

Quindi, vi sono dei problemi decidibili efficientemente, 
dei problemi decidibili molto difficilmente e verificabili facilmente 
e dei problemi decidibili molto difficilmente e verificabili difficilmente; 
da qui parte una gerarchia di classi di complessità infinita, ma a noi interesseranno 
solo le prime due. 
 
Per essere più precisi, si fissa un modello astratto di computazione, che fornisca 
la nozione di \textit{passo elementare}, in modo da poter ragionare precisamente 
sull'efficienza dei problemi. 
Qualunque modello di computazione che costituisca un modello realistico 
di calcolatore, con associata una nozione di passo elementare per costruire 
algoritmi, classifica i problemi nello stesso modo. 

\begin{oss}[Tesi di Church-Turing]
  Ogni modello ragionevole di computazione è equivalente.
\end{oss}

Uno dei modelli di computazione è quello delle Macchine di Turing (MdT): un'idea 
della Macchina di Turing è immaginarla come una macchina che lavora su un 
nastro infinito in lettura e scrittura, mantenendo uno stato e operando su 
una singola porzione di nasto in lettura utilizzando un programma per 
muoversi tra gli stati e scrivere sul nastro. 

Si può passare ora alla definizione formale delle classi dei problemi: 
\begin{defi}[Classe $\mathbb{P}$]
  La classe dei problemi ``efficientemente decidibili'' è definita, con una 
  certa ideologia sottesa, come tutti quei problemi che possono essere 
  risolti in tempo polinomiale, ossia esiste una certa Macchina di Turing T 
  e un polinomio $p: \mathbb{N}\rightarrow \mathbb{N}$ per i quali per ogni 
  $w \in \Sigma^*$  la computazione della MdT sull'input $w$, 
  denotato $T(w)$ termina entro $p(||w||)$ passi e per ogni $w \in L$ si 
  ha che $T(w)$ accetta il problema e per ogni $w \notin L$  si ha 
  che $T(w)$ non accetta, ossia risolve il problema.
\end{defi}

\begin{defi}[Classe $\mathbb{NP}$]
  La classe dei problemi ``verificabili efficientemente'' è definita come 
  tutti quei problemi tali per cui esiste una Macchina di Turing deterministica 
  e due polinomi $p,q:\mathbb{N}\rightarrow \mathbb{N}$ tali che per ogni 
  $w \in \Sigma^*$ e ogni $z \in \Gamma^*$, ossia un \textbf{certificato}, 
  che si può immaginare prodotto oracolarmente, si ha 
  che $T(w,z)$ termina entro $p(||w||)$ passi e si ha che per ogni 
  $w \in L$ esiste $z \in \Gamma^*$ tale che $||z|| \leq q(||w||)$ (ossia 
  il certificato è sufficientemente corto) e $T(w,z)$ accetta e per ogni 
  $w \notin L$ si ha che per ogni possibile $z \in \Gamma^*$ sufficientemente corto
  $T(w,z)$ rifiuta, ossia ``valida'' il certificato.
\end{defi}

\begin{defi}[Riducibilità]
        Un problema $L_1 \subseteq \Sigma^*$ è \textbf{riducibile in tempo polinomiale} 
        a un altro problema $L_2 \subseteq \Gamma^*$ se e solo se esistono
        una Macchina di Turing $T_{{L_1}, {L_2}}$ e un polinomio
        $p: \mathbb{N} \rightarrow \mathbb{N}$ tale che per ogni $w \in L_1$
        $T(w)$ trasfroma $w$ in $w' \in \Gamma^*$ in un numero di passi 
        minore o uguale a $p(||w||)$. 
\end{defi}

Per indicare la relazione di riducibilità tra due problemi si indica la notazione 
$L_1 \preceq_p L_2$ per indicare che il primo problema è riducibile polinomialmente 
al secondo; la nozione di riducibilità è utile poiché rende possibile risolvere 
istanze del problema $L_1$ ``riscrivendole'' come se fossero istanze di $L_2$
(chiaramente questa utilità si verifica quando è più facile risolvere $L_2$ di 
$L_1$). 
Un esempio di riducibilità polinomiale è 
$$
TAUTO \preceq_p UNSAT
$$
poiché la trasformazione $w \rightarrow \neg w$ è semplice e si sa che 
$w$ è tautologica se e solo se $\neg w$ è insoddisfacibile.

\begin{defi}[Problemi $\mathbb{NP}$-completi]
        Un problema appartenente alla classe $\mathbb{NP}_c$ è un problema 
        $L \subseteq \Sigma^*$ se e solo se 
        \begin{itemize}
                \item $L \in \mathbb{NP}$
                \item ogni $L' \in \mathbb{NP}$ è tale che $L' \preceq_p L$ 
        \end{itemize}
        La seconda proprietà si chiama $\mathbb{NP}$-hardness. 
\end{defi}
Come corollario della definizione dei problemi $\mathbb{NP}_c$, si ha che risolvendo 
polinomialmente un problema di tale classe si dimostra che 
$$
\mathbb{P} = \mathbb{NP}
$$

\begin{teo}[di Cook-Levin]
        $SAT \in \mathbb{NP}$-completo. $CNFSAT \in \mathbb{NP}$-completo. 
\end{teo}

Per dimostrare che $CNFSAT \preceq SAT$, basta realizzare che una formula 
in CNF è comunque ancora una formula e pertanto la trasformazione è ovviamente 
polinomiale, essendo la funzione identità. Dimostrare che $SAT \preceq CNFSAT$ 
è tutt'altra questione benché sia ovvio dal Teorema di Cook; tuttavia, questa 
proprietà che invece non vale per le DNF, ossia $SAT \npreceq DNFSAT$, è una 
buona motivazione per concentrarsi sulle CNF. 
Si mostra ora, esplicitamente, che SAT si riduce polinomialmente a CNFSAT, 
conservando non l'equivalenza logica ma la relazione di equisoddisfacibilità. 

\subsection{Equisoddisfacibilità}
\begin{defi}[Equisoddisfacibilità]
        Siano $A, B \in F_L$. Le due formule sono equisoddisfacibili se e solo 
        se $A$ è soddisfacibile se e solo se $B$ è soddisfacibile, ossia 
        se $A$ e $B$ sono entrambe soddisfsacibili o entrambe insoddisfacibili. 
\end{defi}

\subsubsection{Esempi}
\paragraph{1}
Date le due formule $\neg (A \land B)$ e $\neg A \lor \neg B$, è noto 
che sono equivalenti grazie alle leggi di De Morgan e sono, di conseguenza, 
anche equisoddisfacibili.
\paragraph{2} $\neg (A \land B)$ e $(\neg A \land \neg B)$ non sono equivalenti, 
infatti l'assegnamento $A=1$ e $B=1$ soddisfa solo una delle due, tuttavia 
sono equisoddisfacibili.
\paragraph{3} $A \land \neg A$ e $\neg A \neg B$ non sono né equivalenti né
equisoddisfacibili.

Da questi esempi si può chiaramente notare che l'equivalenza implica l'equisoddisfacibilità 
mentre il contrario non è affatto verificato.

\paragraph{4}
L'equisoddisfacibilità è un tipo di relazione d'equivalenza, in quanto 
valgono le proprietà di simmetria, transitività e riflessività.
\paragraph{5} L'equisoddisfacibilità è una congruenza rispetto ai 
connettivi, pensati come operazioni? Non lo è. Sia, per esempio 
$\models A$, quindi $A \in [\top]$ e $B$ soddisfacibile ma non tautologica; 
sono equisoddisfacibili, in quanto sono entrambi chiaramente soddisfacibili. 
Se fosse una congruenza, rispetto alle varie operazioni l'equisoddisfacibilità 
dovrebbe essere mantenuta, invece $\neg A$ è $\bot$, mentre $\neg B$ è 
ancora soddisfacibile, pertanto non sono equisoddisfacibili.

\paragraph{6} Le classi di equivalenza delle formule, per esempio su 
due variabili, sono costruite nelle forme come $F_L^{(2)}/\equiv$, ossia 
$16$ diverse classi ($2^{2^2}$). Per l'equisoddisfacibilità vi saranno unicamente 
due classi, ossia l'insieme delle formule insoddisfacibili ($[\bot]$) e 
tutte le rimanenti, ossia tutte quelle almeno soddisfacibili.  

\subsection{Riduzione $SAT \preceq CNFSAT$}
Sia $A \in F_L$ tale che $A$ sia in Negation Normal Form, ossia $A \in NNF$. 
Se $A \notin CNF$, allora contiene almeno una sottoformula del tipo 
$C \lor (D_1 \land D_2)$ oppure $(D_1 \land D_2) \lor C$. Questo si dimostra 
semplicemente confermando che $A \in NNF$ ma non $A \in CNF$, quindi la parte 
che non rispetta la CNF deve essere una disgiunzione di congiunzioni. Trattiamo, 
d'ora in poi, solo il primo dei due casi, ossia quello in cui nella formula 
appare la sottoformula $C \lor (D_1 \land D_2)$, senza perdita di generalità.
Sia data $A \in NNF$ e sia $B = C \lor (D_1 \land D_2)$ una sua violazione; 
per ogni violazione si introduce una nuova lettera proposizionale $a \in L$ che 
ancora non è presente in $A$. 
Si definisce ora 
$$
B' := B[a/D_1 \land D_2] \land (\neg a \lor D_1) \land (\neg a \lor D_2)
$$
dove 
$$
B'' := B[a/D_1\land D_2]
$$
è la formula ottenuta rimpiazzando ogni occorrenza di $D_1 \land D_2$ con $a$ 
in $B$. 
Come prima osservazione, si ha che $(\neg a \lor D_1) \land (\neg a \lor D_2)$ 
è equivalente a $a \rightarrow (D_1 \land D_2)$. 
Si mostra che $B'$ e $B$ sono equisoddisfacibili, ma in genere non 
sono equivalenti: 
\begin{proof}[$B \in SAT \rightarrow B' \in SAT$]
        Per ipotesi, dato che $B$ è soddisfacibile, sia $\mathfrak{v}:L \rightarrow \{0,1\}$
        tale che $\mathfrak{v}(B) = 1$, ossia $\mathfrak{v} \models B$. 
        Si definisce 
        $$
        \mathfrak{v}_a: L \rightarrow \{0,1\} = 
        \begin{cases}
                \mathfrak{v}_a(p) = \mathfrak{v}(p) & \forall p \in L, p \neq a \\
                \mathfrak{v}_a(a) = \mathfrak{v}(D_1 \land D_2) 
        \end{cases} 
        $$
        L'assegnamento $\mathfrak{v}(a)$ è ben definito, nel senso che non 
        da alla stessa lettera proposizionale due assegnamenti diversi.
        Si nota che $\mathfrak{v}_a \models a \rightarrow (D_1 \land D_2)$, 
        dato che per definizione $\mathfrak{v}_a(a) = \mathfrak{v}(D_1 \land D_2)$
        e per interpretazione dell'implicazione $0 \rightarrow 0 = 1$ e 
        $1 \rightarrow 1 = 1$; dunque, $\mathfrak{v}_a(B') = \mathfrak{v}_a(B'')$, 
        in quanto la ``coda'' di $B'$, ossia $(\neg a \lor D_1) \land (\neg a \lor D_2)$, 
        ha un assegnamento uguale a $1$, ossia 
        $\mathfrak{v}_a(B') = \mathfrak{v}_a(B'') \land 1$.

        $B$ si può riscrivere come 
        $$
        B = B''[D_1 \land D_2/a]
        $$
        dato che $a$ non appare in $B$, e quindi 
        $$
        B'' = (B[a/D_1 \land D_2])[a/a]
        $$
        e grazie al Lemma di Sostituzione si può affermare che i due valori 
        di verità di $B$ e $B''$ sono uguali in quanto i valori di verità di $a$ 
        e $D_1 \land D_2$ sono uguali. Ora si può scrivere 
        \begin{align*}
                1 &= \mathfrak{v}(B) \text{ per ipotesi } \\
                  &= \mathfrak{v}_a(B) \text { poiché } a \text{ non occore in } B \\
                  &= \mathfrak{v}_a(B'') \text{ per il lemma di sostituzione}  \\
                  &= \mathfrak{v}_a(B') 
        \end{align*}
        ossia l'assegnamento che soddisfa $B$ soddisfa anche $B'$, ossia 
        sono equisoddisfacibili. 
\end{proof}
\begin{proof}[$B' \in SAT \rightarrow B \in SAT$]
        La sequenza di derivazioni utilizzata per la dimostrazione precedente 
        è ancora buona, tuttavia c'è un vincolo all'inizio, 
        ossia l'assunzione che se $\mathfrak{v}(B) = 1$ allora $\mathfrak{v}_a(B) = 1$; 
        per definizione 
        $$
        \mathfrak{v}_a: L \rightarrow \{0,1\} = 
        \begin{cases}
                \mathfrak{v}_a(p) = \mathfrak{v}(p) & \forall p \in L, p \neq a \\
                \mathfrak{v}_a(a) = \mathfrak{v}(D_1 \land D_2) 
        \end{cases} 
        $$
        che non è la stessa cosa di dire che $B'$ è soddisfacibile, 
        ossia $\mathfrak{v}_a \models B'$, poiché $B'$ potrebbe essere soddisfatto 
        da assegnamenti che non sono nella forma $\mathfrak{v}_a$. Il ragionamento 
        è un po' più complicato. 

        Si supponga che $B'$ sia soddisfacibile da un assegnamento della forma 
        $\mathfrak{v}_a$, ossia tale che $\mathfrak{v}_a(a) = \mathfrak{v}(D_1 \land D_2)$; 
        allora, in questo caso si può tranquillamente ribaltare la catena di uguaglianze 
        precedente: 
        \begin{align*}
                1 &= \mathfrak{v}_a(B') \\
                  &= \mathfrak{v}_a(B'') \text{ per il lemma di sostituzione}  \\
                  &= \mathfrak{v}_a(B) \text { poiché } a \text{ non occore in } B 
        \end{align*}
        dunque $\mathfrak{v}_a \models B$. 
        Si supponga, ora, che $B'$ sia soddisfatto solo da assegnamenti 
        $\mathfrak{w}$ che non sono della forma $\mathfrak{v}_a$, 
        ossia $\mathfrak{w}(a) \neq \mathfrak{v}(D_1 \land D_2)$. 
        Vi sono allora due casi: nel primo $\mathfrak{w}(a)= 0 \neq \mathfrak{v}(D_1 \land D_2) =  1$, 
        nel secondo accade il contrario, ossia 
        $\mathfrak{w}(a) = 1 \neq \mathfrak{v}(D_1 \land D_2) = 0$ tuttavia 
        quest'ultimo caso è impossibile, poiché 
        $\mathfrak{w}(a \rightarrow D_1 \land D_2) = \mathfrak{w}(1 \rightarrow 0) = 0$
        e quindi non può soddisfare $B'$. Quindi rimane 
        il primo caso e bisogna mostrare che anche 
        tale assegnamento effettivamente soddisfa $B'$ e non è un assurdo; come 
        informazione di carattere generale utile per risolvere 
        questo problema, sia E un'espressione formata solo da $0, 1, \land, \lor$ 
        interpretati come $\min$ e $\max$. Il valore di $E$ è $0$ o $1$. 
        Sia $E'$ ottenuta da $E$ rimpiazzando nessuna o più occorrenze 
        del simbolo $0$ con $1$. Allora $E \leq E'$. Questo si dimostra 
        affermando che $0, 1, \min, \max$ sono tutte funzioni 
        non decrescenti e $E$ ed $E'$ sono composizioni di funzioni 
        non decrescenti, pertanto sono non decrescenti (un'altra prova è 
        per induzione strutturale su $E$).

        Ora, tornando al problema principale, si ha che  $\mathfrak{w}(B') = 1$, 
        $\mathfrak{w}(a) = 0$ e $\mathfrak{w}(D_1 \land D_2) = 1$. Dato 
        che la formula iniziale era in $NNF$, anche $B'$ e $B$ sono in 
        NNF. Dunque, considerando $\mathfrak{w}(B)$ e $\mathfrak{w}(B'')$ 
        si possono esprimere entrambe come funzioni termine 
        di $B''$, ossia  
        $$
        \hat{B}''(\mathfrak{w}(p_1), \cdots, \mathfrak{w}(p_n), \mathfrak{w}(D_1 \land D_2))
        $$
        e
        $$
        \hat{B}''(\mathfrak{w}(p_1), \cdots, \mathfrak{w}(p_n), \mathfrak{w}(a))
        $$
        rispettivamente. Come prima osservazione, $\mathfrak{w}(B'')$ e $\mathfrak{w}(B)$
        sono considerabili come espressioni costruite su $0,1,\land,\lor$, poiché 
        sono entrambi in $NNF$. Si può concludere, quindi, che 
        $$
        \mathfrak{w}(B'') \leq \mathfrak{w}(B) \rightarrow 1 \leq \mathfrak{w}(B) \rightarrow \mathfrak{w}(B) = 1
        $$
        applicando le sostituzioni sui valori di verità di $a$ e $D_1 \land D_2$. 
\end{proof}

\begin{oss}[Nota finale]
        Si sarebbe potuto definire $B'$, alternativamente, come segue: 
        \begin{align*}
                B' &:= B[a/D_1 \land D_2] \land (a \iff (D_1 \land D_2))^c \\
                   &=  B[a/D_1 \land D_2] \land (\neg a \lor D_1) \land (\neg A \lor D_2) \land ((D_1 \land D_2) \rightarrow a)^c \\
                   &=  B[a/D_1 \land D_2] \land (\neg a \lor D_1) \land (\neg A \lor D_2) \land (a \lor \neg D_1 \lor \neg D_2)
        \end{align*}
        Adottando tale definizione per $B'$, allora nella prova che $B'$ soddisfacibile 
        implica $B$ soddisfacibile si sarebbe potuto mostrare che $B'$ è 
        soddisfatto solo da assegnamenti nella forma $\mathfrak{v}_a$, 
        ossia $\mathfrak{v}_a(a) = \mathfrak{v}(D_1\land D_2)$.
\end{oss}

Nel passaggio da $B$ a $B'$ si osserva una dilatazione nella lunghezza della 
formula, ossia $B'$ è \textit{più lungo} di $B$; data la formula 
generale 
$$
B' \rightarrow B[a/D_1 \land D_2] \land (\neg a \lor D_1) \land (\neg a \lor D_2)
$$
la parte della formula $B$ viene al massimo accorciata, però sia aggiunge una 
decina di simboli: si può affermare che la lunghezza di 
$B'$ sia $|B'| = |B| + k$, con $k$ una costante dipendente dal 
modo in cui si conta la lunghezza (si contano le parentesi come simboli eccetera). 
Per ogni ``violazione'' nella formula originale, la formula risultante equisoddisfacibile
senza tale violazione è più lunga di $k$ caratteri e, se vi sono $v$ violazioni, 
dopo $v$ passi la formula risultante sarà CNF e avrà lunghezza $|B| + v * k$, 
non esponenziale rispetto alla lunghezza iniziale. 

La tecnica usata per ridurre $SAT \preceq_p CNFSAT$ che consiste nel rimpiazzare 
$B$ con $B'$ equisoddisfacibile è ispirata alla riduzione 
$$
SAT \preceq 3CNFSAT
$$
dovuta a Karp. Il passaggio $B \rightarrow B'$ è un esempio del cosiddetto 
``Tseytin's Trick'', che si usa per ridurre $F \in F_L$ a una 
in $3CNFSAT$ ad essa equisoddisfacibile. 

\begin{defi}[Problemi $3CNFSAT$]
        $3CNFSAT \subseteq CNFSAT \subseteq F_L$ è costituito da tutte 
        e sole le CNF dove ogni clausola contiene o esattamente $3$ 
        letterali. $3CNFSAT \in \mathbb{NP}$ ed è addirittura $\mathbb{NP}$-completo.
\end{defi}
\begin{oss}[Problemi $2CNFSAT$]
        $2CNFSAT \in \mathbb{P}$.
\end{oss}
\subsubsection{Algoritmo di riduzione a CNF equisoddisfacibili}
Sia data $F \in F_L$ e sia $A \in NNF$ $A \equiv F$, che si 
ottiene in tempo polinomiale rispetto alla lunghezza di $F$. 
Se $A \in CNF$, allora l'algoritmo termina, altrimenti è necessario 
individuare la violazione $B$ sottoformula di $A$ e si sostituisce con 
$B'$ e si ritorna al check su $A \in CNF$. 
\subsubsection{Esempi}
\paragraph{1} Sia $A := p \lor (q \land r)$. Si trasforma ora in una 
formula equisoddisfacibile in CNF: 
$$
A' := (p \lor a) \land (\neg a \lor q) \land (\neg a \lor r)
$$
Queste due formule non sono equivalenti, infatti vi sono assegnamenti 
che portano a risultati diversi; tuttavia sono equisoddisfacibili.

\paragraph{2} 
Sia $A := (p_1 \land p_2) \lor (p_2 \land q_2) \lor (p_3 \land q_3)$. Si 
ha $A'$ uguale a 
\[
        ((p_1 \land q_1) \lor (p_2 \land q_2) \lor a_3) \land (\neg a_2 \lor p_3) \land (\neg a_3 \lor q_3)
\]
e, applicando di nuovo la trasformazione, si ottiene 
\[
        ((p_1 \land q_1) \lor a_2 \lor a_3) \land (\neg a_3 \lor p_3) \land (\neg a_3 \lor q_3) \land (\neg a_2 \lor p_2) \land (\neg a_2 \lor p_2) 
\]
e si conclude definendo $A'''$
\[
(a_1 \lor a_2 \lor a_3) \land (\neg a_3 \lor p_3) \land (\neg a_3 \lor q_3) \land (\neg a_2 \lor p_2) \land (\neg a_2 \lor q_2) \land (\neg a_1 \lor p_1) \land (\neg a_1 \lor q_1)
\]
Data una formula con $n$ letterali si generano $2*n +1 $ clausole, 
di cui una con $n$ letterali e $2n$ con due letterali, laddove utilizzando la 
distributività se ne generavano $2^n$ ognuna con $n$ letterali. 


\section{Deduzione Automatica}
Vogliamo studiare i problemi $CNFSAT$, che è $\mathbb{NP}$-completo, e 
il suo complemento $CNFUNSAT$, che è un problema $\mathbb{NP}$-completo, 
in quanto la riduzione polinomiale $SAT \preceq_p CNFSAT$ è valida e pertanto 
studiare $CNFSAT$ è uguale a studiare $SAT$. Analogamente si può studiare 
$TAUTO \preceq_p SAT^c \preceq_p CNFUNSAT$. In realtà siamo nella posizione 
di studiare $\Gamma \models A$ tramite la risoluzione di $CNFSAT$ e $CNFUNSAT$.
Il motivo per cui non studiare invece $DNFSAT$ e $DNFUNSAT$ è che non 
c'è un algoritmo per ``tradurre'' una formula arbitraria equisoddisfacibile 
in DNF in modo che sia sufficientemente corta: i metodi conosciuti allungano
esponenzialmente la formula. 

\begin{defi}[Variazione notazionale]
        Date $F_i \in F_L$, le formule 
        $$
        F_1 \land F_2 \land \cdots \land F_k
        $$
        e 
        $$
        F_1 \lor F_2 \lor \cdots \lor F_k
        $$
        si possono scrivere senza operatori grazie all'associatività, 
        e inoltre le formule interne nelle formule esterne si possono 
        scambiare ($F_1 \land F_2 \equiv F_2 \land F_1$) grazie 
        alla commutatività e si possono espandere le singole formule 
        ($F_1 \land F_1  \equiv F_1$) grazie all'idempotenza. 

        In questo frangente si possono anche ``tralasciare'' gli operatori 
        $\land$ e $\lor$, chiaramente nel caso in cui ci sia un utilizzo uniforme. 
        La notazione può diventare puramente insiemistica, ossia 
        $$
        \{F_1, F_2, \cdots, F_k\}
        $$
        per indicare entrambe le formule, specificando il connettivo che le 
        unisce. D'ora in poi una CNF sarò un \textbf{insieme} di clausole, 
        dove ogni clausola sarà un insieme di letterali. 
        Per esempio 
        $$
        (p \lor q \lor \neg r) \land (q \lor r \lor a) \land p
        $$
        diventa 
        $$
        \{ \{p, q, \neg r\}, \{q, r, a\}, \{p\}\}
        $$
        ossia una CNF in forma insiemistica, che contiene clausole in 
        forma insiemistica. 
        Una teoria costruita da CNF è l'insieme di tutte le clausole appartenenti 
        a qualche CNF della teoria. 
        La CNF vuota è $\emptyset$, mentre la clausola vuota $\qedsymbol$ 
        e la CNF che contiene la clausola vuota avrà la forma 
        $\{\cdots, \qedsymbol, \cdots \}$.
\end{defi}

Il problema della conseguenza logica di una teoria 
$$
\Gamma \models A
$$
si può ridurre a calcolare la soddisfacibilità di 
$$
S := \{\Gamma^c \cup \{\neg A\}^c\} \text{ dove } \Gamma^c := \{\gamma^c | \gamma \in \Gamma\}
$$
unione di CNF. 
Si è quindi ridotto il problema $\Gamma \models A$ al problema $S \in CNFSAT$,
dove $S$ è un insieme di clausole considerate come insiemi di letterali, 
eventualmente infinito. 


\subsection{Metodi refutazionali}
$S$ è soddisfacibile se esiste $\mathfrak{v}:L \rightarrow \{0,1\}$ tale 
che $\mathfrak{v} \models C$ per ogni clausola $C \in S$, in altre parole 
in ogni $C \in S$ esiste un letterale $l \in C$ tale che $\mathfrak{v} \models l$. 

Se l'obiettivo è dimostrare che $S$ è insoddisfacibile, una strategia 
può essere ampliare $S$ in un nuovo insieme $S \subseteq S'$ in modo 
tale che $S'$ è logicamente equivalente o almeno equisoddisfacibile a $S$.
Ad esempio, se ampliando iterativamente $S$ in $S'$, $S''$ fino a $S^{(u)}$ 
e alla fine la clausola vuota $\qedsymbol$ appartiene a $S^{(u)}$ allora quest'ultimo 
è insoddisfacibile, e quindi anche $S$ lo è. Questa idea può essere sfruttata 
disegnando \textbf{metodi refutazionali}, ossia metodi che hanno l'obiettivo di 
provare l'insoddisfacibilità di un insieme di clausole, in questo frangente, 
ma più in generale anche di una formula o una teoria, i quali sono basati 
sulla regola di inferenza chiamata \textbf{principio di risoluzione}, 
il quale è utile per un tipo di calcolo particolarmente adatto a essere automatizzato, 
ossia SAT solver e Theorem Prover. 

\begin{defi}[Principio di Risoluzione]
        Date due clausole $C_1$ e $C_2$ si dice che $D$ è la \textbf{risolvente}
        di $C_1$ e $C_2$ sul pivot $\ell$ se e solo se
        \begin{itemize}
                \item $\ell \in C_1$
                \item $\bar{\ell} \in C_2$
                \item $D := (C_1 \setminus \{\ell\}) \cup (C_2 \setminus \{\bar{\ell}\})$
        \end{itemize}
       e si scriverà $D = \mathbb{R}(C_1,C_2; \ell, \bar{\ell})$. 
\end{defi}

\subsubsection{Esercizio}
Mostrare che $(C_1 \setminus \{\ell\}) \cup (C_2 \setminus \{\bar{\ell}\})$ può essere 
diverso da $(C_1 \cup C_2) \setminus \{\ell, \bar{\ell}\}$. 

\subsubsection{Esempio}
Sia $C_1 := \{x, y, \neg t\}$ e $C_2 := \{u, \neg y, t\}$; 
si ha $D_1 = \mathbb{R}(C_1, C_2; y, \neg y) = \{x, \neg t, u\}$ e 
$D_2 = \mathbb{R}(C_1, C_2; \neg t, t) = \{x, y, y\}$. 

\begin{lem}[di correttezza della risoluzione]
        Sia $D = \mathbb{R}(C_1, C_2; \ell, \bar{\ell})$, allora 
        $$
        \{C_1, C_2\} \models D
        $$
\end{lem}

\begin{proof}
        Bisogna dimostrare che ogni $\mathfrak{v}$ tale che $\mathfrak{v} \models C_1$ 
        e $\mathfrak{v} \models C_2$ implica $\mathfrak{v} \models D$. Sia allora 
        $\mathfrak{v}$ tale che $\mathfrak{v} \models C_1$ 
        e $\mathfrak{v} \models C_2$, allora $\mathfrak{v} \models m$ e 
        $\mathfrak{v} \models n$ per due letterali $m \in C_1$ e $n \in C_2$; 
        se fosse il caso che $m = \ell$ e  $n = \bar{\ell}$ allora 
        $\mathfrak{v}(\ell) = 1$ e $\mathfrak{v}(\bar\ell) = 1$, impossibile. 
        Allora,almeno uno tra $m$ e $n$ è tale che $m \neq \ell$ o $n \neq \ell$ 
        e assumendo senza perdita di generalità $m \neq l$ si ha 
        $m \in D$, dunque $\mathfrak{v}(D) = 1$. 
\end{proof}

\begin{cor}
        Se $\mathbb{R}(C_1, C_2; \ell, \bar{\ell}) = \qedsymbol$ allora 
        $\{C_1, C_2\}$ è insoddisfacibile, in quanto $\{C_1, C_2\} \models \qedsymbol$.
\end{cor}
\begin{cor}
        Sia $S$ un insieme di clausole e sia $S := S_0, S_1, \cdots, S_k$ una 
        successione di insiemi tale che per ogni $i = 0, \cdots, k-1$ 
        si ha che $S_{i+1}$ si ottiene unendo a $S_i$ una o più risolventi 
        di clausole in $S_i$ e che $\qedsymbol \in S_k$. $S$ è 
        insoddisfacibile.
\end{cor}
L'obiettivo è mostare che i calcoli refutazionali basati su applicazioni 
ripetute della risoluzione sono corretti e completi (refutazionalmente).
In genere, per un calcolo logico si studiano infatti le due
seguenti proprietà:
\begin{defi}[Correttezza]
        Un calcolo si definisce \textbf{corretto} se i certificati che produce 
        testimoniano il vero, ossia sono corretti, in altre parole 
        non produce certificati fasulli. 
\end{defi}

Per esempio, nel frangente attuale 
$S_0, S_1, \cdots, S_k \ni \qedsymbol$ è un certificato 
corretto dell'insoddisfacibilità di $S = S_0$. 

\begin{defi}[Completezza]
        Un calcolo è completo se non omette alcun certificato. 
\end{defi}
Nella situazione attuale vuol dire che se $S$ è insoddisfacibile, allora 
esiste $S_0, S_1, \cdots, S_k$ tale che si produce $\qedsymbol \in S_k$.


Prima di mostrare che il calcolo refutazionale è sia corretto (anche se 
il lemma di correttezza è già stato esplicitato) e completo, è necessario 
prepararsi la strada, poiché non sarà banale. 

\begin{teo}[Teorema di Completezza del principio di risoluzione (o di Robinson)]
       Un insieme $S$ di clausole è insoddisfacibile se e solo se 
       $\qedsymbol \in R^*(S)$, dove $R^*(S)$ è definito come segue: 
       \begin{align*}
               R(S) &:= S \cup \{D: D = \mathbb{R}(C_1, C_2; \ell, \bar{\ell}) C_1, C_2 \in S \land \ell \in C_1 \land \bar{\ell} \in C_2\} \\
               R^2(S) &= R(R(S)) \\
               \cdots \\
               R^{t+1}(S) &:= R(R^{t}(S)) \\
               \cdots \\
               R^*(S) &= \cup_{i \in \omega} R^i(S)
       \end{align*}
       dato $R^0(S) := S$. 
\end{teo}

Il calcolo $R^*$ è un metodo \textit{a forza bruta}, applica il 
Principio di Risoluzione calcolando ciecamente 
tutte le risoluzioni possibili e non c'è da sperare che faccia molto meglio 
delle tavole di verità. 

\begin{proof}[$\qedsymbol \in R^*(S) \rightarrow S$ insoddisfacibile (Correttezza del calcolo R)]
        Si supponga $\qedsymbol \in R^*(S)$, allora c'è un $i \in \omega$ tale 
        per cui, per definizione, $\qedsymbol \in R^i(S)$ e pertanto 
        $R^i(S)$ è insoddisfacibile, per definizione di clausola vuota; 
        $R^i(S)$ è equivalente a $R^{i-1}(S)$ per il lemma di correttezza, 
        arrivando fino a $R^0(S) = S$, che è pertanto insoddisfacibile.
\end{proof}
\begin{proof}[$S \text{ insoddisfacibile } \rightarrow \qedsymbol \in R^*(S)$ (Completezza Refutazionale del calcolo R)]
        Dato che $S$ è insoddisfacibile, per il Teorema di Compattezza esiste 
        un $S_{fin} \subseteq_{\omega} S$ finito e $S_{fin}$ è insoddisfacibile; 
        bisogna capire come sia fatto $S_{fin}$: l'insieme finito $S_{fin}$ contiene 
        un numero finito di lettere proposizionali e si definisce, come è stato fatto 
        precedentemente,
        $Var(S_{fin}) \subseteq \{p_1, p_2, \cdots, p_n\}$ per qualche $n \in  \omega$, 
        $p_i \in L$.
        La notazione $C^{n}_L$ indica l'insieme di tutte le clausole scrivibili 
        sulle prime $n$ lettere proposizionali, ossia $\{p_1, p_2, \cdots, p_n\}$  
        e si ha $S_{fin} \subseteq C^{n}_L$. 
        Si osserva che $C^{0}_L = \{ \qedsymbol \}$. A fortiori, dato che 
        $S_{fin} \subseteq C^{n}_L$ si ha $S_{fin} \subseteq (C^{n}_L \cap S) \subseteq (C_L^{n} \cap R^*(S))$
        e pertanto $C^{n}_L \cap R^*(S)$ è insoddisfacibile, dato che $S_{fin}$ è 
        insoddisfacibile. 

        Se si riesce a dimostrare che per ogni $ k = n, \cdots, 1,0$ si ha 
        $$
        C^k_L \cap R^*(S) \text{ è insoddisfacibile}
        $$
        allora si ha che per $k = 0$
        $$
        C^{0}_L \cap R^*(S) \text{ è insoddisfacibile }
        $$
        poichè se si dimostra che $C^0_L \cap R^*(S)$ 
        insoddisfacibile, allora $\qedsymbol \in R^*(S)$: 
        $$
        C^{0}_L \cap R^*(S) = 
        \begin{cases}
        \emptyset  & \rightarrow C^0_L \cap R^*(S) \text{soddisfacibile, assurdo.} \\
        \qedsymbol & \text{ unico caso possibile }
        \end{cases}
        $$
        dunque $\qed \in R^*(S)$. Per dimostrare che 
        per ogni $k = n, \cdots, 1, 0$ 
        $$
        C^k_L \cap R^*(S) \text{ è insoddisfacibile} 
        $$
        si usa l'induzione decrescente su $k$, ossia l'induzione aritmetica 
        su $t = n - k$; se $k = n \rightarrow t = 0$ e se $k = 0 \rightarrow t = n$. 
        La base dell'induzione è ``regalata'' dal Teorema di Compattezza, 
        ossia che $C^k_L \cap R^*(S)$ sia insoddisfacibile è dovuto all'esistenza $S_{fin}$ 
        insoddisfacibile. Si assume l'asserto vero per $n-0, n-1, \cdots,n-k-1 $ e 
        si prova per $t=n-k$, ossia 
        $$
        C^n_L \cap R^*(S), C^{n-1}_L \cap R^*(S), \cdots, C^{k+1}_L \cap R^*(S)
        $$
        sono insoddisfacibili e si vuole dimostrare per $C^k_L \cap R^*(S)$ sia 
        insoddisfacibile. 

        Per assurdo, si assume $C^k_L \cap R^*(S)$ 
        soddisfacibile, dunque esiste $\mathfrak{v} \models C$ per ogni 
        $C \in C^k_L \cap R^*(S)$; si definiscono ora 
        $$
        \mathfrak{v}^+ : L \rightarrow \{0,1\} ~ ~ ~ \mathfrak{v}^+(p_{k+1}) = 1 
        $$
        $$
        \mathfrak{v}^- : L \rightarrow \{0,1\} ~ ~ ~ \mathfrak{v}^-(p_{k+1}) = 0
        $$
        mentre per ogni altra $i$ $\mathfrak{v}(i) = \mathfrak{v}^+(i) = \mathfrak{v}^-(i)$;
        si noti che $p_{k+1} \notin C^{(k)}_L$. 
        Per ipotesi induttiva esiste $C_1 \in C^{k+1}_K \cap R^*(S)$ tale che 
        $\mathfrak{v}^+(C_1) = 0$; analogamente esiste $C_2 \in C^{k+1}_L \cap R^*(S)$ 
        tale che $\mathfrak{v}^-(C_2) = 0$, in quanto è insoddisfacibile.
       
        La lettera proposizionale $p_{k+1}$ occorre in $C_1$ e più precisamente come 
        letterale $\neg p_{k+1}$, infatti $p_{k+1}$ non può occorrere come letterale 
        positivo poiché $\mathfrak{v}^+(p_{k+1}) =1$ per definizione e dunque $ \mathfrak{v}^+(C_1) = 1$, 
        assurdo. Deve, però, apparire nel letterale $\neg p_{k+1}$, poiché altrimenti 
        si ottiene che $p_{k+1}, \neg p_{k+1} \notin C_1$ dunque $C_1 \in C^k_L \cap R^*(S)$ 
        e dunque $\mathfrak{v}(C_1) = 1$. Analogamente, 
        $p_{k+1}$ occorre in $C_2$ e più precisamente come letterale $p_{k+1}$ 
        e la prova induttiva è identica, mutandis mutandis. 


        Dunque, esiste $D = R(C_1, C_2; \neg p_{k+1}, p_{k+1})$, ossia 
        $$
        D = (C_1 \setminus \{ \neg p_{k+1}\}) \cup (C_2 \setminus \{p_{k+1}\})
        $$
        e $p_{k+1}, \neg p_{k+1} \notin D$ e $D \in C^{k}_L \cap R^*(S)$ 
        e $\mathfrak{v}(D) = 1$. Vi sono due casi: il primo è che 
        $\mathfrak{v}$ soddisfi qualche letterale in $C_1 \setminus \{p_{k+1}\}$, 
        ma allora $\mathfrak{v}(C_1)  = 1$ e $\mathfrak{v}^+(C_1) = 1$, 
        assurdo.
        Il secondo caso è che $\mathfrak{v}$ soddisfi qualche letterale 
        in $C_2 \setminus \{p_{k+1}\}$, e allora $\mathfrak{v}^-(c_2) = 1$, 
        assurdo.
        Non essendoci altri casi, si è raggiunta la contraddizione 
        che conclude la dimostrazione per assurdo, dunque 
        $C^k_L \cap R^*(S)$ è
        insoddisfacibile per ogni $k$, che chiude la prova per induzione; dunque, 
        per ogni $k = n, n-1, \cdots, 1, 0$, $C^k_L \cap R^*(S)$ è 
        insoddisfacibile e $C^0_L \cap R^*(S)$ anche, pertanto 
        $C^0\cap R^*(S) = \{\qedsymbol\}$ e $\qedsymbol \in R^*(S)$.
\end{proof}
\begin{oss}
        Se $S$ è un insieme finito di clausole, la costruzione di $R^*(S)$ 
        costituisce una procedura di decisione, cioè sia che $S$ sia insodd. 
        sia che $S$ sia sodd. termina in tempo finito, dando la risposta 
        corretta. Infatti, se $S$ è finito menziona solo un numero finito $n = |Var(S)|$ di 
        lettere proposizionali diverse e i letterali scrivibili su $\{p_1, \cdots, p_n\}$ 
        sono $2*n$, dunque in $S$ occorrono al più $2*n$ letterali diversi; le 
        clasuole scrivibili su $2*n$ letterali sono al più $2^{2*n}$ dunque 
        in $S$ occorrono al più $2^{2*n}$ clausole. 
        Si nota, ora, che la risoluzione non introduce mai nuovi letterali, 
        dunque la sequenza $R^{(0)}(S) \subseteq R(S) \subseteq \cdots R^{k}(S) \subseteq \cdots$ 
        è tale che  che ogni $R^{i}(S)$ è un sottoinsieme delle $2^{2*n}$ clausole 
        scrivibili su $\{p_1, \cdots, p_n\}$, dunque esiste un $t$ tale che 
        $R^{t}(S) = R^{t+1}(S)$, poiché la sequenza non può crescere all'infinito; 
        pertanto
        $$
        R^{t}(S) = R^{t+1}(S) = \cdots = R^*(S)
        $$
        Se si trova $\qedsymbol \in R^i(S)$, si può concludere che $S$ sia 
        insoddisfacibile. Se, al contrario, $R^t(S) = R^{t+1}(S)$ e $\qedsymbol \notin R^t(S)$, 
        si può concludere che $S$ sia soddisfacibile. 
\end{oss}
\begin{oss}
        Dato che $S_{fin}$ è finito, esiste $t \in \omega$ tale 
        che $R^*(S_{fin})=R^t(S_{fin}) = R^{t+1}(S_{fin})$, 
        tuttavia questo non fornisce una procedura di decisione per 
        gli $S$ infiniti, ma esclusivamente di semidicesione, in quanto 
        in genere non si sa \textit{scegliere} $S_{fin}$ e 
        il meglio che si può fare è descrivere una successione infinita 
        $$
        S_1 \subseteq S_2 \subseteq \cdots \subseteq S_k  \subseteq \cdots 
        $$
        di sottoinsiemi finiti di $S$ tali che $\bigcup S_i = S$ e 
        per ogni $i$ si calcola $R^*(S_i)$: se $\qedsymbol \in S_i$ allora 
        $S$ insoddisfacibile, 
        altrimenti si aumenta $i$ e si procede al passo successivo.
\end{oss}

La Completezza Refutazionale non è la Completezza \textit{tout-court}. 
\`E qualcosa che è più debole, e in realtà proprio per questo è una 
proprietà desiderabile dal punto di vista computazionale. 

\begin{defi}[Deduzione per Risoluzione]
        Una deduzione per risoluzione di una clausola $C$ da un insieme di clausole 
        $S$, indicato 
        $$
                S \vdash_R C
        $$
        è una sequenza finita di clausole 
        $$
        C_1,C_2, \cdots, C_n
        $$ 
        tale che 
        \begin{itemize}
                \item $C_n = C$
                \item $\forall C_i ~~~ C_i \in S \text{ oppure } C_i = R(C_j, C_k, \ell, \bar{\ell})$
        \end{itemize}
\end{defi}

In particolare, una deduzione per risoluzione della clausola vuota 
($S \vdash_R \qedsymbol$) è detta \textbf{refutazione} di $S$. 
\begin{teo}[Teorema di Completezza Refutazionale]
        Un insieme di clausole $S$ è insoddisfacibile se e solo se $S \vdash_R \qedsymbol$.
\end{teo}
Al momento, la refutazione al momento la sappiamo costruire solo tramite 
il metodo $R^*(S)$. 

\paragraph{Esempio}
$$
\{\{a, b, \neg c\}, \{a, b, c\}, \{a, \neg b\}\} \models \{a\}?
$$
Si costruisce una refutazione: 
inizialmente, si trasforma il problema in un problema di insoddisfacibilità, 
ossia 
$$
\{\{a, b, \neg c\}, \{a, b, c\}, \{a, \neg b\}, \{\neg a\}\} \text{ è soddisfacibile?}
$$
Non è soddisfacibile, poiché 
\begin{align*}
        (\{a,b,\neg c\}, \{a, b, c\}) &\vdash_R \{a,b\} \\
        (\{a,b\}, \{a, \neg b\}) &\vdash_R \{a\}\\
        (\{a\}, \{\neg a\}) &\vdash_R \qedsymbol
\end{align*}

\noindent
Genericamente, chiedersi se $\Gamma \models A$ è uguale a chiedersi 
se $\Gamma, \neg A$ sia insoddisfacibile, che è 
 uguale a $\Gamma ^c, (\neg A)^c \vdash_R \qedsymbol$ per Completezza 
 Refutazionale, ma che è diverso da 
$\Gamma^c \vdash_R A^c$; è questa la debolezza 
della completezza refutazionale rispetto alla Completezza tout court
($\Gamma \models A \iff \Gamma^c \vdash_c A$). Quindi, per un calcolo 
refutazionale $R$ per il quale vale la Completezza Refutazionale si ha 
$$
\Gamma \models A \iff \Gamma, \neg A \text{ è insoddisfacibile } \iff \Gamma^C, (\neg A)^C \vdash_R \qedsymbol \rightarrow \nleftarrow \Gamma^C \vdash_R A^C
$$
Si consideri, per esempio, l'insieme di clausole 
$$
\{\{b\}, \{\neg b\}, \{\neg a\}\} \vdash_R \qedsymbol
$$
ma per il fatto che la Risoluzione non introduce mai nuovi letterali, 
non si potrà mai a provare 
$$
\{\{b\},\{\neg b\}\} \vdash_R \{a\}
$$
\subsection{Sistemi Assiomatici (Calcoli alla Hilbert)}
I Sistemi Assiomatici sono un tipo di calcolo tradizionale completo 
tout court. Hanno una formalizzazione tra le più semplici, ma non 
sono in particolar modo adatti alla ricerca \textit{human-oriented} e nemmeno 
alla ricerca di prove tramite algoritmi. 

\begin{defi}[Calcoli alla Hilbert]
        Dato un insieme di assiomi (che sono tautologie della Logica), come per esempio 
il seguente, che è corretto e completo per la Logica Proposizionale classica
$$
\begin{cases}
        A \rightarrow (B \rightarrow A) \\
        (A \rightarrow (B \rightarrow C)) \rightarrow ((A \rightarrow B) \rightarrow (A \rightarrow C)) \\
        ( \neg B \rightarrow \neg A) \rightarrow (A \rightarrow B) 
\end{cases}
$$
con $A, B, C \in F_L$ e avendo regole di inferenza come il \textit{modus ponens} 

\begin{prooftree}
        \AxiomC{$A$}
        \AxiomC{$A\rightarrow B$}
        \BinaryInfC{$B$}
\end{prooftree}

una prova di $A$ da una teoria $\Gamma$ nel calcolo definito Calcolo H o 
Calcolo alla Hilbert, ossia 
$$
\Gamma \vdash_H A
$$
è una successione finita di formule 
$$
A_1, A_2, \cdots, A_u
$$
tale che: 
\begin{enumerate}
        \item $A_u = A$ 
        \item ogni $A_i$ per $i = 1, \cdots, u$ è tale che: 
                \begin{enumerate}
                        \item $A_i \in \Gamma$
                        \item $A_i$ è un'istanza di assioma 
                        \item esistono $j,k < i$ tali che $A_j= A_k \rightarrow A_i$, quindi 
                                \begin{prooftree}
                                        \AxiomC{$A_k$}
                                        \AxiomC{$A_k \rightarrow A_i$}
                                        \BinaryInfC{$A_i$}
                                \end{prooftree}
                \end{enumerate}
\end{enumerate}
\end{defi}

\begin{teo}[Completezza Forte del Calcolo H]
$\Gamma \models A \iff \Gamma \vdash_H A$, 
anche per teorie $\Gamma$ infinite. 
\end{teo}

Il Calcolo H non è particolarmente adatto alla deduzione automatica, come 
sottolineato precedentemente. I vari tipi di calcolo corrispondono ad esigenze 
diverse: quando la necessità è la deduzione automatica, vi sono ottime 
ragioni per preferire il Calcolo Refutazionale. Diamo ora qualche evidenza 
del perché non sia così saggio utilizzare il Calcolo H. Quest'ultimo è 
semplice dal punto di vista concettuale, ma tirar fuori prove è più complicato. 

\subsubsection{Differenze tra i due Calcoli}
Verificare se  $\neg \neg A \models A$ è vera. 

\paragraph{Calcolo R}
Bisogna inizialmente trasportare $A \in F_L$ in lettere proposizionali: 
$$
\neg \neg p \models p 
$$
con $p \in L$; nel Calcolo H questo passaggio non è necessario, si potrà 
tranquillamente ragionare sulle metavariabili.
Si studia l'insoddisfacibilità di 
$$
\{ \neg \neg p, \neg p\} \models \bot
$$
che si traduce, in refutazione, in 
$$
\{\{p\}, \{\neg p \}\} \vdash_R \qedsymbol ?
$$
si ha: $\mathbb{R}(\{p\}, \{\neg p\}; p, \neg p) =  \qedsymbol$, pertanto la prima 
formula è vera. 

\paragraph{Calcolo H}
Per il Calcolo di Hilbert bisogna ``inventarsi'' una dimostrazione; si 
comincia a scrivere 
\begin{align*}
\neg \neg A \rightarrow (\neg \neg \neg \neg A \rightarrow \neg \neg A), \neg \neg A ( \in \Gamma)\\
\therefore \neg \neg \neg \neg A \rightarrow \neg \neg A, ( \neg \neg \neg \neg A \rightarrow \neg \neg A) \rightarrow (\neg A \rightarrow \neg \neg \neg \neg A)\\
\therefore \neg A \rightarrow \neg \neg \neg \neg A, (\neg A \rightarrow \neg \neg \neg \neg A) \rightarrow ( \neg \neg A \rightarrow A) \\
\therefore  \neg \neg A \rightarrow A, \neg \neg A \\
\therefore A
\end{align*}

Il calcolo H non presenta un calcolo estremamente meccanico. 
\paragraph{Esercizio}
Dimostrare che $\models A \rightarrow A$

Si prenda una formula $B$ soddisfacibile, già provata, ossia $\models_H B$. 
\begin{prooftree}
        \AxiomC{$B$}
        \AxiomC{$B \rightarrow ( A \rightarrow B)$}
        \BinaryInfC{$A \rightarrow B$}
\end{prooftree}
e 
\begin{prooftree}
        \AxiomC{$A \rightarrow (B \rightarrow A)$}
        \AxiomC{$(A \rightarrow (B \rightarrow A)) \rightarrow (A \rightarrow A)$}
        \BinaryInfC{$(A \rightarrow B) \rightarrow (A \rightarrow A)$}
\end{prooftree}
da cui si può concludere 
$$
(A \rightarrow A)
$$
\subsection{Procedura refutazionale di Davis-Putnam}
Si supponga che sia stato definito un insieme di clausole
$$
S = \{\{p,q\}, \{p, \neg q\}, \{\neg p, q\}, \{\neg p, \neg q\}\} 
$$
e ci si chiede se sia refutabile.
Si ha 
\begin{prooftree}
        \AxiomC{$p,q$}
        \AxiomC{$p,\neg q$}
        \BinaryInfC{$p$}
        \AxiomC{$\neg p,q$}
        \AxiomC{$\neg p,\neg q$}
        \BinaryInfC{$\neg p$}
        \BinaryInfC{$\qedsymbol$}
\end{prooftree}

Utilizzando il calcolo refutazionale $R^*(S)$ si utilizzano molti più passaggi: 
il problema che si vuole affronare e risolvere, per quanto possibile, è 
designare una tecnica che permetta di mantenere la completezza 
refutazionale selezionando, in qualche modo, i risolventi da calcolare. 

\begin{defi}[Sussunzione]
        Una clausola $C_1$ \textbf{sussume} $C_2$ se e solo se 
        $C_1 \subset C_2$, ossia è un insieme proprio di $C_2$. 
        In termini logici, si ha $\models C_1 \rightarrow C_2$, 
        ossia $\{C_1, C_2\} \equiv \{C_1\}$.
\end{defi}        

\begin{defi}[Regola di Sussunzione]
        Sia $S$ un insieme di clausole e sia $S'$ ottenuto da 
        $S$ eliminando tutte le clausole sussunte. Allora si può 
        concludere che 
        $S \equiv S'$. 
\end{defi}

\begin{oss}
        $S'$ non contiene sussunte, ossia non è ulteriormente riducibile 
        per sussunzione.
\end{oss}

\begin{defi}[Clausola Banale]
        Una clausola $C$ è detta \textbf{banale} (trivial) se contiene 
        sia $L$ che  il sua opposto $\bar{L}$ per qualche letterale 
        $l$, ossia $C \equiv \top$ o $\models C$. 
\end{defi}

\begin{defi}[Regola di rimozione banali]
        Rimuovendo da $S$ tutte le banali costruendo $S'$, 
        si ha $S \equiv S'$, e $S'$ non contiene banali.
\end{defi}

\begin{oss}
        Risolvendo tra loro le clausole non banali $C_1$ e $C_2$, 
        allora in $D = R(C_1, C_2; \ell, \bar{\ell})$ si ha 
        che $\ell \notin D$ e $\bar{\ell} \notin D$, 
        cioè $(C_1 \setminus \{\ell\}) \cup (C_2 \setminus \{ \bar{\ell}\}) = (C_1 \cup C_2) \setminus \{\ell, \bar{\ell}\}$
\end{oss}

\begin{oss}
        $R(\{a, \neg a\}, \{a, \neg a\}; a, \bar{a}) = ( \{a, \neg a\} \setminus \{a\}) \cup (\{a, \neg a\} \setminus \{\neg a\}) = \{\neg a\} \cup \{a\} = \{a, \neg a\}$
\end{oss}


\begin{defi}[Passo di DPP]
        Dato un  input $S$ finito, già ripulito di banali e sussunte, 
        un passo di David-Putnam procedure 
        si articola nei seguenti \textit{micropassi}: 
        \begin{enumerate}
                \item Scegliere una lettera proposizionale $p \in L$ 
                        tra quelle che appaiono in $S$. $p$ sarà detto 
                        il \textit{pivot} del passo.
                \item $S'$ è l'insieme delle $p$-esonerate, ossia 
                        l'insieme delle clausole in $S$ in cui non 
                        occorre $p$ come lettera proposizionale, quindi 
                        né $p$ né $\neg p$.
                \item $S''$ è l'insieme delle $p$-risolventi ed è l'insieme 
                        delle clausole ottenute per risoluzione sul pivot 
                        $p$ in $S\setminus S'$. 
                \item $S'''$ è l'insieme  $S' \cup S''$ ripulito, che è 
                        l'output del passo di DPP.
        \end{enumerate}
\end{defi}

\subsubsection{Esempi di DPP}
\paragraph{1}
Sia $S_0 = \{\{a,b,\neg c\}, \{a, \neg b, \neg d\}, \{a, \neg b, \neg c, \neg d\}, 
\{\neg a, d\}, \{\neg a, \neg c, \neg d\}, \{c\}\}$. 
\begin{itemize}
        \item $S_0$ è ripulito. Si sceglie, come pivot, $c$. 
        Le $c$-esonerate sono la seconda e la terza clausola. 
Le $c$-risolventi sono, per esempio, $\{a,b\}, \{a, \neg b, \neg d\}, \{\neg a, \neg d\}$. 
L'output ripulito è 
$$
\{\{a, \neg b, d\}, \{\neg a, d\}, \{a, b\}, \{a, \neg b, \{d\}, \{\neg a, \neg d\}\}
        $$
\item Per il secondo passo, si sceglie come pivot $a$. 
Le $a$-esonerate sono $\emptyset$, mentre le $a$-risolventi sono 
$$
\{\neg b, d\}, \{\neg b, d, \neg d\}, \{b, d\}, \{\neg b, d\}, \{b, \neg d\}, \{\neg b, \neg d\}
$$
L'output ripulito è: 
$$
\{\{\neg b, d\}, \{b, d\}, \{b, \neg d\}, \{\neg b, \neg d\}\}
$$
\item Come terzo passo si prende come pivot $b$. Le $b$-esonerate sono 
$$
\emptyset
$$
e le $b$-risolventi sono 
$$
\{d\},\{d, \neg d\}, \{\neg d, d\} \{\neg d\}
$$
mentre l'output ripulito è 
$$
\{\{d\}, \{\neg d\}\}
$$
\item Per l'ultimo passo si è obbligati a scegliere come pivot $d$. 
Si ha che le $d$-esonerate è nuovamente l'insieme vuoto $\emptyset$, 
mentre le $d$-risolventi sono 
$$
\qedsymbol
$$
e l'output è 
$$
\{\qedsymbol\}
$$
\end{itemize}
Dato che è stata ottenuta la clausola vuota, si dichiara la clausola 
iniziale $S_0$ insoddisfacibile. 
\paragraph{2}
Sia $S_0 = \{\{a, \neg b, c\}, \{b, \neg c\}, \{a,c,\neg d\}, \{\neg b, \neg d\}, \{a,b,d\}, \{\neg a, d,b\} \{b, \neg c, d\}\{b, c, d\}\}$. 
\begin{itemize}
        \item Ripulito, si trovano le clausole 
$$ 
\{\{a, \neg b, c\}, \{b, \neg c\}, \{a,c,\neg d\}, \{\neg b, \neg d\}, \{a,b,d\}, \{\neg a, d,b\}\}
$$
Si sceglie, come pivot, $b$. 
le $b$- esonerate sono
$$
\{a, c, \neg d\}
$$
e le $b$-risolventi sono: 
$$
\{a, c, \neg c\}, \{a,c,d\}, \{a, \neg a, c, d\}, \{\neg c, \neg d\}, \{a, \neg d, d\}, \{\neg a, \neg d, d\}
$$
e l'output è 
$$
S_1 := \{a, c,d\}, \{\neg c, \neg d\}, \{a, c, \neg d\}
$$
\item Al secondo passo si sceglie, come pivot, $c$. 
Non vi sono $c$-esonerate, mentre 
le $c$-risolventi sono 
$$
\{a, d, \neg d\}, \{a, \neg d, \neg d\}
$$
e l'output è 
$$
S_2 := \{a, \neg d\}
$$
\item Si sceglie, come pivot, $a$. 
Non vi sono $a$-esonerate e non vi sono $a$-risolventi; 
L'output di questo passaggio è l'insieme vuoto $\emptyset$ e si dimostra 
che questo implica che $S_0$ è soddisfacibile.
La lettera proposizionale $d$ non è mai stata scelta come lettera pivot, quindi è 
una \textit{fuoriuscita}. 
\end{itemize}

\begin{defi}
        In un'esecuzione di DPP le lettere che non sono mai pivot si 
        definiscono \textit{fuoriuscite}.
\end{defi}

Delineamo, prima di affrontare la dimostrazione, la strategia che con questi 
dati costruisce un assegnamento che soddisfa $S$.
Sia $\mathfrak{v}: L \rightarrow \{0,1\}$ tale che $\mathfrak{v} \models S_0$. 
Si definisce
$$
\mathfrak{v} := 
\begin{cases}
        a & \mathfrak{v}(a) = 1\\
        b & \mathfrak{v}(b) = 1\\
        c & \mathfrak{v}(c) = 1\\
        d & \mathfrak{v}(d) = 0 \text{ per convenzione per le fuoriuscite, arbitrario}\\
\end{cases}
$$
Si definisce $\mathfrak{v}(a)$ come un valore che soddisfi $S_2$, ossia $\{a, \neg d\}$.
Si definisce $\mathfrak{v}(c)$ come un valore che soddisfi $S_1$ (si è liberi, in 
quanto tutte le clausole sono già soddisfatte dagli altri assegnamenti). 
Si definisce $\mathfrak{v}(b)$ come un valore che soddisfi $S_0$.

\subsubsection{Teorema di completezza (e correttezza) refutazionale di DPP}
\begin{teo}
        Un insieme \textbf{finito} $S$ di clausole nelle lettere proposizionali 
        $p_1, p_2, \cdots, p_n$ ($:= Var(S)$) è insoddisfacibile se e solo se 
        entro al più $n$ passi si ottiene la clausola vuota $\qedsymbol$. 
        $S$ è soddisfacibile se e solo se entro al più $n$ passi si ottiene 
        l'insieme vuoto di clausole $\emptyset$.
        Non vi sono altri modi di terminare. 
\end{teo}

\begin{proof}[Terminazione]
        Sia $DPP(S) := S_0, S_1, \cdots, S_k$ una sequenza di clausole, dove
        $S_0$ è la ripulitura di $S$ e $S_i$ è ottenuto da $S_{i-1}$ attraverso un 
        passo di DPP sul pivot $q_i \in \{p_1, \cdots, p_n\}$.
        Si osserva che $S_i$ non conterrà più alcuna occorrenza di $q_i$ e, quindi, 
        dopo al più $k \leq n$ passi si ottiene $Var(S_k) = \emptyset$, dunque 
        o $S_k = \{ \qedsymbol \}$ oppure $S_k = \emptyset$. 
\end{proof}

\begin{proof}[Correttezza e Completezza Refutazionale di DPP]
        La correttezza di DPP si può specificare in due modi: 
        $$
        \begin{cases}
                S_k = \{\qedsymbol\} \implies S_0 \text{ insoddisfacibile } \\
                S_0 \text{ soddisfacibile } \implies S_k = \emptyset
        \end{cases}
        $$
        La prima delle due versioni è ``gratuitamente'' provata dal Lemma di Correttezza, 
        ancora valido, poiché il calcolo dei risolventi non è cambiato
        (pertanto la correttezza è provata.)

        La completezza di DPP si può specificare nei due modi rimanenti: 
        $$
        \begin{cases}
                S_0 \text{ insoddisfacibile } \implies S_k = \{\qedsymbol\} \\
                S_k = \emptyset \implies S_0 \text{ soddisfacibile } 

        \end{cases}
        $$
        Ci basta dimostrare quest'ultima parte, ossia che se 
        $S_k = \emptyset$ allora $S_0$ è soddisfacibile: 
        questo fatto è chiamato anche \textit{Lemma di Model Building}, poiché 
        la dimostrazione si occupa di mostrare come costruire l'assegnamento 
        che soddisfa $S$. 
        
        Si dimostra per induzione decrescente su $k$, ossia induzione 
        crescente su $t=k -i$:
        \paragraph{base}
        $$
        S_k = \emptyset \rightarrow S_k \text{ soddisfacibile.}
        $$
        \paragraph{passo induttivo} si assume che $S_{i+1}$ soddisfacibile e si mostra $S_i$ 
        soddisfacibile, ossia esiste $\mathfrak{v}: L \rightarrow \{0,1\}$ tale 
        che $\mathfrak{v} \models S_i$. Dato $\mathfrak{v} : L \rightarrow \{0,1\}$ 
        tale che $\mathfrak{v} \models S_{i+1}$ si definiscono le due varianti 
        $$
        \mathfrak{v}^+ : L \rightarrow \{0,1\} ~ ~ ~ \mathfrak{v}^+(p) = 1 
        $$
        $$
        \mathfrak{v}^- : L \rightarrow \{0,1\} ~ ~ ~ \mathfrak{v}^-(p) = 0
        $$
        dove $p$ è il pivot di $S_i \leadsto S_{i+1}$, 
        mentre 
        $$
        \mathfrak{v}(q) = \mathfrak{v}^+(q) = \mathfrak{v}^-(q) \forall q \in L q \neq p
        $$
        Si mostrerà che una delle due varianti soddisfa $S_i$. 
        Per assurdo, si assume $\mathfrak{v}^+, \mathfrak{v}^-$ non verifichino 
        $S_{i}$. 

        Allora esistono $C_1, C_2 \in S_i$ tali che $\mathfrak{v}^+(C_1) = 0$ 
        e $\mathfrak{v}^-(C_2) = 0$. Si assume che
        $\neg p \in C_1$ (senza perdita di generalità), poiché altrimenti se $p \in C_1$ allora 
        $\mathfrak{v}^+(C_1) = 1$, assurdo, e se 
        $p \notin C_1 e \neg p_1 \notin C_1$, allora $C_1$ sarebbe $p$-esonerata, 
        dunque $C_1 \in S_{i+1}$ e $\mathfrak{v} \models C_1$ e $\mathfrak{v}^+ \models C_1$, 
        assurdo. Analogamente per $p \in C_2$. 

        Si conclude che 
        $D = R(C_1, C_2; \neg p, p)$ e $D \in p$-risolventi implica $D \in S_{i+1}$, 
        che implica $\mathfrak{v} \models D$ e allora 
        $$
        \mathfrak{v} \models D \rightarrow \mathfrak{v}^+ \models D \text{ oppure } \mathfrak{v}^- \models D \rightarrow \mathfrak{v}^+ \models C_1 \text{ oppure } \mathfrak{v}^- \models C_2
        $$
        che è assurdo, dunque $S_i$ è soddisfacibile, e pertanto $S_0$ è soddisfacibile.
\end{proof}

\begin{oss}
Sia $S$ un insieme finito di clausole tale che $Var(S) \subseteq \{p_1, \cdots, p_n\}$. 
Allora $DPP(S)$ termina in $k \leq n$ passi. Se $k \leq n \leq ||S||$, 
si potrebbe concludere che si ha una procedura che funziona in tempo 
polinomiale e decide se $S$ è soddisfacibile o meno, e quindi $P = NP$. 
\end{oss}

Se si potesse garantire che $||S_i||$ sia sempre polinomiale rispetto 
a $||S||$ allora davvero $k \leq n$ passi di DPP porterebbero a 
un tempo di esecuzione complessivo che è polinomiale rispetto a $||S||$ 
e dunque $SAT \in P$, e dunque $P = NP$, ma purtroppo questo non si 
può garantire.

Un risultato, dovuto ad Haken, afferma che ogni prova per refutazione del 
\textit{Principio della Piccionaia} su $n$ ``piccioni''
richiede tempo almeno $T(n) = \Omega(2^n)$. 


\begin{defi}[Principio della Piccionaia]
        Il Principio della Piccionaia, notazionalmente espresso su un 
        numero naturale $n$ come 
        $$
        PHP(n)
        $$
        è esresso affermando che $n+1$ piccioni non possono occupare 
        $n$ posti nella piccionaia in modo che ogni posto abbia al più un 
        piccione. 
\end{defi}

Come formalizzare in clausole il $PHP(n)$? Definendo $P$ i ``piccioni'' e 
$H$ i ``posti'', si può definire
$$
\bigwedge_{i \in P} \bigvee_{j \in H} p_{ij} \text{ (ogni piccione ha trovato posto }
$$
In altre parole, per ogni piccione, il piccione ha trovato almeno un posto, 
ossia il piccione $i$ sta nel posto $j$. 
Questo, però va unito al fatto che ogni piccione ha al più un posto: 
$$
\bigwedge_{i \neq j ~ i,j \in P}\bigwedge_{k \in H} (\neq p_{ik} \lor \neq p_{jk}) \text{ nessuna coppia di piccioni 
sta nel posto }k 
$$
in altre parole non c'è nessun posto che contenga contemporaneamente due piccioni. 
Quindi, la formalizzazione finale sarà 
$$
PHP(n) := \bigwedge_{i \in P} \bigvee_{j \in H} p_{ij} ~~ \land ~~ \bigwedge_{i \neq j ~ i,j \in P}\bigwedge_{k \in H} (\neq p_{ik} \lor \neq p_{jk})
$$
Mostrare che il Principio della Piccionaia è vero consiste nel mostrare che 
$PHP(n)$ è insoddisfacibile, ossia che $PHP(n) \in CNFUNSAT \forall n \in \mathbb{N}$. 
Haken ha dimostrato che la risoluzione di questo problema impiega, per ogni $n$, 
tempo esponenziale. 


\subsubsection{Complessità di DPP}
Dato che i passi di DPP sono ``pochi'', $k \leq |Var(S)|$, anche se nel caso 
peggiore DPP richiede tempo esponenziale, nella pratica è un 
algoritmo molto più efficiente di $R^*$ e delle tavole di verità. 
Alcuni frammenti, ossia sottolinguaggi di CNFSAT, sono noti avere tempo di 
decisione, attraverso la DPP, polinomiale, come KROMSAT e HORNSAT; 
una CNF è HORN se e solo se ogni sua clausola è una clausola di Horn, ossia 
una clausola che ha la forma della clausola vuota o è un'unità $\{p\}$ o 
$\{p, \neg p_1, \neg p_2, \cdots\}$ o più genericamente contiene al più un 
letterale positivo; una clausola è di Krom se ha al più due letterali 
(quindi $KROMSAT = 2CNFSAT$). 

\subsubsection{Davis Putnam Logemann Loveland Procedure}
La DPLL è un metodo di ``reingegnerizzazione'' della DPLL. Su un insieme $S$ di 
clausole finito, DPLL cerca di costruire un assegnamento che soddisfi $S$. 
L'idea è di costruire un assegnamento parziale che viene esteso ad ogni passo 
ad una nuova lettera proposizionale $p$, chiamata pivot. Se si giunge ad assegnare 
tutte le lettere in $Var(S)$ si è ottenuto un assegnamento che mostra $S$ soddisfacibile: 
se non si giunge a tal punto, il teorema di completezza e correttezza di DPLL 
dimostra che $S$ è insoddisfacibile. Si propaga, ad ogni passo, tutta l'informazione 
che si guadagna estendendo l'assegnamento al pivot. Sostanzialmente, quello che era 
il passo finale della DPP si utilizza come azione fondamentale nella DPLL. 

Può accadere che non vi siano informazioni sfruttabili per quanto riguarda 
la scelta del pivot ad un certo passo codificata nelle regole che verranno 
date, allora si ``spezza'' la procedura in due parti: in una si assegna al pivot 
il valore di verità $1$ e alla seconda si assegna al pivot il valore di verità 
$0$ e la prova risulta dunque in una 
struttura ad albero i cui rami si visitano in ``backtracking'', la cui implementazione
(anche nel caso di backtracking non cronologico) è soggetto di ampie ricerche; 
se in un ramo si raggiunge $\emptyset$ si è costruito un assegnamento che soddisfa 
l'insieme iniziale e se su \textit{tutti} i rami si raggiunge la clausola 
vuota, allora $S$ è insoddisfacibile.

Le regole utilizzate dalla DPPL sono quelle della DPP adattate a questo contesto; 
il punto iniziale è l'assegnamento vuoto. 

\begin{defi}[Assegnamento Parziale]
        Un assegnamento parziale è una funzione parziale 
        $$
        \mathfrak{v}: L \rightarrow \{0,1,?\}
        $$
        Un assegnamento vuoto è un assegnamento parziale tale che 
        $\mathfrak{v}(p_i) = ? \forall p_i \in \{p_1, \cdots, p_n\}$.
        Un assegnamento completo o totale sulle prime $n$ lettere proposizionali 
        è una mappa $\mathfrak{v}: \{p_1, \cdots, p_n\} \rightarrow \{0,1,?\}$ 
        tale che $\mathfrak{v}(p_i) \neq ? \forall p_i \in \{p_1, \cdots, p_n\}$.
\end{defi}

\paragraph{Regole di DPLL}
\begin{itemize}
        \item Regola iniziale: $\emptyset \vdash S$ è la radice della prova. 
        \item Sussunzione: se $\mathfrak{v}(p_i) = 1$ allora si possono cancellare 
                da $S$ tutte le clausole che contengono  il letterale $p_i$
                \begin{prooftree}
                        \AxiomC{$\mathfrak{v}, \mathfrak{v}(p_i) = 1 \vdash S \cup \{\{p_i\} \cup C\}$}
                                \UnaryInfC{$\mathfrak{v}, \mathfrak{v}(p_i) = 1 \vdash S$}
                \end{prooftree}
                analogamente, se $\mathfrak{v}(p_i) = 0$, allora si possono cancellare 
                da $S$ tutte le clausole che contengono il letterale $\neg p_i$. 
        \item Risoluzione unitaria: se $\mathfrak{v}(p_i) = 0$ si cancella 
                da ogni clausola di $S$ il letterale $p_i$
                \begin{prooftree}
                        \AxiomC{$\mathfrak{v}(p_i) = 0$}
                        \AxiomC{$(\{\neg p_i, \{p_i,\cdots\})$}
                        \BinaryInfC{$C$}
                \end{prooftree}
                Ossia, in termini di ``nodi'' DPLL
                \begin{prooftree}
                        \AxiomC{$\mathfrak{v}, \mathfrak{v}(p_i) = 0 \vdash S \cup \{\{p_i\}\cup C\}$}
                                \UnaryInfC{$\mathfrak{v},\mathfrak{v}(p_i) = 0 \vdash S \cup \{C\}$}
                \end{prooftree}
               mentre se $\mathfrak{v}(p_i) = 1$, 
        allora si elimina da ogni clausola di $S$ il letterale $\neg p_i$.  
        \item Asserzione: se $S$ contiene la clausola $\{p_i\}$, allora 
                si estende $\mathfrak{v}$ ponendo $\mathfrak{v}(p_i) = 1$, 
                cancellando $\{p_i\}$ da $S$
                \begin{prooftree}
                        \AxiomC{$\mathfrak{v}\vdash S \cup \{\{p_i\}\}$}
                        \UnaryInfC{$\mathfrak{v}, \mathfrak{v}(p_i) = 1 \vdash S$}
                \end{prooftree}
                mentre si fa al contrario se $S$ contiene $\{\neg p_i\}$.
        \item Letterale puro: Se il letterale $p_i$ occorre in $S$ e $\neg p_i$ non 
                occorre, allora si estende $\mathfrak{v}$ ponendo $\mathfrak{v}(p_i) = 1$, 
                \begin{prooftree}
                        \AxiomC{$\mathfrak{v}\models S$}
                        \UnaryInfC{$\mathfrak{v}, \mathfrak{v}(p_i)=1 \vdash S$}
                \end{prooftree}
                (chiaramente se $p_i \in C \in S, \neg p_i \notin C \forall C \in S$),
                mentre si fa contrario se $\neg p_i$ occorre in $S$ e $p_i$ 
                non occorre in $S$, ponendo $\mathfrak{v}(p_i) = 0$. 
        \item Spezzamento: in ogni momento, si può biforcare la prova in due 
                sottoprove, dando origine ad una struttura ad albero, in cui 
                in una si pone per il pivot scelto $\mathfrak{v}(p) = 1$ 
                e nell'altra si pone $\mathfrak{v}(p)= 0$. 
                \begin{prooftree}
                        \AxiomC{$\mathfrak{v} \vdash S$} 
                        \UnaryInfC{$\mathfrak{v}, \mathfrak{v}(p_i) = 0 \vdash S ~~~~ ||  ~~~~ \mathfrak{v}, \mathfrak{v}(p_i) = 1 \vdash S$}
                \end{prooftree}
        \item Terminazione: se in un ramo si ottiene $\mathfrak{v} \vdash \emptyset$ 
                allora si prova che $\mathfrak{v}$ è completo su 
                $Var(S)$ e $\mathfrak{v} \models S$. Al contrario, se su tutti 
                i rami si ottiene $\mathfrak{v} \vdash \qedsymbol$, 
                allora $S$ è insoddisfacibile. 
        \item Ogni ramo o ha come foglia $\mathfrak{v} \vdash \emptyset$ 
                oppure $\mathfrak{v} \vdash \qedsymbol$.
\end{itemize}

\backmatter 
\end{document}
