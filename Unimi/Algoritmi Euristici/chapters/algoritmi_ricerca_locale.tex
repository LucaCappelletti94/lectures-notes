\providecommand{\main}{..}
\documentclass[\main/main.tex]{subfiles}
\providecommand{\f}{\mathcal{F}}
\begin{document}
\chapter{Algoritmi di ricerca locale}
\section{Gli algoritmi di scambio}
\begin{multicols}{2}
\begin{definition}[Euristica di scambio]
    In Ottimizzazione Combinatoria ogni soluzione \(x\) è un sotto-insieme di \(B\). Un'\textbf{euristica di scambio} aggiorna passo per passo un sotto-insieme \(x^{\rnd{t}}\):
\begin{enumerate}
    \item Parte da una soluzione ammissibile \(x^{\rnd{0}} \in X\) trovata in qualche modo.
    \item Scambia coppie \(\rnd{A, D}\) di sotto-insiemi, \(A\) esterno e \(D\) interno a \(x^{\rnd{t}}\), generando una famiglia di soluzioni ammissibili.
    \[
        x'_{A, D} = x \cup A \setminus D \quad A \subseteq B \setminus X\;\land\;D \subseteq x
    \]
    \item Ad ogni passo \(t\), sceglie quali sotto-insiemi scambiare in base a un opportuno criterio di scelta \(\phi\rnd{x, A, D}\).
    \[
        \rnd{A^*, D^*} = \argmin_{A, D} \phi\rnd{x, A, D}
    \]
    \item Genera la nuova soluzione corrente eseguendo le modifiche scelte:
    \[
        x^{\rnd{t+1}} = x^{\rnd{t}}\cup A^* \setminus D^*
    \]
    \item Se vale una condizione di fine l'euristica termina, altrimenti ricomincia dal secondo step.
\end{enumerate}

I due elementi fondamentali di un'euristica di scambio sono le coppie di sotto-insiemi disponibili per uno scambio a partire da \(x\) ed il criterio di scelta \(\phi\).
\end{definition}
    \begin{definition}[Intorno]
        \(N: X \rightarrow 2^x\) è una funzione che associa a ogni soluzione ammissibile \(x \in X\) un sotto-insieme di soluzioni ammissibile \(N(x) \subseteq X\).
        
        Una tipica definizione di intorno di \(x\), con un parametro intero \(k\), è l'insieme delle soluzioni con distanza di Hamming non superiore a \(k\) da \(x\).
        \[
            N_{H_k}\rnd{x} = \crl{x' \in X: d_H\rnd{x, x'} \leq k}
        \]
        
        Un'altra definizione comune è il numero di operazioni, quali \textbf{aggiunta}, \textbf{eliminazione} e \textbf{scambio}, che separano due soluzioni.
    \end{definition}
    \begin{definition}[Grafo di ricerca]
        Un grafo i cui nodi rappresentano le soluzioni ammissibili \(x \in X\) e gli archi collegano ogni soluzione \(x\) a quelle del suo intorno \(N(x)\)
        
        Un arco viene detto \textbf{mossa} perché trasforma una soluzione in un'altra muovendo elementi.
    \end{definition}
    \begin{definition}[Distanza di Hamming]
        La distanza di Hamming tra due soluzioni \(x\) ed \(x'\) è pari al numero di elementi per cui i loro vettori di incidenza differiscono:
        \[
            d_H\rnd{x, x'} = \sum_{i \in B} \abs{x_i - x'_i}
        \]
    \end{definition}
\begin{observation}[Quando un'euristica di scambio può trovare una soluzione ottima?]
    Un'euristica di scambio può trovare soluzione ottima solo se almeno una soluzione ottima è raggiungibile da ogni soluzione iniziale.
\end{observation}
\begin{definition}[Grado connesso all'ottimo]
Si dice che il grafo di ricerca è \textbf{connesso all'ottimo} quando contiene per ogni soluzione \(x \in X\) un cammino da \(x\) a \(X^*\)
\end{definition}
\begin{definition}[Grafo fortemente connesso]
Si dice che il grafo di ricerca è fortemente connesso quando contiene per ogni coppia di soluzioni \(x, y \in X\) un cammino da \(x\) a \(y\).
\end{definition}
\begin{definition}[Euristica steepest descent]
    Molto spesso il criterio di scelta \(\phi\) della nuova soluzione in \(N(x)\) è la funzione obiettivo, cioè ad ogni passo dell'euristica ci si sposta dalla soluzione corrente alla miglior soluzione del suo intorno. Per evitare comportamenti ciclici, si accettano solo soluzioni miglioranti.
    
    In generale l'euristica  steepest descent non trova l'ottimo globale.
\end{definition}
\begin{definition}[Intorno Esatto]
    Un intorno esatto è una funzione intorno \(N: X \rightarrow 2^X\) tale che ogni ottimo locale è anche ottimo globale:
    \[
        \bar{X}_N = X^*
    \]
    
    Sono estremamente rari. Un caso rilevante è lo scambio fra variabili di base e fuori base usato dall'\textbf{algoritmo del simplesso}.
\end{definition}
\begin{observation}[Fitness-Distance correlation]
    Se la correlazione tra qualità e vicinanza agli ottimi globali è forte conviene costruire buone soluzioni iniziali perché avvicinano la ricerca locale a buoni ottimi locali, conviene quindi intensificare piuttosto che diversificare.

    Al contrario, se la correlazione è debole, una buona inizializzazione è meno importante e conviene diversificare piuttosto che intensificare.
\end{observation}
\begin{definition}[Landscape]
    Si definisce \textbf{landscape} la terna \(\rnd{X, N , f}\) dove:
    \begin{description}
        \item[\(X\)] è lo spazio di ricerca.
        \item[\(N: X \rightarrow 2^X\)] è la funzione intorno.
        \item[\(f: X \rightarrow \N\)] è la funzione obbiettivo.
    \end{description}
    
    Si può vedere come il grafo di ricerca pesato sui nodi con l'obbiettivo.
\end{definition}
\begin{observation}[Come si può stimare la complessità di un landscape?]
    La complessità di un landscape si può stimare empiricamente in vari modi:
\begin{enumerate}
    \item Eseguendo una random walk nel grafo di ricerca.
    \item Determinando la sequenza di valori dell'obbiettivo.
    \item Calcolandone il valore medio campionario.
    \item Calcolando il \textbf{coefficiente empirico di auto-correlazione}.
\end{enumerate}
\end{observation}

\begin{definition}[Coefficiente di autocorrelazione]
    Il coefficiente empirico di auto-correlazione:
    \[
        r(i) = \frac{\frac{\sum_{t=1}^{t_{\max}-1} \rnd{f^{(t)} - \bar{f}}\rnd{f^{(t+i)} - \bar{f}}}{t_{\max}-i}}{\frac{\sum_{t=1}^{t_{\max}} \rnd{f^{(t)} - \bar{f}}^2}{t_{\max}}}
    \]
    Si tratta di una funzione di \(i\) che parte da \(r(0) = 1\), e in genere va calando.
    
    Se rimane a circa \(1\) il landscape risulta \textbf{liscio}, se varia bruscamente il landscape è \textbf{accidentato}.
\end{definition}
\begin{definition}[Plateau]
    Un plateau di valore \(f\) è ciascun sottoinsieme di soluzioni di valore \(f\) che siano adiacenti nel grafo di ricerca.
    
    Plateau molto ampi ostacolano la scelta della soluzione dell'euristica steepest descent, perché fanno dipendere dall'ordine di visita di \(N(x)\).
    
    Essi sono utili siccome è possibile analizzare il grafo di ricerca dividendolo per livelli di obiettivo.
\end{definition}
\begin{definition}[Bacini di attrazione]
    Un \textbf{bacino di attrazione} di una soluzione di ottimo locale \(\bar{x}\) è l'insieme delle soluzioni \(x^{\rnd{0}} \in X\) tali che l'euristica steepest descent partendo da \(x^{\rnd{0}}\) produca come risultato \(\bar{x}\).
    
    L'euristica ha tipicamente risultati buoni se i bacini sono pochi e ampi.
\end{definition}
\begin{observation}[Come si può esplorare l'intorno?]
    Due strategie sono possibili: la \textbf{ricerca esaustiva} in cui si valutano tutte le soluzioni dell'intorno, che spesso richiede molto tempo, o l'\textbf{esplorazione efficiente dell'intorno}, in cui anziché visitare tutto l'intorno si trova la sua soluzione ottima risolvendo qualche forma di problema ausiliario.
\end{observation}
\begin{definition}[Conservazione parziale dell'intorno]
    Eseguendo un'operazione \(o \in \mathcal{O}\) su due soluzioni simili \(x, x' \in X\) spesso:
\begin{enumerate}
    \item L'operazione è ammissibile per entrambe le soluzioni
    \item La variazione della funzione obbiettivo è la stessa
\end{enumerate}
In tali casi conviene:
\begin{enumerate}
    \item Calcolare la variazione della funzione obbiettivo per ogni operazione ammissibile e conservare i valori a parte.
    \item Eseguire l'operazione \(o^*\) migliore, generando la nuova soluzione \(x'\).
    \item Calcolare la variazione della funzione obbiettivo della nuova soluzione \(x'\) solo per le operazioni valide tranne quelle utilizzate per arrivare in questa soluzione e recuperare i valori salvati per \(o \in \mathcal{O}_{xx'}\) che sono ancora validi.
    \item tornare al punto 2.
\end{enumerate}
\end{definition}
\end{multicols}
\end{document}