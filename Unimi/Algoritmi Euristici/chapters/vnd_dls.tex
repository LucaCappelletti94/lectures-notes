\providecommand{\main}{..}
\documentclass[\main/main.tex]{subfiles}
\begin{document}
\chapter{Variable Neighbourhood Descent e Dynamic Local Search}
\section{Variable Neighbourhood Descent (VND)}
\begin{multicols}{2}
\begin{observation}[Cosa sfrutta la VND?]
    La \textbf{Variable NeighbourHood Descent} sfrutta il fatto che gli ottimi locali sono relativi all'intorno scelto: cambiando intorno, in genere un ottimo locale non è più tale.
\end{observation}
\begin{observation}[Come procede la VND?]
    Nella VND, una volta definito un insieme di intorni \(N_1, \ldots, N_{k_{\max}}\):
    \begin{enumerate}
        \item Si parte con \(k=1\)
        \item Si trova un ottimo locale \(\bar{x}\) rispetto a \(N_k\) con un'euristica \textbf{steepest descent}
        \item Si aggiorna \(k\)
        \item Se \(\bar{x}\) è un ottimo locale per tutti gli \(N_k\) si termina, altrimenti si torna al punto 2.
    \end{enumerate}
\end{observation}
\begin{observation}[Quali sono le differenze tra VND e VNS?]
    Vi è una stretta relazione fra algoritmi VND e VNS: le differenze fondamentali sono che nella VND:
    \begin{enumerate}
        \item Ad ogni passo la soluzione corrente è la migliore nota.
        \item Gli intorno sono esplorati anziché usati per estrarre soluzioni casuali, quindi non sono mai enormi.
        \item Gli intorni non formano necessariamente una gerarchia, quindi l'aggiornamento di \(k\) può non essere un incremento.
        \item Quando si raggiunge un ottimo locale per ogni \(N_k\), si termina.
    \end{enumerate}
\end{observation}
\begin{observation}[Quali strategie per lo scorrimento di intorno esistono nella VND?]
    Ci sono due categorie principali di metodi VND: i metodi ad \textbf{intorno gerarchico} ed i metodi con \textbf{intorno eterogeneo}. 
\end{observation}
\begin{definition}[Metodi ad intorno gerarchico]
    Nei metodi con \textbf{intorno gerarchico} si vuole:
    \begin{enumerate}
        \item Sfruttare a fondo gli intorni piccoli e rapidi.
        \item Ricorrere a quelli grandi e lenti solo per uscire dagli ottimi locali
    \end{enumerate}
    Di conseguenza l'aggiornamento di \(k\) funziona come nella VNS:
    \begin{enumerate}
        \item quando non si trovano miglioramenti in \(N_k\), si incrementa \(k\).
        \item quando si trovano miglioramenti in \(N_k\), \(k\) torna a \(1\).
    \end{enumerate}
\end{definition}
\begin{definition}[Metodi a intorno eterogeneo]
    Nei metodi a \textbf{intorno eterogeneo} si vuole sfruttare la potenzialità di intorni topologicamente diversi tra loro, di conseguenza \(k\) scorre progressivamente i valori da \(1\) a \(k_{\max}\).
\end{definition}
\begin{observation}[Quando termina lo scorrimento degli intorni nella VND?]
    Si termina quando la soluzione corrente è ottimo locale per tutti gli \(N_k\):
    \begin{enumerate}
        \item Nel caso gerarchico, si termina quando fallisce \(N_{k_{\max}}\).
        \item Nel caso eterogeneo, occorre un segnalatore di miglioramento (flag).
    \end{enumerate}
\end{observation}
\end{multicols}

\clearpage
\section{Dynamic Local Search (DLS)}
\begin{multicols}{2}
\begin{observation}[Come si relaziona la DLS rispetto alla VND?]
    La \textbf{Dynamic Local Search} è un approccio complementare alla VND, conserva l'intorno iniziale e modifica la funzione obiettivo.
\end{observation}
\begin{observation}
    Il Dynamic Local Search (DLS) si usa spesso nei problemi in cui l'obbiettivo è poco utile (per esempio nel caso di ampi \textit{plateau}).
\end{observation}
\begin{observation}[Quale è l'idea fondamentale del Dynamic Local Search (DLS)?]
    L'idea fondamentale del Dynamic Local Search (DLS) è di:
    \begin{itemize}
        \item Associare all'insieme una funzione di penalità \(w: B \rightarrow \N\).
        \item Costruire una funzione ausiliaria \(\bar{f}\rnd{f\rnd{x}, w\rnd{x}}\) che combina la funzione obbiettivo \(f\) con la penalità \(w\).
        \item Applicare un'euristica di scambio \textbf{steepest descent} che ottimizza \(\bar{f}\).
        \item Aggiornare la penalità \(w\) in base ai risultati e riapplicare l'euristica.
    \end{itemize}
\end{observation}
\begin{definition}[DLS con penalità additiva]
    L'idea fondamentale del Dynamic Local Search possiede molte varianti, un esempio è utilizzare una \textbf{penalità additiva}:
     \[
        \bar{f}\rnd{x} = f\rnd{x} + \sum_{i \in x} w_i
    \]
\end{definition}
\begin{definition}[DLS con penalità moltiplicativa]
    L'idea fondamentale del Dynamic Local Search possiede molte varianti, un esempio è utilizzare una \textbf{penalità moltiplicative}:
     \[
       \bar{f}\rnd{x} = \sum_{i \in x} w_i \phi_i
    \]
\end{definition}
\begin{observation}[Nella DLS le penalità possono subire quale tipo di aggiornamento?]
    Le penalità possono subire:
    \begin{enumerate}
        \item Un aggiornamento casuale: perturbazione "rumorosa" dei costi
        \item Un aggiornamento basato sulla memoria, che favorisca gli elementi più frequenti (intensificazione) o rari (diversificazione).
    \end{enumerate}
\end{observation}
\begin{observation}[Nella DLS le penalità quando possono subire un aggiornamento?]
    L'aggiornamento delle penalità nella DLS può avvenire:
    \begin{enumerate}
        \item Ad ogni singola esplorazione di intorno.
        \item Quando si arriva a un ottimo locale per \(\bar{f}\).
        \item Quando la miglior soluzione nota \(x^*\) non cambia per parecchio.
    \end{enumerate}
\end{observation}
\begin{definition}[Come si applica il DLS al MCP?]
    Dato un grafo non orientato, si cerca una clique di cardinalità massima:
    \begin{itemize}
        \item L'euristica di scambio è una VND che usa gli intorni.
        \begin{description}
            \item[\(N_{A_1}\) aggiunta di un vertice] La soluzione migliora sempre, ma l'intorno è molto piccolo e spesso vuoto.
            \item[\(N_{S_1}\) scambio di un vertice interno con un esterno] L'intorno è più grande, ma forma un \textit{plateau}.
        \end{description}
        \item L'obbiettivo non fornisce una direzione utile in nessuno dei due intorni.
        \item Si associa a ogni vertice \(i\) una penalità \(w_i\), inizialmente nulla.
        \item L'euristica di scambio minimizza la penalità totale, all'interno dell'intorno.
        \item Si aggiorna la penalità:
        \begin{itemize}
            \item Quando termina l'esplorazione di \(N_{S_1}\): la penalità dei vertici della clique corrente aumenta di \(1\).
            \item Dopo un dato numero di esplorazioni: tutte le penalità non nulle diminuiscono di \(1\).
        \end{itemize}
    \end{itemize}
    La logica del metodo consiste nel tendere a:
    \begin{itemize}
        \item espellere i vertici interni.
        \item in particolare, quelli rimasti interni più a lungo.
    \end{itemize}
\end{definition}
\end{multicols}
\clearpage
\begin{multicols}{2}
\begin{definition}[DLS per il MAX-SAT]
    Date \(m\) disgiunzioni logiche dipendenti da \(n\) variabili logiche, si cerca un assegnamento di verità che soddisfi il massimo numero di formule.
    \begin{itemize}
        \item L'intorno \(N_{F_1}\) è generato invertendo il valore di una variabile.
        \item Si associa a ogni formula logica una penalità \(w_j\) inizialmente pari a \(1\).
        \item L'euristica di scambio massimizza il numero di formule soddisfatte.
        \item La funzione modificata massimizza il numero più la penalità totale.
        \item Si aggiorna la penalità:
        \begin{itemize}
            \item Quando si raggiunge un ottimo locale:
            \[
                w_j = \a_{\text{us}}w_j \quad \forall j \in U\rnd{x}, \text{ le formule insoddisfatte}
            \]
            con \(\a_{\text{us}}>1\) per favorire le formule attualmente insoddisfatte.
            \item Con una certa probabilità o dopo un certo numero di aggiornamenti:
            \[
                w_j = \rnd{1-\rho} w_j + \rho \quad \forall j
            \]
            per riuniformare le penalità al valore minimo unitario.
        \end{itemize}
    \end{itemize}
\end{definition}
\begin{observation}[Logica del DLS per il MAX-SAT]
    La logica del metodo consiste nel tendere a:
    \begin{itemize}
        \item Soddisfare le formule attualmente insoddisfatte (diversificazione).
        \item In particolare, quelle rimaste insoddisfatte più a lungo e più di recente (memoria).
        \item Valori bassi di \(\rho\), il parametro dell'oblio, conservano le penalità (intensificazione), valori alti le cancellano (diversificazione).
    \end{itemize}
\end{observation}
\begin{observation}[Taratura dei parametri del DLS per il MAX-SAT]
    La taratura dei parametri può essere reattiva:
    \begin{itemize}
        \item Applicare per lo stesso tempo diverse configurazioni.
        \item Replicare il procedimento dando più tempo alle configurazioni sperimentalmente migliori.
    \end{itemize}
\end{observation}
\end{multicols}
\end{document}