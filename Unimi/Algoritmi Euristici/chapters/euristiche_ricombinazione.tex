\providecommand{\main}{..}
\documentclass[\main/main.tex]{subfiles}
\begin{document}
\chapter{Euristiche di ricombinazione}
\begin{multicols}{2}
\begin{definition}[Euristiche di ricombinazione]
    Le euristiche di ricombinazione, a differenza delle euristiche costruttive e di scambio (almeno tipicamente), gestiscono molte soluzioni in parallelo.
    
    Esse partono da un insieme, detto \textbf{popolazione}, di soluzioni, detti \textbf{individui}, e ricombinano questi individui producendo una nuova popolazione.
\end{definition}
\begin{observation}[Quale è l'aspetto originale delle euristiche di ricombinazione?]
    Il loro aspetto originale è l'uso di operazioni che lavorano su più soluzioni.
\end{observation}
\begin{observation}[Qual'è l'idea generale delle euristiche di ricombinazione?]
    L'idea di fondo è che:
    \begin{enumerate}
        \item Soluzioni buone condividono componenti con l'ottimo globale.
        \item Soluzioni diverse possono condividere componenti diverse.
        \item Combinando soluzioni diverse è possibile fondere componenti ottime più facilmente che costruendole un passo per volta.
    \end{enumerate}
\end{observation}
\begin{observation}[Come procedono le euristiche di ricombinazione?]
    Tipicamente le euristiche di ricombinazione procedono nelle modalità seguenti:
    \begin{enumerate}
        \item Costruire una popolazione iniziale di soluzioni.
        \item Finché non si verifica un'opportuna condizione di termine produrre popolazioni successive, dette \textbf{generazioni}.
        \item Per ogni generazione:
        \begin{enumerate}
            \item Estrarre sotto-insiemi di individui.
            \item Applicare operazioni di scambio agli individui singoli.
            \item Applicare operazioni di ricombinazione ai sotto-insiemi.
            \item Raccogliere gli individui generati dalle operazioni.
            \item Scegliere se accettare o no ogni nuovo individuo (e in quante copie) producendo così una nuova popolazione.
        \end{enumerate}
    \end{enumerate}
\end{observation}
\begin{definition}[Scatter Search]
    La Scatter Search è una euristica di ricombinazione che genera una popolazione iniziale di scambio e quindi la migliora iterativamente con una procedura di scambio.
    \begin{enumerate}
        \item Costruisce un insieme di riferimento (reference set \(\rnd{R = R_B \cup R_D}\)).
        \begin{itemize}
            \item Il sottoinsieme \(R_B\) contiene le migliori soluzioni note.
            \item Il sottoinsieme \(R_D\) contiene le soluzioni più distanti tra loro e da \(R_B\).
        \end{itemize}
        \item Per ogni coppia di soluzioni \(x, y \in R_B \times \rnd{R_B \cup R_D}\):
        \begin{itemize}
            \item Si procede a ricombinare \(x\) e \(y\), generando \(z\).
            \item Migliora \(z\) ottenenendo \(z'\) con una procedura di scambio
            \item Se \(z' \not\in R_B\) e in \(R_B\) c'è una soluzione peggiore, la sostituisce con \(z'\).
            \item Se \(z' \not\in R_D\) e in \(R_D\) c'è una soluzione più vicina, la sostituisce con \(z'\).
        \end{itemize}
        \item La procedura termina quando \(R\) rimane invariato.
    \end{enumerate}
    La logica è che le ricombinazioni \(R_B \times R_B\) \textbf{intensificano} la ricerca, mentre le ricombinazioni \(R_B \times R_D\) \textbf{diversificano} la ricerca.
\end{definition}
\begin{definition}[Procedura di ricombinazione]
    La procedura di ricombinazione spesso dipende dalla natura del problema. Spesso ci si limita a trattare \(x\) e \(y\) come sotto-insiemi:
    \begin{enumerate}
        \item Prima \(z\) include tutti gli elementi condivisi da \(x\) e \(y\): \(z = x \cap y\).
        \item Poi si procede ad aggiungere a \(z\) elementi estratti a caso alternativamente da \(x \setminus y\) e da \(y \setminus x\) fino a ottenere una soluzione ammissibile.
    \end{enumerate}
    
    Se il sottoinsieme \(z\) ottenuto è inammissibile, un'euristica di scambio ausiliaria, detta \textbf{procedura di riparazione}, riporta \(z\) all'ammissibilità.
\end{definition}
\end{multicols}
\clearpage
\begin{definition}[Path relinking]
    L'algoritmo di Path Relinking usato generalmente come procedura finale di intensificazione più che come metodo a sé. Dato un intorno \(N\) su cui si basa un'euristica ausiliaria di scambio:
    \begin{enumerate}
        \item Raccoglie in un insieme di riferimento \(R\) le migliori soluzioni generate dall'euristica ausiliaria, dette \textbf{soluzioni di élite}.
        \item Per ogni coppia di soluzioni \(x, y \in R\):
        \begin{itemize}
            \item Costruisce un cammino da \(x\) a \(y\) nello spazio di ricerca dell'intorno \(N\) applicando a \(z^{\rnd{0}} = x\) l'euristica ausiliaria di scambio, ma scegliendo a ogni passo la soluzione più vicina alla destinazione \(y\):
            \[
                z^{z+1} = \argmin_{z \in N\rnd{z^{\rnd{k}}}} D\rnd{z, y}
            \]
            dove \(R_D\) è un'opportuna funzione metrica sulle soluzioni. A pari distanza, ottimizza la funzione obbiettivo \(f\) del problema.
            \item Trova la miglior soluzione \(z^*_{xy}\) lungo il cammino:
            \[
                z_{xy}^* = \argmin_k f\rnd{z^{\rnd{k}}}
            \]
            \item Se \(z_{xy}^* \not\in R\) ed è migliore di una di quelle di \(R\), la inserisce in R.
        \end{itemize}
    \end{enumerate}
\end{definition}
\begin{definition}[Cammini di Relinking]
    I cammini di relinking esplorati in questo modo:
    \begin{itemize}
        \item Intensificano la ricerca, perché collegano soluzioni buone.
        \item Diversificano la ricerca, perché in genere sono diversi da quelli seguiti dall'euristica di scambio, soprattutto se gli estremi sono lontani.
        \item Poiché la distanza di \(z^{\rnd{k}}\) da \(y\) cala via via, si possono esplorare:
        \begin{itemize}
            \item Soluzioni peggioranti senza il rischio di comportamenti ciclici.
            \item Sotto-insiemi inammissibili senza il rischio di non riuscire a riottenere soluzioni ammissibili.
        \end{itemize}
    \end{itemize}
\end{definition}
\begin{observation}[Quali varianti del Path relinking esistono?]
    Esistono diverse varianti al Path Relinking, tra cui:
    \begin{description}
    \item[Backward path relinking] costruisce il cammino all'indietro da \(y\) a \(x\).
    \item[Back-and-forward path relinking] costruisce entrambi i cammini.
    \item[Mixed path relinking] costruisce un cammino facendo un passo per volta alternativamente da ogni estremo (aggiornando la destinazione).
    \item[Truncated path relinking] costruisce solo il principo del cammino.
    \end{description}    
\end{observation}
\end{document}