\providecommand{\main}{..}
\documentclass[\main/main.tex]{subfiles}
\begin{document}
\chapter{Matching Covers}

\section{Matching}
\setlength\columnsep{25pt}
\begin{multicols}{2}
  \begin{definition}[Matching o Accoppiamento]
    Dato un grafo \(G=\rnd{V, E}\), un \textbf{matching} è un sottoinsieme \(M\subseteq E\) di archi \textit{a due a due non adiacenti}.
  \end{definition}
  \begin{definition}[Matching massimo]
    Matching \(M^*\) di cardinalità massima.
  \end{definition}
  \begin{definition}[Matching ripartito]
    Se il grafo \(G\) è \textbf{bipartito}, allora anche \(M\) si dice \textbf{bipartito}.
  \end{definition}
  \begin{definition}[Matching perfetto]
    Se la cardinalità del matching è pari a metà del numero di vertici, allora si dice \textbf{perfetto}:
    \[
      \abs{M} = \frac{\abs{V}}{2}
    \]
  \end{definition}
  \begin{definition}[Matching massimale]
    Un matching \(M\) si dice \textbf{massimale} se ogni elemento di \(E\setminus M\) è adiacente ad almeno un elemento di \(M\).

    Un matching massimale \textbf{non} necessariamente è massimo, mentre un matching massimo è sempre massimale.
  \end{definition}
\end{multicols}

\section{Insieme stabile}
\begin{definition}[Insieme stabile o indipendente]
  Dato un grafo simmetrico \(G=\rnd{V, E}\), un qualunque sottoinsieme \(S\) di vertici si dice \textbf{indipendente} o \textbf{stabile} se esso è costituito da elementi a due a due non adiacenti.
\end{definition}
\begin{definition}[Insieme stabile massimo]
  Un insieme stabile \(S^*\) si dice \textbf{massimo} se \(\abs{S^*} \geq \abs{S}\), per ogni insieme stabile \(S\) di G.
\end{definition}
\begin{definition}[Insieme stabile massimale]
  Un insieme stabile \(S\) si dice \textbf{massimale} se ogni elemento di \(V\setminus S\) è adiacente ad almeno un elemento di \(S\).
\end{definition}

\section{Copertura}
\begin{definition}[Copertura]
  Dato un grafo simmetrico \(G = \rnd{V, E}\), un qualunque sottoinsieme \(T\) di vertici (\(F\) di archi) tale che ogni arco di \(E\) (vertice di \(V\)) incide su almeno un elemento di \(T\) (di \(F\)) si dice \textbf{copertura}. In particolare, l'insieme \(T\) è detto \textbf{trasversale} o \textbf{vertex cover} mentre l'insieme \(F\) è detto \textbf{edge cover}.
\end{definition}
\begin{definition}[Copertura minima]
  Una copertura \(X^*\) si dice \textbf{minima} se \(\abs{X^*} \leq \abs{X}\), per ogni insieme copertura \(X\) di \(G\).
\end{definition}
\begin{definition}[Copertura minimale]
  Una copertura \(X\) si dice \textbf{minimale} se \(X \setminus \crl{x}\) non è una copertura per ogni \(x \in X\).
\end{definition}

\section{Disuguaglianze duali deboli}
\begin{theorem}[Disuguaglianze duali deboli]
  Indichiamo con \(\a(G)\) l'\textbf{insieme stabile massimo} di \(G\), con \(\mu(G)\) il \textbf{matching massimo} di \(G\), con \(\rho(G)\) l'\textbf{edge cover minimo} di \(G\) e \(\tau(G)\) \textbf{trasversale minimo} di \(G\).
  Per un grafo \(G\) valgono le seguenti due disuguaglianze:
  \begin{align*}
    \a(G)  & \leq \rho(G) \\
    \mu(G) & \leq \tau(G)
  \end{align*}
\end{theorem}
\begin{proof}[Disuguaglianze duali deboli]
  Siano \(X\) l'\textbf{insieme stabile} di \(G\) e \(Y\) l'\textbf{edge cover} di \(G\).

  Poiché \(Y\) copre \(V\), ogni elemento di \(X\) incide su almeno un elemento di \(Y\).

  D'altra parte, nessun elemento di \(Y\) copre contemporaneamente due elementi di \(X \) altrimenti i due elementi sarebbero adiacenti e quindi non potrebbero appartenere all'insieme stabile \(X\).

  Pertanto, per ogni \(x \in X\) esiste un distinto \(y \in Y\) che lo copre, e quindi \(\abs{X} \leq \abs{Y}\).

  Riscrivendo la precedente relazione per gli insiemi massimi \(X^*\) e \(Y^*\) si ottiene:
  \[
    \a(G) \leq \rho(G)
  \]
  Scambiando il ruolo di \(V\) ed \(E\), si ottiene \(\mu(G) \leq \tau(G)\).
\end{proof}
\clearpage
\section{Teorema di Gallai}
\begin{theorem}[Teorema di Gallai]
  Per ogni grafo \(G\) con \(n\) nodi si ha:
  \[
    \a(G) + \tau(G) = n
  \]
  Se inoltre \(G\) non ha nodi isolati
  \[
    \mu(G) + \rho(G) = n
  \]
\end{theorem}
\begin{proof}[Teorema di Gallai]
  \textbf{Iniziamo ottenendo la prima equazione:}
  Sia \(S\) un insieme stabile di \(G\). Allora \(V\setminus S\) è un insieme trasversale. In particolare, \(\abs{V\setminus S} \geq \tau(G)\).
  Se consideriamo l'insieme stabile massimo \(S^*\), otteniamo:
  \[
    \tau(G) \geq \abs{V\setminus S^*} = n - \alpha(G)
  \]
  da cui ricaviamo:
  \[
    \alpha(G) + \tau(G) \leq n
  \]

  Viceversa, sia \(T\) un insieme trasversale di \(G\). Allora \(V\setminus T\) è un insieme stabile.

  In particolare, \(\abs{V-T} \leq \alpha(G)\).

  Se consideriamo l'insieme trasversale minimo \(T^*\), otteniamo:
  \[
    \a(G) \geq \abs{V\setminus T^*} = n - \tau(G)
  \]
  da cui ricaviamo
  \[
    \a(G) + \tau(G) \geq n
  \]
  Considerando la condizione ottenuta precedentemente possiamo concludere che:
  \[
    \a(G) + \tau(G) = n
  \]
  \textbf{Procediamo a dimostrare la seconda equazione}
  Sia \(G\) un grafo privo di nodi isolati e sia \(M^*\) il matching massimo di \(G\). Indichiamo con \(V_{M^*}\) i nodi che sono estremi degli archi in \(M^*\).

  Sia \(H\) un insieme minimale di archi tale che ogni nodo in \(V\setminus V_{M^*}\) è estremo di qualche arco in \(H\).

  Segue che:
  \[
    \abs{H} = \abs{V\setminus V_{M^*}} = n - 2\abs{M^*}
  \]
  Osserviamo che l'insieme \(C=H\cup M^*\) è un edge-cover di \(G\).

  Sicuramente, \(\abs{C} \geq \rho(G)\), quindi:
  \[
    \rho(G) \leq \abs{C} = \abs{M^*} + \abs{H} = \abs{M^*} + n - 2\abs{M^*} = n - \abs{M^*} = n - \mu(G)
  \]
  da cui ricaviamo:
  \[
    \rho(G) + \mu(G) \leq n
  \]
  Sia \(C\) il minimo edge-cover su \(G\), cioè tale che \(\abs{C} = \rho(G)\) e sia \(H = (V, C)\) il sottografo indotto da C. Valgono quindi le seguenti proprietà:

  \begin{enumerate}
    \item \(H\) è un grafo aciclico.
    \item Ogni cammino di \(H\) è composto al più da due archi.
  \end{enumerate}

  Dalle proprietà precedenti concludiamo che il grafo \(H = (V, C)\) ha \(\abs{V} = n\) vertici e \(\abs{C} = \rho(G)\) archi. Può infine essere decomposto in \(N\) componenti connesse aventi la forma di stella.

  Consideriamo l'\(i\)-esima componente connessa di \(H\). Indichiamo con \(s_i\) il numero di nodi della componente connessa e con \(s_i-1\) il numero di archi della componente connessa. Pertanto:
  \[
    n = \sum_{i=1}^N s_i \quad \text{e} \quad \rho(G) = \sum_{i=1}^N (s_i -1) = n - N \Rightarrow N = n - \rho(G)
  \]
  Sia \(M\) un matching con un arco per ogni componente di \(H\). Si ottiene:
  \[
    \mu(G) \geq \abs{M} = n - \rho(G) \Rightarrow \rho(G) + \mu(G) \geq n
  \]
  Considerando la condizione ottenuta precedentemente, possiamo concludere che:
  \[
    \rho(G) + \mu(G) = n
  \]

\end{proof}

\section{Cammino alternante e aumentante}
\begin{multicols}{2}[Sia \(M\) un matching di \(G = \rnd{V, E}\).]
  \begin{definition}[Arco accoppiato]
    Un arco \(\rnd{i,j} \in E\) si dice \textbf{accoppiato} se:
    \[
      \rnd{i,j} \in M
    \]
    Altrimenti è detto \textbf{libero}.
  \end{definition}
  \begin{definition}[Vertice accoppiato]
    Un vertice \(i \in V\) si dice \textbf{accoppiato} se su di esso incide un arco di \(M\). Altrimenti si dice che \textbf{non incide}.
  \end{definition}
  \begin{definition}[Cammino alternante]
    Un cammino \(P\) sul grafo \(G\) si dice \textbf{alternante} rispetto a \(M\) se esso è costituito alternativamente da archi accoppiati e liberi.
  \end{definition}
  \begin{definition}[Cammino aumentante]
    Un cammino \(P\) \textit{alternante} rispetto ad \(M\) che abbia entrambi gli estremi esposti si dice \textbf{aumentante}.
  \end{definition}
\end{multicols}

\begin{theorem}
  Sia \(M\) un matching di \(G\) e sia \(P\) un cammino aumentante rispetto a \(M\). La differenza simmetrica:
  \[
    M' = \rnd{M\setminus P} \cup \rnd{P\setminus M}
  \]
  È un matching di cardinalità \(\abs{M} + 1\).
\end{theorem}

\begin{proof}
  Sia \(M\) un matching di \(G\) e sia \(P\) un cammino aumentante rispetto a \(M\). L'insieme \(M' = \rnd{M\setminus P} \cup \rnd{P\setminus M}\) gode delle seguenti proprietà:
  \begin{enumerate}
    \item \(M'\) è un matching:
          \begin{enumerate}
            \item I nodi che non sono toccati da \(P\) non è cambiato nulla: su di essi incide un solo arco di \(M\) che ora appartiene anche ad \(M'\).
            \item Sui nodi intermedi di \(P\) incide soltanto un arco di \(P\setminus M\), e quindi di \(M'\).
            \item I nodi estremi di \(P\) prima erano esposti e adesso sono accoppiati e su di essi incide soltanto un arco di \(P\setminus M\).
          \end{enumerate}
    \item \(M'\) ha un elemento in più di \(M\):
          \begin{enumerate}
            \item Sia \(\abs{M} = m_1 + m_2\) con \(m_1 = \abs{M \setminus P}\) ed \(m_2 = \text{numero di archi del matching appartenenti al cammino.}\)
            \item Poiché \(P\) è aumentante, \(\abs{P} = m_2 + \rnd{m_2 + 1}\) dove \(m_2 + 1 = \abs{P\setminus M}\).
            \item \(\abs{M'} = \abs{M \setminus P} + \abs{P \setminus M} = m_1 + m_2 + 1 = \abs{M} +1\)
          \end{enumerate}
  \end{enumerate}
\end{proof}

\begin{theorem}[Teorema di Berge]
  Un matching \(M\) di \(G\) è massimo \textbf{se e solo se} \(G\) non ammette cammini aumentanti rispetto a \(M\).
\end{theorem}
\begin{proof}[Teorema di Berge]
  La condizione sufficiente segue dal teorema precedente. Per la condizione necessaria, facciamo vedere che, se non esistono cammini aumentanti rispetto a un certo matching \(M\), allora quel matching \(M\) è massimo:

  Supponiamo che \(G\) ammetta un matching \(M'\) con un elemento in più di \(M\). Vogliamo dimostrare che allora esiste un cammino aumentante per \(M\).

  Consideriamo l'insieme di archi:
  \[
    F = \crl{M'\cup M}\setminus\crl{M' \cap M}
  \]
  e sia \(G'\) il sottografo di \(G\) avente gli stessi nodi di \(G\) ma contenente solo l'insieme di archi di \(F\). Analizziamo il grado di ciascun nodo di \(G'\), considerando tutti i casi possibili:
  \begin{enumerate}
    \item Un nodo su cui incide lo stesso arco appartenente sia ad \(M\) che ad \(M'\) è un nodo isolato su \(G'\) e quindi ha grado 0.
    \item Un nodo su cui incide sia un arco di \(M\) sia un arco di \(M'\) è un nodo che ha grado \(2\) su \(G'\).
    \item Un nodo su cui incide un arco di \(M\) e nessun arco di \(M'\) o viceversa è un nodo che ha grado \(1\) su \(G'\).
  \end{enumerate}
  Considerando ora \(G'\), il sott

  TODO FINIRE DIMOSTRAZIONEEEE!!!
\end{proof}

\begin{theorem}[Teorema del cammino aumentante]
  Sia \(v\) un vertice esposto in un matching \(M\). Se non esiste un cammino aumentante per \(M\) che parte da \(v\), allora esiste un matching massimo avente \(v\) esposto.
\end{theorem}

\begin{proof}[Teorema del cammino aumentante]
  Sia \(M^*\) un matching massimo in cui \(v\) è accoppiato. Consideriamo \(\crl{M^*\cup M}\setminus\crl{M^* \cap M}\):

  TODO FINIRE DIMOSTRAZIONE
\end{proof}

\begin{theorem}[Teorema di König]
  Se \(G=\rnd{X, Y, E}\) è un grafo bipartito, allora \(\mu(G) = \tau(G)\).
\end{theorem}

\begin{proof}[Teorema di König]
  Sia \(M^*\) un matching massimo, e siano

  TODO FINIRE DIMOSTRAZIONE
\end{proof}

\end{document}