\providecommand{\main}{..}
\documentclass[\main/main.tex]{subfiles}
\begin{document}
\chapter{Matching Covers}

\section{Matching}
\setlength\columnsep{25pt}
\begin{multicols}{2}
  \begin{definition}[Matching o Accoppiamento]
    Dato un grafo \(G=\rnd{V, E}\), un \textbf{matching} è un sottoinsieme \(M\subseteq E\) di archi \textit{a due a due non adiacenti}.
  \end{definition}
  \begin{definition}[Matching massimo]
    Matching \(M^*\) di cardinalità massima.
  \end{definition}
  \begin{definition}[Matching ripartito]
    Se il grafo \(G\) è \textbf{bipartito}, allora anche \(M\) si dice \textbf{bipartito}.
  \end{definition}
  \begin{definition}[Matching perfetto]
    Se la cardinalità del matching è pari a metà del numero di vertici, allora si dice \textbf{perfetto}:
    \[
      \abs{M} = \frac{\abs{V}}{2}
    \]
  \end{definition}
  \begin{definition}[Matching massimale]
    Un matching \(M\) si dice \textbf{massimale} se ogni elemento di \(E\setminus M\) è adiacente ad almeno un elemento di \(M\).

    Un matching massimale \textbf{non} necessariamente è massimo, mentre un matching massimo è sempre massimale.
  \end{definition}
\end{multicols}

\section{Insieme stabile}
\begin{definition}[Insieme stabile o indipendente]
  Dato un grafo simmetrico \(G=\rnd{V, E}\), un qualunque sottoinsieme \(S\) di vertici si dice \textbf{indipendente} o \textbf{stabile} se esso è costituito da elementi a due a due non adiacenti.
\end{definition}
\begin{definition}[Insieme stabile massimo]
  Un insieme stabile \(S^*\) si dice \textbf{massimo} se \(\abs{S^*} \geq \abs{S}\), per ogni insieme stabile \(S\) di G.
\end{definition}
\begin{definition}[Insieme stabile massimale]
  Un insieme stabile \(S\) si dice \textbf{massimale} se ogni elemento di \(V\setminus S\) è adiacente ad almeno un elemento di \(S\).
\end{definition}

\section{Copertura}
\begin{definition}[Copertura]
  Dato un grafo simmetrico \(G = \rnd{V, E}\), un qualunque sottoinsieme \(T\) di vertici (\(F\) di archi) tale che ogni arco di \(E\) (vertice di \(V\)) incide su almeno un elemento di \(T\) (di \(F\)) si dice \textbf{copertura}. In particolare, l'insieme \(T\) è detto \textbf{trasversale} o \textbf{vertex cover} mentre l'insieme \(F\) è detto \textbf{edge cover}.
\end{definition}
\begin{definition}[Copertura minima]
  Una copertura \(X^*\) si dice \textbf{minima} se \(\abs{X^*} \leq \abs{X}\), per ogni insieme copertura \(X\) di \(G\).
\end{definition}
\begin{definition}[Copertura minimale]
  Una copertura \(X\) si dice \textbf{minimale} se \(X \setminus \crl{x}\) non è una copertura per ogni \(x \in X\).
\end{definition}

\section{Disuguaglianze duali deboli}
\begin{theorem}[Disuguaglianze duali deboli]
  Indichiamo con \(\a(G)\) l'\textbf{insieme stabile massimo} di \(G\), con \(\mu(G)\) il \textbf{matching massimo} di \(G\), con \(\rho(G)\) l'\textbf{edge cover minimo} di \(G\) e \(\tau(G)\) \textbf{trasversale minimo} di \(G\).
  Per un grafo \(G\) valgono le seguenti due disuguaglianze:
  \begin{align*}
    \a(G)  & \leq \rho(G) \\
    \mu(G) & \leq \tau(G)
  \end{align*}
\end{theorem}
\begin{proof}[Disuguaglianze duali deboli]
  Siano \(X\) l'\textbf{insieme stabile} di \(G\) e \(Y\) l'\textbf{edge cover} di \(G\).

  Poiché \(Y\) copre \(V\), ogni elemento di \(X\) incide su almeno un elemento di \(Y\).

  D'altra parte, nessun elemento di \(Y\) copre contemporaneamente due elementi di \(X \) altrimenti i due elementi sarebbero adiacenti e quindi non potrebbero appartenere all'insieme stabile \(X\).

  Pertanto, per ogni \(x \in X\) esiste un distinto \(y \in Y\) che lo copre, e quindi \(\abs{X} \leq \abs{Y}\).

  Riscrivendo la precedente relazione per gli insiemi massimi \(X^*\) e \(Y^*\) si ottiene:
  \[
    \a(G) \leq \rho(G)
  \]
  Scambiando il ruolo di \(V\) ed \(E\), si ottiene \(\mu(G) \leq \tau(G)\).
\end{proof}
\clearpage
\section{Teorema di Gallai}
\begin{theorem}[Teorema di Gallai]
  Per ogni grafo \(G\) con \(n\) nodi si ha:
  \[
    \a(G) + \tau(G) = n
  \]
  Se inoltre \(G\) non ha nodi isolati
  \[
    \mu(G) + \rho(G) = n
  \]
\end{theorem}
\begin{proof}[Teorema di Gallai]
  \textbf{Iniziamo ottenendo la prima equazione:}
  Sia \(S\) un insieme stabile di \(G\). Allora \(V\setminus S\) è un insieme trasversale. In particolare, \(\abs{V\setminus S} \geq \tau(G)\).
  Se consideriamo l'insieme stabile massimo \(S^*\), otteniamo:
  \[
    \tau(G) \geq \abs{V\setminus S^*} = n - \alpha(G)
  \]
  da cui ricaviamo:
  \[
    \alpha(G) + \tau(G) \leq n
  \]

  Viceversa, sia \(T\) un insieme trasversale di \(G\). Allora \(V\setminus T\) è un insieme stabile.

  In particolare, \(\abs{V-T} \leq \alpha(G)\).

  Se consideriamo l'insieme trasversale minimo \(T^*\), otteniamo:
  \[
    \a(G) \geq \abs{V\setminus T^*} = n - \tau(G)
  \]
  da cui ricaviamo
  \[
    \a(G) + \tau(G) \geq n
  \]
  Considerando la condizione ottenuta precedentemente possiamo concludere che:
  \[
    \a(G) + \tau(G) = n
  \]
  \textbf{Procediamo a dimostrare la seconda equazione}
  Sia \(G\) un grafo privo di nodi isolati e sia \(M^*\) il matching massimo di \(G\). Indichiamo con \(V_{M^*}\) i nodi che sono estremi degli archi in \(M^*\).

  Sia \(H\) un insieme minimale di archi tale che ogni nodo in \(V\setminus V_{M^*}\) è estremo di qualche arco in \(H\).

  Segue che:
  \[
    \abs{H} = \abs{V\setminus V_{M^*}} = n - 2\abs{M^*}
  \]
  Osserviamo che l'insieme \(C=H\cup M^*\) è un edge-cover di \(G\).

  Sicuramente, \(\abs{C} \geq \rho(G)\), quindi:
  \[
    \rho(G) \leq \abs{C} = \abs{M^*} + \abs{H} = \abs{M^*} + n - 2\abs{M^*} = n - \abs{M^*} = n - \mu(G)
  \]
  da cui ricaviamo:
  \[
    \rho(G) + \mu(G) \leq n
  \]
  Sia \(C\) il minimo edge-cover su \(G\), cioè tale che \(\abs{C} = \rho(G)\) e sia \(H = (V, C)\) il sottografo indotto da C. Valgono quindi le seguenti proprietà:

  \begin{enumerate}
    \item \(H\) è un grafo aciclico.
    \item Ogni cammino di \(H\) è composto al più da due archi.
  \end{enumerate}

  Dalle proprietà precedenti concludiamo che il grafo \(H = (V, C)\) ha \(\abs{V} = n\) vertici e \(\abs{C} = \rho(G)\) archi. Può infine essere decomposto in \(N\) componenti connesse aventi la forma di stella.

  Consideriamo l'\(i\)-esima componente connessa di \(H\). Indichiamo con \(s_i\) il numero di nodi della componente connessa e con \(s_i-1\) il numero di archi della componente connessa. Pertanto:
  \[
    n = \sum_{i=1}^N s_i \quad \text{e} \quad \rho(G) = \sum_{i=1}^N (s_i -1) = n - N \Rightarrow N = n - \rho(G)
  \]
  Sia \(M\) un matching con un arco per ogni componente di \(H\). Si ottiene:
  \[
    \mu(G) \geq \abs{M} = n - \rho(G) \Rightarrow \rho(G) + \mu(G) \geq n
  \]
  Considerando la condizione ottenuta precedentemente, possiamo concludere che:
  \[
    \rho(G) + \mu(G) = n
  \]

\end{proof}

\end{document}