\providecommand{\main}{..}
\documentclass[\main/main.tex]{subfiles}
\begin{document}

\chapter{Minimum spanning trees (MST)}
\section{Minimum spanning trees}
\begin{problem}[Connector Problem]
Dato un grafo connesso \(\Gr \) ed un costo positivo \(c_e \in \R^+ \forall e \in E\), trovare il sottografo connesso di \(G\) a costo minimo.
\end{problem}
\begin{lemma}
  Un lato \(e = uv\) di \(G\) è un lato di un circuito in \(G\) se e solo se esiste un cammino in \(G \setminus \crl{e}\) da \(u\) a \(v\).
\end{lemma}
Di conseguenza se eliminiamo un lato in un circuito di un grafo connesso, il grafo risultante rimane connesso ed ogni soluzione ottima al \textit{connector problem} non avrà circuiti.
\begin{definition}[Albero e Foresta]
  Un grafo privo di circuiti è detto \textbf{foresta}, mentre una foresta connessa è detta \textbf{albero}.
\end{definition}
È possibile risolvere il Connector Problem risolvendo il problema del Minimum Spanning Tree (MST):
\begin{problem}[Minimum Spanning Tree Problem]
Dato un grafo connesso \(\Gr \) ed un costo \(c_e \in \R \forall e \in E\), trovare l'albero ricoprente a costo minimo in \(G\).
\end{problem}
\begin{lemma}
  Un sottografo ricoprente connesso di \(G\) è un albero ricoprente se e solo se ha esattamente \(n -1\) lati.
\end{lemma}
\clearpage
\section{Algoritmi che risolvono MST}
\subsection{Algoritmo di Kruskal}
Sia \(H = (V, F)\) una foresta ricoprente di \(G\), con \(F=\emptyset \) inizialmente.

A ogni passo viene aggiunto ad \(F\) il lato a costo minimo \(e \not\in F\) tale che \(H\) rimane una foresta.

Il procedimento si interrompe quando \(H\) diviene un albero ricoprente.

\subsection{Algoritmo di Prim}
Sia \(H = (V(H), T)\) un albero di \(\Gr \) con inizialmente \(V(H) = \crl{r}\) per qualche \(r \in V \) e \(T = \emptyset \).

A ogni passo viene aggiunto a \(T\) il lato a costo minimo \(e \not\in T\) tale che \(H\) rimanga un albero.

Il procedimento si interrompe quando \(H\) diviene un albero ricoprente.

\subsection{Validità degli algoritmi su MST}
\begin{definition}[Taglio]
  Dato un grafo \(\Gr \), chiamiamo \(\delta(A)\) con \(A \subseteq V\) l'insieme:
  \[
    \delta(A) = \crl{e \in E: \text{entrambi i vertici di \(e\) sono in \(A\)}}
  \]
  Questo insieme per qualche \(A\) è detto \textbf{taglio} di \(G\).
\end{definition}

\begin{theorem}[Tagli di grafi connessi]
  Un grafo \(\Gr \) è connesso se e solo se non esiste un insieme non vuoto \(A \subset V\) tale che:
  \[
    \delta(A) = \emptyset
  \]
\end{theorem}
\begin{proof}[Tagli di grafi connessi]
  È chiaro che se \(\delta(A) = \emptyset \), \(u \in A\) e \(v \not\in A\), allora non ci può essere un cammino da \(u\) a \(v\) e quindi se \(A \neq \emptyset \) e \(A \subset V\) allora G non è connesso.

  Dobbiamo dimostrare che, se \(G\) non è connesso, allora non esiste un tale set \(A\): scegliamo quindi due vertici \(u, v \in V\) tali che non esiste un cammino tra \(u\) a \(v\).

  Definiamo \(A\) come:
  \[
    A = \crl{w \in V: \text{esiste un cammino tra \(u\) a \(w\)}}
  \]
  Quindi \(u \in A\) e \(v \not\in A\), pertanto \(A \neq \emptyset \) e \(A \subset V\).

  Per dimostrare che \(\delta(A) = \emptyset \) procediamo per assurdo: supponiamo che \(p \in A\), \(q \not\in A\) e che \(e = pq \in E\). Aggiungendo \(e\) a qualsiasi cammino da \(u\) a \(p\) si ottiene un cammino da \(u\) a \(q\), contraddicendo il fatto che \(q \not\in A\).
\end{proof}
\begin{definition}[Sottografo estensibile a MST]
  Un sottoinsieme di lati di \(\Gr \) \(A \subseteq E\) è estensibile a un MST se \(A\) è contenuto in un insieme di lati di qualche MST di \(G\).
\end{definition}
\begin{theorem}[Estensione a MST]
  Dato un grafo \(\Gr \), sia \(B \subseteq E\) un sottoinsieme estendibile a MST e \(e\) sia un lato a costo minimo di qualche taglio \(D\) tale che \(D \cap B = \emptyset \). Allora \(B \cup \crl{e}\) è estendibile ad un MST.
\end{theorem}
\begin{lemma}[Alberi ricoprenti]
  Sia \(H = \rnd{V, T}\) un albero ricoprente di \(G\), \(e = vw\) un lato di \(G\) ma non di \(H\) e sia \(f\) un lato di un cammino semplice in \(T\) da \(v\) a \(w\). Allora il sottografo:
  \[
    H' = \rnd{V, \rnd{T \cup \crl{e}}\setminus \crl{f}}
  \]
  è un albero ricoprente di \(G\).
\end{lemma}
\begin{proof}[Estensione a MST]
  Sia \(H = \rnd{V, T}\) un MST tale che \(B \subseteq T\). Se \(e \in T\), allora abbiamo terminato il procedimento.

  Altrimenti, sia \(P\) un cammino semplice appartenente a \(H\) da \(v\) a \(w\), dove \(vw = e\). Siccome non esiste nessun cammino in \(G\setminus D\) da \(v\) a \(w\), esiste un lato \(f \in P\) tale che \(f \in D\).

  Quindi \(c_f \geq c_e\), e per il lemma sugli alberi ricoprenti anche \(\rnd{V, \rnd{T \cup \crl{e}} \setminus \crl{f}}\) è un MST.

  Siccome \(D \cap B = \emptyset \), ne segue che \(f \not\in B\), pertanto \(B \cup \crl{e}\) è estendibile a un MST.
\end{proof}
\clearpage
\section{MST e programmazione lineare}
\begin{theorem}[Legame tra MST e PL]
  Sia \(x^o\) il vettore caratteristico di un MST in rispetto ai costi \(\bmc \). ALlora \(x^o\) è una soluzione ottima del problema di programmazione lineare:
  \begin{align*}
    \min \bmct\bmx                                                                 \\
    x\rnd{\gamma(S)} \leq \abs{S} -1 & \quad \forall S \neq \emptyset, S \subset V \\
    x(E) = \abs{V} -1                                                              \\
    x_e \geq 0                       & \quad \forall e \in E
  \end{align*}
\end{theorem}
\begin{proof}[Legame tra MST e PL]
  Per un sottoinsieme di lati \(A\), sia \(k(A)\) il numero di componenti del sottografo \(\rnd{V, A}\) di \(G\). Prendiamo in considerazione una versione equivalente del problema proposto nel teorema:
  \begin{align*}
    \min \bmct\bmx                                                                   \\
    x\rnd{A} & \leq \abs{V} - k(A) \quad \forall A \subset E                         \\
    x(E)     & = \abs{V} -1                                                          \\
    x_e      & \geq 0                                        & \quad \forall e \in E
  \end{align*}
  Sia \(A \subseteq E\) e siano \(S_1, \ldots, S_k\) gli insiemi di vertici delle componenti del sottografo \(\rnd{V, A}\). Allora:
  \[
    x(A) \leq \sum_{i=1}^k x\rnd{\gamma\rnd{S_i}} \leq \sum_{i=1}^k \rnd{\abs{S_i}-1} = \abs{V} - k
  \]
  Procediamo ora a mostrare che \(x^o\) è la soluzione ottima del problema semplificato proposto e che è sufficiente perché ciò sia vero che essa è il vettore caratteristico di un albero ricoprente \(T\) generato dall'algoritmo di Kruskal.

  Mostriamo che \(x^o\) è ottimale mostrando che l'algoritmo di Kruskal può essere utilizzato per calcolare una soluzione ammissibile al problema PL duale che soddisfa gli scarti complementari con \(x^o\). Riscrivendo la funzione obbiettivo del problema primale come \(\max -\bmct\bmx \), il duale risulta:
  \begin{align*}
    \min \sum_{A \subseteq W} \rnd{\abs{V} - k(A)}\bmy_A           \\
    \sum \rnd{\bmy_a: e \in A} & \geq -c_e, \quad \forall e \in E  \\
    \bmy_A                     & \geq 0, \quad \forall A \subset E
  \end{align*}
  Sia \(e_1, \ldots, e_m\) l'ordine con cui l'algoritmo di Kruskal considera i lati. Sia \(R_i = \crl{e_1, \ldots, e_i}, \forall i \in \N \cap \sqr{1, m}\).

  Sia \(\bmy_A^o=0\) a meno che \(\exists i: A = R_i\), e assegniamo \(\bmy^o_{R_i} = c_{e_{i+1}} - c_{e_{i}} \forall i \in \N \cap \sqr{1, m-1}\). Infine assegniamo \(\bmy^o_{R_m} = -c_{e_m}\).

  Segue dall'ordine dei lati considerati dall'algoritmo che \(\bmy^o_A \geq 0\) se \(A\neq E\). Considerando ora il vincolo a sommatoria del problema, dove \(e=e_i\) abbiamo:
  \[
    \sum_{\bmy^o_A: e \in A} = \sum_{j=i}^m \bmy^o_{R_j} = \sum_{j=i}^{m-1}\rnd{c_{e_{i+1}}-c_{e_i}} - c_{e_m} = -c_{e_i} = -c_e
  \]
  In altre parole, tutte le disuguaglianze considerate valgono come uguaglianze, pertanto la soluzione \(\bmy^o\) è ammissibile e rispetta le condizioni degli scarti complementari: rimane solo una condizione da verificare, cioè se:
  \[
    \bmy^o_A > 0 \Rightarrow \bmx^o \text{ soddisfa le condizioni degli scarti complementari.}
  \]
  Per verificare questo, sappiamo che \(\exists i: A = R_i\). Se i vincoli del problema primale non soddisfano le condizioni degli scarti complementari per \(R_i\), allora esiste qualche lato di \(R_i\) la cui aggiunta a \(T\cap R_i\) porterebbe a ridurre il numero di componenti di \(\rnd{V, T \cap R_i}\). Ma un tale lato avrebbe termini in due componenti diverse di \(\rnd{V, R_i \cup T}\), e quindi sarebbe stato aggiunto a \(T\) dall'algoritmo di Kruskal.

  Per cui, \(\bmx^o\) e \(\bmy^o\) soddisfano le condizioni degli scarti complementari. Ne segue che \(\bmx^o\) è la soluzione ottima del problema di PL.
\end{proof}
\end{document}