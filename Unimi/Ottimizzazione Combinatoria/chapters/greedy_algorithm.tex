 \providecommand{\main}{..}
\documentclass[\main/main.tex]{subfiles}
\begin{document}

\chapter{Matroidi e Algoritmo Greedy}
\section{Sistema di Indipendenza}
\begin{definition}[Sistema di Indipendenza (SI)]
  Sia \(E = \crl{e_1, \ldots, e_n}\) e \(\bm{I} \subseteq 2^E\) tale che per tutti \(I, J \subseteq E\) valga la proprietà:
  \[
    J \subset I \in \bm{I} \rightarrow J \in \bm{I}
  \]
  Allora la coppia \(\rnd{E, \bm{I}}\) è detta \textbf{Sistema di Indipendenza} ed i membri di \(\bm{I}\) vengono detti \textbf{insiemi indipendenti}.
\end{definition}
\section{GREEDSUM}
L'algoritmo riceve un insieme di elementi \(E\) ed una funzione di costo \(c\) ed inizia con l'insieme soluzione \(I = \emptyset \).

Successivamente itera \(\abs{E}\) volte, sceglindo ogni volta l'elemento di \(E\) che massimizza la funzione \(c\).

Questo elemento \(e^* = \max_e c(E)\) viene quindi rimosso dal set degli oggetti: \(E = E - \crl{e^*}\).

Se l'insieme \(I \cup \crl{e^*}\) risulta essere indipendente, allora il nuovo elemento viene aggiunto all'insieme di soluzione: \(I = I \cup \crl{e^*}\).

\section{I teoremi di Rado}
I teoremi di Rado consentono di fare un'importante affermazione su GREEDSUM: ``GREEDSUM fornisce l'ottimo se e solo se il SI è un matroide''.
\begin{theorem}[Primo teorema di Rado]
  Sia \(\rnd{E, \bm{I}}\) un Sistema di Indipendenza (SI). Se per due qualunque indipendenti \(I\) e \(J\) tali che \(\abs{I} = \abs{J} + 1\) vale la proprietà:
  \[
    \exists e \in I: J \cup \crl{e} \in I
  \]
  allora \textbf{GREEDSUM} fornisce un membro di \(\bm{I}\) di valore massimo per qualunque funzione di valutazione non-negativa: \(e: E \rightarrow \R^+\).
  \label{primo_rado}
\end{theorem}
\begin{proof}[Primo teorema di Rado]
  La proprietà del teorema implica che se \(A \subseteq E\) e \(I, I' \subseteq A\) sono insiemi indipendenti massimali, allora \(\abs{I} = \abs{I'}\).

  Infatti se \textbf{per assurdo}
\end{proof}
\begin{theorem}[Secondo teorema di Rado]
  Sia \(\rnd{E, \bm{I}}\) un Sistema di Indipendenza (SI). Se per qualunque funzione di valutazione non negativa \(c: E \rightarrow \R^+\) degli elementi di \(E\), \textbf{GREEDSUM} fornisce sempre un membro di \(\bm{I}\) di valore massimo, allora il SI gode della proprietà del primo teorema di Rado \ref{primo_rado}.
\end{theorem}
\begin{proof}[Secondo teorema di Rado]
\end{proof}
\begin{definition}[Matroide]
  Un SI che gode della proprietà del primo teorema di Rado \ref{primo_rado} si dice \textbf{matroide}.
\end{definition}
\end{document}