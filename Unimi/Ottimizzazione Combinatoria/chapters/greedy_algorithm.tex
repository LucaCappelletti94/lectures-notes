 \providecommand{\main}{..}
\documentclass[\main/main.tex]{subfiles}
\begin{document}

\chapter{Matroidi e Algoritmo Greedy}
\section{Sistema di Indipendenza}
\begin{definition}[Sistema di Indipendenza (SI)]
  Sia \(E = \crl{e_1, \ldots, e_n}\) e \(F \subseteq 2^E\) tale che per tutti \(I, J \subseteq E\) valga la proprietà:
  \[
    J \subset I \in F \rightarrow J \in F
  \]
  Allora la coppia \(\rnd{E, F}\) è detta \textbf{Sistema di Indipendenza} ed i membri di \(F\) vengono detti \textbf{insiemi indipendenti}.
\end{definition}
\section{GREEDSUM}
L'algoritmo riceve un insieme di elementi \(E\) ed una funzione di costo \(c\) ed inizia con l'insieme soluzione \(I = \emptyset \).

Successivamente itera \(\abs{E}\) volte, scegliendo ogni volta l'elemento di \(E\) che massimizza la funzione \(c\).

Questo elemento \(e^* = \max_e c_{E}\) viene quindi rimosso dal set degli oggetti: \(E = E - \crl{e^*}\).

Se l'insieme \(I \cup \crl{e^*}\) risulta essere indipendente, allora il nuovo elemento viene aggiunto all'insieme di soluzione: \(I = I \cup \crl{e^*}\).
\begin{definition}[Matroide]
  Un SI che gode della proprietà del primo teorema di Rado \ref{primo_rado} si dice \textbf{matroide}.
\end{definition}
\clearpage
\section{I teoremi di Rado}
I teoremi di Rado consentono di fare un'importante affermazione su GREEDSUM: ``GREEDSUM fornisce l'ottimo se e solo se il SI è un \textbf{matroide}''.
\begin{theorem}[Primo teorema di Rado]
  Sia \(\rnd{E, F}\) un Sistema di Indipendenza (SI). Se per due insiemi indipendenti \(I\) e \(J\) qualunquetali che \(\abs{I} = \abs{J} + 1\) vale la proprietà:
  \[
    \exists e \in I: J \cup \crl{e} \in F
  \]
  allora \textbf{GREEDSUM} fornisce un membro di \(F\) di valore massimo per qualunque funzione di valutazione: \(e: E \rightarrow \R^+\).
  \label{primo_rado}
\end{theorem}
\begin{proof}[Primo teorema di Rado]
  \textbf{Iniziamo dimostrando che gli insiemi sono indipendenti massimali di pari cardinalità:} la proprietà del teorema implica che se \(A \subseteq E\) e \(I, I' \subseteq A\) sono insiemi indipendenti massimali, allora \(\abs{I} = \abs{I'}\).

  Infatti se \textbf{per assurdo} \(\abs{I} < \abs{I'}\), si potrebbe trovare un \(I'' \subseteq I'\) tale che \(\abs{I''} = \abs{I} + 1\) eliminando da \(I'\) un numero di elementi pari a \(\abs{I'} - \abs{I} - 1\), ma per la proprietà del teorema esiste \(e \in I'' \setminus I\) tale che \(I \cup \crl{e} \in F\) e \(I\) non sarebbe massimale.

  \textbf{Quindi dimostriamo che GREEDSUM dia il risultato ottimo:} supponiamo adesso ancora per assurdo che \textbf{GREEDSUM} non dia un risultato ottimo, cioè che per una certa funzione di valutazione \(c: E \rightarrow \R^+\) l'algoritmo fornisce un insieme indipendente \(I = \crl{e_1, \ldots, e_i}\) mentre ne esiste un altro \(J = \crl{e'_1, \ldots, e'_j}\) tale che \(c_J > c_I\).

  Supponiamo senza perdita di generalità di aver ordinato gli elementi di \(I\) e \(J\) in ordine di valore non crescente.

  Per costruzione \(I\) è massimale e quindi anche \(J\) per quanto visto sopra, per cui \(i=j\). Vedremo che per \(m=1, 2, \ldots, i\) vale che \(c_{e_m} \geq c_{e'_m}\), contraddicendo l'ipotesi per cui \(c_J > c_I\).

  Procediamo quindi per \textbf{induzione}:

  \textbf{Caso base:} quando \(m=1\), la proprietà è vera per come \textbf{GREEDSUM} sceglie il primo elemento.

  \textbf{Passo induttivo:} Supponiamo allora che per qualche \(m>1\) si abbia \(c_{e_m} < c_{e'_m}\) mentre \(c_{e_s} \geq c_{e'_s}\) per \(s = 1, \ldots, m-1\):
  \[A = \crl{e \in E: c_{e} \geq c_{e'_m}}\]

  Allora l'insieme \(\crl{e_1, \ldots, e_{m-1}}\) è un indipendente massimale di \(A\) dato che se \(\crl{e_1, \ldots, e_{m-1}, e} \in I\) e \(c_{e} \geq c_{e'_m} > c_{e_m}\), \textbf{GREEDSUM} avrebbe scelto \(e\) invece di \(e_m\) come successivo elemento da aggiungere.

  Ciò contraddice quanto dimostrato nella prima parte dato che \(\crl{e'_1, \ldots, e'_m}\) è un altro insieme indipendente di \(A\) ed avrebbe cardinalità maggiore.
\end{proof}
\clearpage
\section{Diagramma della dimostrazione}
\begin{figure}
  \includegraphics[width=0.8\textwidth]{proofs/Primo_Teorema_di_Rado}
\end{figure}
\clearpage
\begin{theorem}[Secondo teorema di Rado]
  Sia \(\rnd{E, F}\) un Sistema di Indipendenza (SI). Se per qualunque funzione di valutazione non negativa \(c: E \rightarrow \R^+\) degli elementi di \(E\), \textbf{GREEDSUM} fornisce sempre un membro di \(F\) di valore massimo, allora il SI gode della proprietà del primo teorema di Rado \ref{primo_rado}.
\end{theorem}
\begin{proof}[Secondo teorema di Rado]
  Procediamo \textbf{per assurdo} e supponiamo che la proprietà del primo teorema di Rado \ref{primo_rado} non valga.

  Ciò significa che esistono due \textbf{insiemi indipendenti} \(I\) e \(J\) con \(\abs{I} = p\) e \(\abs{J} = p+1\) tali che per nessun elemento \(e \in J \setminus I\) accade che \(I \cup \crl{e} \in I\).

  Consideriamo allora la seguente maniera di valutare gli elementi di \(E\):
  \[
    c_{e} = \begin{cases}
      p+2 & e \in I            \\
      p+1 & e \in J\setminus I \\
      0   & e \not\in I \cup J
    \end{cases}
  \]
  Se ne deduce che \(c_J \geq \rnd{p+1}^2 > p(p+2) - c_I\) e quindi \(I\) non è ottimo.

  \textbf{GREEDSUM} d'altra parte per sua natura inizierà a inserire nella soluzione per primi tutti gli elementi di \(I\) e non potrà poi aumentare il valore della soluzione, dato che sarà costretto a selezione solo elementi di valore nullo.

  Ma allora abbiamo trovato una funzione di valutazione per cui l'algoritmo non fornisce l'ottimo, contrariamente a quanto assunto.
\end{proof}
\clearpage
\section{Diagramma della dimostrazione}
\begin{figure}
  \includegraphics[width=0.6\textwidth]{proofs/Secondo_Teorema_di_Rado}
\end{figure}
\end{document}