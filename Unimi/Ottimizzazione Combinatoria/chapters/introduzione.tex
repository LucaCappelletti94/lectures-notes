 \providecommand{\main}{..}
\documentclass[\main/main.tex]{subfiles}
\begin{document}

\begin{multicols}{2}
  [
    \chapter{Introduzione}
    L' \textbf{Ottimizzazione combinatoria} propone modelli di soluzioni ad innumerevole problemi, tra i quali vi sono:
  ]
  \paragraph*{Matching covers}
  Consideriamo due insiemi \(A\) e \(B\), di cardinalità \(n\): ad ogni coppia di valori del prodotto cartesiano dei due insiemi è associato un valore positivo che descrive la compatibilità tra i due valori. Si vanno a scegliere \(n\) coppie, senza che gli elementi vengano ripetuti, in modo da massimizzare la compatibilità totale.

  \paragraph*{Set Covering}
  Data una \textit{matrice binaria} ed un vettore di costi associati alle colonne si va a realizzare il sottoinsieme di costo minimo che copra tutte le righe.

  \paragraph*{Set Packing}
  Data una \textit{matrice binaria} ed un vettore di valori associati alle colonne, si cerca il sottoinsieme di colonne di valore massimo tali che non coprino entrambe una stessa riga.

  \paragraph*{Set Partitioning}
  Data una \textit{matrice binaria} ed un vettore di cosi associati alle colonne, si cerca il sottoinsieme di colonne di costo minimo che copra tutte le righe senza conflitti.

  \paragraph*{Vertex Cover}
  Dato un grafo non orientato \(G=\rnd{V, E}\) si cerca il sottoinsieme di vertici di cardinalità minima tale che ogni lato del grafo vi incida.

  \paragraph*{Maximum Clique Problem}
  Dato un grafo non orientato e una funzione peso definita sui vertici, si cerca il sottoinsieme di vertici fra loro adiacenti di peso massimo.

  \paragraph*{Maximum Independent Set Problem}
  Dato un grafo non orientato e una funzione di peso definita sui vertici, si cerca il sottoinsieme di vertici fra loro non adiacenti di peso massimo.

  \paragraph*{Minimum Steiner Tree}
  Dato un grafo non orientato e una funzione costo definita sui lati, si cerca un albero ricoprente di costo minimo.

  \paragraph*{Boolean satisfiability problem or SAT}
  Data una forma normale congiunta (CNF), si cerca un assegnamento di verità alle variabili logiche che la soddisfi.

  \subparagraph*{Versione pesata (MAX-SAT)}
  Viene considerata anche una funzione peso associata alle formule che compongono la CNF. L'obbiettivo è massimizzare il peso totale delle formule soddisfatte.
\end{multicols}

\end{document}