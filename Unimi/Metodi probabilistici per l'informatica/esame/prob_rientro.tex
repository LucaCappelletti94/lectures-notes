\providecommand{\main}{..}
\documentclass[\main/main.tex]{subfiles}
\begin{document}

\section{Probabilità di ingresso}
\begin{definition}[Probabilità di ingresso]
  Il coefficiente  \(f_{ij}\) denota la probabilità di ingresso in uno stato \(j\) partendo da uno stato \(i\). Essi possiedono alcune caratteristiche:
  \begin{align*}
    f^{(0)}(i,j) & = 0                                                                                            \\
    f^{(n)}(i,j) & = \prob{\tau_j=n}{}{i} \quad \forall n \in \N_+                                                \\
    f(i,j)       & = \sum_{n\geq 1} f^{(n)}(i,j) = \prob{\tau_j < \infty}{}{i}  = 1 - \prob{\tau_j = \infty}{}{i}
  \end{align*}
  I coefficienti sulla diagonale principale \(f_{ii}\) sono detti \textbf{coefficienti di probabilità di rientro}.
\end{definition}
\end{document}