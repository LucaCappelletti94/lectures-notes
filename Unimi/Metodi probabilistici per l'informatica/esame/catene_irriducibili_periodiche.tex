\providecommand{\main}{..}
\documentclass[\main/main.tex]{subfiles}
\begin{document}

\section{Esempio: catena irriducibile non primitiva}
Prendendo in considerazione una catena formata da soli due stati con una matrice di transizione:
\[
  \matr{P} = \begin{bmatrix}
    0 & 1 \\
    1 & 0
  \end{bmatrix}
\]
Chiaramente \(\matr{P}\) è irriducibile di periodo \(2\) e si verifica subito che \(p^{(n)}_{12} = 1\) per \(n\) dispari e \(p^{(n)}_{12} = 0\) per \(n\) pari.

Al crescere di \(n\) quindi la sequenza non ammette limite., per quanto è possibile dimostrare una forma di convergenza più debole, data da una media dei valori di \(p^{(n)}_{ij}\).
\end{document}