\providecommand{\main}{..}
\documentclass[\main/main.tex]{subfiles}
\begin{document}

\section{Tempi medi di rientro}
\begin{definition}[Tempi medi di rientro]
  Supponendo \(X_0 = i\) ricorrente, definiamo il tempo medio di rientro nello stato \(i\):
  \[
    \mean{\tau_i}{}{i} = \sum_{n \geq 1} n \prob{\tau_i = n}{}{i} = \sum_{n \geq 1} n f^{(n)}_{ii}
  \]
\end{definition}

\begin{theorem}[Il tempo medio di nodi ricorrenti è finito]
  Per ogni nodo \(i \in S\) ricorrente vale che:
  \[
    \mean{\tau_i}{}{i} < + \infty
  \]
\end{theorem}

\begin{proof}[Il tempo medio di nodi ricorrenti è finito]
  Per il \textbf{teorema della probabilità di rientro per stati transienti}, possiamo limitarci a provare che \(f^{(n)}_{ii} = \O{\epsilon^n}\).

  Essendo inoltre \(i\) essenziale, per la \textbf{relazione tra stati essenziali e ricorrenti}, possiamo restringere l'insieme degli stati \(S\) alla classe \(C\) di \(i\).

  Definiamo una nuova catena di Markov sugli stati \(C\), rendendo \(i\) assorbente. Definiamo la nuova matrice di transizione \(\tilde{\matr{P}}\) della catena come:

  \begin{align*}
    \tilde{p}_{kj} & = p_{kj} \quad k \neq i     \\
    \tilde{p}_{ij} & = \begin{cases}
      1 & j = i    \\
      0 & j \neq i
    \end{cases}
  \end{align*}

  Tutti gli stati \(j \neq i\) sono \textbf{transienti} e quindi:
  \[
    \sum_{j \neq i} \tilde{p}^{(n)}_{kj} = \O{\epsilon^n}
  \]
  Ne segue allora che, per ogni intero \(n\geq 2\):
  \begin{align*}
    f^{(n)}_{ii} & = \sum_{k, j \in Cm k, j \neq i} p_{ik} \tilde{p}^{(n-2)}_{kj}p_{ji}                 \\
                 & = \O{\epsilon^n} \cdot \sum_{k, j \in Cm k, j \neq i} p_{ik} p_{ji} = \O{\epsilon^n} \\
  \end{align*}
\end{proof}

\begin{theorem}[Tempi medi di rientro di matrici irriducibili]
  Se \(\matr{P}\) è \textbf{irriducibile}, per ogni nodo \(i, j \in S\) vale che:
  \[
    \mean{\tau_i}{}{i} = \sum_{n \geq 1} n f^{(n)}_{ij} < + \infty
  \]
\end{theorem}
\end{document}