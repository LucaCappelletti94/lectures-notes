\providecommand{\main}{..}
\documentclass[\main/main.tex]{subfiles}
\begin{document}

\section{Proprietà di autovalore unitario delle matrici stocastiche}
\begin{theorem}[Proprietà di autovalore unitario delle matrici stocastiche]
  Sia \(\matr{P}\) una matrice stocastica di dimensione \(r\). Allora \(1\) è autovalore di \(\matr{P}\) e ammette come corrispondente autovettore destro il vettore \(\bme \), di dimensione \(r\) tale che \(\bmet = \begin{bmatrix}
    1 & 1 & \ldots & 1
  \end{bmatrix}\).
\end{theorem}
\begin{proof}[Proprietà di autovalore unitario delle matrici stocastiche]
  Sia \(\matr{P} = \begin{bmatrix}
    p_{ij}
  \end{bmatrix}\). Le righe di una \textbf{matrice stocastica} hanno somma unitaria, per cui vale che:
  \[
    \matr{P}\bme = \bme
  \]
  Da cui segue che \(1\) è un autovalore di \(\matr{P}\) e che \(\bme \) è un autovettore destro di \(\matr{P}\) corrispondente a \(1\).
\end{proof}

\end{document}