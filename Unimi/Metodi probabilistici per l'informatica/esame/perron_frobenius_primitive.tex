\providecommand{\main}{..}
\documentclass[\main/main.tex]{subfiles}
\begin{document}

\section{Teorema di Perron-Frobenius per le matrici primitive}
\begin{theorem}[Teorema di Perron-Frobenius per le matrici primitive]
  Sia \(\matr{A} \geq 0\) una matrice primitiva. Allora esiste un autovalore di \(\matr{A}\) \(\lambda \) tale che:
  \begin{enumerate}
    \item \(\lambda \) è reale e strettamente positivo.
    \item Tutti gli autovalori diversi da \(\lambda \) sono minori in modulo.
    \item \(\lambda \) ammette autovettori destri e sinistri strettamente positivi.
    \item \(\lambda \) è radice semplice dell'equazione caratteristica (ha molteplicità algebrica unitaria).
  \end{enumerate}
\end{theorem}

\end{document}