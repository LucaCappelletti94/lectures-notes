\providecommand{\main}{..}
\documentclass[\main/main.tex]{subfiles}
\begin{document}

\section{Campionatori di Gibbs}
Dati due insiemi finiti \(V, R\) di cardinalità \(k, q > 1\), denotiamo con \(R^V\) il set delle funzioni a valori da \(V\) a \(R\), di cardinalità \(q^k\): un campionatore di Gibbs estrae secondo una distribuzione fissata \(\bm{\pi} \) un elemento da tale insieme.

Viene definito tramite una catena di Markov sull'insieme di stati:
\[
  S = \crl{A \in R^V: \pi(A) > 0}
\]
Il nuovo stato della catena viene ottenuto estraendo a caso in modo uniforme un elemento \(v \in V\) e scegliendo un nuovo valore \(c \in R\) con probabilità \(\bm{\pi} \) condizionata a conservare uguali ad \(A\) i valori degli altri elementi di \(V\).

Il nuovo stato \(B\) differisce da \(A\) al più per il solo valore attribuito all'elemento \(v\) estratto, ed esiste sempre una probabilità non nulla che \(B\) coincida con \(A\).

\begin{theorem}[La distribuzione di un campionatore di Gibbs è reversibile]
  La distribuzione \(\bm{\pi} \) è \textbf{reversibile} per la catena di Markov con matrice di transizione \(\matr{P}\) definita per un campionatore di Gibbs.

  Chiamando \(\pi_v\) la probabilità \(\pi(C \in R^V: C(u) = A(u) \, \forall u \neq v)\)
  \[
    p_{AB} = \begin{cases}
      0                                  & \exists u, v \in V: A(u) \neq B(U) \, \land \, A(v) \neq B(v) \\
      \frac{\pi_B}{k\pi_v}               & !\exists v \in V: A(v) \neq B(v)                              \\
      \frac{\pi_A}{k\sum_{v \in V}\pi_v} & A = B
    \end{cases}
  \]
\end{theorem}
\end{document}