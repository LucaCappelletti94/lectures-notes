\providecommand{\main}{..}
\documentclass[\main/main.tex]{subfiles}
\begin{document}

\section{Teorema di Perron-Frobenius per le matrici stocastiche}
\begin{theorem}[Teorema di Perron-Frobenius per le matrici stocastiche]
  Se \(\matr{P}\) è una matrice stocastica primitiva, allora \(1\) è il suo autovettore di \textbf{Perron-Frobenius} ed inoltre, per qualche \(0 \leq \epsilon < 1\), abbiamo che:
  \[
    \matr{P}^n = \bme\bmvt + \O{\epsilon^n}
  \]
  dove \(\bmv \) è l'autovettore sinistro di \(\matr{P}\) corrispondente a \(1\) mentre \(\bme \) ne è l'autovettore destro.
\end{theorem}

\end{document}