\providecommand{\main}{..}
\documentclass[\main/main.tex]{subfiles}
\begin{document}

\section{Relazione tra matrici primitive e irriducibili}
\begin{theorem}[Relazione tra matrici primitive e irriducibili]
  Una matrice non negativa \(\matr{A}\) è primitiva se e solo se \(\matr{A}\) è irriducibile e aperiodica.
\end{theorem}
\begin{proof}[Relazione tra matrici primitive e irriducibili]

  \textbf{Se matrice è primitiva allora è irriducibile e aperiodica}

  Per definizione, tutte le matrici primitive sono irriducibili e aperiodiche.

  \textbf{Se una matrice è irriducibile e aperiodica allora è primitiva}

  Supponiamo che \(\matr{A}\) sia irriducibile e aperiodica. Utilizzando le \textbf{proprietà dei coefficienti di matrici irriducibili}, notiamo che tutti i coefficienti \(r_j\) in questo caso sono \textbf{nulli}. Per la seconda proprietà, essendo \(r_j = 0\) e \(d=1\) deve valere che:
  \[
    a_{ij}^{(n)} > 0 \quad \forall n \geq q_j
  \]
  È quindi possibile scegliere un \(n\) tale per cui:
  \[
    \matr{A}^n > 0
  \]
  Per cui \(\matr{A}\) è \textbf{primitiva}.
\end{proof}

\end{document}