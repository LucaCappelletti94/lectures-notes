\providecommand{\main}{..}
\documentclass[\main/main.tex]{subfiles}
\begin{document}

\section{Catena di Markov reversibile}
\begin{definition}[Catena di Markov reversibile]
  Una catena di Markov \(\crl{X_n}_n\) definita sugli spazi \(S\) con matrice di transizione \(\matr{P}\) si dice \textbf{reversibile} se esiste un \textbf{vettore stocastico} \(\bm{\pi} \) definito in \(S\) tale che:
  \[
    \pi_i p_{ij} = \pi_j p_{ji}
  \]
  O equivalentemente per qualsiasi sequenza di finita di stati:
  \[
    \prob{X_0 = i, X_1 = j}{}{\bm{\pi}} = \prob{X_0 = j, X_1 = i}{}{\bm{\pi}}
  \]
  Il vettore \(\bm{\pi} \) è detto \textbf{distribuzione reversibile} per la catena \(\crl{X_n}_n\).
\end{definition}

\begin{theorem}[Distribuzioni reversibili stazionarie]
  Se \(\bm{\pi} \) è una distribuzione reversibile per una catena \(\crl{X_n}_n\) allora \(\bm{\pi} \) è anche una distribuzione stazionaria per \(\crl{X_n}_n\).
\end{theorem}
\begin{proof}[Distribuzioni reversibili stazionarie]
  Siano \(S\) e \(\matr{P}\) rispettivamente, l'insieme degli stati e la matrice di transizione della catena. Poiché \(\bm{\pi} \) è una \textbf{distribuzione reversibile}, per ogni \(j \in S\) abbiamo che:
  \[
    \crl{\bm{\pi}^T \matr{P}}_j = \sum_{i=1}^m \pi_i p_{ij} = \sum_{i=1}^m \pi_i p_{ji} = \pi_j
  \]
  Di conseguenza \(\bm{\pi}^T \matr{P} = \bm{\pi}^T\) e quindi \(\bm{\pi} \) risulta stazionaria.
\end{proof}

\begin{theorem}[Matrice simmetrica irriducibile bistocastica]
  Sia \(\crl{X_n}_n\) una catena di Markov con matrice di transizione \(\matr{P}\) irriducibile e bistocastica. Se \(\crl{X_n}_n\) è reversibile allora \(\matr{P}\) è simmetrica.
\end{theorem}
\begin{proof}[Matrice simmetrica irriducibile bistocastica]
  Infatti, essendo \(\matr{P}\) irriducibile e bistocastica, la sua unica distribuzione stazionaria è quella uniforme \(\bm{\pi} = \begin{bmatrix}
    \frac{1}{k} & \frac{1}{k} & \cdots & \frac{1}{k}
  \end{bmatrix}\) dove \(k\) è la dimensione della matrice \(\matr{P}\).

  Inoltre per la reversibilità della catena \(\bm{\pi} \) è anche l'unica distribuzione reversibile.

  Di conseguenza, \(\pi_i \matr{P}_{ij} = \pi_j \matr{P}_{ji}\) per ogni coppia di indici \(i,j\) il che implica \(\matr{P}_{ij} = \matr{P}_{ji}\).
\end{proof}

\end{document}