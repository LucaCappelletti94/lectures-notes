\providecommand{\main}{../..}
\documentclass[\main/main.tex]{subfiles}
\begin{document}

% \section{Proprietà di Grafi, matrici e vettori stocastici}
% \subsection{Cos'è una matrice stocastica?}
% \subsection{Cos'è una matrice primitiva?}
% \subsection{Proprietà di autovalori ed autovettori di una matrice stocastica}
% \subsection{Teoremi sulla periodicità}

\section{Matrici e grafi}
\subsection{Grafo orientato}
Si tratta di una coppia \(\rnd{\mathcal{V}, \mathcal{E}}\), dove \(\mathcal{V}\) rappresenta un insieme finito di elementi detti \textbf{nodi} (o vertici) mentre \(\mathcal{E}\) rappresenta un insieme ordinato di nodi che vengono chiamati \textbf{lati} (edges).

\subsection{Matrice associata a grafo}
Ogni matrice non negativa può essere associata a un grafo orientato nel quale i lati sono pesati con valori positivi.

\subsection{Cammino}
Si tratta di una sequenza di nodi del tipo \(c = {v_0, v_1, \ldots, v_n}\) non necessariamente distinti tali che la tupla \(\rnd{v_{i-1}, v_{v_i}} \in \mathcal{E}\). Se esiste un cammino da un nodo \(v_0\) a \(v_k\) la sua lunghezza è \(k\).

\subsection{Cammino semplice}
Un cammino è detto semplice se tutti i suoi nodi sono distinti tranne al più il primo e l'ultimo.

\subsection{Ciclo}
Si tratta di un cammino con lunghezza non nulla nel quale primo e ultimo nodo coincidono.

\subsection{Grafo fortemente connesso}
Una grafo in cui per ogni coppia di nodi esiste una arco.

\subsection{Componente fortemente connessa}
Un sotto-grafo formato da una classe \(C\) e dai lati di \(C_e\) che connettono tali nodi.

\subsection{Classe essenziale}
Una classe \(C\) è detta essenziale se \(\forall v_i \in C \quad v_i \rightarrow v_j \Rightarrow v_j \in C\).

Non è possibile uscire da una classe essenziale seguendo un cammino che porta ad uno dei suoi nodi.

\subsection{Nodo essenziale}
Ogni nodo appartenente ad una classe essenziale viene detto essenziale \(\Leftrightarrow \forall v_j: v_i \rightarrow v_j \Rightarrow v_j \rightarrow v_i\).

\subsection{Periodicità}
Dato un grado \(\mathcal{G}\), consideriamo \(v_i\), un nodo per il quale passa almeno un ciclo. Il \textbf{periodo} \(d(v_i)\) di \(v_i\) è il massimo comun divisore delle lunghezze dei cicli che passano per \(v_i\).

\begin{figure}
  \[
    d(v_i) = \mcd\crl{n \in \N: v_i\xrightarrow{n}v_j}
  \]
  \caption{Periodo di un nodo}
\end{figure}

Ogni nodo con un \textbf{auto-anello} è detto \textbf{aperiodico}. I nodi appartenenti alla stessa classe hanno lo stesso periodo.

\subsection{Matrice di adiacenza}
Matrice che assume solo valori binari: \(\rnd{0,1}\).

\subsection{P}
Si tratta di una matrice quadrata non negativa con grafo \(\mathcal{G}\) associato che \(\forall n \in \N, \forall \rnd{i,j}: a_{ij}^{\rnd{n}} = \sum \text{passi di tutti i cammini di lunghezza \(n\) da \(v_i\) a \(v_j\)}\).

Nelle matrici di \textbf{adiacenza} \(a_{ij}^{\rnd{n}}\) denota il numero di cammini da \(v_i \rightarrow v_j\) con lunghezza \(n\).

\subsection{Matrice irriducibile}
Una matrice non-negativa \(\matr{A}\) si dice \textbf{irriducibile} se \(\forall \rnd{i,j} \exists n \in \N: a_{ij}^n > 0\).

\subsection{Proprietà di una matrice irriducibile}
\begin{enumerate}
  \item \(\forall i = 1, \ldots, k \exists N(i): \forall n \in N(i) \quad a_{ij}^{\rnd{n}} > 0\).
  \item \(\forall i = 1, \ldots, k \exists r \in \N: 0 \leq r \leq d \land a_{ij}^{\rnd{s}}>0 \Rightarrow s = r\mod\rnd{d}\)
  \item \(\forall i,j = 1, \ldots, k \exists q(i,j) \in \N: \forall n \geq q a_{ij}^{\rnd{r+nd}}>0\)
\end{enumerate}

\subsection{Matrice riducibile}
Quando non è irriducibile, cioè se:
\begin{enumerate}
  \item Se e solo se il grafo associato possiede almeno due componenti totalmente connesse, o equivalentemente,
  \item Se e solo se mediante una permutazione di righe e delle colonne può essere trasformata in una matrice triangolare a blocchi.
\end{enumerate}

\subsection{Matrici primitive}
\begin{definition}[Matrice primitiva]
  Ogni matrice \(\matr{A} \in \R^{k\times k}_{+}\) è primitiva se e solo se:
  \begin{enumerate}
    \item È aperiodica.
    \item È irriducibile
  \end{enumerate}
\end{definition}

\begin{proof}[Matrice primitiva]
  Supponiamo che \(\matr{A}\) sia irriducibile e abbia periodo \(d=1\). Sappiamo che \(\forall i,j \in 1, \ldots, k \exists q(i,j) \in \N: a_{ij}^{(r+nd)} >0 \Rightarrow a_{ij}^{(n)} > 0\).
  Chiamiamo \(M = \max\crl{q(i,j): i,j = 1, \ldots, k} \Rightarrow \forall n > M, a_{ij}^{(n)} > 0 \forall i,j\). Questo significa che \(\matr{A}^n>0\) e quindi \(\matr{A}\) è primitiva.
\end{proof}

\subsection{Proprietà delle matrici irriducibili (Teorema di Perron-Frobenius)}

\begin{theorem}[Teorema di Perron-Frobenius]
  Sia \(\matr{A} \geq 0\) una matrice primitiva. Allora esiste un autovalore \(\lambda \) di \(\matr{A}\), detto autovalore di Perron-Frobenius, che gode delle seguenti proprietà:
  \begin{enumerate}
    \item \(\lambda > 0\)
    \item \(\forall \mu \neq \lambda \Leftrightarrow \abs{\mu} < \lambda \)
    \item \(\lambda \) ammette autovettori destri e sinistri strettamente positivi.
    \item \(\lambda \) è radice di \(\det\rnd{\matr{I}\bmx - \matr{A}}\)
  \end{enumerate}
\end{theorem}

\subsection{Matrice stocastica}
Una matrice è stocastica se e solo se:
\begin{enumerate}
  \item \(a_{ij} \geq 0\)
  \item \(\forall i, \sum_{i=1}^n a_{ij} = 1\)
\end{enumerate}

\subsection{Proprietà delle matrici stocastiche}
\begin{enumerate}
  \item Ogni matrice stocastica possiede autovalore \(1\), ed è il suo autovalore di
        Perron-Frobenius.
  \item La matrice risultante dal prodotto di due matrici stocastiche è stocastica.
  \item Una matrice è \textbf{bi-stocastica} se e solo se l'autovettore sinistro di \(\matr{A}\) corrisponde all'autovettore dell'autovalore\(1\):
        \[
          \sum_{i=1}^n = 1 \quad \land \quad \sum_{j=1}^n a_{ij} = 1
        \]
  \item Se \(\matr{A} > 0\) è stocastica e primitiva allora \(1\) è l'unico autovalore di modulo massimo per \(\matr{A}\).
  \item Se \(\mu \) è un autovettore di una matrice stocastica, allora \(\abs{\mu} < 1\).
        \begin{proof}
          \(\exists \bmv' = \begin{bmatrix}
            v_1, \ldots, v_k
          \end{bmatrix}\) autovettore sinistro di \(\matr{A}\) corrispondente a \(\mu, \bmv' \matr{A} = \mu \bmv'\):
          \[
            \mu \cdot \sum_{i=1}^k v_i = \sum_{i=1}^k \abs{\mu v_i} = \sum_{i=1}^k \abs{\sum_{j=1}^k v_j\cdot a_{ij}} = \sum_{i=1}^k \sum{j=1}^k \abs{v_j \cdot a_{ij}} = \sum_{j=1}^k \abs{v_j} \sum_{i=1}^k a_{ij} = \abs{\mu} \cdot \sum_{i=1}^k \abs{v_i} \leq \sum_{j=1} \abs{v_i} \leq \sum_{j=1}^k \abs{v_j} \Rightarrow \abs{\mu} \leq 1
          \]
        \end{proof}
\end{enumerate}

\end{document}