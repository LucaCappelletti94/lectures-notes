\providecommand{\main}{..}
\documentclass[\main/main.tex]{subfiles}
\begin{document}

\section{Algoritmo di Metropolis}
si tratta di una procedimento per generare a caso un elemento secondo una distribuzione fissata.

Sia \(S\) un insieme finito e \(\bm{\pi} \) una distribuzione definita su \(S\) tale che \(\pi(i) > 0 \, \forall i \in S\). Definiamo inoltre un grafo non orientato \(G = (S, E)\) che gode delle seguenti proprietà:
\begin{enumerate}
  \item \(S\) sia l'insieme dei nodi di \(G\).
  \item \(G\) sia connesso.
  \item Il grado di \(G\) sia limitato da una opportuna costante.
\end{enumerate}

Possiamo procedere a definire la matrice di transizione \(\matr{P}\) come:
\[
  p_{ij} = \begin{cases}
    0                                                                                & i \neq j \, \land \, \crl{i, j} \not\in E \\
    \frac{1}{d_i}\min\crl{\frac{\pi_j d_i}{\pi_di d_j}, 1}                           & \crl{i,j} \in E                           \\
    1-\frac{1}{d_i}\sum_{l:\crl{i,l} \in E}\min\crl{\frac{\pi_l d_i}{\pi_di d_l}, 1} & i==j
  \end{cases}
\]
Dove:
\begin{enumerate}
  \item Se i nodi non hanno un arco non vi è possibilità di transitarvi.
  \item Se i nodi hanno un arco tra loro, la probabilità è pari all'uniforme di entrare in \(i\), \(\frac{1}{d_i}\) moltiplicata per il rapporto dell'estrazione dell'elemento \(j\) e dell'elemento \(i\) limitato a 1, cioè \(\min\crl{\frac{\pi_j d_i}{\pi_i d_j}, 1}\).
  \item Se i nodi coincidono, la probabilità di rimanere nello stesso è \(1\) meno la somma delle probabilità di andare in uno qualsiasi dei nodi adiacenti.
\end{enumerate}
dove \(d_i\) è il grado del generico nodo \(i\).

La catena è \textbf{irriducibile} Poiché il grafo associato \(G\) è connesso, ed inoltre la distribuzione \(\bm{\pi} \) è reversibile per la catena, infatti:
\[
  \pi_i p_{ij} = \pi_j p_{ji}
\]

\subsection{La procedura}
Si inizia da un nodo \(i\) e si estrae a caso in modo uniforme un nodo \(l \in S\) tale che \(\crl{i,l} \in E\), quindi si confrontano le probabilità \(\pi_l d_i \geq \pi_i d_l\): se la disuguaglianza è vera, il prossimo nodo \(j=l\), altrimenti si sceglie a caso un nuovo nodo \(j \in {l, i}\) con le seguenti probabilità:
\[
  \prob{l} = \frac{\pi_l d_i}{\pi_i d_l} \qquad \prob{i} = 1- \prob{l}
\]

\end{document}