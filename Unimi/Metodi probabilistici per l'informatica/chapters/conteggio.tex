\providecommand{\main}{..}
\documentclass[\main/main.tex]{subfiles}
\begin{document}

\section{Conteggio approssimato di colorazioni}
\subsection{Cosa è un RPTAS}
\begin{definition}[RPTAS]
  Si definisce \textbf{RPTAS} uno schema di approssimazione probabilistico pienamente polinomiale per una funzione \(F: I \rightarrow \N \) è un algoritmo probabilistico che su input \(\rnd{x, \epsilon}\) che soddisfa le seguenti condizioni:
  \begin{enumerate}
    \item Per ogni \(\epsilon>0\) esiste un polinomio \(p_{\epsilon}(y)\) tale che l'algoritmo su input \(\rnd{x, \epsilon}\) richiede un tempo minore o uguale a \(p_{\epsilon}(n)\) dove \(n = \abs{x}\).
    \item Il valore \(R(x, \epsilon)\) restituito dall'algoritmo su input \(\rnd{x, \epsilon}\) verifica la relazione:
          \[
            \prob{\rnd{1-\epsilon}F(x) \leq R(x, \epsilon) \leq \rnd{1+\epsilon}F(x)} \geq \frac{2}{3}
          \]
  \end{enumerate}
\end{definition}

\subsection{Applicazione}
Si considera un \(G=(V, E)\) e da un numero \(q\) di colori. Denotiamo con \(k\) il numero dei nodi del grafo, \(d\) il suo grado ed \(m\) il numero dei suoi lati. La coppia \(\rnd{k, m}\) rappresenta la dimensione dell'istanza del problema.

Consideriamo l'insieme dei lati \(E = \crl{e_1, e_2, \ldots, e_m}\) del grafo e per ogni \(j \in \crl{0, 1, \ldots, m}\) definiamo:
\[
  G_j = \rnd{V, \crl{e_1, \ldots, e_j}} \quad \ldots \quad G_0 = \rnd{V, \emptyset}
\]
Denotiamo inoltre \(Z_j\) come il numero di \(q\)-colorazioni di \(G_j\), in particolare \(Z_0 = q^k\), poiché nel grafo \(G_0\) ogni assegnamento di colori ai nodi è una \(q\)-colorazione.

Il valore di \(Z_m\) che vogliamo calcolare può essere ottenuto mediante:
\[
  Z_m = Z_0 \cdot \frac{Z_1}{Z_0} \cdot \frac{Z_2}{Z_1} \cdots \frac{Z_m}{Z_{m-1}}
\]
Denotiamo ora con \(x_j\) e \(y_j\) i due nodi del lato \(e_j\), per ciascun \(j = \crl{1, 2, \ldots, m}\). Allora:
\[
  Z_j = \#\crl{\left. f \in Q^V \right\rvert f \text{ è }q-\text{colorazione di }G_{j-1}\text{ tale che }f(x_j)\neq f(y_j)}
\]
Quindi \(\frac{Z_j}{Z_{j-1}}\) è la frequenza di colorazioni di \(G_{j-1}\) che sono valide anche per \(G_j\).
Ne segue che:
\[
  \frac{Z_j}{Z_{j-1}} = \prob{f(x_j)\neq f(y_j)}
\]
\subsection{La sequenza}
Per ogni grafo \(G_{j-1}\) si generano diverse \(q\)-colorazioni casuali usando il metodo dei campionatore di Gibbs e si verifica quante di queste attribuiscono colori diversi ai nodi \(x_j\) e \(y_j\). Si ottiene così una stima \(Y_j\) della probabilità \(\frac{Z_j}{Z_{j-1}}\).

Moltiplicando tutti questi valori tra di loro insieme alla quantità \(Z_0 = q^k\) si ottiene \(R\), un'approssimazione di \(Z_m\):

\[
  R = q^k \prod_{j=1}^m Y_j = \prod_{j=1}^m \frac{H_j}{l}
\]
Dove \(H_j\) è il numero di \(q\)-colorazioni di \(G_{j-1}\) che attribuiscono colori diversi ai nodi \(x_j\) e \(y_j\)

Il procedimento descritto dipende da due parametri, \(l\) e \(n\):
\begin{description}
  \item[\(l\)] numero di \(q\)-colorazioni casuali di \(G_{j-1}\).
  \item[\(n\)] numero di passi principali eseguiti da ciascun campionatore di Gibbs per determinare la \(q\)-colorazione casuale corrispondente.
\end{description}

\begin{theorem}[RPTAS per \(q\)-colorazioni]
  Assumendo la notazione illustrata sopra, supponiamo che \(d\geq2\) e \(q>2d^2\). Allora esiste un \textbf{RPTAS} per il problema di determinare il numero minimo di colorazioni di \(G\).
\end{theorem}

\end{document}