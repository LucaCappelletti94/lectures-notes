\providecommand{\main}{..}
\documentclass[\main/main.tex]{subfiles}
\begin{document}

\section{Catene riducibili}
Nel caso di catene con un grado di connessione che contiene più componenti fortemente connesse non è possibile dare una risposta univoca ed il comportamento dipende dalla relazione di connessione tra le varie componenti e dalla periodicità dei loro elementi. È possibile provare che se esiste una sola classe essenziale e quest'ultima è aperiodica allora la catena è ergodica, viceversa allora la catena non può essere ergodica.

\begin{theorem}[Catene riducibili ergodiche]
  Se la catena di Markov \(\crl{M_k}\) possiede una sola classe essenziale \(C\) e quest'ultima è \textbf{aperiodica} allora, per ogni \(j \in C\) e ogni \(i \in S\):
  \[
    \lim_{n\rightarrow+\infty} p^{(n)}_{ij} = \frac{1}{\sum_{n\geq 1} n f^{(n)}_{ij}}
  \]
  La catena risulta pertanto \textbf{ergodica}.
\end{theorem}

\begin{theorem}[Relazione tra classi essenziali e catene ergodiche]
  Una catena di Markov finita è \textbf{ergodica} se e solo se possiede un'unica classe essenziale e quest'ultima è \textbf{aperiodica}.
\end{theorem}
\end{document}