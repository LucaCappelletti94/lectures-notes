\providecommand{\main}{..}
\documentclass[\main/main.tex]{subfiles}
\begin{document}

\subsection{Problemi complessi}

\begin{figure}[H]
	\[
		P = (X, \Omega, F, f, D, \Pi)
	\]
	\caption{Definizione formale di problema di decisione.}
\end{figure}

Queste variabili rappresentano:

\begin{enumerate}
	\item $X$ rappresenta l'insieme delle \textbf{alternative}, o delle \textbf{soluzioni} o anche delle \textbf{soluzioni ammissibili}.
	\item $\Omega$ rappresenta insieme degli \textbf{scenari} o \textbf{esiti}.
	\item $F$ rappresenta l'insieme degli \textbf{impatti}.
	\item $f$ rappresenta la \textbf{funzione dell'impatto}.
	\item $D$ rappresenta l'insieme dei \textbf{decisori}, tipicamente un insieme finito e di dimensione bassa. Un decisore è un'entità umana, modellata quanto possibile matematicamente.
	\item $\Pi$ insieme delle \textbf{preferenze}.
\end{enumerate}

$X$ viene definito come: \[X \subseteq \mathbb{R}^n \text{se } \bm{x} \in X \Rightarrow \bm{x} = \begin{bmatrix}x_1\\x_2\\...\\x_n \end{bmatrix}\]
con ogni termine $x_i$ viene chiamato o \textbf{elemento di alternativa} o \textbf{variabile di decisione}.

$\Omega$ viene definito come: \[\Omega \subseteq \mathbb{R}^r \text{se } \bm{\omega} \in \Omega \Rightarrow \bm{\omega} = \begin{bmatrix}\omega_1\\\omega_2\\...\\\omega_r \end{bmatrix}\]
con ogni termine $\omega_i$ viene chiamato o \textbf{elemento di scenario} o \textbf{variabile esogene}, cioè variabili che influiscono sulla configurazione del nostro sistema, non decise arbitrariamente ma provenienti dall'esterno.

$F$ viene definito come: \[F \subseteq \mathbb{R}^p \text{se } \bm{f} \in F \Rightarrow \bm{f} = \begin{bmatrix}f_1\\ f_2\\...\\ f_p \end{bmatrix}\]
Le $f_l \in \mathbb{R}$ vengono ipotizzate ad essere intere e vengono chiamate \textbf{indicatore}, \textbf{attributo}, \textbf{criterio} o \textbf{obbiettivo}. Un \textbf{indicatore} per esempio potrebbe essere un \textit{valore ottimo}.

La $f$ viene definita come: \[ f(\bm{x},\bm{\omega}): X\times\Omega \rightarrow F \]
La matrice di tutte le combinazioni viene chiamata \textbf{matrice delle valutazioni}.

La $\Pi$ viene definita come \[\Pi: D \rightarrow 2^{F\times F}\], dove $\pi_d \subseteq F\times F$. $F\times F$ rappresenta l'insieme delle \textbf{coppie ordinate di impatti}, mentre $2^{F \times F}$ rappresenta l'insieme delle \textbf{relazioni binarie}.

Per esempio, ponendo $F = \{f, f', f''\}$, otteniamo un prodotto cartesiano: \[F \times F = \{ (f,f'), (f,f''), (f', f), (f, f''), (f'', f), (f'', f'), (f, f), (f', f'), (f'', f'') \}\]

La \textbf{preferenza} è la volontà per cui il decisore risulta disponibile a fare uno scambio.

Un esempio di preferenza è: \[ f' \leq_d f' \Leftrightarrow (f', f'')\in \Pi_d \]. In un ambiente ingegneristico si usa il $\leq_d$, minimizzando i costi, mentre in un ambiente economico si cerca di massimizzare i costi $\geq_d$.

\begin{definition}[indifferenza]
Due preferenze $f'$ e $f''$ sono dette \textbf{indifferenti} quando:

\[
	f' ~ f'' \Leftrightarrow \begin{cases} f' \leq_d f'' \\ f' \geq_d f'' \end{cases}
\] 
\end{definition}

\begin{definition}[Preferenza Stretta]
Una preferenza $f'$ è detta \textbf{preferenza stretta} quando:

\[
	f' <_d f'' \Leftrightarrow \begin{cases} f' \leq_d f'' \\ f' \ngeq_d f'' \end{cases}
\] 
\end{definition}

\begin{definition}[Incomparabilità]
Due preferenze $f'$ e $f''$ sono dette \textbf{incomparabili} quando:

\[
	f' \Join_d f'' \Leftrightarrow \begin{cases} f' \nleq_d f'' \\ f' \ngeq_d f'' \end{cases}
\] 
\end{definition}


\subsection{Proprietà delle preferenze}
\subsubsection{Proprietà riflessiva}
\[
	f \leq f \qquad \forall f \in F
\]

\subsubsection{Proprietà di completezza}
Un decisore può sempre concludere una decisione (ipotesi molto forte che talvolta porta a risultati impossibili):
\[
	f \nleq f' \Rightarrow f' \leq f \qquad \forall f, f' \in F
\]

\subsubsection{Proprietà di anti-simmetria}
\[
	f \leq f' \wedge f' \leq f \Rightarrow f' = f \qquad \forall f, f' \in F
\]

\subsubsection{Proprietà Transitiva}
Solitamente i decisori non possiedono questa proprietà, anche perché è necessario modellare lo scorrere del tempo, per cui le proprietà valgono potenzialmente solo in un determinato periodo temporale.
\[
	f \leq f' \wedge f' \leq f'' \Rightarrow f \leq f'' \qquad \forall f, f', f'' \in F
\]

\subsection{Ipotesi funzione del valore}
Un decisore che ha in mente una funzione valore $v$, ha in mente una relazione di preferenza $\Pi$ \textbf{riflessiva}, \textbf{completa}, \textbf{non necessariamente anti simmetrica} e \textbf{transitiva}. Quando una relazione possiede queste proprietà viene chiamata \textbf{ordine debole}, debole perché possono esiste dei \textit{pari merito}. Un campo di applicazione sono i campionati sportivi.
\[
	\exists v: F\rightarrow \mathbb{R}: f \leq f' \Leftrightarrow v_{(f)} \geq v_{(f')}
\]

\end{document}