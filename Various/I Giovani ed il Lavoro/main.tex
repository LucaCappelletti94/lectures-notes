\providecommand{\main}{.}
\documentclass{report} % We use report to allow for chapers etc...
\usepackage[utf8]{inputenc} % This allows for utf support.
\usepackage[T1]{fontenc}  % Defines true type fonts
\usepackage[italian, english]{babel}   % Set up supported languages. Last one is default.
\usepackage{csquotes} % Package required by babel
\usepackage{geometry} % More rich support for page layout.
\geometry{
 a4paper,
 total={170mm,250mm},
 left=20mm,
 top=25mm,
 }
\usepackage{emptypage} % When a page is empty, Latex won't generate page number or other page elements.
\usepackage{multicol} % For the possibility of using columns with  \begin{multicols}{n}.
\usepackage[colorlinks=true,urlcolor=blue,pdfpagelabels,hyperindex=false]{hyperref}  % Enable table of contents and links.
\usepackage{microtype} % for automatic micro fitting of characters
\usepackage{amsmath} % for math rendering
\usepackage{float} % to enable floating graphics
\usepackage{graphicx} % for images and generally graphics
\usepackage{caption} % enabling of nice captions
\usepackage{subcaption} % and subcaptions of images
\graphicspath{ {images/} } % We'll put all the images in the folder "images"
\usepackage[style=authoryear,sorting=ynt, backend=biber]{biblatex} % Package to handle the bibliography
\nocite{*} % This allows for having entried in the bib file that do not have to be necesseraly used
\usepackage{subfiles} % To use subfiles without cruxifying saints
\usepackage{chngcntr} % Has functions to reset counters

\usepackage{parskip} % To leave spaces in paragraphs
\usepackage{bm} % To have bold vectors

\usepackage{xcolor} % To highlight text
\usepackage[cache=false]{minted} % For highlighting code
\definecolor{mintedbackground}{rgb}{0.95,0.95,0.95}
\setminted{
bgcolor=mintedbackground,
fontfamily=tt,
linenos=true,
numberblanklines=true,
numbersep=5pt,
gobble=0,
frame=leftline,
framerule=0.4pt,
framesep=2mm,
funcnamehighlighting=true,
tabsize=4,
obeytabs=false,
mathescape=false
samepage=false, %with this setting you can force the list to appear on the same page
showspaces=false,
showtabs =false,
texcl=false,
}

\usepackage{paralist} % For compacted enumerations
\usepackage[automake,style=long,nonumberlist,toc,xindy,acronym,nomain]{glossaries} % for glossaries and acronyms
\makeglossaries % Enables the package above

\usepackage{imakeidx} % instead of makeidx, so you don't need to run MakeIndex
\makeindex[program=makeindex,columns=2,intoc=true,options={-s ../../general/pyro.ist}] % Enables the package above
\indexsetup{firstpagestyle=empty, othercode=\small} % No page number in the first page of analytical index

\usepackage{bookmark} % Enables bookmarks in generated pdf

\usepackage{fourier} % For icons such as \bomb, \noway, \danger and various others. For more info, go here: http://ctan.mirror.garr.it/mirrors/CTAN/fonts/fourier-GUT/doc/latex/fourier/fourier-orns.pdf

\usepackage{mathtools} % To use custom delimiters that Latex did not implement cos duh

\usepackage{fancyhdr} % This allows for the headings in the chapters
\pagestyle{fancy} % This activates it

\DeclarePairedDelimiter\abs{\lvert}{\rvert}%
\DeclarePairedDelimiter\norm{\lVert}{\rVert}%

\newtheorem{theorem}{Theorem}[section]
\newtheorem{corollary}{Corollary}[theorem]
\newtheorem{lemma}[theorem]{Lemma}
\newtheorem{definition}[theorem]{Definizione}

\usepackage{enumitem,amssymb} % For a todo check list
\newlist{todolist}{itemize}{2} % For a  todo check list
\setlist[todolist]{label=$\square$}  % For a todo check list

\counterwithin*{equation}{section}% Reset equation at \section
\counterwithin*{equation}{subsection}% Reset equation at \subsection
%\counterwithin*{equation}{subsubsection}% Reset equation at \subsubsection // Doesn't work for god knows why.

%%%%%%%%%%%%%%%%%%%%%%%%%
% IMPORTANT NOTE ON BIBLIOGRAPHY USE %
%%%%%%%%%%%%%%%%%%%%%%%%%

% To use the bibliography, run BibTex on this file. If everything is correct, it will terminate with no errors.
% The warning "Using fall-back BibTeX(8) backend: (biblatex) functionality may be reduced/unavailable." is expected.

\newacronym{fol}{FOL}{First Order Logic}

\begin{document}

\hypersetup{pageanchor=false}
\begin{titlepage}
	\begin{center}
		\ifLuaTeX
			\directlua{dofile(deapness.."/general/lua/title.lua")}
		\else
			\large{TITLEPAGE NOT RENDERED!\\RECOMPILE WITH LUATEX!}
		\fi
	\end{center}
\end{titlepage}
\hypersetup{pageanchor=true}

{\hypersetup{hidelinks}
  \tableofcontents  % Generates the table of contents
}

\chapter{I Giovani e il Lavoro}
Guardare per ``Laboratorio delle idee'', una servizio per verificare se un'idea sia sensata o meno.

\section{Punto nuova impresa}
Si tratta di un servizio per aiutare l'avvio di imprese nei primi 3 anni di vita.

\section{Business Plan}

\begin{tabular}{|c|c|c|c|}
  \hline
  \textbf{Parte Introduttiva} & \textbf{Parte di analisi del mercato} & \textbf{Parte tecnico operativa} & \textbf{Parte Monetaria}         \\
  \hline
  Descrizione dell'idea       & Analisi di fattibilità                & Analisi organizzazione           & Previsione economico finanziaria \\
  \hline
  Attidudini                  & Macroambiente                         & Attività organizzativa           & Contabilità                      \\
  Aspirazioni                 & Mercato                               & Logistica                        & Bilancio preventivo              \\
  Motivazioni                 & Settore                               & Localizzazione                   &                                  \\
  Creatività                  & Concorrenza                           & Procedure burocratiche           &                                  \\
  Studi ed esperienze         & Clienti                               & Forma Giuridica                  &                                  \\
                              & Prodotti                              &                                  &                                  \\
                              & Punti di forza e debolezze            &                                  &                                  \\
                              & Piano di marketing e marketing mix    &                                  &                                  \\
                              & Piano commerciale                     &                                  &                                  \\
  \hline
\end{tabular}

\subsection{Dove trovare dati e informazioni sulle imprese esistenti}
Le union camere / camere di commercio sono buone fonti di dati.

\subsection{Per informazioni sui clienti}
Fare richieste all'istat.

\subsection{Concorrenza}
Quando non si ha concorrenza bisogna sempre chiedersi:

\begin{enumerate}
  \item Hanno fallito altre imprese? Perchè? Il mercato era acerbo?
  \item Ora il mercato è maturo?
  \item Perchè nessuno ci ha pensato?
\end{enumerate}

Nelle camere di commercio è anche possibile verificare la presenza di società concorrenti ed identificare elementi di differenza tra questi e la propria azienda.

\subsection{Analisi organizzazione}
Si tratta della parte che si occupa di coprire il ``come'' l'organizzazione è avviata ed il progetto gestito. In questa fase è necessario quantificare i costi, definire la neccessità di collaboratori e dipendenti (quindi analisi del costo del lavoro) e delle competenze specifiche di cui si avranno bisogno. Inoltre esiste anche l'aspetto delle norme che vincolano l'apertura di alcune specifiche aziende (tabaccherie, bar).

\subsubsection{Forma giuridica}
L'apertura della partita IVA viene effettuata all'agenzia delle entrate mentre quando uno si iscrive in camera di commercia e quindi diventa un'impresa deve andare a scegliere una forma giuridica.

\paragraph{Società di capitale}
Diviene preferibile quando è alto il rischio di perdere denaro.

\paragraph{Impresa individuale}
Viene considerata quando si è un'unica persona che rappresenta l'azienda. I fornitori possono rifarsi sul patrimonio personale, però è estramamente snella e costa solo €150 euro ad aprirsi. Tendenzialmente ha costi limitati di gestione.

\paragraph{Società a responsabilità limitata unipersonale}
Viene considerata quando si è un'unica persona che rappresenta l'azienda. Viene creato un capitale aziendale su cui i fornitori si possono fare rivalere. Questi saranno i soldi con cui posso investire. Nelle SRL semplificate il capitale sociale può essere di 1 euro, ma è praticamente un'idiozia. Costa molto farla (dai €1500 ai €10000) e costa molto mantenerla, e sono costi legati ad un commercialista che deve realizzare e depositare i bilanci. Le startup innovative possono nascere solo come società di capitali.

\paragraph{Società in accomandita semplice}
Deve esserci almeno un socio che lavora e poi uno o più soci accomandanti che forniscono soldi / brevetti ma non lavorano a pieno regime nell'attività.

L'iter tende ad essere lungo (qualche settimana) perchè coinvolge un notaio e commercialista.

\subsubsection{Procedure burocratiche}
Per creare un'impresa individuale basta procedere tramite la \textbf{comunicazione unica} (costo €150) che realizza partita IVA, iscrizione agli albi, apertura posizione INAIL e INPS ed eventuale SCIA in comune.

\subsection{Parte monetaria}
Un conto economico deve arrivare a parità nei primi 3 anni di attività.

\textbf{Il flusso di cassa} invece si occupa di definire, mese per mese, come va l'azienda e serve e individuare delle \textbf{stagionalità del mercato}.

Bisogna pensare a come i clienti pagano e come vengono pagati i fornitori ed identificare come varia la liquidità dell'azienda, negoziando la liquidità in uscita con i fornitori.

\subsubsection{Finanziamenti}
Deve esserci almeno un 40\% del capitale che deriva dal capitale personale o da fondi perduti (aiuti genitoriali), altrimenti i finanziamenti bancari non esistono. Quando si hanno buone basi, i finanziamenti bancari si ricevono in meno di un mese. La finanza pubblica invece tende ad avere tempi biblici e sono offerti tramite bandi.

\section{Procedure}
Si possono andare a vedere le procedure per aprire un iter di impresa alla pagina \href{milomb.camcom.it}{milomb}.

\section{Chiusura d'impresa}
I costi della chiusura di un'impresa sono paragonabili ai costi di apertura, forse un po' ridotti.

\section{Intraprendo}
Si tratta di un'iniziativa diretta a aziende under 35, over 50 o a elevata innovatività.


\input{\main/../../general/footer.tex}
\end{document}
